\documentclass[twocolumn,a4j,10pt]{ltjtarticle}
\title{所得税法施行令}
\author{}
\date{}
\renewcommand{\baselinestretch}{0.8}
\setlength{\columnseprule}{.4pt}
\usepackage{enumitem}
\setlist[description]{topsep=3pt,parsep=0pt,partopsep=0pt,itemsep=3pt,leftmargin=10pt,labelsep=5pt,labelsep=10pt}
\makeatletter
\newcommand{\subsubsubsection}{\@startsection{paragraph}{4}{\z@}%
  {1.0\Cvs \@plus.5\Cdp \@minus.2\Cdp}%
  {.1\Cvs \@plus.3\Cdp}%
  {\reset@font\sffamily\normalsize}
}
\makeatother
\setcounter{secnumdepth}{4}
\begin{document}
\maketitle
\tableofcontents
\part*{第一編 総則}
\addcontentsline{toc}{part}{第一編 総則}
\section*{第一章 通則}
\addcontentsline{toc}{section}{第一章 通則}
\noindent\hspace{10pt}(定義)
\begin{description}
\item[第一条]この政令において「国内」、「国外」、「居住者」、「非永住者」、「非居住者」、「内国法人」、「外国法人」、「人格のない社団等」、「株主等」、「法人課税信託」、「恒久的施設」、「公社債」、「預貯金」、「合同運用信託」、「貸付信託」、「投資信託」、「証券投資信託」、「オープン型の証券投資信託」、「公社債投資信託」、「公社債等運用投資信託」、「公募公社債等運用投資信託」、「特定目的信託」、「特定受益証券発行信託」、「棚卸資産」、「有価証券」、「固定資産」、「減価償却資産」、「繰延資産」、「各種所得」、「各種所得の金額」、「変動所得」、「臨時所得」、「純損失の金額」、「雑損失の金額」、「災害」、「障害者」、「特別障害者」、「寡婦」、「寡夫」、「勤労学生」、「同一生計配偶者」、「控除対象配偶者」、「源泉控除対象配偶者」、「扶養親族」、「控除対象扶養親族」、「特別農業所得者」、「予定納税額」、「確定申告書」、「期限後申告書」、「修正申告書」、「青色申告書」、「更正請求書」、「確定申告期限」、「出国」、「更正」、「決定」、「源泉徴収」、「附帯税」、「充当」又は「還付加算金」とは、それぞれ所得税法(以下「法」という。)第二条第一項(定義)に規定する国内、国外、居住者、非永住者、非居住者、内国法人、外国法人、人格のない社団等、株主等、法人課税信託、恒久的施設、公社債、預貯金、合同運用信託、貸付信託、投資信託、証券投資信託、オープン型の証券投資信託、公社債投資信託、公社債等運用投資信託、公募公社債等運用投資信託、特定目的信託、特定受益証券発行信託、棚卸資産、有価証券、固定資産、減価償却資産、繰延資産、各種所得、各種所得の金額、変動所得、臨時所得、純損失の金額、雑損失の金額、災害、障害者、特別障害者、寡婦、寡夫、勤労学生、同一生計配偶者、控除対象配偶者、源泉控除対象配偶者、扶養親族、控除対象扶養親族、特別農業所得者、予定納税額、確定申告書、期限後申告書、修正申告書、青色申告書、更正請求書、確定申告期限、出国、更正、決定、源泉徴収、附帯税、充当又は還付加算金をいう。
\item[\rensuji{2}]この政令において、次の各号に掲げる用語の意義は、当該各号に定めるところによる。
\begin{description}
\item[一]利子所得、配当所得、不動産所得、事業所得、給与所得、退職所得、山林所得、譲渡所得、一時所得又は雑所得
\item[二]利子所得の金額、配当所得の金額、不動産所得の金額、事業所得の金額、給与所得の金額、退職所得の金額、山林所得の金額、譲渡所得の金額、一時所得の金額又は雑所得の金額
\item[三]総所得金額、退職所得金額又は山林所得金額
\item[四]雑損控除、医療費控除、社会保険料控除、小規模企業共済等掛金控除、生命保険料控除、地震保険料控除、寄附金控除、障害者控除、寡婦(寡夫)控除、勤労学生控除、配偶者控除、配偶者特別控除、扶養控除又は基礎控除
\item[五]課税総所得金額、課税退職所得金額又は課税山林所得金額
\item[六]配当控除、分配時調整外国税相当額控除又は外国税額控除
\item[七]予定納税基準額又は申告納税見積額
\end{description}
\item[\rensuji{3}]この政令において、「相続人」には、包括受遺者を含むものとし、「被相続人」には、包括遺贈者を含むものとする。
\end{description}
\noindent\hspace{10pt}(恒久的施設の範囲)
\begin{description}
\item[第一条の二]法第二条第一項第八号の四イ(定義)に規定する政令で定める場所は、国内にある次に掲げる場所とする。
\begin{description}
\item[一]事業の管理を行う場所、支店、事務所、工場又は作業場
\item[二]鉱山、石油又は天然ガスの坑井、採石場その他の天然資源を採取する場所
\item[三]その他事業を行う一定の場所
\end{description}
\item[\rensuji{2}]法第二条第一項第八号の四ロに規定する政令で定めるものは、非居住者又は外国法人の国内にある長期建設工事現場等(非居住者又は外国法人が国内において長期建設工事等(建設若しくは据付けの工事又はこれらの指揮監督の役務の提供で一年を超えて行われるものをいう。以下この項及び第六項において同じ。)を行う場所をいい、非居住者又は外国法人の国内における長期建設工事等を含む。第六項において同じ。)とする。
\item[\rensuji{3}]前項の場合において、二以上に分割をして建設若しくは据付けの工事又はこれらの指揮監督の役務の提供(以下この項及び第五項において「建設工事等」という。)に係る契約が締結されたことにより前項の非居住者又は外国法人の国内における当該分割後の契約に係る建設工事等(以下この項において「契約分割後建設工事等」という。)が一年を超えて行われないこととなつたとき(当該契約分割後建設工事等を行う場所(当該契約分割後建設工事等を含む。)を前項に規定する長期建設工事現場等に該当しないこととすることが当該分割の主たる目的の一つであつたと認められるときに限る。)における当該契約分割後建設工事等が一年を超えて行われるものであるかどうかの判定は、当該契約分割後建設工事等の期間に国内における当該分割後の他の契約に係る建設工事等の期間(当該契約分割後建設工事等の期間と重複する期間を除く。)を加算した期間により行うものとする。
\item[\rensuji{4}]非居住者又は外国法人の国内における次の各号に掲げる活動の区分に応じ当該各号に定める場所(当該各号に掲げる活動を含む。)は、第一項に規定する政令で定める場所及び第二項に規定する政令で定めるものに含まれないものとする。
\begin{description}
\item[一]当該非居住者又は外国法人に属する物品又は商品の保管、展示又は引渡しのためにのみ施設を使用すること
\item[二]当該非居住者又は外国法人に属する物品又は商品の在庫を保管、展示又は引渡しのためにのみ保有すること
\item[三]当該非居住者又は外国法人に属する物品又は商品の在庫を事業を行う他の者による加工のためにのみ保有すること
\item[四]その事業のために物品若しくは商品を購入し、又は情報を収集することのみを目的として、第一項各号に掲げる場所を保有すること
\item[五]その事業のために前各号に掲げる活動以外の活動を行うことのみを目的として、第一項各号に掲げる場所を保有すること
\item[六]第一号から第四号までに掲げる活動及び当該活動以外の活動を組み合わせた活動を行うことのみを目的として、第一項各号に掲げる場所を保有すること
\end{description}
\item[\rensuji{5}]前項の規定は、次に掲げる場所については、適用しない。
\begin{description}
\item[一]第一項各号に掲げる場所(国内にあるものに限る。以下この項において「事業を行う一定の場所」という。)を使用し、又は保有する前項の非居住者又は外国法人が当該事業を行う一定の場所において事業上の活動を行う場合において、次に掲げる要件のいずれかに該当するとき(当該非居住者又は外国法人が当該事業を行う一定の場所において行う事業上の活動及び当該非居住者又は外国法人(国内において当該非居住者又は外国法人に代わつて活動をする場合における当該活動をする者を含む。)が当該事業を行う一定の場所以外の場所(国内にあるものに限る。イ及び第三号において「他の場所」という。)において行う事業上の活動(ロにおいて「細分化活動」という。)が一体的な業務の一部として補完的な機能を果たすときに限る。)における当該事業を行う一定の場所
\begin{description}
\item[イ]当該他の場所(当該他の場所において当該非居住者又は外国法人が行う建設工事等及び当該活動をする者を含む。)が当該非居住者又は外国法人の恒久的施設に該当すること。
\item[ロ]当該細分化活動の組合せによる活動の全体がその事業の遂行にとつて準備的又は補助的な性格のものでないこと。
\end{description}
\item[二]事業を行う一定の場所を使用し、又は保有する前項の非居住者又は外国法人及び当該非居住者又は外国法人と特殊の関係にある者(国内において当該者に代わつて活動をする場合における当該活動をする者(イ及び次号イにおいて「代理人」という。)を含む。以下この項において「関連者」という。)が当該事業を行う一定の場所において事業上の活動を行う場合において、次に掲げる要件のいずれかに該当するとき(当該非居住者又は外国法人及び当該関連者が当該事業を行う一定の場所において行う事業上の活動(ロにおいて「細分化活動」という。)がこれらの者による一体的な業務の一部として補完的な機能を果たすときに限る。)における当該事業を行う一定の場所
\begin{description}
\item[イ]当該事業を行う一定の場所(当該事業を行う一定の場所において当該関連者(代理人を除く。イにおいて同じ。)が行う建設工事等及び当該関連者に係る代理人を含む。)が当該関連者の恒久的施設(当該関連者が居住者又は内国法人である場合にあつては、恒久的施設に相当するもの)に該当すること。
\item[ロ]当該細分化活動の組合せによる活動の全体が当該非居住者又は外国法人の事業の遂行にとつて準備的又は補助的な性格のものでないこと。
\end{description}
\item[三]事業を行う一定の場所を使用し、又は保有する前項の非居住者又は外国法人が当該事業を行う一定の場所において事業上の活動を行う場合で、かつ、当該非居住者又は外国法人に係る関連者が他の場所において事業上の活動を行う場合において、次に掲げる要件のいずれかに該当するとき(当該非居住者又は外国法人が当該事業を行う一定の場所において行う事業上の活動及び当該関連者が当該他の場所において行う事業上の活動(ロにおいて「細分化活動」という。)がこれらの者による一体的な業務の一部として補完的な機能を果たすときに限る。)における当該事業を行う一定の場所
\begin{description}
\item[イ]当該他の場所(当該他の場所において当該関連者(代理人を除く。イにおいて同じ。)が行う建設工事等及び当該関連者に係る代理人を含む。)が当該関連者の恒久的施設(当該関連者が居住者又は内国法人である場合にあつては、恒久的施設に相当するもの)に該当すること。
\item[ロ]当該細分化活動の組合せによる活動の全体が当該非居住者又は外国法人の事業の遂行にとつて準備的又は補助的な性格のものでないこと。
\end{description}
\end{description}
\item[\rensuji{6}]非居住者又は外国法人が長期建設工事現場等を有する場合には、当該長期建設工事現場等は第四項第四号から第六号までに規定する第一項各号に掲げる場所と、当該長期建設工事現場等に係る長期建設工事等を行う場所(当該長期建設工事等を含む。)は前項各号に規定する事業を行う一定の場所と、当該長期建設工事現場等を有する非居住者又は外国法人は同項各号に規定する事業を行う一定の場所を使用し、又は保有する第四項の非居住者又は外国法人と、当該長期建設工事等を行う場所において事業上の活動を行う場合(当該長期建設工事等を行う場合を含む。)は前項各号に規定する事業を行う一定の場所において事業上の活動を行う場合と、当該長期建設工事等を行う場所において行う事業上の活動(当該長期建設工事等を含む。)は同項各号に規定する事業を行う一定の場所において行う事業上の活動とそれぞれみなして、前二項の規定を適用する。
\item[\rensuji{7}]法第二条第一項第八号の四ハに規定する政令で定める者は、国内において非居住者又は外国法人に代わつて、その事業に関し、反復して次に掲げる契約を締結し、又は当該非居住者若しくは外国法人によつて重要な修正が行われることなく日常的に締結される次に掲げる契約の締結のために反復して主要な役割を果たす者(当該者の国内における当該非居住者又は外国法人に代わつて行う活動(当該活動が複数の活動を組み合わせたものである場合にあつては、その組合せによる活動の全体)が、当該非居住者又は外国法人の事業の遂行にとつて準備的又は補助的な性格のもの(当該非居住者又は外国法人に代わつて行う活動を第五項各号の非居住者又は外国法人が同項各号の事業を行う一定の場所において行う事業上の活動とみなして同項の規定を適用した場合に同項の規定により当該事業を行う一定の場所につき第四項の規定を適用しないこととされるときにおける当該活動を除く。)のみである場合における当該者を除く。次項において「契約締結代理人等」という。)とする。
\begin{description}
\item[一]当該非居住者又は外国法人の名において締結される契約
\item[二]当該非居住者又は外国法人が所有し、又は使用の権利を有する財産について、所有権を移転し、又は使用の権利を与えるための契約
\item[三]当該非居住者又は外国法人による役務の提供のための契約
\end{description}
\item[\rensuji{8}]国内において非居住者又は外国法人に代わつて行動する者が、その事業に係る業務を、当該非居住者又は外国法人に対し独立して行い、かつ、通常の方法により行う場合には、当該者は、契約締結代理人等に含まれないものとする。
\item[\rensuji{9}]第五項第二号及び前項ただし書に規定する特殊の関係とは、一方の者が他方の法人の発行済株式(投資信託及び投資法人に関する法律(昭和二十六年法律第百九十八号)第二条第十二項(定義)に規定する投資法人にあつては、発行済みの投資口(同条第十四項に規定する投資口をいう。以下この項において同じ。))又は出資(当該他方の法人が有する自己の株式(投資口を含む。以下この項において同じ。)又は出資を除く。)の総数又は総額の百分の五十を超える数又は金額の株式又は出資を直接又は間接に保有する関係その他の財務省令で定める特殊の関係をいう。
\end{description}
\noindent\hspace{10pt}(預貯金の範囲)
\begin{description}
\item[第二条]法第二条第一項第十号(預貯金の意義)の預貯金は、銀行その他の金融機関に対する預金及び貯金のほか、次に掲げるものとする。
\begin{description}
\item[一]労働基準法(昭和二十二年法律第四十九号)第十八条(貯蓄金の管理等)又は船員法(昭和二十二年法律第百号)第三十四条(貯蓄金の管理等)の規定により管理される労働者又は船員の貯蓄金
\item[二]国家公務員共済組合法(昭和三十三年法律第百二十八号)第九十八条(福祉事業)若しくは地方公務員等共済組合法(昭和三十七年法律第百五十二号)第百十二条第一項(福祉事業)に規定する組合に対する組合員の貯金又は私立学校教職員共済法(昭和二十八年法律第二百四十五号)第二十六条第一項(福祉事業)に規定する事業団に対する加入者の貯金
\item[三]金融商品取引法(昭和二十三年法律第二十五号)第二条第九項(定義)に規定する金融商品取引業者(同法第二十八条第一項(通則)に規定する第一種金融商品取引業を行う者に限る。)に対する預託金で、勤労者財産形成促進法(昭和四十六年法律第九十二号)第六条第一項、第二項又は第四項(勤労者財産形成貯蓄契約等)に規定する勤労者財産形成貯蓄契約、勤労者財産形成年金貯蓄契約又は勤労者財産形成住宅貯蓄契約に基づく有価証券の購入のためのもの
\end{description}
\end{description}
\noindent\hspace{10pt}(委託者が実質的に多数でない信託)
\begin{description}
\item[第二条の二]法第二条第一項第十一号(合同運用信託の意義)に規定する政令で定める信託は、信託の効力が生じた時において、当該信託の委託者(当該信託の委託者となると見込まれる者を含む。以下この項において同じ。)の全部が委託者の一人(以下この項において「判定対象委託者」という。)及び次に掲げる者である場合(当該信託の委託者の全部が信託財産に属する資産のみを当該信託に信託する場合を除く。)における当該信託とする。
\begin{description}
\item[一]次に掲げる個人
\begin{description}
\item[イ]当該判定対象委託者の親族
\item[ロ]当該判定対象委託者と婚姻の届出をしていないが事実上婚姻関係と同様の事情にある者
\item[ハ]当該判定対象委託者の使用人
\item[ニ]イからハまでに掲げる者以外の者で当該判定対象委託者から受ける金銭その他の資産によつて生計を維持しているもの
\item[ホ]ロからニまでに掲げる者と生計を一にするこれらの者の親族
\end{description}
\item[二]当該判定対象委託者と他の者との間にいずれか一方の者(当該者が個人である場合には、これと法人税法施行令(昭和四十年政令第九十七号)第四条第一項(同族関係者の範囲)に規定する特殊の関係のある個人を含む。)が他方の者(法人に限る。)を直接又は間接に支配する関係がある場合における当該他の者
\item[三]当該判定対象委託者と他の者(法人に限る。)との間に同一の者(当該者が個人である場合には、これと法人税法施行令第四条第一項に規定する特殊の関係のある個人を含む。)が当該判定対象委託者及び当該他の者を直接又は間接に支配する関係がある場合における当該他の者
\end{description}
\item[\rensuji{2}]前項第二号又は第三号に規定する直接又は間接に支配する関係とは、一方の者と他方の者との間に当該他方の者が次に掲げる法人に該当する関係がある場合における当該関係をいう。
\begin{description}
\item[一]当該一方の者が法人を支配している場合(法人税法施行令第十四条の二第二項第一号(委託者が実質的に多数でない信託)に規定する法人を支配している場合をいう。)における当該法人
\item[二]前号若しくは次号に掲げる法人又は当該一方の者及び前号若しくは次号に掲げる法人が他の法人を支配している場合(法人税法施行令第十四条の二第二項第二号に規定する他の法人を支配している場合をいう。)における当該他の法人
\item[三]前号に掲げる法人又は当該一方の者及び同号に掲げる法人が他の法人を支配している場合(法人税法施行令第十四条の二第二項第三号に規定する他の法人を支配している場合をいう。)における当該他の法人
\end{description}
\end{description}
\noindent\hspace{10pt}(公社債等運用投資信託の範囲等)
\begin{description}
\item[第二条の三]法第二条第一項第十五号の二(公社債等運用投資信託の意義)に規定する政令で定める資産は、次に掲げる資産とする。
\begin{description}
\item[一]公社債
\item[二]手形
\item[三]金銭債権(民法(明治二十九年法律第八十九号)第三編第一章第七節第一款(指図証券)に規定する指図証券、同節第二款(記名式所持人払証券)に規定する記名式所持人払証券、同節第三款(その他の記名証券)に規定するその他の記名証券及び同節第四款(無記名証券)に規定する無記名証券に係る債権並びに電子記録債権法(平成十九年法律第百二号)第二条第一項(定義)に規定する電子記録債権を除く。)
\item[四]合同運用信託
\end{description}
\item[\rensuji{2}]法第二条第一項第十五号の二に規定する政令で定めるものは、証券投資信託以外の投資信託のうち次に掲げる要件を満たすものとする。
\begin{description}
\item[一]その信託財産を前項第一号から第三号までに掲げる資産に対する投資として運用することを目的とする投資信託で、その信託財産を同項各号に掲げる資産にのみ運用するものであること。
\item[二]当該投資信託の投資信託約款(投資信託及び投資法人に関する法律第四条第一項(投資信託契約の締結)に規定する委託者指図型投資信託約款又は同法第四十九条第一項(投資信託契約の締結)に規定する委託者非指図型投資信託約款をいう。次条において同じ。)その他これに類する書類に当該投資信託が前号に規定する投資信託である旨の定めがあること。
\end{description}
\end{description}
\noindent\hspace{10pt}(公募の要件)
\begin{description}
\item[第二条の四]法第二条第一項第十五号の三(公募公社債等運用投資信託の意義)に規定する政令で定める取得勧誘は、同号の受益権の募集が国内において行われる場合にあつては、当該募集に係る金融商品取引法第二条第三項(定義)に規定する取得勧誘(以下この条において「取得勧誘」という。)が同項第一号に掲げる場合に該当し、かつ、投資信託約款にその取得勧誘が同号に掲げる場合に該当するものである旨の記載がなされて行われるものとし、当該受益権の募集が国外において行われる場合にあつては、当該募集に係る取得勧誘が同号に掲げる場合に該当するものに相当するものであり、かつ、目論見書(同法第二条第十項に規定する目論見書をいう。)その他これに類する書類にその取得勧誘が同号に掲げる場合に該当するものに相当するものである旨の記載がなされて行われるものとする。
\end{description}
\noindent\hspace{10pt}(棚卸資産の範囲)
\begin{description}
\item[第三条]法第二条第一項第十六号(棚卸資産の意義)に規定する政令で定める資産は、次に掲げる資産とする。
\begin{description}
\item[一]商品又は製品(副産物及び作業くずを含む。)
\item[二]半製品
\item[三]仕掛品(半成工事を含む。)
\item[四]主要原材料
\item[五]補助原材料
\item[六]消耗品で貯蔵中のもの
\item[七]前各号に掲げる資産に準ずるもの
\end{description}
\end{description}
\noindent\hspace{10pt}(有価証券に準ずるものの範囲)
\begin{description}
\item[第四条]法第二条第一項第十七号(有価証券の意義)に規定する政令で定める有価証券は、次に掲げるものとする。
\begin{description}
\item[一]金融商品取引法第二条第一項第一号から第十五号まで(定義)に掲げる有価証券及び同項第十七号に掲げる有価証券(同項第十六号に掲げる有価証券の性質を有するものを除く。)に表示されるべき権利(これらの有価証券が発行されていないものに限る。)
\item[二]合名会社、合資会社又は合同会社の社員の持分、法人税法(昭和四十年法律第三十四号)第二条第七号(定義)に規定する協同組合等の組合員又は会員の持分その他法人の出資者の持分
\item[三]株主又は投資主(投資信託及び投資法人に関する法律第二条第十六項(定義)に規定する投資主をいう。)となる権利、優先出資者(協同組織金融機関の優先出資に関する法律(平成五年法律第四十四号)第十三条第一項(優先出資者となる時期等)の優先出資者をいう。)となる権利、特定社員(資産の流動化に関する法律(平成十年法律第百五号)第二条第五項(定義)に規定する特定社員をいう。)又は優先出資社員(同法第二十六条(社員)に規定する優先出資社員をいう。)となる権利その他法人の出資者となる権利
\end{description}
\end{description}
\noindent\hspace{10pt}(固定資産の範囲)
\begin{description}
\item[第五条]法第二条第一項第十八号(固定資産の意義)に規定する政令で定める資産は、たな卸資産、有価証券及び繰延資産以外の資産のうち次に掲げるものとする。
\begin{description}
\item[一]土地(土地の上に存する権利を含む。)
\item[二]次条各号に掲げる資産
\item[三]電話加入権
\item[四]前三号に掲げる資産に準ずるもの
\end{description}
\end{description}
\noindent\hspace{10pt}(減価償却資産の範囲)
\begin{description}
\item[第六条]法第二条第一項第十九号(減価償却資産の意義)に規定する政令で定める資産は、棚卸資産、有価証券及び繰延資産以外の資産のうち次に掲げるもの(時の経過によりその価値の減少しないものを除く。)とする。
\begin{description}
\item[一]建物及びその附属設備(暖冷房設備、照明設備、通風設備、昇降機その他建物に附属する設備をいう。)
\item[二]構築物(ドック、橋、岸壁、桟橋、軌道、貯水池、坑道、煙突その他土地に定着する土木設備又は工作物をいう。)
\item[三]機械及び装置
\item[四]船舶
\item[五]航空機
\item[六]車両及び運搬具
\item[七]工具、器具及び備品(観賞用、興行用その他これらに準ずる用に供する生物を含む。)
\item[八]次に掲げる無形固定資産
\begin{description}
\item[イ]鉱業権(租鉱権及び採石権その他土石を採掘し又は採取する権利を含む。)
\item[ロ]漁業権(入漁権を含む。)
\item[ハ]ダム使用権
\item[ニ]水利権
\item[ホ]特許権
\item[ヘ]実用新案権
\item[ト]意匠権
\item[チ]商標権
\item[リ]ソフトウエア
\item[ヌ]育成者権
\item[ル]営業権
\item[ヲ]専用側線利用権(鉄道事業法(昭和六十一年法律第九十二号)第二条第一項(定義)に規定する鉄道事業又は軌道法(大正十年法律第七十六号)第一条第一項(軌道法の適用対象)に規定する軌道を敷設して行う運輸事業を営む者(以下この号において「鉄道事業者等」という。)に対して鉄道又は軌道の敷設に要する費用を負担し、その鉄道又は軌道を専用する権利をいう。)
\item[ワ]鉄道軌道連絡通行施設利用権(鉄道事業者等が、他の鉄道事業者等、独立行政法人鉄道建設・運輸施設整備支援機構、独立行政法人日本高速道路保有・債務返済機構又は国若しくは地方公共団体に対して当該他の鉄道事業者等、独立行政法人鉄道建設・運輸施設整備支援機構若しくは独立行政法人日本高速道路保有・債務返済機構の鉄道若しくは軌道との連絡に必要な橋、地下道その他の施設又は鉄道若しくは軌道の敷設に必要な施設を設けるために要する費用を負担し、これらの施設を利用する権利をいう。)
\item[カ]電気ガス供給施設利用権(電気事業法(昭和三十九年法律第百七十号)第二条第一項第八号(定義)に規定する一般送配電事業、同項第十号に規定する送電事業若しくは同項第十四号に規定する発電事業又はガス事業法(昭和二十九年法律第五十一号)第二条第五項(定義)に規定する一般ガス導管事業を営む者に対して電気又はガスの供給施設(同条第七項に規定する特定ガス導管事業の用に供するものを除く。)を設けるために要する費用を負担し、その施設を利用して電気又はガスの供給を受ける権利をいう。)
\item[ヨ]水道施設利用権(水道法(昭和三十二年法律第百七十七号)第三条第五項(定義)に規定する水道事業者に対して水道施設を設けるために要する費用を負担し、その施設を利用して水の供給を受ける権利をいう。)
\item[タ]工業用水道施設利用権(工業用水道事業法(昭和三十三年法律第八十四号)第二条第五項(定義)に規定する工業用水道事業者に対して工業用水道施設を設けるために要する費用を負担し、その施設を利用して工業用水の供給を受ける権利をいう。)
\item[レ]電気通信施設利用権(電気通信事業法(昭和五十九年法律第八十六号)第九条第一号(電気通信事業の登録)に規定する電気通信回線設備を設置する同法第二条第五号(定義)に規定する電気通信事業者に対して同条第四号に規定する電気通信事業の用に供する同条第二号に規定する電気通信設備の設置に要する費用を負担し、その設備を利用して同条第三号に規定する電気通信役務の提供を受ける権利(電話加入権及びこれに準ずる権利を除く。)をいう。)
\end{description}
\item[九]次に掲げる生物(第七号に掲げるものに該当するものを除く。)
\begin{description}
\item[イ]牛、馬、豚、綿羊及びやぎ
\item[ロ]かんきつ樹、りんご樹、ぶどう樹、梨樹、桃樹、桜桃樹、びわ樹、くり樹、梅樹、柿樹、あんず樹、すもも樹、いちじく樹、キウイフルーツ樹、ブルーベリー樹及びパイナップル
\item[ハ]茶樹、オリーブ樹、つばき樹、桑樹、こりやなぎ、みつまた、こうぞ、もう宗竹、アスパラガス、ラミー、まおらん及びホップ
\end{description}
\end{description}
\end{description}
\noindent\hspace{10pt}(繰延資産の範囲)
\begin{description}
\item[第七条]法第二条第一項第二十号(繰延資産の意義)に規定する政令で定める費用は、個人が支出する費用(資産の取得に要した金額とされるべき費用及び前払費用を除く。)のうち次に掲げるものとする。
\begin{description}
\item[一]開業費(不動産所得、事業所得又は山林所得を生ずべき事業を開始するまでの間に開業準備のために特別に支出する費用をいう。)
\item[二]開発費(新たな技術若しくは新たな経営組織の採用、資源の開発又は市場の開拓のために特別に支出する費用をいう。)
\item[三]前二号に掲げるもののほか、次に掲げる費用で支出の効果がその支出の日以後一年以上に及ぶもの
\begin{description}
\item[イ]自己が便益を受ける公共的施設又は共同的施設の設置又は改良のために支出する費用
\item[ロ]資産を賃借し又は使用するために支出する権利金、立ちのき料その他の費用
\item[ハ]役務の提供を受けるために支出する権利金その他の費用
\item[ニ]製品等の広告宣伝の用に供する資産を贈与したことにより生ずる費用
\item[ホ]イからニまでに掲げる費用のほか、自己が便益を受けるために支出する費用
\end{description}
\end{description}
\item[\rensuji{2}]前項に規定する前払費用とは、個人が一定の契約に基づき継続的に役務の提供を受けるために支出する費用のうち、その支出する日の属する年の十二月三十一日(年の中途において死亡し又は出国をした場合には、その死亡又は出国の時)においてまだ提供を受けていない役務に対応するものをいう。
\end{description}
\noindent\hspace{10pt}(変動所得の範囲)
\begin{description}
\item[第七条の二]法第二条第一項第二十三号(変動所得の意義)に規定する政令で定める所得は、漁獲若しくはのりの採取から生ずる所得、はまち、まだい、ひらめ、かき、うなぎ、ほたて貝若しくは真珠(真珠貝を含む。)の養殖から生ずる所得、原稿若しくは作曲の報酬に係る所得又は著作権の使用料に係る所得とする。
\end{description}
\noindent\hspace{10pt}(臨時所得の範囲)
\begin{description}
\item[第八条]法第二条第一項第二十四号(臨時所得の意義)に規定する政令で定める所得は、次に掲げる所得その他これらに類する所得とする。
\begin{description}
\item[一]職業野球の選手その他一定の者に専属して役務の提供をする者が、三年以上の期間、当該一定の者のために役務を提供し、又はそれ以外の者のために役務を提供しないことを約することにより一時に受ける契約金で、その金額がその契約による役務の提供に対する報酬の年額の二倍に相当する金額以上であるものに係る所得
\item[二]不動産、不動産の上に存する権利、船舶、航空機、採石権、鉱業権、漁業権又は工業所有権その他の技術に関する権利若しくは特別の技術による生産方式若しくはこれらに準ずるものを有する者が、三年以上の期間、他人(その者が非居住者である場合の法第百六十一条第一項第一号(国内源泉所得)に規定する事業場等を含む。)にこれらの資産を使用させること(地上権、租鉱権その他の当該資産に係る権利を設定することを含む。)を約することにより一時に受ける権利金、頭金その他の対価で、その金額が当該契約によるこれらの資産の使用料の年額の二倍に相当する金額以上であるものに係る所得(譲渡所得に該当するものを除く。)
\item[三]一定の場所における業務の全部又は一部を休止し、転換し又は廃止することとなつた者が、当該休止、転換又は廃止により当該業務に係る三年以上の期間の不動産所得、事業所得又は雑所得の補償として受ける補償金に係る所得
\item[四]前号に掲げるもののほか、業務の用に供する資産の全部又は一部につき鉱害その他の災害により被害を受けた者が、当該被害を受けたことにより、当該業務に係る三年以上の期間の不動産所得、事業所得又は雑所得の補償として受ける補償金に係る所得
\end{description}
\end{description}
\noindent\hspace{10pt}(災害の範囲)
\begin{description}
\item[第九条]法第二条第一項第二十七号(災害の意義)に規定する政令で定める災害は、冷害、雪害、干害、落雷、噴火その他の自然現象の異変による災害及び鉱害、火薬類の爆発その他の人為による異常な災害並びに害虫、害獣その他の生物による異常な災害とする。
\end{description}
\noindent\hspace{10pt}(障害者及び特別障害者の範囲)
\begin{description}
\item[第十条]法第二条第一項第二十八号(障害者の意義)に規定する政令で定める者は、次に掲げる者とする。
\begin{description}
\item[一]精神上の障害により事理を弁識する能力を欠く常況にある者又は児童相談所、知的障害者更生相談所(知的障害者福祉法(昭和三十五年法律第三十七号)第九条第六項(更生援護の実施者)に規定する知的障害者更生相談所をいう。次項第一号及び第三十一条の二第十四号(障害者等の範囲)において同じ。)、精神保健福祉センター(精神保健及び精神障害者福祉に関する法律(昭和二十五年法律第百二十三号)第六条第一項(精神保健福祉センター)に規定する精神保健福祉センターをいう。次項第一号において同じ。)若しくは精神保健指定医の判定により知的障害者とされた者
\item[二]前号に掲げる者のほか、精神保健及び精神障害者福祉に関する法律第四十五条第二項(精神障害者保健福祉手帳の交付)の規定により精神障害者保健福祉手帳の交付を受けている者
\item[三]身体障害者福祉法(昭和二十四年法律第二百八十三号)第十五条第四項(身体障害者手帳の交付)の規定により交付を受けた身体障害者手帳に身体上の障害がある者として記載されている者
\item[四]前三号に掲げる者のほか、戦傷病者特別援護法(昭和三十八年法律第百六十八号)第四条(戦傷病者手帳の交付)の規定により戦傷病者手帳の交付を受けている者
\item[五]前二号に掲げる者のほか、原子爆弾被爆者に対する援護に関する法律(平成六年法律第百十七号)第十一条第一項(認定)の規定による厚生労働大臣の認定を受けている者
\item[六]前各号に掲げる者のほか、常に就床を要し、複雑な介護を要する者
\item[七]前各号に掲げる者のほか、精神又は身体に障害のある年齢六十五歳以上の者で、その障害の程度が第一号又は第三号に掲げる者に準ずるものとして市町村長又は特別区の区長(社会福祉法(昭和二十六年法律第四十五号)に定める福祉に関する事務所が老人福祉法(昭和三十八年法律第百三十三号)第五条の四第二項各号(福祉の措置の実施者)に掲げる業務を行つている場合には、当該福祉に関する事務所の長。次項第六号において「市町村長等」という。)の認定を受けている者
\end{description}
\item[\rensuji{2}]法第二条第一項第二十九号に規定する政令で定める者は、次に掲げる者とする。
\begin{description}
\item[一]前項第一号に掲げる者のうち、精神上の障害により事理を弁識する能力を欠く常況にある者又は児童相談所、知的障害者更生相談所、精神保健福祉センター若しくは精神保健指定医の判定により重度の知的障害者とされた者
\item[二]前項第二号に掲げる者のうち、同号の精神障害者保健福祉手帳に精神保健及び精神障害者福祉に関する法律施行令(昭和二十五年政令第百五十五号)第六条第三項(精神障害の状態)に規定する障害等級が一級である者として記載されている者
\item[三]前項第三号に掲げる者のうち、同号の身体障害者手帳に身体上の障害の程度が一級又は二級である者として記載されている者
\item[四]前項第四号に掲げる者のうち、同号の戦傷病者手帳に精神上又は身体上の障害の程度が恩給法(大正十二年法律第四十八号)別表第一号表ノ二の特別項症から第三項症までである者として記載されている者
\item[五]前項第五号又は第六号に掲げる者
\item[六]前項第七号に掲げる者のうち、その障害の程度が第一号又は第三号に掲げる者に準ずるものとして市町村長等の認定を受けている者
\end{description}
\end{description}
\noindent\hspace{10pt}(寡婦の範囲)
\begin{description}
\item[第十一条]法第二条第一項第三十号イ又はロ(寡婦の意義)に規定する夫の生死の明らかでない者で政令で定めるものは、次に掲げる者の妻とする。
\begin{description}
\item[一]太平洋戦争の終結の当時もとの陸海軍に属していた者で、まだ国内に帰らないもの
\item[二]前号に掲げる者以外の者で、太平洋戦争の終結の当時国外にあつてまだ国内に帰らず、かつ、その帰らないことについて同号に掲げる者と同様の事情があると認められるもの
\item[三]船舶が沈没し、転覆し、滅失し若しくは行方不明となつた際現にその船舶に乗つていた者若しくは船舶に乗つていてその船舶の航行中に行方不明となつた者又は航空機が墜落し、滅失し若しくは行方不明となつた際現にその航空機に乗つていた者若しくは航空機に乗つていてその航空機の航行中に行方不明となつた者で、三月以上その生死が明らかでないもの
\item[四]前号に掲げる者以外の者で、死亡の原因となるべき危難に遭遇した者のうちその危難が去つた後一年以上その生死が明らかでないもの
\item[五]前各号に掲げる者のほか、三年以上その生死が明らかでない者
\end{description}
\item[\rensuji{2}]法第二条第一項第三十号イに規定するその者と生計を一にする親族で政令で定めるものは、その者と生計を一にする子(他の者の同一生計配偶者又は扶養親族とされている者を除く。)でその年分の総所得金額、退職所得金額及び山林所得金額の合計額が四十八万円以下のものとする。
\end{description}
\noindent\hspace{10pt}(寡夫の範囲)
\begin{description}
\item[第十一条の二]法第二条第一項第三十一号(寡夫の意義)に規定する妻の生死の明らかでない者で政令で定めるものは、前条第一項各号に掲げる者の夫とする。
\item[\rensuji{2}]法第二条第一項第三十一号に規定するその者と生計を一にする親族で政令で定めるものは、その者と生計を一にする子(他の者の同一生計配偶者又は扶養親族とされている者を除く。)でその年分の総所得金額、退職所得金額及び山林所得金額の合計額が四十八万円以下のものとする。
\end{description}
\noindent\hspace{10pt}(勤労学生の範囲)
\begin{description}
\item[第十一条の三]法第二条第一項第三十二号ロ(定義)に規定する政令で定める者は、次に掲げる者とする。
\begin{description}
\item[一]独立行政法人国立病院機構、独立行政法人労働者健康安全機構、日本赤十字社、商工会議所、健康保険組合、健康保険組合連合会、国民健康保険団体連合会、国家公務員共済組合連合会、社会福祉法人、宗教法人、一般社団法人及び一般財団法人並びに農業協同組合法(昭和二十二年法律第百三十二号)第十条第一項第十一号(事業)に掲げる事業を行う農業協同組合連合会及び医療法人
\item[二]学校教育法(昭和二十二年法律第二十六号)第百二十四条(専修学校)に規定する専修学校又は同法第百三十四条第一項(各種学校)に規定する各種学校のうち、教育水準を維持するための教員の数その他の文部科学大臣が定める基準を満たすものを設置する者(前号に掲げる者を除く。)
\end{description}
\item[\rensuji{2}]法第二条第一項第三十二号ロ又はハに規定する政令で定める課程は、当該課程が次の各号に掲げる課程のいずれの区分に属するかに応じ当該各号に掲げる事項に該当する課程とする。
\begin{description}
\item[一]学校教育法第百二十四条に規定する専修学校の同法第百二十五条第一項(専修学校の課程)に規定する高等課程及び専門課程
\begin{description}
\item[イ]職業に必要な技術の教授をすること。
\item[ロ]その修業期間が一年以上であること。
\item[ハ]その一年の授業時間数が八百時間以上であること(夜間その他特別な時間において授業を行う場合には、その一年の授業時間数が四百五十時間以上であり、かつ、その修業期間を通ずる授業時間数が八百時間以上であること。)。
\item[ニ]その授業が年二回を超えない一定の時期に開始され、かつ、その終期が明確に定められていること。
\end{description}
\item[二]前号に掲げる課程以外の課程
\begin{description}
\item[イ]前号イ及びニに掲げる事項
\item[ロ]その修業期間(普通科、専攻科その他これらに類する区別された課程があり、それぞれの修業期間が一年以上であつて一の課程に他の課程が継続する場合には、これらの課程の修業期間を通算した期間)が二年以上であること。
\item[ハ]その一年の授業時間数(普通科、専攻科その他これらに類する区別された課程がある場合には、それぞれの課程の授業時間数)が六百八十時間以上であること。
\end{description}
\end{description}
\item[\rensuji{3}]文部科学大臣は、第一項第二号の基準を定めたときは、これを告示する。
\end{description}
\noindent\hspace{10pt}(農業の範囲)
\begin{description}
\item[第十二条]法第二条第一項第三十五号(特別農業所得者の意義)に規定する政令で定める事業は、次に掲げる事業とする。
\begin{description}
\item[一]米、麦その他の穀物、馬鈴しよ、甘しよ、たばこ、野菜、花、種苗その他のほ場作物、果樹、樹園の生産物又は温室その他特殊施設を用いてする園芸作物の栽培を行なう事業
\item[二]繭又は蚕種の生産を行なう事業
\item[三]主として前二号に規定する物の栽培又は生産をする者が兼営するわら工品その他これに類する物の生産、家畜、家きん、毛皮獣若しくは蜂はち
の育成、肥育、採卵若しくはみつの採取又は酪農品の生産を行なう事業
\end{description}
\end{description}
\noindent\hspace{10pt}(国内に住所を有するものとみなされる公務員から除かれる者)
\begin{description}
\item[第十三条]法第三条第一項(居住者及び非居住者の区分)に規定する政令で定める者は、日本の国籍を有する者で、現に国外に居住し、かつ、その地に永住すると認められるものとする。
\end{description}
\noindent\hspace{10pt}(国内に住所を有する者と推定する場合)
\begin{description}
\item[第十四条]国内に居住することとなつた個人が次の各号のいずれかに該当する場合には、その者は、国内に住所を有する者と推定する。
\begin{description}
\item[一]その者が国内において、継続して一年以上居住することを通常必要とする職業を有すること。
\item[二]その者が日本の国籍を有し、かつ、その者が国内において生計を一にする配偶者その他の親族を有することその他国内におけるその者の職業及び資産の有無等の状況に照らし、その者が国内において継続して一年以上居住するものと推測するに足りる事実があること。
\end{description}
\item[\rensuji{2}]前項の規定により国内に住所を有する者と推定される個人と生計を一にする配偶者その他その者の扶養する親族が国内に居住する場合には、これらの者も国内に住所を有する者と推定する。
\end{description}
\noindent\hspace{10pt}(国内に住所を有しない者と推定する場合)
\begin{description}
\item[第十五条]国外に居住することとなつた個人が次の各号のいずれかに該当する場合には、その者は、国内に住所を有しない者と推定する。
\begin{description}
\item[一]その者が国外において、継続して一年以上居住することを通常必要とする職業を有すること。
\item[二]その者が外国の国籍を有し又は外国の法令によりその外国に永住する許可を受けており、かつ、その者が国内において生計を一にする配偶者その他の親族を有しないことその他国内におけるその者の職業及び資産の有無等の状況に照らし、その者が再び国内に帰り、主として国内に居住するものと推測するに足りる事実がないこと。
\end{description}
\item[\rensuji{2}]前項の規定により国内に住所を有しない者と推定される個人と生計を一にする配偶者その他その者の扶養する親族が国外に居住する場合には、これらの者も国内に住所を有しない者と推定する。
\end{description}
\section*{第一章の二 法人課税信託の受託者等に関する通則}
\addcontentsline{toc}{section}{第一章の二 法人課税信託の受託者等に関する通則}
\noindent\hspace{10pt}(法人課税信託の併合又は分割等)
\begin{description}
\item[第十六条]信託の併合に係る従前の信託又は信託の分割に係る分割信託(信託の分割によりその信託財産の一部を他の信託又は新たな信託に移転する信託をいう。次項において同じ。)が法人課税信託(法人税法第二条第二十九号の二イ又はハ(定義)に掲げる信託に限る。以下この項において「特定法人課税信託」という。)である場合には、当該信託の併合に係る新たな信託又は当該信託の分割に係る他の信託若しくは新たな信託(法人課税信託を除く。)は、特定法人課税信託とみなす。
\item[\rensuji{2}]信託の併合又は信託の分割(一の信託が新たな信託に信託財産の一部を移転するものに限る。以下この項及び次項において「単独新規信託分割」という。)が行われた場合において、当該信託の併合が法人課税信託を新たな信託とするものであるときにおける当該信託の併合に係る従前の信託(法人課税信託を除く。)は当該信託の併合の直前に法人課税信託に該当することとなつたものとみなし、当該単独新規信託分割が集団投資信託(法第十三条第三項第一号(信託財産に属する資産及び負債並びに信託財産に帰せられる収益及び費用の帰属)に規定する集団投資信託をいう。以下この項において同じ。)又は受益者等課税信託(同条第一項に規定する受益者(同条第二項の規定により同条第一項に規定する受益者とみなされる者を含む。)がその信託財産に属する資産及び負債を有するものとみなされる信託をいう。以下この項において同じ。)を分割信託とし、法人課税信託を承継信託(信託の分割により分割信託からその信託財産の一部の移転を受ける信託をいう。以下この項及び次項において同じ。)とするものであるときにおける当該承継信託は当該単独新規信託分割の直後に集団投資信託又は受益者等課税信託から法人課税信託に該当することとなつたものとみなす。
\item[\rensuji{3}]他の信託に信託財産の一部を移転する信託の分割(以下この項において「吸収信託分割」という。)又は二以上の信託が新たな信託に信託財産の一部を移転する信託の分割(以下この項において「複数新規信託分割」という。)が行われた場合には、当該吸収信託分割又は複数新規信託分割により移転する信託財産をその信託財産とする信託(以下この項において「吸収分割中信託」という。)を承継信託とする単独新規信託分割が行われ、直ちに当該吸収分割中信託及び承継信託(複数新規信託分割にあつては、他の吸収分割中信託)を従前の信託とする信託の併合が行われたものとみなして、前二項の規定を適用する。
\item[\rensuji{4}]前三項に定めるもののほか、受託法人又は法人課税信託の委託者若しくは受益者についての法又はこの政令の規定の適用に関し必要な事項は、財務省令で定める。
\end{description}
\section*{第二章 課税所得の範囲}
\addcontentsline{toc}{section}{第二章 課税所得の範囲}
\subsection*{第一節 課税所得の範囲}
\addcontentsline{toc}{subsection}{第一節 課税所得の範囲}
\noindent\hspace{10pt}(非永住者の課税所得の範囲)
\begin{description}
\item[第十七条]法第七条第一項第二号(課税所得の範囲)に規定する国外にある有価証券の譲渡により生ずる所得として政令で定めるものは、有価証券でその取得の日がその譲渡(租税特別措置法(昭和三十二年法律第二十六号)第三十七条の十第三項若しくは第四項(一般株式等に係る譲渡所得等の課税の特例)又は第三十七条の十一第三項若しくは第四項(上場株式等に係る譲渡所得等の課税の特例)の規定によりその額及び価額の合計額が同法第三十七条の十第一項に規定する一般株式等に係る譲渡所得等又は同法第三十七条の十一第一項に規定する上場株式等に係る譲渡所得等に係る収入金額とみなされる金銭及び金銭以外の資産の交付の基因となつた同法第三十七条の十第三項(第八号及び第九号に係る部分を除く。)若しくは第四項第一号から第三号まで又は第三十七条の十一第四項第一号及び第二号に規定する事由に基づく同法第三十七条の十第二項第一号から第五号までに掲げる株式等(同項第四号に掲げる受益権にあつては、公社債投資信託以外の証券投資信託の受益権及び証券投資信託以外の投資信託で公社債等運用投資信託に該当しないものの受益権に限る。)についての当該金銭の額及び当該金銭以外の資産の価額に対応する権利の移転又は消滅を含む。以下この条において同じ。)の日の十年前の日の翌日から当該譲渡の日までの期間(その者が非永住者であつた期間に限る。)内にないもの(次項において「特定有価証券」という。)のうち、次に掲げるものの譲渡により生ずる所得とする。
\begin{description}
\item[一]金融商品取引法第二条第八項第三号ロ(定義)に規定する外国金融商品市場において譲渡がされるもの
\item[二]外国金融商品取引業者(国外において金融商品取引法第二条第九項に規定する金融商品取引業者(同法第二十八条第一項(通則)に規定する第一種金融商品取引業又は同条第二項に規定する第二種金融商品取引業を行う者に限る。)と同種類の業務を行う者をいう。以下この項において同じ。)への売委託(当該外国金融商品取引業者が当該業務として受けるものに限る。)により譲渡が行われるもの
\item[三]外国金融商品取引業者又は国外において金融商品取引法第二条第十一項に規定する登録金融機関若しくは投資信託及び投資法人に関する法律第二条第十一項(定義)に規定する投資信託委託会社と同種類の業務を行う者の営業所、事務所その他これらに類するもの(国外にあるものに限る。)に開設された口座に係る国外における社債、株式等の振替に関する法律(平成十三年法律第七十五号)に規定する振替口座簿に類するものに記載若しくは記録がされ、又は当該口座に保管の委託がされているもの
\end{description}
\item[\rensuji{2}]非永住者が譲渡をした有価証券(以下この項において「譲渡有価証券」という。)が当該譲渡の時において特定有価証券に該当するかどうかの判定は、当該譲渡の前に取得をした当該譲渡有価証券と同一銘柄の有価証券のうち先に取得をしたものから順次譲渡をしたものとした場合に当該譲渡をしたものとされる当該同一銘柄の有価証券の取得の日により行うものとする。
\item[\rensuji{3}]個人の有する有価証券(以下この項において「従前の有価証券」という。)について次に掲げる事由が生じた場合には、当該事由により取得した有価証券(以下この項において「取得有価証券」という。)はその者が引き続き所有していたものと、当該従前の有価証券のうち当該取得有価証券の取得の基因となつた部分は当該取得有価証券と同一銘柄の有価証券とそれぞれみなして、前二項の規定を適用する。
\begin{description}
\item[一]株式(出資を含む。)を発行した法人の行つた法第五十七条の四第一項(株式交換等に係る譲渡所得等の特例)に規定する株式交換又は同条第二項に規定する株式移転
\item[二]法第五十七条の四第三項第一号に規定する取得請求権付株式、同項第二号に規定する取得条項付株式、同項第三号に規定する全部取得条項付種類株式、同項第四号に規定する新株予約権付社債、同項第五号に規定する取得条項付新株予約権又は同項第六号に規定する取得条項付新株予約権が付された新株予約権付社債のこれらの号に規定する請求権の行使、取得事由の発生、取得決議又は行使
\item[三]株式(出資及び投資信託及び投資法人に関する法律第二条第十四項に規定する投資口を含む。以下この項において同じ。)又は投資信託若しくは特定受益証券発行信託の受益権の分割又は併合
\item[四]株式を発行した法人の第百十一条第二項(株主割当てにより取得した株式の取得価額)に規定する株式無償割当て(当該株式無償割当てにより当該株式と同一の種類の株式が割り当てられる場合の当該株式無償割当てに限る。)
\item[五]株式を発行した法人の第百十二条第一項(合併により取得した株式等の取得価額)に規定する合併
\item[六]第百十二条第三項に規定する投資信託等(以下この号において「投資信託等」という。)の受益権に係る投資信託等の同項に規定する信託の併合
\item[七]株式を発行した法人の第百十三条第一項(分割型分割により取得した株式等の取得価額)に規定する分割型分割
\item[八]特定受益証券発行信託の受益権に係る特定受益証券発行信託の第百十三条第六項に規定する信託の分割
\item[九]株式を発行した法人の第百十三条の二第一項(株式分配により取得した株式等の取得価額)に規定する株式分配
\item[十]株式を発行した法人の第百十五条(組織変更があつた場合の株式等の取得価額)に規定する組織変更
\item[十一]新株予約権(投資信託及び投資法人に関する法律第二条第十七項に規定する新投資口予約権を含む。次号において同じ。)又は新株予約権付社債を発行した法人を第百十六条(合併等があつた場合の新株予約権等の取得価額)に規定する被合併法人、分割法人、株式交換完全子法人又は株式移転完全子法人とする同条に規定する合併等
\item[十二]新株予約権の行使
\end{description}
\item[\rensuji{4}]法第七条第一項第二号に規定する国外源泉所得(以下この項において「国外源泉所得」という。)で国内において支払われ、又は国外から送金されたものの範囲については、次に定めるところによる。
\begin{description}
\item[一]非永住者が各年において国外から送金を受領した場合には、その金額の範囲内でその非永住者のその年における国外源泉所得に係る所得で国外の支払に係るものについて送金があつたものとみなす。
\item[二]前号に規定する所得の金額は、非永住者の国外源泉所得に係る所得で国外の支払に係るもの及び非国外源泉所得に係る所得で国外の支払に係るものについてそれぞれ法第二十三条から第三十五条まで(所得の種類及び各種所得の金額)及び第六十九条(損益通算)の規定に準じて計算した各種所得の金額の合計額に相当する金額とする。
\item[三]法第七条第一項第二号及び前二号の規定を適用する場合において、国外源泉所得に係る各種所得又は非国外源泉所得に係る各種所得について国内及び国外において支払われたものがあるときは、その各種所得の金額(前号後段に規定する所得については、同号後段の規定により計算した金額)に、その各種所得に係る収入金額のうちに国内で支払われた金額又は国外で支払われた金額の占める割合を乗じて計算した金額をそれぞれその各種所得の金額のうち国内の支払に係るもの又は国外の支払に係るものとみなす。
\item[四]第一号の場合において、国外源泉所得に係る各種所得で国外の支払に係るものが二以上あるときは、それぞれの各種所得について、同号の規定により送金があつたものとみなされる国外源泉所得に係る送金額に当該各種所得の金額(第二号後段に規定する所得については、同号後段の規定により計算した金額)がその合計額のうちに占める割合を乗じて計算した金額に相当する金額の送金があつたものとみなす。
\item[五]非永住者の国外源泉所得に係る所得で国外の支払に係るもののうち、前各号の規定により送金があつたものとみなされたものに係る各種所得については、それぞれその各種所得と、これと同一種類の国外源泉所得に係る所得で国内の支払に係るもの及び非国外源泉所得に係る所得とを合算してその者の総所得金額、退職所得金額及び山林所得金額を計算する。
\item[六]年の中途において、非永住者以外の居住者若しくは非居住者が非永住者となり、又は非永住者が非永住者以外の居住者若しくは非居住者となつたときは、その者がその年において非永住者であつた期間内に生じた国外源泉所得又は非国外源泉所得に係る所得で国外の支払に係るもの及び当該期間内に国外から送金があつた金額について前各号の規定を適用する。
\end{description}
\end{description}
\subsection*{第二節 非課税所得}
\addcontentsline{toc}{subsection}{第二節 非課税所得}
\noindent\hspace{10pt}(非課税とされない当座預金の利子)
\begin{description}
\item[第十八条]法第九条第一項第一号(非課税所得)に規定する政令で定める利子は、年一パーセントを超える利率の利子を付された当座預金の利子とする。
\end{description}
\noindent\hspace{10pt}(非課税とされる児童又は生徒の預貯金の利子等)
\begin{description}
\item[第十九条]法第九条第一項第二号(非課税所得)に規定する政令で定める預貯金又は合同運用信託は、同号に規定する学校の児童又は生徒が、その学校の長の指導を受けて、財務省令で定めるところにより、当該児童又は生徒の代表者の名義で預入し又は信託した預貯金又は合同運用信託とする。
\end{description}
\noindent\hspace{10pt}(非課税とされる業務上の傷害に基づく給付等)
\begin{description}
\item[第二十条]法第九条第一項第三号イ(非課税所得)に規定する政令で定める給付は、次に掲げる給付とする。
\begin{description}
\item[一]恩給法の一部を改正する法律(昭和二十八年法律第百五十五号)附則第二十二条第一項(旧軍人等に対する増加恩給等の給付等)の規定による傷病年金
\item[二]労働基準法第八章(災害補償)の規定により受ける療養の給付若しくは費用、休業補償、障害補償、打切補償又は分割補償(障害補償に係る部分に限る。)
\item[三]船員法第十章(災害補償)の規定により受ける療養の給付若しくは費用、傷病手当、予後手当又は障害手当
\item[四]条例の規定により地方公共団体から支払われる給付で法第九条第一項第三号イに規定する増加恩給又は傷病賜金に準ずるもの
\end{description}
\item[\rensuji{2}]法第九条第一項第三号ハに規定する政令で定める共済制度は、地方公共団体の条例において精神又は身体に障害のある者(以下この項において「心身障害者」という。)を扶養する者を加入者とし、その加入者が地方公共団体に掛金を納付し、当該地方公共団体が心身障害者の扶養のための給付金を定期に支給することを定めている制度(脱退一時金(加入者が当該制度から脱退する場合に支給される一時金をいう。)の支給に係る部分を除く。)で、次に掲げる要件を備えているものとする。
\begin{description}
\item[一]心身障害者の扶養のための給付金(その給付金の支給開始前に心身障害者が死亡した場合に加入者に対して支給される弔慰金を含む。)のみを支給するものであること。
\item[二]前号の給付金の額は、心身障害者の生活のために通常必要とされる費用を満たす金額(同号の弔慰金にあつては、掛金の累積額に比して相当と認められる金額)を超えず、かつ、その額について、特定の者につき不当に差別的な取扱いをしないこと。
\item[三]第一号の給付金(同号の弔慰金を除く。次号において同じ。)の支給は、加入者の死亡、重度の障害その他地方公共団体の長が認定した特別の事故を原因として開始されるものであること。
\item[四]第一号の給付金の受取人は、心身障害者又は前号の事故発生後において心身障害者を扶養する者とするものであること。
\item[五]第一号の給付金に関する経理は、他の経理と区分して行い、かつ、掛金その他の資金が銀行その他の金融機関に対する運用の委託、生命保険への加入その他これらに準ずる方法を通じて確実に運用されるものであること。
\end{description}
\end{description}
\noindent\hspace{10pt}(非課税とされる通勤手当)
\begin{description}
\item[第二十条の二]法第九条第一項第五号(非課税所得)に規定する政令で定めるものは、次の各号に掲げる通勤手当(これに類するものを含む。)の区分に応じ当該各号に定める金額に相当する部分とする。
\begin{description}
\item[一]通勤のため交通機関又は有料の道路を利用し、かつ、その運賃又は料金(以下この条において「運賃等」という。)を負担することを常例とする者(第四号に規定する者を除く。)が受ける通勤手当(これに類する手当を含む。以下この条において同じ。)
\item[二]通勤のため自動車その他の交通用具を使用することを常例とする者(その通勤の距離が片道二キロメートル未満である者及び第四号に規定する者を除く。)が受ける通勤手当
\begin{description}
\item[イ]その通勤の距離が片道十キロメートル未満である場合
\item[ロ]その通勤の距離が片道十キロメートル以上十五キロメートル未満である場合
\item[ハ]その通勤の距離が片道十五キロメートル以上二十五キロメートル未満である場合
\item[ニ]その通勤の距離が片道二十五キロメートル以上三十五キロメートル未満である場合
\item[ホ]その通勤の距離が片道三十五キロメートル以上四十五キロメートル未満である場合
\item[ヘ]その通勤の距離が片道四十五キロメートル以上五十五キロメートル未満である場合
\item[ト]その通勤の距離が片道五十五キロメートル以上である場合
\end{description}
\item[三]通勤のため交通機関を利用することを常例とする者(第一号に掲げる通勤手当の支給を受ける者及び次号に規定する者を除く。)が受ける通勤用定期乗車券(これに類する乗車券を含む。以下この条において同じ。)
\item[四]通勤のため交通機関又は有料の道路を利用するほか、併せて自動車その他の交通用具を使用することを常例とする者(当該交通用具を使用する距離が片道二キロメートル未満である者を除く。)が受ける通勤手当又は通勤用定期乗車券
\end{description}
\end{description}
\noindent\hspace{10pt}(非課税とされる職務上必要な給付)
\begin{description}
\item[第二十一条]法第九条第一項第六号(非課税所得)に規定する政令で定めるものは、次に掲げるものとする。
\begin{description}
\item[一]船員法第八十条第一項(食料の支給)の規定により支給される食料その他法令の規定により無料で支給される食料
\item[二]給与所得を有する者でその職務の性質上制服を着用すべき者がその使用者から支給される制服その他の身回品
\item[三]前号に規定する者がその使用者から同号に規定する制服その他の身回品の貸与を受けることによる利益
\item[四]国家公務員宿舎法(昭和二十四年法律第百十七号)第十二条(無料宿舎)の規定により無料で宿舎の貸与を受けることによる利益その他給与所得を有する者でその職務の遂行上やむを得ない必要に基づき使用者から指定された場所に居住すべきものがその指定する場所に居住するために家屋の貸与を受けることによる利益
\end{description}
\end{description}
\noindent\hspace{10pt}(非課税とされる在外手当)
\begin{description}
\item[第二十二条]法第九条第一項第七号(非課税所得)に規定する政令で定める手当は、国外で勤務する者がその勤務により国内で勤務した場合に受けるべき通常の給与に加算して支給を受ける給与のうち、その勤務地における物価、生活水準及び生活環境並びに勤務地と国内との間の為替相場等の状況に照らし、加算して支給を受けることにより国内で勤務した場合に比して利益を受けると認められない部分の金額とする。
\end{description}
\noindent\hspace{10pt}(職員の給与が非課税とされる国際機関の範囲)
\begin{description}
\item[第二十三条]法第九条第一項第八号(非課税所得)に規定する政令で定める国際機関は、国際間の取極に基づき設立された機関のうち日本国が構成員となつているものその他国を構成員とするもので、財務大臣が指定するものとする。
\item[\rensuji{2}]財務大臣は、前項の指定をしたときは、これを告示する。
\end{description}
\noindent\hspace{10pt}(給与が非課税とされる外国政府職員等の要件)
\begin{description}
\item[第二十四条]法第九条第一項第八号(非課税所得)に規定する政令で定める要件は、外国政府又は外国の地方公共団体に勤務する者については次の各号に掲げる要件とし、前条第一項に規定する国際機関に勤務する者については第一号に掲げる要件とする。
\begin{description}
\item[一]その者が日本の国籍を有しない者であり、かつ、日本国に永住する許可を受けている者(日本国に長期にわたり在留することを認められている者を含む。)として財務省令で定めるものでないこと。
\item[二]その者のその外国政府又は外国の地方公共団体のために行なう勤務が日本国又はその地方公共団体の行なう業務に準ずる業務で収益を目的としないものに係る勤務であること。
\end{description}
\end{description}
\noindent\hspace{10pt}(譲渡所得について非課税とされる生活用動産の範囲)
\begin{description}
\item[第二十五条]法第九条第一項第九号(非課税所得)に規定する政令で定める資産は、生活に通常必要な動産のうち、次に掲げるもの(一個又は一組の価額が三十万円を超えるものに限る。)以外のものとする。
\begin{description}
\item[一]貴石、半貴石、貴金属、真珠及びこれらの製品、べつこう製品、さんご製品、こはく製品、ぞうげ製品並びに七宝製品
\item[二]書画、こつとう及び美術工芸品
\end{description}
\end{description}
\noindent\hspace{10pt}(非課税とされる資力喪失による譲渡所得)
\begin{description}
\item[第二十六条]法第九条第一項第十号(非課税所得)に規定する政令で定める所得は、資力を喪失して債務を弁済することが著しく困難であり、かつ、国税通則法(昭和三十七年法律第六十六号)第二条第十号(定義)に規定する強制換価手続の執行が避けられないと認められる場合における資産の譲渡による所得で、その譲渡に係る対価が当該債務の弁済に充てられたものとする。
\end{description}
\noindent\hspace{10pt}(オープン型の証券投資信託の収益の分配のうち非課税とされるもの)
\begin{description}
\item[第二十七条]法第九条第一項第十一号(非課税所得)に規定する政令で定めるものは、オープン型の証券投資信託の契約に基づき収益調整金のみに係る収益として分配される特別分配金とする。
\end{description}
\noindent\hspace{10pt}(非課税とされる金品の交付を行う財団法人日本オリンピック委員会に加盟している団体)
\begin{description}
\item[第二十八条]法第九条第一項第十四号(非課税所得)に規定する政令で定める団体は、オリンピック競技大会において実施される競技に関する業務を行う一般社団法人又は一般財団法人のうち、その運営組織が適正であり、かつ、同号の金品の交付を適正に行うことができると認められるものとして文部科学大臣が財務大臣と協議して指定するものとする。
\item[\rensuji{2}]文部科学大臣は、前項の規定により一般社団法人又は一般財団法人を指定したときは、これを告示する。
\end{description}
\noindent\hspace{10pt}(学資に充てるため給付される金品が非課税とされない特別の関係がある者の範囲)
\begin{description}
\item[第二十九条]法第九条第一項第十五号ロ(非課税所得)に規定する当該使用人と政令で定める特別の関係がある者は、次に掲げる者とする。
\begin{description}
\item[一]当該使用人(法第九条第一項第十五号ロに規定する使用人をいう。以下この項において同じ。)の親族
\item[二]当該使用人と婚姻の届出をしていないが事実上婚姻関係と同様の事情にある者及びその者の直系血族
\item[三]当該使用人の直系血族と婚姻の届出をしていないが事実上婚姻関係と同様の事情にある者
\item[四]前三号に掲げる者以外の者で、当該使用人から受ける金銭その他の財産によつて生計を維持しているもの及びその者の直系血族
\item[五]前各号に掲げる者以外の者で、当該使用人の直系血族から受ける金銭その他の財産によつて生計を維持しているもの
\end{description}
\item[\rensuji{2}]前項の規定は、法第九条第一項第十五号ニに規定する当該使用人と政令で定める特別の関係がある者について準用する。
\end{description}
\noindent\hspace{10pt}(非課税とされる保険金、損害賠償金等)
\begin{description}
\item[第三十条]法第九条第一項第十七号(非課税所得)に規定する政令で定める保険金及び損害賠償金(これらに類するものを含む。)は、次に掲げるものその他これらに類するもの(これらのものの額のうちに同号の損害を受けた者の各種所得の金額の計算上必要経費に算入される金額を補てんするための金額が含まれている場合には、当該金額を控除した金額に相当する部分)とする。
\begin{description}
\item[一]損害保険契約(保険業法(平成七年法律第百五号)第二条第四項(定義)に規定する損害保険会社若しくは同条第九項に規定する外国損害保険会社等の締結した保険契約又は同条第十八項に規定する少額短期保険業者(以下この号において「少額短期保険業者」という。)の締結したこれに類する保険契約をいう。以下この条において同じ。)に基づく保険金、生命保険契約(同法第二条第三項に規定する生命保険会社若しくは同条第八項に規定する外国生命保険会社等の締結した保険契約又は少額短期保険業者の締結したこれに類する保険契約をいう。以下この号において同じ。)又は旧簡易生命保険契約(郵政民営化法等の施行に伴う関係法律の整備等に関する法律(平成十七年法律第百二号)第二条(法律の廃止)の規定による廃止前の簡易生命保険法(昭和二十四年法律第六十八号)第三条(政府保証)に規定する簡易生命保険契約をいう。)に基づく給付金及び損害保険契約又は生命保険契約に類する共済に係る契約に基づく共済金で、身体の傷害に基因して支払を受けるもの並びに心身に加えられた損害につき支払を受ける慰謝料その他の損害賠償金(その損害に基因して勤務又は業務に従事することができなかつたことによる給与又は収益の補償として受けるものを含む。)
\item[二]損害保険契約に基づく保険金及び損害保険契約に類する共済に係る契約に基づく共済金(前号に該当するもの及び第百八十四条第四項(満期返戻金等の意義)に規定する満期返戻金等その他これに類するものを除く。)で資産の損害に基因して支払を受けるもの並びに不法行為その他突発的な事故により資産に加えられた損害につき支払を受ける損害賠償金(これらのうち第九十四条(事業所得の収入金額とされる保険金等)の規定に該当するものを除く。)
\item[三]心身又は資産に加えられた損害につき支払を受ける相当の見舞金(第九十四条の規定に該当するものその他役務の対価たる性質を有するものを除く。)
\end{description}
\end{description}
\subsection*{第三節 障害者等の少額預金の利子所得等の非課税}
\addcontentsline{toc}{subsection}{第三節 障害者等の少額預金の利子所得等の非課税}
\noindent\hspace{10pt}(用語の意義)
\begin{description}
\item[第三十一条]この節において、次の各号に掲げる用語の意義は、当該各号に定めるところによる。
\begin{description}
\item[一]障害者等、金融機関の営業所等、特定公募公社債等運用投資信託、有価証券、預入等、非課税貯蓄申込書、合同運用信託等、剰余金の配当、額面金額等又は非課税貯蓄申告書
\item[二]預貯金等
\item[三]金融機関の振替口座簿
\end{description}
\end{description}
\noindent\hspace{10pt}(障害者等の範囲)
\begin{description}
\item[第三十一条の二]法第十条第一項(障害者等の少額預金の利子所得等の非課税)に規定する政令で定める個人は、次に掲げる者とする。
\begin{description}
\item[一]国民年金法(昭和三十四年法律第百四十一号)第十五条第二号(給付の種類)に掲げる障害基礎年金を受けている者
\item[二]厚生年金保険法(昭和二十九年法律第百十五号)第三十二条第二号(保険給付の種類)に規定する障害厚生年金を受けている者又は同条第三号に掲げる遺族厚生年金を受けている同法第五十九条第一項(遺族)に規定する遺族(妻に限る。)である者
\item[三]恩給法第二条第一項(恩給の種類)に規定する増加恩給を受けている者又は同項に規定する扶助料を受けている同法第七十二条第一項(遺族)に規定する遺族(妻に限る。)である者
\item[四]労働者災害補償保険法(昭和二十二年法律第五十号)第十二条の八第一項第六号(業務災害に関する保険給付の種類)に掲げる傷病補償年金、同法第十五条第一項(障害補償給付)に規定する障害補償年金、同法第二十二条の三第二項(障害給付)に規定する障害年金若しくは同法第二十三条第一項(傷病年金)に規定する傷病年金を受けている者又は同法第十六条(遺族補償給付)に規定する遺族補償年金若しくは同法第二十二条の四第二項(遺族給付)に規定する遺族年金を受けている同法第十六条の二第一項(遺族)(同法第二十二条の四第三項において準用する場合を含む。)に規定する遺族(妻に限る。)である者
\item[五]船員保険法(昭和十四年法律第七十三号)第八十七条第一項(障害年金及び障害手当金の支給要件)に規定する障害年金を受けている者又は同法第九十七条(遺族年金の支給要件)に規定する遺族年金を受けている同法第三十五条第一項(遺族年金を受ける遺族の範囲及び順位)に規定する遺族(妻に限る。)である者
\item[六]国家公務員災害補償法(昭和二十六年法律第百九十一号)第九条第三号(補償の種類)に掲げる傷病補償年金若しくは同条第四号イに掲げる障害補償年金を受けている者又は同条第六号イに掲げる遺族補償年金を受けている同法第十六条第一項(遺族補償年金)に規定する遺族(妻に限る。)である者
\item[七]地方公務員災害補償法(昭和四十二年法律第百二十一号)第二十五条第一項第三号(補償の種類等)に掲げる傷病補償年金若しくは同項第四号イに掲げる障害補償年金を受けている者又は同項第六号イに掲げる遺族補償年金を受けている同法第三十二条第一項(遺族補償年金)に規定する遺族(妻に限る。)である者
\item[八]公害健康被害の補償等に関する法律(昭和四十八年法律第百十一号)第三条第一項第二号(補償給付の種類等)に掲げる障害補償費を受けている者又は同項第三号に掲げる遺族補償費を受けている同法第三十条第一項(遺族補償費を受けることができる遺族の範囲及び順位)に規定する遺族(妻に限る。)である者
\item[九]独立行政法人医薬品医療機器総合機構法(平成十四年法律第百九十二号)第十五条第一項第一号イ若しくは第二号イ(業務の範囲)に規定する障害年金を受けている者又は同項第一号イ若しくは第二号イに規定する遺族年金を受けている同法第十六条第一項第四号(副作用救済給付)若しくは第二十条第一項第四号(感染救済給付)に定める遺族(妻に限る。)である者
\item[十]戦傷病者戦没者遺族等援護法(昭和二十七年法律第百二十七号)第五条第一号(援護の種類)に規定する障害年金を受けている者又は同条第二号に規定する遺族年金若しくは遺族給与金を受けている同法第二十四条(遺族の範囲)に規定する遺族(妻に限る。)である者
\item[十一]児童扶養手当法(昭和三十六年法律第二百三十八号)第四条第一項(支給要件)に規定する児童扶養手当を受けている同項に規定する児童の母である者
\item[十二]予防接種法(昭和二十三年法律第六十八号)第十六条第一項第三号若しくは第二項第三号(給付の範囲)に掲げる障害年金を受けている者又は同項第四号に掲げる遺族年金を受けている同号に規定する遺族(妻に限る。)である者
\item[十三]特別児童扶養手当等の支給に関する法律(昭和三十九年法律第百三十四号)第十七条(支給要件)に規定する障害児福祉手当又は同法第二十六条の二(支給要件)に規定する特別障害者手当を受けている者
\item[十四]都道府県知事又は地方自治法(昭和二十二年法律第六十七号)第二百五十二条の十九第一項(指定都市の事務)の指定都市の長から療育手帳(知的障害者の福祉の充実を図るため、児童相談所又は知的障害者更生相談所において知的障害と判定された者に対して支給される手帳で、その者の障害の程度その他の事項の記載があるものをいう。)の交付を受けている者
\item[十五]精神保健及び精神障害者福祉に関する法律第四十五条第二項(精神障害者保健福祉手帳の交付)の規定により精神障害者保健福祉手帳の交付を受けている者
\item[十六]原子爆弾被爆者に対する援護に関する法律第二十四条第一項(医療特別手当の支給)に規定する医療特別手当、同法第二十五条第一項(特別手当の支給)に規定する特別手当、同法第二十六条第一項(原子爆弾小頭症手当の支給)に規定する原子爆弾小頭症手当、同法第二十七条第一項(健康管理手当の支給)に規定する健康管理手当又は同法第二十八条第一項(保健手当の支給)に規定する保健手当の支給を受けている者
\item[十七]戦傷病者特別援護法第四条(戦傷病者手帳の交付)の規定により戦傷病者手帳の交付を受けている者
\item[十八]前各号に掲げる者に準ずる者として財務省令で定める者
\end{description}
\end{description}
\noindent\hspace{10pt}(金融機関等の範囲)
\begin{description}
\item[第三十二条]法第十条第一項(障害者等の少額預金の利子所得等の非課税)に規定する政令で定める金融機関その他の預貯金の受入れ若しくは信託の引受けをする者、金融商品取引業者又は登録金融機関は、次に掲げる者とする。
\begin{description}
\item[一]銀行、信託会社(信託業法(平成十六年法律第百五十四号)第三条(信託会社の免許)又は第五十三条第一項(外国信託会社の免許)の免許を受けたものに限る。)、信用金庫、信用金庫連合会、労働金庫、労働金庫連合会、信用協同組合、信用協同組合連合会(中小企業等協同組合法(昭和二十四年法律第百八十一号)第九条の九第一項第一号(協同組合連合会)の事業を行う協同組合連合会をいう。以下この節において同じ。)、農林中央金庫及び株式会社商工組合中央金庫並びに貯金の受入れをする農業協同組合、農業協同組合連合会、漁業協同組合、漁業協同組合連合会、水産加工業協同組合及び水産加工業協同組合連合会
\item[二]労働基準法第十八条(貯蓄金の管理等)又は船員法第三十四条(貯蓄金の管理等)の規定によりこれらの規定に規定する労働者又は船員の貯蓄金をその委託を受けて管理する者
\item[三]国家公務員共済組合法第九十八条(福祉事業)若しくは地方公務員等共済組合法第百十二条第一項(福祉事業)の規定によりこれらの規定に規定する組合員の貯金の受入れをする者又は私立学校教職員共済法第二十六条第一項(福祉事業)の規定により同項に規定する加入者の貯金の受入れをする者
\item[四]金融商品取引法第二条第九項(定義)に規定する金融商品取引業者(同法第二十八条第一項(通則)に規定する第一種金融商品取引業を行う者に限る。)
\item[五]金融商品取引法第三十三条の二(金融機関の登録)の登録を受けた生命保険会社及び損害保険会社
\end{description}
\end{description}
\noindent\hspace{10pt}(利子所得等について非課税とされる預貯金等の範囲)
\begin{description}
\item[第三十三条]法第十条第一項(障害者等の少額預金の利子所得等の非課税)に規定する政令で定める預貯金は、本邦通貨以外の通貨で預入される預貯金とする。
\item[\rensuji{2}]法第十条第一項に規定する政令で定める合同運用信託は、本邦通貨以外の通貨により引き受けられる金銭信託に係る合同運用信託とする。
\item[\rensuji{3}]法第十条第一項に規定する政令で定める公募公社債等運用投資信託は、本邦通貨以外の通貨により引き受けられる金銭信託に係る公募公社債等運用投資信託とする。
\item[\rensuji{4}]法第十条第一項に規定する政令で定める公社債及び投資信託又は特定目的信託の受益権は、次に掲げるもの(第一号から第五号までに掲げるものにあつては国内において発行されたものに限るものとし、第六号及び第七号に掲げるものにあつてはその募集が国内において行われる受益権で当該受益権に係る信託の設定(追加設定を含む。)があつた日において購入されたものに限る。)で本邦通貨で表示されたものとする。
\begin{description}
\item[一]国債及び地方債
\item[二]特別の法令により設立された法人が当該法令の規定により発行する債券
\item[三]長期信用銀行法(昭和二十七年法律第百八十七号)第八条(長期信用銀行債の発行)の規定による長期信用銀行債、金融機関の合併及び転換に関する法律(昭和四十三年法律第八十六号)第八条第一項(特定社債の発行)(同法第五十五条第四項(長期信用銀行が普通銀行となる転換)において準用する場合を含む。)の規定による特定社債(会社法の施行に伴う関係法律の整備等に関する法律(平成十七年法律第八十七号)第二百条第一項(金融機関の合併及び転換に関する法律の一部改正に伴う経過措置)の規定によりなお従前の例によることとされる同法第百九十九条(金融機関の合併及び転換に関する法律の一部改正)の規定による改正前の金融機関の合併及び転換に関する法律第十七条の二第一項(債券の発行の特例)に規定する普通銀行で同項(同法第二十四条第一項第七号(合併に関する規定の準用)において準用する場合を含む。以下この号において同じ。)の認可を受けたものの発行する同法第十七条の二第一項の債券(第三十七条第二項(有価証券の記録等)において「旧法債券」という。)を含む。)、信用金庫法(昭和二十六年法律第二百三十八号)第五十四条の二の四第一項(全国連合会債の発行)の規定による全国連合会債又は株式会社商工組合中央金庫法(平成十九年法律第七十四号)第三十三条(商工債の発行)の規定による商工債(同法附則第三十七条(商工債に関する経過措置)の規定により同法第三十三条の規定により発行された商工債とみなされたもの(第三十七条第二項において「旧商工債」という。)を含む。)
\item[四]その債務について政府が保証している社債
\item[五]内国法人の発行する社債のうち、その発行に際して金融商品取引法第二十一条第四項(元引受契約)に規定する元引受契約が前条第四号に掲げる金融商品取引業者により締結されたもの
\item[六]公社債投資信託(投資信託及び投資法人に関する法律第二条第二十四項(定義)に規定する外国投資信託(次号において「外国投資信託」という。)を除く。)の受益権
\item[七]公募公社債等運用投資信託(投資信託及び投資法人に関する法律第二条第一項に規定する委託者指図型投資信託に限るものとし、外国投資信託を除く。)の受益権
\item[八]法第六条の三第四号(受託法人等に関するこの法律の適用)に規定する社債的受益権(当該受益権の募集が公募(金融商品取引法第二条第三項に規定する取得勧誘のうち同項第一号に掲げる場合に該当するものとして財務省令で定めるものをいう。)により行われたものに限る。)
\item[九]外国、外国の地方公共団体その他の外国法人(財務省令で定める国際機関を除く。)の発行する債券のうち、その発行に際して第五号に規定する元引受契約が同号に規定する金融商品取引業者により締結されたもの
\end{description}
\end{description}
\noindent\hspace{10pt}(非課税貯蓄申込書の記載事項及び提出)
\begin{description}
\item[第三十四条]非課税貯蓄申込書には、法第十条第一項(障害者等の少額預金の利子所得等の非課税)の規定の適用を受けようとする旨及び次に掲げる事項を記載しなければならない。
\begin{description}
\item[一]提出者の氏名、生年月日及び住所
\item[二]障害者等に該当する事実
\item[三]預貯金等のうち、提出者がその金融機関の営業所等を経由して提出した非課税貯蓄申告書に記載したものの種別
\item[四]預入等をする前号の預貯金等で法第十条第一項の規定の適用を受けようとするものの金額(当該預貯金等が有価証券である場合には、その額面金額等)
\item[五]その他参考となるべき事項
\end{description}
\item[\rensuji{2}]非課税貯蓄申込書は、法第十条第一項の規定の適用を受けようとする預貯金等の預入等をする都度、その預入等をする金融機関の営業所等に提出しなければならない。
\item[\rensuji{3}]金融機関の営業所等は、個人の提出する非課税貯蓄申込書に記載された氏名、生年月日及び住所並びに障害者等に該当する事実と法第十条第二項の規定により提示又は送信を受けた同項に規定する書類又は署名用電子証明書等に記載又は記録がされた氏名、生年月日及び住所並びに障害者等に該当する事実並びにその者に係る非課税貯蓄申告書に記載された氏名、生年月日及び住所(第四十三条第一項(非課税貯蓄に関する異動申告書)に規定する申告書の提出があつた場合には、当該申告書に記載された変更後の氏名及び住所)とが異なるときは、当該非課税貯蓄申込書を受理してはならない。
\end{description}
\noindent\hspace{10pt}(普通預金契約等についての非課税貯蓄申込書の特例)
\begin{description}
\item[第三十五条]個人が法第十条第一項(障害者等の少額預金の利子所得等の非課税)の規定の適用を受けようとする預貯金等の預入等をする場合において、その預入等が普通預金その他の財務省令で定める預貯金等に係る契約(以下この条において「普通預金契約等」という。)に基づくものであるときは、その者がその預入等に際して提出する非課税貯蓄申込書には、前条第一項第四号に掲げる事項に代えて、その普通預金契約等に基づいて預入等をする当該財務省令で定める預貯金等の区分及びその預貯金等の現在高(有価証券については、額面金額等により計算した現在高。以下この条において同じ。)に係る限度額を記載することができる。
\item[\rensuji{2}]前項の規定による記載をした非課税貯蓄申込書を提出した場合において、その預貯金等の現在高に係る限度額を変更する必要が生じたときは、その後に提出する非課税貯蓄申込書に変更後の限度額を記載するものとする。
\item[\rensuji{3}]法第十条第一項の規定の適用を受けようとする預貯金等につき第一項の規定による記載をした非課税貯蓄申込書を提出した場合には、その預貯金等については、前条第二項の規定にかかわらず、その現在高がその記載をしたその預貯金等の現在高に係る限度額(前項の規定による記載をした非課税貯蓄申込書を提出した場合には、その提出後においては、変更後の限度額)に達するまでの間は、非課税貯蓄申込書の提出を要しない。
\item[\rensuji{4}]第一項又は第二項の規定による記載をした非課税貯蓄申込書を提出した個人が、その提出後において障害者等に該当しないこととなつた場合には、その者は、遅滞なく、当該申込書を提出した金融機関の営業所等の長に、障害者等に該当しなくなつた旨その他財務省令で定める事項を記載した届出書を提出しなければならない。
\end{description}
\noindent\hspace{10pt}(障害者等の少額預金の利子所得等が非課税とされない場合等)
\begin{description}
\item[第三十六条]個人が次の各号に掲げる場合に該当することとなつたとき(次項及び第三項に規定する場合に該当する場合を除く。)は、その者が当該各号に規定する契約に基づいて預入等をした預貯金等の利子、収益の分配又は剰余金の配当でその該当することとなつた後に支払を受けるものについては、法第十条第一項(障害者等の少額預金の利子所得等の非課税)の規定は、適用しない。
\begin{description}
\item[一]法第十条第一項の規定の適用を受けようとする預貯金等に係る契約に基づいて預入等をする預貯金等の一部につき非課税貯蓄申込書の提出をしなかつた場合(前条第三項の規定に該当する場合を除く。)
\item[二]前条第一項の規定による記載をした非課税貯蓄申込書を提出した場合において、その記載をした同項に規定する預貯金等の現在高に係る限度額(同条第二項の規定による記載をした非課税貯蓄申込書を提出した場合には、その提出後においては、変更後の限度額)を超えて同条第一項に規定する普通預金契約等に基づく預入等をしたとき。
\end{description}
\item[\rensuji{2}]預貯金等に係る契約に基づいて預入等をする預貯金等につき非課税貯蓄申込書を提出した個人が、その提出の後障害者等に該当しないこととなり、かつ、当該該当しないこととなつた後において当該契約に基づき当該預貯金等の預入等をする場合における当該該当しないこととなつた日以後に当該預入等をした法第十条第一項の規定の適用がない預貯金等に係る部分の利子、収益の分配又は剰余金の配当の計算については、財務省令で定める。
\item[\rensuji{3}]普通預金その他の財務省令で定めるもの(以下この項において「普通預金等」という。)につき非課税貯蓄申込書を提出した個人が、その提出の後障害者等に該当しないこととなつた場合には、当該該当しないこととなつた日の属する利子の計算期間に係る利子に対する法第十条の規定の適用については、当該計算期間内における当該普通預金等の預入は、同条第二項の規定に従つて行われたものとみなし、当該計算期間後最初の利子の計算期間に係る利子に対する同条又は前項の規定の適用については、当該計算期間の初日における当該普通預金等の現在高は、同日においてその預入が行われたものとみなす。
\end{description}
\noindent\hspace{10pt}(有価証券の記録等)
\begin{description}
\item[第三十七条]法第十条第一項第二号(障害者等の少額預金の利子所得等の非課税)に規定する政令で定める方法は、個人が同号の金融機関の営業所等において同項の規定の適用を受けようとする貸付信託又は特定公募公社債等運用投資信託の信託をする際に、その貸付信託又は特定公募公社債等運用投資信託の受益権につき、当該金融機関の営業所等に係る金融機関の振替口座簿に記載又は記録を受ける方法(その受益権を表示する受益証券が記名式である場合には、その受益証券につき、当該金融機関の営業所等において第四十八条第三項(金融機関の営業所等における非課税貯蓄に関する帳簿の整理保存)の帳簿に法第十条第一項の規定の適用がある旨の記載又は記録を受ける方法)とする。
\item[\rensuji{2}]法第十条第一項第三号に規定する政令で定める方法は、個人が同号の金融機関の営業所等において同項の規定の適用を受けようとする有価証券の購入をする際に、その有価証券につき、当該金融機関の営業所等に係る金融機関の振替口座簿に記載又は記録を受ける方法とする。
\item[\rensuji{3}]個人が、法第十条第一項の規定の適用を受けようとする前項ただし書に規定する有価証券の購入をする場合において、同項の支払事務取扱者に保管を委託するときは、その保管の取次ぎをする同項の金融機関の営業所等の長は、当該支払事務取扱者に対し、その保管の取次ぎをする際、その有価証券が同条第一項の規定の適用に係るものである旨を通知しなければならない。
\item[\rensuji{4}]第一項の金融機関の営業所等の長又は第二項の金融機関の営業所等(同項の保管の取次ぎをするものを除く。)の長若しくは前項の通知を受けた支払事務取扱者は、貸付信託若しくは特定公募公社債等運用投資信託の受益権若しくは有価証券の振替に関する帳簿又は有価証券の保管に関する帳簿に、その受益権又は有価証券が法第十条第一項の規定の適用に係るものである旨を記載し、又は記録しなければならない。
\end{description}
\noindent\hspace{10pt}(金融機関の営業所等の長の支払事務取扱者に対する通知等)
\begin{description}
\item[第三十八条]前条第一項又は第二項の金融機関の営業所等(貸付信託若しくは特定公募公社債等運用投資信託の受益権又は有価証券に係る支払事務取扱者でないものに限る。)の長は、当該受益権又は有価証券が法第十条(障害者等の少額預金の利子所得等の非課税)に規定する要件を満たすものである場合には、その支払事務取扱者に対し、その収益の分配、利子又は剰余金の配当の支払期ごとに、当該受益権又は有価証券が同条第一項の規定の適用に係るものである旨を通知しなければならない。
\item[\rensuji{2}]前条第二項の金融機関の営業所等(同項の保管の取次ぎをするものに限る。)の長は、次の各号に掲げる場合には、同項の支払事務取扱者に対し、当該各号に規定する事由が生じた都度、当該各号に掲げる事項を通知しなければならない。
\begin{description}
\item[一]法第十条第一項の規定の適用を受ける有価証券につき個人から提出された第四十三条第一項から第三項まで(非課税貯蓄に関する異動申告書)に規定する申告書又は第四十五条第一項(非課税貯蓄廃止申告書)に規定する非課税貯蓄廃止申告書を受理した場合
\item[二]前号に規定する個人の相続人から提出された第四十六条第一項(非課税貯蓄者死亡届出書等)に規定する届出書を受理した場合
\item[三]第一号に規定する個人につき第四十五条第五項又は第四十六条第二項に規定する書類を提出する場合(前号に掲げる場合に該当する場合を除く。)
\item[四]第一号に規定する個人がその金融機関の営業所等において非課税貯蓄申込書を提出して購入した有価証券の額面金額等の合計額が、その者がその金融機関の営業所等を経由して提出した非課税貯蓄申告書に記載された法第十条第三項第三号に掲げる最高限度額(同条第四項の申告書の提出があつた場合には、その提出の日以後においては、変更後の最高限度額)を超えることとなり、又はその超えた後再び当該最高限度額を超えないこととなつた場合
\end{description}
\item[\rensuji{3}]次に掲げる申告書若しくは届出書又は前項第一号若しくは第二号の申告書若しくは届出書の受理をした金融機関の営業所等(前条第二項の保管の取次ぎをするものを除く。)の長はこれらの申告書又は届出書に記載された事項を、前項の規定による通知を受けた支払事務取扱者は当該通知の内容を、貸付信託若しくは特定公募公社債等運用投資信託の受益権若しくは有価証券の振替に関する帳簿又は有価証券の保管に関する帳簿に、記載し、又は記録しなければならない。
\begin{description}
\item[一]法第十条第一項の規定の適用を受ける貸付信託又は特定公募公社債等運用投資信託の受益権につき個人から提出された第四十三条第一項から第三項までに規定する申告書又は第四十五条第一項に規定する非課税貯蓄廃止申告書
\item[二]前号に規定する個人の相続人から提出された第四十六条第一項に規定する届出書
\end{description}
\end{description}
\noindent\hspace{10pt}(非課税限度額の計算等)
\begin{description}
\item[第三十九条]法第十条第一項第三号(障害者等の少額預金の利子所得等の非課税)に規定する政令で定めるものは、投資信託(同項に規定する委託者非指図型投資信託を除く。)については、その設定又は追加設定があつた時において当該投資信託につき信託又は追加信託がされた金額をその時における当該信託又は追加信託についての受益権の口数で除して計算した金額とし、特定目的信託については、第三十三条第四項第八号(利子所得等について非課税とされる預貯金等の範囲)に掲げる社債的受益権に係る元本の額(資産の流動化に関する法律施行令(平成十二年政令第四百七十九号)第五十二条第二項第三号(社債的受益権を定める特定目的信託契約に付すべき条件)に規定する元本の額をいう。)をその受益権の口数で除して計算した金額とする。
\item[\rensuji{2}]第三十五条第一項(普通預金契約等についての非課税貯蓄申込書の特例)の規定による記載がされた非課税貯蓄申込書に係る同項に規定する普通預金契約等に基づいて預入等をされた預貯金等については、当該申込書の提出のあつた日以後においては、当該申込書を提出した者が引き続き当該申込書に記載された預貯金等の現在高(有価証券については、額面金額等により計算した現在高。次項において同じ。)に係る限度額(同条第二項の規定による記載がされた非課税貯蓄申込書が提出された場合には、その提出があつた日以後においては、変更後の限度額)に相当する金額の当該申込書に係る預貯金等を有しているものとみなして、法第十条第一項各号に規定する元本の合計額又は額面金額等の合計額を計算するものとする。
\item[\rensuji{3}]個人が非課税貯蓄申込書を提出して預入等をした預貯金等の法第十条第一項各号に規定する元本の合計額又は額面金額等の合計額が、その預貯金等の利子、収益の分配又は剰余金の配当の計算期間を通じて当該各号に規定する最高限度額を超えないかどうかは、その計算期間中のいずれの日においてもその預貯金等(その日以前に第三十六条第一項各号(障害者等の少額預金の利子所得等が非課税とされない場合)の規定に該当するに至つたものを除く。)の最終の現在高の合計額が当該最高限度額を超えていないかどうかにより、判定するものとする。
\end{description}
\noindent\hspace{10pt}(非課税貯蓄申告書)
\begin{description}
\item[第四十条]国内に住所を有する個人が非課税貯蓄申告書を提出する場合には、当該申告書に記載する法第十条第三項第三号(非課税貯蓄申告書の記載事項)に掲げる最高限度額は、一万円に整数を乗じた金額で、かつ、三百万円(当該申告書に記載すべき同項第四号に掲げる最高限度額がある場合には、三百万円から当該最高限度額の合計額を控除した残額)以下の金額としなければならない。
\end{description}
\noindent\hspace{10pt}(非課税貯蓄限度額変更申告書)
\begin{description}
\item[第四十一条]法第十条第四項(障害者等の少額預金の利子所得等の非課税)の規定による申告書(以下この節において「非課税貯蓄限度額変更申告書」という。)には、次に掲げる事項を記載しなければならない。
\begin{description}
\item[一]提出者の氏名、生年月日、住所及び個人番号(行政手続における特定の個人を識別するための番号の利用等に関する法律(平成二十五年法律第二十七号)第二条第五項(定義)に規定する個人番号をいう。以下同じ。)
\item[二]障害者等に該当する事実
\item[三]その金融機関の営業所等の名称及び所在地
\item[四]預貯金等のうち提出者がその金融機関の営業所等を経由して提出した非課税貯蓄申告書に記載したものの種別
\item[五]前号の非課税貯蓄申告書に記載した法第十条第三項第三号に掲げる最高限度額(当該申告書につき既に非課税貯蓄限度額変更申告書を提出している場合には、当該申告書に記載した変更後の最高限度額)
\item[六]変更後の最高限度額
\item[七]他の金融機関の営業所等を経由して非課税貯蓄申告書を提出している場合には、当該申告書に記載した法第十条第三項第四号に掲げる最高限度額の合計額
\item[八]第四号の非課税貯蓄申告書の提出年月日その他参考となるべき事項
\end{description}
\item[\rensuji{2}]非課税貯蓄限度額変更申告書に記載することができる前項第六号の変更後の最高限度額は、一万円に整数を乗じた金額で、かつ、三百万円(当該申告書に記載すべき同項第七号に掲げる最高限度額の合計額がある場合には、三百万円から当該合計額を控除した残額)以下の金額とする。
\item[\rensuji{3}]非課税貯蓄限度額変更申告書は、その提出しようとする際に、国内に住所を有しない個人及び障害者等に該当しない個人については、その提出をすることができない。
\end{description}
\noindent\hspace{10pt}(障害者等に該当する旨を証する書類の範囲等)
\begin{description}
\item[第四十一条の二]法第十条第二項(障害者等の少額預金の利子所得等の非課税)に規定する政令で定める書類は、障害者等の身体障害者手帳、遺族基礎年金の年金証書その他の財務省令で定める書類のうちいずれかの書類(以下この項、第三項及び第五項において「障害者等確認書類」という。)(当該障害者等確認書類に当該障害者等の生年月日又は住所が記載されていない場合には、当該障害者等確認書類及び住所等確認書類(当該障害者等の氏名、生年月日及び住所を証する住民票の写し、健康保険の被保険者証、運転免許証その他の財務省令で定める書類のうちいずれかの書類をいう。次項において同じ。))とする。
\item[\rensuji{2}]法第十条第二項に規定する政令で定めるところにより行う同項に規定する署名用電子証明書等の送信は、住所等確認書類の提示に代えて行う当該署名用電子証明書等の送信とする。
\item[\rensuji{3}]法第十条第五項に規定する政令で定める書類は、障害者等確認書類及び本人確認書類(行政手続における特定の個人を識別するための番号の利用等に関する法律第二条第七項(定義)に規定する個人番号カードその他の財務省令で定める書類をいう。以下この条及び第四十三条第一項(非課税貯蓄に関する異動申告書)において同じ。)とする。
\item[\rensuji{4}]法第十条第五項に規定する政令で定めるところにより行う同項に規定する署名用電子証明書等の送信は、本人確認書類の提示に代えて行う当該署名用電子証明書等の送信とする。
\item[\rensuji{5}]金融機関の営業所等の長が、財務省令で定めるところにより、非課税貯蓄申告書を提出した者の氏名、生年月日、住所及び個人番号並びに障害者等に該当する事実その他の事項を記載した帳簿(その者からその者の障害者等確認書類及び本人確認書類の写しを添付した申請書又はその提出の際にその者の法第十条第五項に規定する署名用電子証明書等(この項を除き、以下この節において「署名用電子証明書等」という。)の送信を受けている申請書(その者の障害者等確認書類の写しを添付したものに限る。)の提出を受けて作成されたものに限る。)を備えているときは、その者は、同条第二項の規定にかかわらず、当該金融機関の営業所等に対して提出する非課税貯蓄申込書にその旨の記載をすることにより同項の書類の提示(第二項に定めるところにより行う同項に規定する署名用電子証明書等の送信を含む。第四十七条第二項(非課税貯蓄相続申込書)において同じ。)に代えることができる。
\end{description}
\noindent\hspace{10pt}(非課税貯蓄申告書への確認した旨の証印等)
\begin{description}
\item[第四十一条の三]金融機関の営業所等の長は、法第十条第五項(障害者等の少額預金の利子所得等の非課税)の規定による告知があつた場合には、その告知に係る非課税貯蓄申告書又は非課税貯蓄限度額変更申告書に、当該告知があつた事項につき確認した旨の証印をし、財務省令で定める事項を記載しなければならない。
\item[\rensuji{2}]金融機関の営業所等の長は、前項の規定により非課税貯蓄申告書又は非課税貯蓄限度額変更申告書に確認した旨の証印をする場合には、第四十八条第四項(金融機関の営業所等における非課税貯蓄に関する帳簿書類の整理保存等)の規定により作成するこれらの申告書の写しに当該確認した旨の証印をした事実を記録しておかなければならない。
\end{description}
\noindent\hspace{10pt}(同一金融機関の営業所等を経由して重ねて提出できる非課税貯蓄申告書の範囲)
\begin{description}
\item[第四十二条]法第十条第七項(障害者等の少額預金の利子所得等の非課税)に規定する政令で定める非課税貯蓄申告書は、次に掲げるものとする。
\begin{description}
\item[一]既に提出した非課税貯蓄申告書の提出の際に経由した金融機関の営業所等が、次に掲げる金融機関の営業所又は事務所(次項において「信託銀行の営業所等」という。)である場合において、預貯金等のうち当該申告書に記載したもの以外の種別の預貯金等につき提出する非課税貯蓄申告書
\begin{description}
\item[イ]金融機関の信託業務の兼営等に関する法律(昭和十八年法律第四十三号)により同法第一条第一項(兼営の認可)に規定する信託業務を営む同項に規定する金融機関、長期信用銀行法第二条(定義)に規定する長期信用銀行、金融機関の合併及び転換に関する法律第八条第一項(特定社債の発行)に規定する普通銀行で同項(同法第五十五条第四項(長期信用銀行が普通銀行となる転換)において準用する場合を含む。)の認可を受けたもの(会社法の施行に伴う関係法律の整備等に関する法律第二百条第一項(金融機関の合併及び転換に関する法律の一部改正に伴う経過措置)の規定によりなお従前の例によることとされる同法第百九十九条(金融機関の合併及び転換に関する法律の一部改正)の規定による改正前の金融機関の合併及び転換に関する法律第十七条の二第一項(債券の発行の特例)に規定する普通銀行で同項(同法第二十四条第一項第七号(合併に関する規定の準用)において準用する場合を含む。)の認可を受けたものを含む。)、信用金庫法第五十四条の二の四第一項(全国連合会債の発行)に規定する全国を地区とする信用金庫連合会で同条第三項により認可を受けたもの、農林中央金庫又は株式会社商工組合中央金庫
\item[ロ]金融商品取引法第三十三条の二(金融機関の登録)の登録を受けた銀行、信用金庫、信用金庫連合会、労働金庫、労働金庫連合会、信用協同組合、信用協同組合連合会、農業協同組合、農業協同組合連合会、漁業協同組合、漁業協同組合連合会、水産加工業協同組合又は水産加工業協同組合連合会(イに掲げる金融機関に該当するものを除く。)
\end{description}
\item[二]既に第四十五条第一項(非課税貯蓄廃止申告書)に規定する非課税貯蓄廃止申告書を提出している場合又は同条第四項の規定により当該申告書の提出があつたとみなされる場合において、同条第一項又は第五項の金融機関の営業所等を経由して再び当該申告書に係る種別の預貯金等につき提出する非課税貯蓄申告書
\end{description}
\item[\rensuji{2}]信託銀行の営業所等を経由して提出する非課税貯蓄申告書に係る法第十条第三項の規定及び第四十一条第一項(非課税貯蓄限度額変更申告書)の規定の適用については、法第十条第三項第三号中「預貯金、合同運用信託、特定公募公社債等運用投資信託又は有価証券で」とあるのは「預貯金、合同運用信託、特定公募公社債等運用投資信託又は有価証券ごとに」と、同項第四号中「既に」とあるのは「既に当該金融機関の営業所等又は」と、「当該他の」とあるのは「当該金融機関の営業所等及び他の」と、第四十一条第一項第七号中「他の」とあるのは「当該金融機関の営業所等又は他の」とする。
\end{description}
\noindent\hspace{10pt}(非課税貯蓄に関する異動申告書)
\begin{description}
\item[第四十三条]非課税貯蓄申告書を提出した個人が、その提出後、次に掲げる場合に該当することとなつた場合には、その者は、遅滞なく、その旨その他財務省令で定める事項を記載した申告書を、当該非課税貯蓄申告書の提出をした金融機関の営業所等(次項若しくは第三項又は次条第一項に規定する場合に該当するときは、これらの規定に規定する移管先の営業所等)を経由し、その者の住所地(国内における住所の変更についてはその変更前の住所地とし、国外の場所から従前の住所地以外の国内の場所への住所の変更についてはその従前の住所地とする。)の所轄税務署長に提出しなければならない。
\begin{description}
\item[一]その者の氏名又は住所の変更をした場合(住所の変更については、国内における住所の変更及び国外の場所から従前の住所地以外の国内の場所への住所の変更に限る。)
\item[二]その者の個人番号の変更をした場合
\end{description}
\item[\rensuji{2}]非課税貯蓄申告書を提出した個人が、その提出後、その者の法第十条第一項(障害者等の少額預金の利子所得等の非課税)の規定の適用を受ける預貯金等の受入れ又は引受けをしている金融機関の営業所等(以下この条において「移管前の営業所等」という。)に対して当該預貯金等に関する事務の全部を移管前の営業所等以外の金融機関の営業所等(当該申告書に記載した移管前の営業所等に係る第三十二条各号(金融機関等の範囲)に掲げる者又はその者と預貯金に係る債務の承継に関する契約を締結している者の営業所、事務所その他これらに準ずるものに限る。以下この項において「移管先の営業所等」という。)に移管すべきことを依頼し、かつ、その移管がされることとなつた場合において、当該預貯金等につき引き続き移管先の営業所等において法第十条第一項の規定の適用を受けようとするときは、当該個人は、当該移管を依頼する際、その旨、その者の氏名、生年月日、住所及び個人番号その他財務省令で定める事項を記載した申告書を、移管前の営業所等及び移管先の営業所等を経由して、その者の住所地の所轄税務署長に提出しなければならない。
\item[\rensuji{3}]非課税貯蓄申告書を提出した個人が、その提出後、その者の法第十条第一項の規定の適用を受ける有価証券(合同運用信託等に係る無記名の貸付信託又は特定公募公社債等運用投資信託の受益証券を含む。以下この条において「特定有価証券」という。)につきその取得をし、かつ、当該特定有価証券につき第三十七条第一項又は第二項(有価証券の記録等)の規定により金融機関の振替口座簿に記載若しくは記録をし、若しくは保管の委託を受け、又は保管の取次ぎをした金融機関の営業所等(以下この条において「特定営業所等」という。)に係る第三十二条各号に掲げる者(以下この項において「特定金融機関」という。)の特定業務(有価証券(合同運用信託等に係る無記名の貸付信託又は特定公募公社債等運用投資信託の受益証券を含む。)の当該個人による特定営業所等における購入に係る業務をいう。以下この項において同じ。)につき次に掲げる事由が生じたことにより、当該事由が生じた日から起算して一年を経過する日(当該事由が第一号に掲げるものであつて、同日前に同号の特定業務の停止につき定められた期間が終了する場合には、その終了の日)までの間に特定営業所等に対してその者の当該特定有価証券に関する事務の全部を特定営業所等以外の金融機関の営業所等(特定金融機関と特定有価証券に関する事務の移管(当該個人が特定営業所等にその取得をした特定有価証券の保管の委託をしている場合には、特定有価証券の保管の委託に係る契約の承継を含む。以下この条において同じ。)に関する契約を締結している者の営業所、事務所その他これらに準ずるものに限る。以下この項において「移管先の営業所等」という。)に移管すべきことを依頼し、かつ、その移管がされることとなつた場合において、その取得をした特定有価証券につき引き続き移管先の営業所等において法第十条第一項の規定の適用を受けようとするときは、当該個人は、当該移管を依頼する際、その旨、その者の氏名、生年月日、住所及び個人番号その他財務省令で定める事項を記載した申告書を特定営業所等及び移管先の営業所等を経由して、その者の住所地の所轄税務署長に提出しなければならない。
\begin{description}
\item[一]法律の規定に基づく措置として当該特定業務の停止を命ぜられたこと。
\item[二]当該特定業務を廃止したこと。
\item[三]当該特定業務に係る免許、認可、承認又は登録が取り消されたこと(既に前号に掲げる事由が生じている場合を除く。)。
\item[四]当該特定業務を行う特定営業所等に係る特定金融機関が解散したこと(既に前二号に掲げる事由が生じている場合を除く。)。
\end{description}
\item[\rensuji{4}]前二項の申告書がこれらの規定に規定する移管先の営業所等に受理されたときは、これらの規定による移管があつた日以後における当該移管があつた預貯金等に係る法第十条及びこの節の規定の適用については、当該預貯金等に係る移管前の営業所等又は特定営業所等の長がした非課税貯蓄申込書の受理、同条第五項の規定による確認した旨の証印その他の手続は、当該移管先の営業所等の長がしたものとみなす。
\item[\rensuji{5}]前項後段の規定の適用を受ける個人は、同項に規定する移管があつた日以後、遅滞なく、法第十条及びこの節に定めるところにより、同項後段の規定により変更があつたものとみなされる変更後の最高限度額につき、非課税貯蓄限度額変更申告書を提出しなければならない。
\item[\rensuji{6}]第一項から第三項までの規定による申告書(以下この節において「非課税貯蓄に関する異動申告書」という。)がこれらの規定に規定する税務署長に提出された場合には、これらの規定に規定する金融機関の営業所等においてこれを受理した日に、その提出がされたものとみなす。
\item[\rensuji{7}]第二項の規定による預貯金等の移管又は第三項の規定による特定有価証券に関する事務の移管があつた後においては、これらの移管に係る預貯金等についての非課税貯蓄申込書は、これらの規定に規定する移管先の営業所等に対してのみ提出することができる。
\end{description}
\noindent\hspace{10pt}(金融機関等において事業譲渡等があつた場合の申告)
\begin{description}
\item[第四十四条]事業の譲渡若しくは合併若しくは分割又は金融機関の営業所等の新設若しくは廃止若しくは業務を行う区域の変更により、非課税貯蓄申告書を提出した個人が預入等をした預貯金等のうち法第十条第一項(障害者等の少額預金の利子所得等の非課税)の規定の適用を受けるものの事務の全部が、その事業の譲渡を受けた第三十二条各号(金融機関等の範囲)に掲げる者(以下この条において「金融機関等」という。)若しくはその合併により設立した金融機関等若しくはその合併後存続する金融機関等若しくはその分割により資産及び負債の移転を受けた金融機関等の営業所、事務所その他これらに準ずるもの又は同一の金融機関等の他の営業所、事務所その他これらに準ずるもの(以下この条において「移管先の営業所等」という。)に移管された場合には、当該移管先の営業所等の長は、遅滞なく、その旨及び当該移管された預貯金等に係る法第十条第三項各号に掲げる事項その他の財務省令で定める事項を記載した書類を当該移管先の営業所等の所在地の所轄税務署長に提出しなければならない。
\item[\rensuji{2}]前項の書類が同項の所轄税務署長において受理されたときは、移管された日以後における当該移管された預貯金等に係る法第十条及びこの節の規定の適用については、当該預貯金等に係る移管前の営業所等(当該預貯金等を移管した金融機関の営業所等をいう。)の長がした非課税貯蓄申込書の受理、同条第五項の規定による確認した旨の証印その他の手続は、当該移管先の営業所等の長がしたものとみなす。
\item[\rensuji{3}]前条第七項の規定は、第一項の移管された預貯金等に係る非課税貯蓄申込書の提出について準用する。
\end{description}
\noindent\hspace{10pt}(非課税貯蓄廃止申告書)
\begin{description}
\item[第四十五条]非課税貯蓄申告書を提出した個人が、その提出後、当該申告書の提出の際に経由した金融機関の営業所等において預入等をした当該申告書に記載した預貯金等につき法第十条第一項(障害者等の少額預金の利子所得等の非課税)の規定の適用を受けることをやめようとする場合には、その者は、その旨その他財務省令で定める事項を記載した申告書(以下この節において「非課税貯蓄廃止申告書」という。)を、当該預貯金等の受入れ又は引受けをする金融機関の営業所等を経由し、その者の住所地の所轄税務署長に提出しなければならない。
\item[\rensuji{2}]非課税貯蓄廃止申告書が前項の税務署長に提出された場合には、同項の金融機関の営業所等においてこれを受理した日に、その提出がされたものとみなす。
\item[\rensuji{3}]非課税貯蓄廃止申告書の提出があつた場合には、その提出があつた日後に支払の確定する第一項に規定する預貯金等の利子、収益の分配又は剰余金の配当については、法第十条第一項の規定は、適用しない。
\item[\rensuji{4}]非課税貯蓄申告書を提出した個人が、当該申告書の提出の際に経由した金融機関の営業所等において預入等をした当該申告書に記載した預貯金等(法第十条第一項の規定の適用を受けるものに限る。以下この項において同じ。)を有しないこととなつた場合において、その有しないこととなつた日以後二年を経過する日の属する年の十二月三十一日までの間に、当該金融機関の営業所等において当該預貯金等の預入等をしなかつたとき(当該預貯金等につき非課税貯蓄廃止申告書を提出した場合を除く。)は、その翌年一月一日に当該預貯金等につき非課税貯蓄廃止申告書の提出があつたものとみなす。
\item[\rensuji{5}]前項の金融機関の営業所等の長は、同項の規定により非課税貯蓄廃止申告書の提出があつたものとみなされる個人の各人別に、当該個人の氏名、生年月日、住所及び個人番号その他の財務省令で定める事項を記載した書類を、当該申告書の提出があつたものとみなされる日の属する月の翌月十日までに当該金融機関の営業所等の所在地の所轄税務署長に提出しなければならない。
\end{description}
\noindent\hspace{10pt}(非課税貯蓄者死亡届出書等)
\begin{description}
\item[第四十六条]非課税貯蓄申告書を提出した個人が死亡したときは、その者の相続人は、当該申告書に係る預貯金等で法第十条第一項(障害者等の少額預金の利子所得等の非課税)の規定の適用に係るものの利子、収益の分配又は剰余金の配当につきその相続の開始があつたことを知つた日以後最初に支払がされる日までに、その旨その他財務省令で定める事項を記載した届出書を、当該預貯金等の受入れ又は引受けをしている金融機関の営業所等の長に提出しなければならない。
\item[\rensuji{2}]前項の金融機関の営業所等の長は、同項の届出書(以下この節において「非課税貯蓄者死亡届出書」という。)を受理した場合又は業務に関連して非課税貯蓄申告書を提出した個人が死亡したことを知つた場合には、当該届出書を提出した者の被相続人又は当該死亡した個人の各人別に、これらの者の氏名、生年月日及び住所その他の財務省令で定める事項を記載した書類を、当該届出書を受理した日又は当該死亡したことを知つた日の属する月の翌月十日までに当該金融機関の営業所等の所在地の所轄税務署長に提出しなければならない。
\end{description}
\noindent\hspace{10pt}(非課税貯蓄相続申込書)
\begin{description}
\item[第四十七条]前条第一項に規定する相続人のうちに同項に規定する預貯金等と同一の種別の預貯金等につき同項に規定する預貯金等の受入れ又は引受けをしている金融機関の営業所等に非課税貯蓄申込書を提出することができる障害者等である者がある場合において、その者が、相続により取得したその被相続人に係る預貯金等で法第十条第一項(障害者等の少額預金の利子所得等の非課税)の規定の適用に係るものにつき引き続き同項の規定の適用を受けたい旨、その適用を受けようとする預貯金等の金額(当該預貯金等が有価証券である場合には、その額面金額等)、障害者等に該当する旨その他財務省令で定める事項を記載した書類(以下この節において「非課税貯蓄相続申込書」という。)を、前条第一項に規定する支払がされる日までに、その金融機関の営業所等に提出したときは、法第十条第一項及びこの節の規定の適用については、その者がその金融機関の営業所等においてその非課税貯蓄相続申込書を提出した日に非課税貯蓄申込書を提出して当該金額に相当する預貯金等の預入等をしたものとみなす。
\item[\rensuji{2}]非課税貯蓄相続申込書を提出する者は、その提出の際、前項の金融機関の営業所等の長にその者の法第十条第二項に規定する書類の提示をしなければならない。
\item[\rensuji{3}]第三十四条第三項(非課税貯蓄申込書の記載事項及び提出)及び第四十一条の二第五項(障害者等に該当する旨を証する書類の範囲等)の規定は、非課税貯蓄相続申込書の受理について準用する。
\end{description}
\noindent\hspace{10pt}(金融機関の営業所等の非課税貯蓄申告書の税務署長への送付等)
\begin{description}
\item[第四十七条の二]金融機関の営業所等の長は、非課税貯蓄申告書、非課税貯蓄限度額変更申告書、非課税貯蓄に関する異動申告書又は非課税貯蓄廃止申告書を受理した場合には、その受理した日の属する月の翌月十日までに、これらの申告書を当該金融機関の営業所等の所在地の所轄税務署長に送付しなければならない。
\end{description}
\noindent\hspace{10pt}(金融機関の営業所等における非課税貯蓄に関する帳簿書類の整理保存等)
\begin{description}
\item[第四十八条]金融機関の営業所等の長は、非課税貯蓄申込書又は非課税貯蓄相続申込書の提出を受けた場合には、これらの申込書を提出して預入等がされた預貯金等に関する通帳、証書、証券その他の書類(第三十七条第一項又は第二項(有価証券の記録等)の規定により金融機関の振替口座簿に記載若しくは記録をし、若しくは保管の委託を受け、又は保管の取次ぎをする預貯金等に係るものを除く。)に、その預貯金等が法第十条第一項(障害者等の少額預金の利子所得等の非課税)の規定の適用に係るものである旨の証印(証印に準ずる表示を含む。次項において同じ。)をし、かつ、これらの申込書を財務省令で定めるところにより保存しなければならない。
\item[\rensuji{2}]金融機関の営業所等の長は、前項の預貯金等に係る非課税貯蓄廃止申告書若しくは非課税貯蓄者死亡届出書を受理した場合又は第四十五条第五項(非課税貯蓄廃止申告書)若しくは第四十六条第二項(非課税貯蓄者死亡届出書等)に規定する書類を提出した場合には、遅滞なく、その預貯金等についてした前項の証印を抹消しなければならない。
\item[\rensuji{3}]金融機関の営業所等の長は、非課税貯蓄申込書を提出して預入等がされた預貯金等につき帳簿を備え、各人別に、その預貯金等の元本又は額面金額等及びその利子、収益の分配又は剰余金の配当の計算に関する事項を明らかにし、かつ、当該帳簿を財務省令で定めるところにより保存しなければならない。
\item[\rensuji{4}]金融機関の営業所等の長は、非課税貯蓄申告書、非課税貯蓄限度額変更申告書若しくは非課税貯蓄に関する異動申告書を受理した場合又は第四十五条第五項若しくは第四十六条第二項に規定する書類を提出する場合には、財務省令で定めるところにより、これらの申告書又は書類の写し(これに準ずるものを含む。)を作成し、これを保存しなければならない。
\item[\rensuji{5}]金融機関の営業所等の長は、第四十一条の二第五項(障害者等に該当する旨を証する書類の範囲等)に規定する帳簿を作成し、又は第三十五条第四項(普通預金契約等についての非課税貯蓄申込書の特例)に規定する届出書、第四十一条の二第五項に規定する申請書(同項に規定する障害者等確認書類及び本人確認書類並びに署名用電子証明書等を含む。)若しくは非課税貯蓄者死亡届出書を受理した場合には、財務省令で定めるところにより、当該帳簿又は届出書若しくは申請書を保存しなければならない。
\item[\rensuji{6}]第三十七条第四項の金融機関の営業所等及び支払事務取扱者は同項に規定する貸付信託若しくは特定公募公社債等運用投資信託の受益権若しくは有価証券の振替に関する帳簿又は有価証券の保管に関する帳簿を、第三十八条第一項(金融機関の営業所等の長の支払事務取扱者に対する通知)の支払事務取扱者は同項に規定する通知の内容を記載した書類を、財務省令で定めるところにより保存しなければならない。
\end{description}
\noindent\hspace{10pt}(非課税貯蓄申告書等の書式)
\begin{description}
\item[第四十九条]非課税貯蓄申告書、非課税貯蓄申込書、非課税貯蓄限度額変更申告書、非課税貯蓄に関する異動申告書、非課税貯蓄廃止申告書及び非課税貯蓄相続申込書の書式は、財務省令で定める。
\end{description}
\noindent\hspace{10pt}(金融機関の営業所等の届出及び営業所番号)
\begin{description}
\item[第五十条]金融機関の営業所等の長は、財務省令で定めるところにより、当該金融機関の営業所等の名称及び所在地並びに当該金融機関の営業所等に係る金融機関等(第三十二条各号(金融機関等の範囲)に掲げる者をいう。)の個人番号又は法人番号(行政手続における特定の個人を識別するための番号の利用等に関する法律第二条第十五項(定義)に規定する法人番号をいう。以下同じ。)(個人番号を有しない個人にあつては、名称及び所在地)その他の事項を記載した届出書を、当該金融機関の営業所等の所在地の所轄税務署長を経由して、国税庁長官に提出しなければならない。
\item[\rensuji{2}]国税庁長官は、前項の届出書の提出があつた場合には、当該届出書に係る金融機関の営業所等の全部又は一部につき、当該金融機関の営業所等ごとの番号(以下この条において「営業所番号」という。)を定め、又は当該営業所番号を変更することができる。
\item[\rensuji{3}]国税庁長官は、前項の規定により営業所番号を定め、又は変更した場合には、当該金融機関の営業所等の長に対し、書面によりその旨及び当該営業所番号を通知するものとする。
\item[\rensuji{4}]営業所番号の通知を受けた金融機関の営業所等の長は、税務署長に提出するこの節に規定する書類には、当該営業所番号を付記するものとする。
\end{description}
\subsection*{第四節 公共法人等及び公益信託等に係る非課税}
\addcontentsline{toc}{subsection}{第四節 公共法人等及び公益信託等に係る非課税}
\noindent\hspace{10pt}(貸付信託の受益権の収益の分配のうち公共法人等が引き続き所有していた期間の金額)
\begin{description}
\item[第五十一条]法第十一条第一項及び第二項(公共法人等及び公益信託等に係る非課税)に規定する政令で定めるところにより計算した金額は、次の各号に掲げる場合の区分に応じ当該各号に定める金額とする。
\begin{description}
\item[一]法第十一条第一項に規定する内国法人(以下この条から第五十一条の四まで(公社債等の利子等に係る非課税申告書の提出)において「公共法人等」という。)又は法第十一条第二項に規定する公益信託若しくは加入者保護信託(以下この条から第五十一条の四までにおいて「公益信託等」という。)の受託者が、その所有し、又はその公益信託等の信託財産に属する貸付信託の受益権の収益の分配の計算期間を通じて第五十一条の三第一項(公社債等に係る有価証券の記録等)の規定により金融機関の振替口座簿(第三十二条第一号、第四号及び第五号(金融機関等の範囲)に掲げる者が社債、株式等の振替に関する法律の規定により備え付ける振替口座簿をいう。以下この条及び第五十一条の三において同じ。)に記載若しくは記録を受け、又は保管の委託をしている場合
\item[二]公共法人等又は公益信託等の受託者が、その所有し、又はその公益信託等の信託財産に属する貸付信託の受益権につきその収益の分配の計算期間の中途において第五十一条の三第一項の規定により金融機関の振替口座簿に記載若しくは記録を受け、又は保管の委託をし、かつ、その記載若しくは記録を受け、又は保管の委託をした日から当該計算期間の終了の日までの期間を通じて金融機関の振替口座簿に記載若しくは記録を受け、又は保管の委託をしている場合
\end{description}
\end{description}
\noindent\hspace{10pt}(公社債等の範囲)
\begin{description}
\item[第五十一条の二]法第十一条第三項(公共法人等及び公益信託等に係る非課税)に規定する政令で定める受益権は、次に掲げる受益権とする。
\begin{description}
\item[一]貸付信託の受益権
\item[二]公社債投資信託の受益権
\item[三]公社債等運用投資信託の受益権
\item[四]法第六条の三第四号(受託法人等に関するこの法律の適用)に規定する社債的受益権
\end{description}
\end{description}
\noindent\hspace{10pt}(公社債等に係る有価証券の記録等)
\begin{description}
\item[第五十一条の三]法第十一条第三項(公共法人等及び公益信託等に係る非課税)に規定する政令で定める方法は、公共法人等又は公益信託等の受託者が所有し、又はその公益信託等の信託財産に属する同項に規定する公社債等(以下この条及び次条において「公社債等」という。)の利子等(同項に規定する利子等をいう。次条において同じ。)につき法第十一条第一項及び第二項の規定の適用を受けようとする次の各号に掲げる公社債等の区分に応じ当該各号に定める方法とする。
\begin{description}
\item[一]公社債及び前条各号に掲げる受益権(次号及び第三号に掲げるものを除く。)
\item[二]公社債及び前条第二号又は第三号に掲げる受益権で投資信託委託会社(投資信託及び投資法人に関する法律第二条第十一項(定義)に規定する投資信託委託会社をいう。次項において同じ。)から取得するもの
\item[三]長期信用銀行法第八条(長期信用銀行債の発行)の規定による長期信用銀行債その他財務省令で定める公社債等、記名式の貸付信託及び公募公社債等運用投資信託(投資信託及び投資法人に関する法律第二条第二項に規定する委託者非指図型投資信託に限る。)の受益証券
\end{description}
\item[\rensuji{2}]前項第一号若しくは第三号の金融機関の営業所等又は同項第二号の投資信託委託会社の営業所(次条において「金融機関等の営業所等」という。)は、金融機関の振替口座簿に記載若しくは記録をし、若しくは保管の委託を受けた公社債等又は振替の取次ぎをした公社債等につき、帳簿を備え、その記載若しくは記録を受け、又は保管の委託をした者の各人別に口座を設け、財務省令で定める事項を記載し、又は記録しなければならない。
\item[\rensuji{3}]前二項に定めるもののほか、前項の帳簿の保存その他公社債等に係る有価証券の記載若しくは記録、振替の取次ぎ又は保管の委託に係る手続に関し必要な事項は、財務省令で定める。
\end{description}
\noindent\hspace{10pt}(公社債等の利子等に係る非課税申告書の提出)
\begin{description}
\item[第五十一条の四]公共法人等又は公益信託等の受託者は、その支払を受けるべき公社債等の利子等につき法第十一条第一項及び第二項(公共法人等及び公益信託等に係る非課税)の規定の適用を受けようとする場合には、当該公社債等の利子等の支払を受けるべき日の前日までに、同条第三項に規定する申告書を金融機関等の営業所等及び当該公社債等の利子等の支払をする者を経由してその支払をする者の当該利子等に係る法第十七条(源泉徴収に係る所得税の納税地)の規定による納税地(法第十八条第二項(納税地の指定)の規定による指定があつた場合には、その指定をされた納税地)の所轄税務署長に提出しなければならない。
\item[\rensuji{2}]前項の金融機関等の営業所等の長は、同項の申告書に記載されている公社債等に係る有価証券の記載若しくは記録、振替の取次ぎ又は保管に関する事項と前条第二項の帳簿に記載されている当該公社債等に係る有価証券の記載若しくは記録、振替の取次ぎ又は保管に関する事項とが異なるときは、当該申告書を受理してはならない。
\item[\rensuji{3}]第一項の場合において、同項の申告書が同項の金融機関等の営業所等に受理されたときは、当該申告書は、その受理された日に同項の税務署長に提出されたものとみなす。
\end{description}
\noindent\hspace{10pt}(公共法人等に該当する農業協同組合連合会の要件等)
\begin{description}
\item[第五十一条の五]法別表第一の農業協同組合連合会の項に規定する政令で定める要件は、当該農業協同組合連合会の定款に次に掲げる定めがあることとする。
\begin{description}
\item[一]当該農業協同組合連合会の行う事業は、農業協同組合法第十条第一項第十一号(医療に関する施設)に掲げる事業(これに附帯する事業を含む。)又は当該事業及び同項第十二号(老人の福祉に関する施設)に掲げる事業(これらに附帯する事業を含む。)に限る旨の定め
\item[二]当該農業協同組合連合会は、剰余金の配当(出資に係るものに限る。)を行わない旨の定め
\item[三]当該農業協同組合連合会が解散したときは、その残余財産が国若しくは地方公共団体又は第一号に規定する事業を行う他の農業協同組合連合会に帰属する旨の定め
\end{description}
\item[\rensuji{2}]農業協同組合連合会は、法別表第一の農業協同組合連合会の項に規定する指定を受けようとするときは、その名称及び主たる事務所の所在地、その設置する病院又は診療所の名称及び所在地その他の財務省令で定める事項を記載した申請書に定款の写しその他の財務省令で定める書類を添付し、これを財務大臣に提出しなければならない。
\item[\rensuji{3}]財務大臣は、法別表第一の農業協同組合連合会の項の規定により農業協同組合連合会を指定したときは、これを告示する。
\end{description}
\section*{第三章 所得の帰属に関する通則}
\addcontentsline{toc}{section}{第三章 所得の帰属に関する通則}
\noindent\hspace{10pt}(信託財産に属する資産及び負債並びに信託財産に帰せられる収益及び費用の帰属)
\begin{description}
\item[第五十二条]法第十三条第二項(信託財産に属する資産及び負債並びに信託財産に帰せられる収益及び費用の帰属)に規定する政令で定める権限は、信託の目的に反しないことが明らかである場合に限り信託の変更をすることができる権限とする。
\item[\rensuji{2}]法第十三条第二項に規定する信託の変更をする権限には、他の者との合意により信託の変更をすることができる権限を含むものとする。
\item[\rensuji{3}]停止条件が付された信託財産の給付を受ける権利を有する者は、法第十三条第二項に規定する信託財産の給付を受けることとされている者に該当するものとする。
\item[\rensuji{4}]法第十三条第一項に規定する受益者(同条第二項の規定により同条第一項に規定する受益者とみなされる者を含む。以下この項において同じ。)が二以上ある場合における同条第一項の規定の適用については、同項の信託の信託財産に属する資産及び負債の全部をそれぞれの受益者がその有する権利の内容に応じて有するものとし、当該信託財産に帰せられる収益及び費用の全部がそれぞれの受益者にその有する権利の内容に応じて帰せられるものとする。
\item[\rensuji{5}]法第十三条第三項第二号に規定する退職年金に関する契約で政令で定めるものは、次に掲げる契約とする。
\begin{description}
\item[一]法人税法施行令第百五十六条の二第十号(用語の意義)に規定する厚生年金基金契約
\item[二]国家公務員共済組合法第二十一条第二項第二号(設立及び業務)に掲げる業務に係る国家公務員共済組合法施行令(昭和三十三年政令第二百七号)第九条の四第一号(厚生年金保険給付積立金等及び退職等年金給付積立金等の管理及び運用に関する契約)に掲げる契約
\item[三]地方公務員等共済組合法第三条の二第一項第三号(組合の業務)に規定する退職等年金給付組合積立金の積立ての業務に係る地方公務員等共済組合法施行令(昭和三十七年政令第三百五十二号)第十六条の三第一号(資金の運用に関する契約)(同令第二十条(準用規定)において準用する場合を含む。)に掲げる契約
\item[四]地方公務員等共済組合法第三十八条の二第二項第四号(地方公務員共済組合連合会)に規定する退職等年金給付調整積立金の管理及び運用に関する事務に係る業務に係る地方公務員等共済組合法施行令第二十一条の三(準用規定)において準用する同令第十六条の三第一号に掲げる契約
\item[五]日本私立学校振興・共済事業団法(平成九年法律第四十八号)第二十三条第一項第八号(業務)に掲げる業務に係る信託の契約
\end{description}
\end{description}
\section*{第四章 納税地}
\addcontentsline{toc}{section}{第四章 納税地}
\noindent\hspace{10pt}(納税地の判定に係る特殊関係者)
\begin{description}
\item[第五十三条]法第十五条第四号(納税地)に規定する政令で定める者は、次に掲げる者及びこれらの者であつた者とする。
\begin{description}
\item[一]納税義務者とまだ婚姻の届出をしないが事実上婚姻関係と同様の事情にある者
\item[二]納税義務者の使用人
\item[三]前二号に掲げる者及び納税義務者の親族以外の者で納税義務者から受ける金銭その他の資産によつて生計を維持しているもの
\end{description}
\end{description}
\noindent\hspace{10pt}(特殊な場合の納税地)
\begin{description}
\item[第五十四条]法第十五条第六号(納税地)に規定する政令で定める場所は、次の各号に掲げる場合の区分に応じ当該各号に掲げる場所とする。
\begin{description}
\item[一]法第十五条第一号から第五号までの規定により納税地を定められていた者がこれらの規定のいずれにも該当しないこととなつた場合(同条第二号の規定により納税地を定められていた者については、同号の居所が短期間の滞在地であつた場合を除く。)
\item[二]前号に掲げる場合を除き、その者が国に対し所得税に関する法律の規定に基づく申告、請求その他の行為をする場合
\item[三]前二号に掲げる場合以外の場合
\end{description}
\end{description}
\noindent\hspace{10pt}(源泉徴収に係る所得税の納税地)
\begin{description}
\item[第五十五条]法第十七条本文(源泉徴収に係る所得税の納税地)に規定する政令で定める場所は、同条に規定する給与等支払者が提出する法第二百二十九条(開業等の届出)若しくは第二百三十条(給与等の支払をする事務所の開設等の届出)に規定する届出書又は法人税法施行令第十八条第一項若しくは第二項(納税地等の異動の届出)に規定する書面(次項において「開業等届出書」と総称する。)に記載すべき当該給与等支払者の移転後の事務所等(法第十七条に規定する事務所等をいう。)の所在地とする。
\item[\rensuji{2}]法第十七条ただし書に規定する政令で定めるものは、次の各号に掲げるものとし、同条ただし書に規定する政令で定める場所は、それぞれその支払の日(支払があつたものとみなされる日を含む。以下この項において「支払日」という。)における当該各号に定める場所(当該支払日以後に当該各号に規定する者(第四号にあつては、同号の法人課税信託の受託者である同号イからハまでに掲げる者とする。以下この項において「利子等支払者」という。)が国内において当該各号に定める場所を移転した場合には、当該利子等支払者が提出する開業等届出書に記載すべき当該利子等支払者の移転後の当該各号に定める場所)とする。
\begin{description}
\item[一]日本国の国債の利子
\item[二]日本の地方公共団体の発行する地方債又は内国法人の発行する債券の利子
\item[三]内国法人の支払う法第二十四条第一項(配当所得)に規定する剰余金の配当、利益の配当、剰余金の分配、金銭の分配及び基金利息
\item[四]法第十七条に規定する受託法人の支払う法人課税信託の収益の分配
\begin{description}
\item[イ]個人
\item[ロ]内国法人
\item[ハ]外国法人
\end{description}
\item[五]投資信託(投資信託及び投資法人に関する法律第二条第一項(定義)に規定する委託者指図型投資信託に限る。)の収益の分配(前号に掲げるものを除く。)
\item[六]特定受益証券発行信託の収益の分配
\item[七]法第百六十一条第一項第四号から第七号まで及び第十号から第十六号まで(国内源泉所得)に掲げる国内源泉所得(次号に掲げるものを除く。)で国外において支払われるもの又は同項第八号ロに掲げる国内源泉所得
\item[八]法第百八十三条第二項(賞与に係る源泉徴収時期の特例)(法第二百十二条第四項(非居住者に対する準用)において準用する場合を含む。以下この号において同じ。)に規定する賞与
\end{description}
\end{description}
\noindent\hspace{10pt}(納税地の指定)
\begin{description}
\item[第五十六条]法第十八条第一項(納税地の指定)に規定する政令で定める場合は、同条の規定により指定されるべき納税地が法第十五条から第十七条まで(納税地)の規定による納税地(既に法第十八条の規定により納税地の指定がされている場合には、その指定をされている納税地)の所轄国税局長の管轄区域以外の地域にある場合とする。
\end{description}
\noindent\hspace{10pt}(納税地の異動の届出)
\begin{description}
\item[第五十七条]法第二十条(納税地の異動の届出)に規定する届出は、同条の納税地の異動があつた後遅滞なく、異動前の納税地及び異動後の納税地を記載した書面をもつてしなければならない。
\end{description}
\part*{第二編 居住者の納税義務}
\addcontentsline{toc}{part}{第二編 居住者の納税義務}
\section*{第一章 課税標準の計算}
\addcontentsline{toc}{section}{第一章 課税標準の計算}
\subsection*{第一節 各種所得の金額の計算}
\addcontentsline{toc}{subsection}{第一節 各種所得の金額の計算}
\subsubsection*{第一款 利子所得及び配当所得}
\addcontentsline{toc}{subsubsection}{第一款 利子所得及び配当所得}
\noindent\hspace{10pt}(投資信託等の収益の分配に係る収入金額)
\begin{description}
\item[第五十八条]投資信託又は特定受益証券発行信託(以下この項において「投資信託等」という。)について信託の終了(当該投資信託等の信託の併合に係るものである場合にあつては、当該投資信託等の受益者に当該信託の併合に係る新たな信託の受益権以外の資産(信託の併合に反対する当該受益者に対するその買取請求に基づく対価として交付される金銭その他の資産を除く。)の交付がされた信託の併合に係るものに限る。)又は信託契約の一部の解約により分配される収益に係る利子所得又は配当所得の収入金額は、当該信託の終了又は当該契約の一部の解約により当該投資信託等の受益権を有する者に対し支払われる金額のうち、当該信託の終了又は当該契約の一部の解約の時において当該投資信託等について信託されている金額で当該受益権に係るものを超える部分の金額とする。
\item[\rensuji{2}]特定受益証券発行信託について信託の分割(分割信託(信託の分割によりその信託財産の一部を受託者を同一とする他の信託又は新たな信託の信託財産として移転する信託をいう。)の受益者に承継信託(信託の分割により受託者を同一とする他の信託からその信託財産の一部の移転を受ける信託をいう。)の受益権以外の資産(信託の分割に反対する当該受益者に対する信託法(平成十八年法律第百八号)第百三条第六項(受益権取得請求)に規定する受益権取得請求に基づく対価として交付される金銭その他の資産を除く。)の交付がされたものに限る。)により分配される収益に係る配当所得の収入金額は、当該信託の分割により当該特定受益証券発行信託の受益権を有する者に対し支払われる金額のうち、当該信託の分割の時において当該特定受益証券発行信託について信託されている金額で当該受益権に係るものを超える部分の金額とする。
\end{description}
\noindent\hspace{10pt}(配当所得の金額の計算上控除する負債の利子)
\begin{description}
\item[第五十九条]法第二十四条第二項(配当所得の金額)に規定する政令で定めるところにより計算した金額は、その年中に支払う同項に規定する負債の利子の額を十二で除し、これにその年において当該負債により取得した元本を有していた期間の月数を乗じて計算した金額とする。
\item[\rensuji{2}]前項の月数は、暦に従つて計算し、一月に満たない端数を生じたときは、これを一月とする。
\end{description}
\begin{description}
\item[第六十条]削除
\end{description}
\noindent\hspace{10pt}(所有株式に対応する資本金等の額又は連結個別資本金等の額の計算方法等)
\begin{description}
\item[第六十一条]法第二十五条第一項第五号(配当等とみなす金額)に規定する政令で定める取得は、次に掲げる事由による取得とする。
\begin{description}
\item[一]金融商品取引法第二条第十六項(定義)に規定する金融商品取引所の開設する市場(同条第八項第三号ロに規定する外国金融商品市場を含む。)における購入
\item[二]店頭売買登録銘柄(株式(出資及び投資信託及び投資法人に関する法律第二条第十四項(定義)に規定する投資口を含む。以下この項において同じ。)で、金融商品取引法第二条第十三項に規定する認可金融商品取引業協会が、その定める規則に従い、その店頭売買につき、その売買価格を発表し、かつ、当該株式の発行法人に関する資料を公開するものとして登録したものをいう。)として登録された株式のその店頭売買による購入
\item[三]金融商品取引法第二条第八項に規定する金融商品取引業のうち同項第十号に掲げる行為を行う者が同号の有価証券の売買の媒介、取次ぎ又は代理をする場合におけるその売買(同号ニに掲げる方法により売買価格が決定されるものを除く。)
\item[四]事業の全部の譲受け
\item[五]合併又は分割若しくは現物出資(適格分割若しくは適格現物出資又は事業を移転し、かつ、当該事業に係る資産に当該分割若しくは現物出資に係る分割承継法人若しくは被現物出資法人の株式が含まれている場合の当該分割若しくは現物出資に限る。)による被合併法人又は分割法人若しくは現物出資法人からの移転
\item[六]適格分社型分割(法人税法第二条第十二号の十一(定義)に規定する分割承継親法人株式が交付されるものに限る。)による分割承継法人からの交付
\item[七]法第五十七条の四第一項(株式交換等に係る譲渡所得等の特例)に規定する株式交換(同項に規定する政令で定める関係がある法人の株式が交付されるものに限る。)による同項に規定する株式交換完全親法人からの交付
\item[八]合併に反対する当該合併に係る被合併法人の株主等の買取請求に基づく買取り
\item[九]会社法(平成十七年法律第八十六号)第百八十二条の四第一項(反対株主の株式買取請求)(資産の流動化に関する法律第三十八条(特定出資についての会社法の準用)又は第五十条第一項(優先出資についての会社法の準用)において準用する場合を含む。)、第百九十二条第一項(単元未満株式の買取りの請求)又は第二百三十四条第四項(一に満たない端数の処理)(会社法第二百三十五条第二項(一に満たない端数の処理)又は他の法律において準用する場合を含む。)の規定による買取り
\item[十]法第五十七条の四第三項第三号に規定する全部取得条項付種類株式を発行する旨の定めを設ける法人税法第十三条第一項(事業年度の意義)に規定する定款等の変更に反対する株主等の買取請求に基づく買取り(その買取請求の時において、当該全部取得条項付種類株式の同号に定める取得決議に係る取得対価の割当てに関する事項(当該株主等に交付する当該買取りをする法人の株式の数が一に満たない端数となるものに限る。)が当該株主等に明らかにされている場合(法第五十七条の四第三項に規定する場合に該当する場合に限る。)における当該買取りに限る。)
\item[十一]法第五十七条の四第三項第三号に規定する全部取得条項付種類株式に係る同号に定める取得決議(当該取得決議に係る取得の価格の決定の申立てをした者でその申立てをしないとしたならば当該取得の対価として交付されることとなる当該取得をする法人の株式の数が一に満たない端数となるものからの取得(同項に規定する場合に該当する場合における当該取得に限る。)に係る部分に限る。)
\item[十二]会社法第百六十七条第三項(効力の発生)若しくは第二百八十三条(一に満たない端数の処理)に規定する一株に満たない端数(これに準ずるものを含む。)又は投資信託及び投資法人に関する法律第八十八条の十九(一に満たない端数の処理)に規定する一口に満たない端数に相当する部分の対価としての金銭の交付
\end{description}
\item[\rensuji{2}]法第二十五条第一項に規定する株式又は出資に対応する部分の金額は、同項に規定する事由の次の各号に掲げる区分に応じ、当該各号に定める金額とする。
\begin{description}
\item[一]法第二十五条第一項第一号に掲げる合併
\item[二]法第二十五条第一項第二号に掲げる分割型分割
\begin{description}
\item[イ]当該分割型分割の日の属する事業年度の前事業年度(当該分割型分割の日以前六月以内に法人税法第七十二条第一項(仮決算をした場合の中間申告書の記載事項等)又は第八十一条の二十第一項(仮決算をした場合の連結中間申告書の記載事項等)に規定する期間についてこれらの規定に掲げる事項を記載した同法第二条第三十号に規定する中間申告書又は同条第三十一号の二に規定する連結中間申告書を提出し、かつ、その提出の日から当該分割型分割の日までの間に同条第三十一号に規定する確定申告書又は同条第三十二号に規定する連結確定申告書を提出していなかつた場合には、当該中間申告書又は連結中間申告書に係るこれらの規定に規定する期間)終了の時の資産の帳簿価額から負債(新株予約権に係る義務を含む。)の帳簿価額を減算した金額(当該終了の時から当該分割型分割の直前の時までの間に資本金等の額若しくは連結個別資本金等の額又は同条第十八号に規定する利益積立金額(第五号イにおいて「利益積立金額」という。)若しくは同条第十八号の三に規定する連結個別利益積立金額(法人税法施行令第九条第一項第一号若しくは第六号(利益積立金額)又は第九条の二第一項第一号若しくは第四号(連結利益積立金額)に掲げる金額を除く。)が増加し、又は減少した場合には、その増加した金額を加算し、又はその減少した金額を減算した金額)
\item[ロ]当該分割型分割の直前の移転資産(当該分割型分割により当該分割法人から分割承継法人に移転した資産をいう。)の帳簿価額から移転負債(当該分割型分割により当該分割法人から当該分割承継法人に移転した負債をいう。)の帳簿価額を控除した金額(当該金額がイに掲げる金額を超える場合(イに掲げる金額が零に満たない場合を除く。)には、イに掲げる金額)
\end{description}
\item[三]法第二十五条第一項第三号に掲げる株式分配
\begin{description}
\item[イ]当該株式分配を前号イの分割型分割とみなした場合における同号イに掲げる金額
\item[ロ]当該現物分配法人の当該株式分配の直前の法人税法第二条第十二号の十五の二に規定する完全子法人の株式の帳簿価額に相当する金額(当該金額が零以下である場合には零とし、当該金額がイに掲げる金額を超える場合(イに掲げる金額が零に満たない場合を除く。)にはイに掲げる金額とする。)
\end{description}
\item[四]法第二十五条第一項第四号に掲げる資本の払戻し又は解散による残余財産の分配(次号に掲げるものを除く。以下この号において「払戻し等」という。)
\begin{description}
\item[イ]当該払戻し等を第二号イの分割型分割とみなした場合における同号イに掲げる金額
\item[ロ]当該資本の払戻しにより減少した資本剰余金の額又は当該解散による残余財産の分配により交付した金銭の額及び金銭以外の資産の価額(法人税法第二条第十二号の十五に規定する適格現物分配に係る資産にあつては、その交付の直前の帳簿価額)の合計額(当該減少した資本剰余金の額又は当該合計額がイに掲げる金額を超える場合には、イに掲げる金額)
\end{description}
\item[五]法第二十四条第一項(配当所得)に規定する出資等減少分配(以下この号において「出資等減少分配」という。)
\begin{description}
\item[イ]当該投資法人の当該出資等減少分配の日の属する事業年度の前事業年度終了の時の当該投資法人の資産の帳簿価額から負債の帳簿価額を減算した金額(当該終了の時から当該出資等減少分配の直前の時までの間に資本金等の額又は利益積立金額(法人税法施行令第九条第一項第一号に掲げる金額を除く。)が増加し、又は減少した場合には、その増加した金額を加算し、又はその減少した金額を減算した金額)
\item[ロ]当該出資等減少分配による出資総額等の減少額として財務省令で定める金額(当該金額がイに掲げる金額を超える場合には、イに掲げる金額)
\end{description}
\item[六]法第二十五条第一項第五号から第七号までに掲げる事由(以下この号において「自己株式の取得等」という。)
\begin{description}
\item[イ]当該自己株式の取得等をした法人が一の種類の株式を発行していた法人(口数の定めがない出資を発行する法人を含む。)である場合
\item[ロ]当該自己株式の取得等をした法人が二以上の種類の株式を発行していた法人である場合
\end{description}
\end{description}
\item[\rensuji{3}]法第二十五条第一項第一号に掲げる合併又は同項第二号に掲げる分割型分割に際して当該合併又は分割型分割に係る被合併法人又は分割法人の株主等に対する株式に係る剰余金の配当、利益の配当又は剰余金の分配として交付がされた金銭その他の資産(法人税法第二条第十二号の九イに規定する分割対価資産を除く。)及び合併に反対する当該株主等に対するその買取請求に基づく対価として交付がされる金銭その他の資産は、同項の金銭その他の資産に含まれないものとする。
\item[\rensuji{4}]法第二十五条第二項に規定する政令で定めるものは、次に掲げる合併又は分割型分割(法第二十四条第一項に規定する分割型分割をいう。第二号及び次項において同じ。)とする。
\begin{description}
\item[一]法人税法施行令第四条の三第二項第一号(適格組織再編成における株式の保有関係等)に規定する無対価合併で同項第二号ロに掲げる関係があるもの
\item[二]法人税法施行令第四条の三第六項第一号イに規定する無対価分割に該当する分割型分割で同項第二号イ(2)に掲げる関係があるもの
\end{description}
\item[\rensuji{5}]法第二十五条第二項に規定する場合には、同項の被合併法人又は分割法人の株主等は、前項第一号に掲げる合併にあつては当該合併に係る被合併法人が当該合併により当該合併に係る合併法人に移転をした資産(営業権にあつては、法人税法施行令第百二十三条の十第三項(非適格合併等により移転を受ける資産等に係る調整勘定の損金算入等)に規定する独立取引営業権(以下この項において「独立取引営業権」という。)に限る。)の価額(法人税法第六十二条の八第一項(非適格合併等により移転を受ける資産等に係る調整勘定の損金算入等)に規定する資産調整勘定の金額を含む。)から当該被合併法人が当該合併により当該合併法人に移転をした負債の価額(法人税法第六十二条の八第二項及び第三項に規定する負債調整勘定の金額を含む。)を控除した金額を当該被合併法人の当該合併の日の前日の属する事業年度又は連結事業年度終了の時の発行済株式等の総数で除して計算した金額に当該被合併法人の株主等が当該合併の直前に有していた当該被合併法人の株式の数を乗じて計算した金額に相当する当該合併法人の株式の交付を受けたものと、前項第二号に掲げる分割型分割にあつては当該分割型分割に係る分割法人が当該分割型分割により当該分割型分割に係る分割承継法人に移転をした資産(営業権にあつては、独立取引営業権に限る。)の価額(法人税法第六十二条の八第一項に規定する資産調整勘定の金額を含む。)から当該分割法人が当該分割型分割により当該分割承継法人に移転をした負債の価額(法人税法第六十二条の八第二項及び第三項に規定する負債調整勘定の金額を含む。)を控除した金額を当該分割法人の当該分割型分割の直前の発行済株式等の総数で除して計算した金額に当該分割法人の株主等が当該分割型分割の直前に有していた当該分割法人の株式の数を乗じて計算した金額に相当する当該分割承継法人の株式の交付を受けたものと、それぞれみなす。
\item[\rensuji{6}]この条において、次の各号に掲げる用語の意義は、当該各号に定めるところによる。
\begin{description}
\item[一]適格分割
\item[二]適格現物出資
\item[三]分割承継法人
\item[四]被現物出資法人
\item[五]被合併法人
\item[六]分割法人
\item[七]現物出資法人
\item[八]適格分社型分割
\item[九]現物分配法人
\item[十]合併法人
\end{description}
\item[\rensuji{7}]第一項又は第四項に規定する合併には、法人課税信託に係る信託の併合を含むものとし、第一項に規定する分割には、法人課税信託に係る信託の分割を含むものとする。
\end{description}
\noindent\hspace{10pt}(企業組合等の分配金)
\begin{description}
\item[第六十二条]次に掲げる分配金の額は、法第二十四条第一項(配当所得)に規定する配当等の収入金額とする。
\begin{description}
\item[一]企業組合の組合員が中小企業等協同組合法第五十九条第三項(企業組合の剰余金の配当)の規定によりその企業組合の事業に従事した程度に応じて受ける分配金
\item[二]協業組合の組合員が中小企業団体の組織に関する法律(昭和三十二年法律第百八十五号)第五条の二十第二項(剰余金の配当)の定款の別段の定めに基づき出資口数に応じないで受ける分配金
\item[三]農業協同組合法第七十二条の十第一項第二号(農業の経営)の事業を行う農事組合法人、漁業生産組合又は生産森林組合でその事業に従事する組合員に対し給料、賃金、賞与その他これらの性質を有する給与を支給するものの組合員が、同法第七十二条の三十一第二項(剰余金の配当)、水産業協同組合法(昭和二十三年法律第二百四十二号)第八十五条第二項(剰余金の配当)又は森林組合法(昭和五十三年法律第三十六号)第九十九条第二項(剰余金の配当)の規定によりこれらの法人の事業に従事した程度に応じて受ける分配金
\item[四]農住組合の組合員が農住組合法(昭和五十五年法律第八十六号)第五十五条第二項(剰余金の配当)の規定により組合事業の利用分量に応じて受ける分配金
\end{description}
\item[\rensuji{2}]農業協同組合法第七十二条の十第一項第二号の事業を行う農事組合法人、漁業生産組合又は生産森林組合でその事業に従事する組合員に対し給料、賃金、賞与その他これらの性質を有する給与を支給しないものの組合員が、同法第七十二条の三十一第二項、水産業協同組合法第八十五条第二項又は森林組合法第九十九条第二項の規定によりこれらの法人の事業に従事した程度に応じて受ける分配金の額は、配当所得、給与所得及び退職所得以外の各種所得に係る収入金額とする。
\item[\rensuji{3}]生計を一にする親族のうちに同一の法人から前項の分配金を受ける者が二人以上ある場合には、これらの者のうち同項に規定する収入金額の最も大きい者以外の者の受ける当該収入金額に係る所得については、これを当該収入金額の最も大きい者の経営する事業から受ける当該所得とみなして、法第五十六条(事業から対価を受ける親族がある場合の必要経費の特例)の規定を適用する。
\item[\rensuji{4}]法人税法第二条第七号(定義)に規定する協同組合等から支払を受ける同法第六十条の二第一項第一号(協同組合等の事業分量配当等の損金算入)に掲げる金額で同項の規定により当該協同組合等の各事業年度の所得の金額の計算上損金の額に算入されるものは、配当所得以外の各種所得に係る収入金額とする。
\end{description}
\subsubsection*{第二款 事業所得}
\addcontentsline{toc}{subsubsection}{第二款 事業所得}
\noindent\hspace{10pt}(事業の範囲)
\begin{description}
\item[第六十三条]法第二十七条第一項(事業所得)に規定する政令で定める事業は、次に掲げる事業(不動産の貸付業又は船舶若しくは航空機の貸付業に該当するものを除く。)とする。
\begin{description}
\item[一]農業
\item[二]林業及び狩猟業
\item[三]漁業及び水産養殖業
\item[四]鉱業(土石採取業を含む。)
\item[五]建設業
\item[六]製造業
\item[七]卸売業及び小売業(飲食店業及び料理店業を含む。)
\item[八]金融業及び保険業
\item[九]不動産業
\item[十]運輸通信業(倉庫業を含む。)
\item[十一]医療保健業、著述業その他のサービス業
\item[十二]前各号に掲げるもののほか、対価を得て継続的に行なう事業
\end{description}
\end{description}
\subsubsection*{第三款 給与所得}
\addcontentsline{toc}{subsubsection}{第三款 給与所得}
\noindent\hspace{10pt}(確定給付企業年金規約等に基づく掛金等の取扱い)
\begin{description}
\item[第六十四条]事業を営む個人又は法人が支出した次の各号に掲げる掛金、保険料、事業主掛金又は信託金等は、当該各号に規定する被共済者、加入者、受益者等、企業型年金加入者、個人型年金加入者又は信託の受益者等に対する給与所得に係る収入金額に含まれないものとする。
\begin{description}
\item[一]独立行政法人勤労者退職金共済機構又は第七十四条第五項(特定退職金共済団体の承認)に規定する特定退職金共済団体が行う退職金共済に関する制度に基づいてその被共済者のために支出した掛金(第七十六条第一項第二号ロからヘまで(退職金共済制度等に基づく一時金で退職手当等とみなさないもの)に掲げる掛金を除くものとし、中小企業退職金共済法(昭和三十四年法律第百六十号)第五十三条(従前の積立事業についての取扱い)の規定により独立行政法人勤労者退職金共済機構に納付した金額を含む。)
\item[二]確定給付企業年金法(平成十三年法律第五十号)第三条第一項(確定給付企業年金の実施)に規定する確定給付企業年金に係る規約に基づいて同法第二十五条第一項(加入者)に規定する加入者のために支出した同法第五十五条第一項(掛金)の掛金(同法第六十三条(積立不足に伴う掛金の拠出)、第七十八条第三項(実施事業所の増減)、第七十八条の二第三号(確定給付企業年金を実施している事業主が二以上である場合等の実施事業所の減少の特例)及び第八十七条(終了時の掛金の一括拠出)の掛金並びにこれに類する掛金で財務省令で定めるものを含む。)のうち当該加入者が負担した金額以外の部分
\item[三]法人税法附則第二十条第三項(退職年金等積立金に対する法人税の特例)に規定する適格退職年金契約に基づいて法人税法施行令附則第十六条第一項第二号(適格退職年金契約の要件等)に規定する受益者等のために支出した掛金又は保険料(第七十六条第二項第二号に規定する受益者等とされた者に係る掛金及び保険料を除く。)のうち当該受益者等が負担した金額以外の部分
\item[四]確定拠出年金法(平成十三年法律第八十八号)第四条第三項(承認の基準等)に規定する企業型年金規約に基づいて同法第二条第八項(定義)に規定する企業型年金加入者のために支出した同法第三条第三項第七号(規約の承認)に規定する事業主掛金(同法第五十四条第一項(他の制度の資産の移換)の規定により移換した確定拠出年金法施行令(平成十三年政令第二百四十八号)第二十二条第一項第五号(他の制度の資産の移換の基準)に掲げる資産を含む。)
\item[五]確定拠出年金法第五十六条第三項(承認の基準等)に規定する個人型年金規約に基づいて同法第六十八条の二第一項(中小事業主掛金)の個人型年金加入者のために支出した同項の掛金
\item[六]勤労者財産形成促進法第六条の二第一項(勤労者財産形成給付金契約等)に規定する勤労者財産形成給付金契約に基づいて同項第二号に規定する信託の受益者等のために支出した同項第一号に規定する信託金等
\end{description}
\item[\rensuji{2}]事業を営む個人が、前項各号に掲げる掛金、保険料、事業主掛金又は信託金等を支出した場合には、その支出した金額(確定給付企業年金法第五十六条第二項(掛金の納付)又は法人税法施行令附則第十六条第二項の規定に基づき、前項第二号に掲げる掛金又は同項第三号に掲げる掛金若しくは保険料の支出を金銭に代えて同法第五十六条第二項に規定する株式又は同令附則第十六条第二項に規定する株式をもつて行つた場合には、その時におけるこれらの株式の価額)は、その支出した日の属する年分の当該事業に係る不動産所得の金額、事業所得の金額又は山林所得の金額の計算上、必要経費に算入する。
\end{description}
\noindent\hspace{10pt}(不適格退職金共済契約等に基づく掛金の取扱い)
\begin{description}
\item[第六十五条]事業を営む個人又は法人が支出した次の各号に掲げる掛金(当該個人のための掛金及び当該各号に規定する者が負担した金額に相当する部分の掛金を除く。)で、当該個人のその事業に係る不動産所得の金額、事業所得の金額若しくは山林所得の金額又は当該法人の各事業年度の所得の金額の計算上必要経費又は損金の額に算入されるものは、当該各号に規定する者に対する給与所得に係る収入金額に含まれるものとする。
\begin{description}
\item[一]前条第一項第一号に規定する制度に該当しない第七十三条第一項第一号(特定退職金共済団体の要件)に規定する退職金共済契約(以下この号において「退職金共済契約」という。)又はこれに類する契約に基づいて被共済者又はこれに類する者のために支出した掛金(第七十五条第一項(特定退職金共済団体の承認の取消し等)の規定による承認の取消しを受けた団体に対しその取消しに係る退職金共済契約に基づき支出し、又は同条第三項の規定により承認が失効をした団体に対しその失効に係る退職金共済契約に基づき支出した掛金については、その取消しの時又はその失効後に支出した掛金)及び第七十六条第一項第二号ロからヘまで(退職金共済制度等に基づく一時金で退職手当等とみなさないもの)に掲げる掛金
\item[二]前条第一項第三号に規定する適格退職年金契約に該当しない第百八十三条第三項第三号(生命保険契約等に基づく年金に係る雑所得の金額の計算上控除する保険料等)に掲げる契約に基づいてその受益者、保険金受取人又は共済金受取人とされた使用人(法人の役員を含む。)のために支出した掛金又は保険料(法人税法施行令附則第十八条第一項(適格退職年金契約の承認の取消し)の規定による承認の取消しを受けた第七十六条第二項第一号に規定する信託会社等に対しその取消しに係る同号に規定する契約に基づき支出した掛金又は保険料については、その取消しの時以後に支出した掛金又は保険料)及び第七十六条第二項第二号に規定する受益者等とされた者に係る掛金又は保険料
\end{description}
\end{description}
\begin{description}
\item[第六十六条]削除
\end{description}
\begin{description}
\item[第六十七条]削除
\end{description}
\begin{description}
\item[第六十八条]削除
\end{description}
\subsubsection*{第四款 退職所得}
\addcontentsline{toc}{subsubsection}{第四款 退職所得}
\noindent\hspace{10pt}(退職所得控除額に係る勤続年数の計算)
\begin{description}
\item[第六十九条]法第三十条第三項第一号(退職所得)に規定する政令で定める勤続年数は、次に定めるところにより計算するものとする。
\begin{description}
\item[一]法第三十条第一項に規定する退職手当等(法第三十一条(退職手当等とみなす一時金)の規定により退職手当等とみなされるものを除く。以下この条及び次条において「退職手当等」という。)については、退職手当等の支払を受ける居住者(以下この項において「退職所得者」という。)が退職手当等の支払者の下においてその退職手当等の支払の基因となつた退職の日まで引き続き勤務した期間(以下この項において「勤続期間」という。)により勤続年数を計算する。
\begin{description}
\item[イ]退職所得者が退職手当等の支払者の下において就職の日から退職の日までに一時勤務しなかつた期間がある場合には、その一時勤務しなかつた期間前にその支払者の下において引き続き勤務した期間を勤続期間に加算した期間により勤続年数を計算する。
\item[ロ]退職所得者が退職手当等の支払者の下において勤務しなかつた期間に他の者の下において勤務したことがある場合において、その支払者がその退職手当等の支払金額の計算の基礎とする期間のうちに当該他の者の下において勤務した期間を含めて計算するときは、当該他の者の下において勤務した期間を勤続期間に加算した期間により勤続年数を計算する。
\item[ハ]退職所得者が退職手当等の支払者から前に退職手当等の支払を受けたことがある場合には、前に支払を受けた退職手当等の支払金額の計算の基礎とされた期間の末日以前の期間は、勤続期間又はイ若しくはロの規定により加算すべき期間に含まれないものとして、勤続期間の計算又はイ若しくはロの計算を行う。
\end{description}
\item[二]法第三十一条の規定により退職手当等とみなされるもの(以下この項において「退職一時金等」という。)については、組合員等であつた期間(退職一時金等の支払金額の計算の基礎となつた期間(当該退職一時金等の支払金額のうちに次に掲げる金額が含まれている場合には、当該金額の計算の基礎となつた期間を含む。)をいい、当該期間の計算が時の経過に従つて計算した期間によらず、これに一定の期間を加算して計算した期間によつている場合には、その加算をしなかつたものとして計算した期間をいう。ただし、当該退職一時金等が第七十二条第三項第六号(退職手当等とみなす一時金)に掲げる一時金に該当する場合には、当該支払金額の計算の基礎となつた期間は、当該支払金額の計算の基礎となつた確定拠出年金法第三十三条第二項第一号(支給要件)に規定する企業型年金加入者期間(同法第四条第三項(承認の基準等)に規定する企業型年金規約に基づいて納付した同法第三条第三項第七号(規約の承認)に規定する事業主掛金に係る当該企業型年金加入者期間に限るものとし、同法第五十四条第二項(他の制度の資産の移換)又は第五十四条の二第二項(脱退一時金相当額等の移換)の規定により同法第三十三条第一項の通算加入者等期間に算入された期間及び当該企業型年金加入者期間に準ずるものとして財務省令で定める期間を含む。以下この号において「企業型年金加入者期間等」という。)と、当該計算の基礎となつた同条第二項第三号に規定する個人型年金加入者期間(同法第五十六条第三項(承認の基準等)に規定する個人型年金規約に基づいて納付した同法第五十五条第二項第四号(規約の承認)に規定する個人型年金加入者掛金に係る当該個人型年金加入者期間に限るものとし、同法第七十四条の二第二項(脱退一時金相当額等の移換)の規定により同法第七十三条(企業型年金に係る規定の準用)において準用する同法第三十三条第一項の通算加入者等期間に算入された期間及び当該個人型年金加入者期間に準ずるものとして財務省令で定める期間を含む。)のうち企業型年金加入者期間等と重複していない期間とを合算した期間をいう。次号において同じ。)により勤続年数の計算を行う。
\begin{description}
\item[イ]中小企業退職金共済法第三十条第一項(退職金相当額の受入れ等)の受入れに係る金額、同法第三十一条の二第六項(退職金共済事業を廃止した団体からの受入金額の受入れ等)において準用する同条第一項の受入れに係る金額又は同法第三十一条の三第六項(資産管理運用機関等からの移換額の移換等)において準用する同条第一項の移換に係る金額
\item[ロ]公的年金制度の健全性及び信頼性の確保のための厚生年金保険法等の一部を改正する法律(平成二十五年法律第六十三号。以下「平成二十五年厚生年金等改正法」という。)附則第三十六条第七項(解散存続厚生年金基金の残余財産の独立行政法人勤労者退職金共済機構への交付)において準用する同条第一項の規定による申出に従い交付された額
\item[ハ]第七十三条第一項第八号ロ(特定退職金共済団体の要件)に規定する退職金に相当する額、同号ニに規定する退職給付金に相当する額又は同号ホに規定する引継退職給付金に相当する額
\end{description}
\item[三]その年に二以上の退職手当等又は退職一時金等の支給を受ける場合には、これらの退職手当等又は退職一時金等のそれぞれについて前二号の規定により計算した期間のうち最も長い期間により勤続年数を計算する。
\end{description}
\item[\rensuji{2}]前項各号の規定により計算した期間に一年未満の端数を生じたときは、これを一年として同項の勤続年数を計算する。
\item[\rensuji{3}]退職手当等の支払者には、その者が相続人である場合にはその被相続人を含むものとし、その者が合併後存続する法人又は合併により設立された法人である場合には合併により消滅した法人を含むものとし、その者が法人の分割により資産及び負債の移転を受けた法人である場合にはその分割により当該資産及び負債の移転を行つた法人を含むものとする。
\end{description}
\noindent\hspace{10pt}(特定役員退職手当等に係る役員等勤続年数の計算)
\begin{description}
\item[第六十九条の二]法第三十条第四項(退職所得)に規定する政令で定める勤続年数は、退職手当等に係る調整後勤続期間(前条第一項第一号の規定により計算した期間をいう。第七十一条の二第五項(特定役員退職手当等と一般退職手当等がある場合の退職所得の金額の計算)において同じ。)のうち、その退職手当等の支払を受ける居住者が法第三十条第四項に規定する役員等として勤務した期間(第七十一条の二第五項において「役員等勤続期間」という。)により計算するものとする。
\item[\rensuji{2}]前条第二項及び第三項の規定は、前項の勤続年数を計算する場合について準用する。
\end{description}
\noindent\hspace{10pt}(退職所得控除額の計算の特例)
\begin{description}
\item[第七十条]法第三十条第五項第一号(退職所得)に規定する政令で定める場合は、次の各号に掲げる場合とし、同項第一号に規定する政令で定めるところにより計算した金額は、当該各号に定める金額とする。
\begin{description}
\item[一]第六十九条第一項第一号ロ(退職所得控除額に係る勤続年数の計算)に規定する場合に該当し、かつ、同号ロに規定する他の者から前に退職手当等(法第三十条第一項に規定する退職手当等をいう。以下この条から第七十一条の二(特定役員退職手当等と一般退職手当等がある場合の退職所得の金額の計算)までにおいて同じ。)の支払を受けている場合又は同号ハただし書に規定する場合に該当する場合
\item[二]その年の前年以前四年内(その年に第七十二条第三項第六号(退職手当等とみなす一時金)に掲げる一時金の支払を受ける場合には、十四年内。以下この号において同じ。)に退職手当等(前号に規定する前に支払を受けた退職手当等を除く。)の支払を受け、かつ、その年に退職手当等の支払を受けた場合において、その年に支払を受けた退職手当等につき第六十九条第一項各号の規定により計算した期間の基礎となつた勤続期間等(同項第三号に規定する勤続期間等をいう。以下この条において同じ。)の一部がその年の前年以前四年内に支払を受けた退職手当等(次項において「前の退職手当等」という。)に係る勤続期間等(次項において「前の勤続期間等」という。)と重複している場合
\end{description}
\item[\rensuji{2}]前項第二号の場合において、前の退職手当等の収入金額が前の退職手当等について同号の規定を適用しないで計算した法第三十条第三項の規定による退職所得控除額に満たないときは、前の退職手当等の支払金額の計算の基礎となつた勤続期間等のうち、前の退職手当等に係る就職の日又は第六十九条第一項第二号に規定する組合員等であつた期間の初日から次の各号に掲げる場合の区分に応じ当該各号に定める数(一に満たない端数を生じたときは、これを切り捨てた数)に相当する年数を経過した日の前日までの期間を前の勤続期間等とみなして、前項第二号に定める金額を計算する。
\begin{description}
\item[一]前の退職手当等の収入金額が八百万円以下である場合
\item[二]前の退職手当等の収入金額が八百万円を超える場合
\end{description}
\item[\rensuji{3}]第一項第一号の期間及び同項第二号の重複している部分の期間に一年未満の端数があるときは、その端数を切り捨てる。
\end{description}
\noindent\hspace{10pt}(退職所得の割増控除が認められる障害による退職の要件)
\begin{description}
\item[第七十一条]法第三十条第五項第三号(退職所得)に規定する政令で定める場合は、退職手当等の支払を受ける居住者が在職中に障害者に該当することとなつたことにより、その該当することとなつた日以後全く又はほとんど勤務に服さないで退職した場合とする。
\end{description}
\noindent\hspace{10pt}(特定役員退職手当等と一般退職手当等がある場合の退職所得の金額の計算)
\begin{description}
\item[第七十一条の二]その年中に特定役員退職手当等(法第三十条第四項(退職所得)に規定する特定役員退職手当等をいう。以下この条において同じ。)と一般退職手当等(特定役員退職手当等以外の退職手当等をいう。以下この条において同じ。)がある場合の退職所得の金額は、次に掲げる金額の合計額(その年中の一般退職手当等の収入金額が第二号に規定する一般退職所得控除額に満たない場合には、その満たない部分の金額を第一号に掲げる金額から控除した残額)とする。
\begin{description}
\item[一]その年中の特定役員退職手当等の収入金額から特定役員退職所得控除額(次に掲げる金額の合計額をいう。次号において同じ。)を控除した残額
\begin{description}
\item[イ]四十万円に特定役員等勤続年数から重複勤続年数を控除した年数を乗じて計算した金額
\item[ロ]二十万円に重複勤続年数を乗じて計算した金額
\end{description}
\item[二]その年中の一般退職手当等の収入金額から一般退職所得控除額(法第三十条第二項に規定する退職所得控除額から特定役員退職所得控除額(前号の収入金額が特定役員退職所得控除額に満たない場合には、当該収入金額)を控除した残額をいう。)を控除した残額の二分の一に相当する金額
\end{description}
\item[\rensuji{2}]前項に規定する特定役員等勤続年数とは、特定役員等勤続期間(特定役員退職手当等につき第六十九条第一項第一号及び第三号(退職所得控除額に係る勤続年数の計算)の規定により計算した期間をいう。以下この項及び第四項において同じ。)により計算した年数をいい、前項に規定する重複勤続年数とは、特定役員等勤続期間と一般勤続期間(一般退職手当等につき同条第一項各号の規定により計算した期間をいう。)とが重複している期間により計算した年数をいう。
\item[\rensuji{3}]第六十九条第二項及び第三項の規定は、前項に規定する特定役員等勤続年数又は重複勤続年数を計算する場合について準用する。
\item[\rensuji{4}]法第三十条第五項(第一号に係る部分に限る。)の規定の適用があり、かつ、次の各号に掲げる場合に該当するときの第一項第一号に規定する特定役員退職所得控除額は、同号の合計額から当該各号に掲げる場合の区分に応じ当該各号に定める金額を控除した金額とする。
\begin{description}
\item[一]第七十条第一項第一号(退職所得控除額の計算の特例)に規定する前に支払を受けた退職手当等の全部又は一部が特定役員退職手当等に該当する場合
\item[二]特定役員等勤続期間の全部又は一部が第七十条第一項第二号に規定する前の勤続期間等と重複している場合
\end{description}
\item[\rensuji{5}]調整後勤続期間のうちに五年以下の役員等勤続期間と当該役員等勤続期間以外の期間がある退職手当等の支払を受ける場合には、当該退職手当等は、次に掲げる退職手当等から成るものとする。
\begin{description}
\item[一]退職手当等の金額から次号に掲げる金額を控除した残額に相当する特定役員退職手当等
\item[二]役員等勤続期間以外の期間を基礎として、他の使用人に対する退職給与の支給の水準等を勘案して相当と認められる金額に相当する一般退職手当等
\end{description}
\item[\rensuji{6}]前項の規定の適用がある場合には、同項の退職手当等の支払を受ける場合は、その年中に特定役員退職手当等と一般退職手当等がある場合とみなして、第一項の規定を適用する。
\end{description}
\noindent\hspace{10pt}(退職手当等とみなす一時金)
\begin{description}
\item[第七十二条]法第三十一条第一号(退職手当等とみなす一時金)に規定する政令で定める一時金(これに類する給付を含む。)は、次に掲げる一時金とする。
\begin{description}
\item[一]国民年金法等の一部を改正する法律(昭和六十年法律第三十四号)第五条(船員保険法の一部改正)の規定による改正前の船員保険法の規定に基づく一時金
\item[二]地方公務員等共済組合法の一部を改正する法律(平成二十三年法律第五十六号)附則の規定に基づく一時金
\item[三]厚生年金保険制度及び農林漁業団体職員共済組合制度の統合を図るための農林漁業団体職員共済組合法等を廃止する等の法律(平成十三年法律第百一号)附則第三十条(特例一時金の支給)の規定に基づく一時金(同条第一項第一号に掲げる者に対して支給するものに限る。)
\end{description}
\item[\rensuji{2}]法第三十一条第二号に規定する政令で定める一時金(これに類する給付を含む。)は、平成二十五年厚生年金等改正法第一条(厚生年金保険法の一部改正)の規定による改正前の厚生年金保険法(以下「旧厚生年金保険法」という。)第九章(厚生年金基金及び企業年金連合会)の規定に基づく一時金で平成二十五年厚生年金等改正法附則第三条第十二号(定義)に規定する厚生年金基金の加入員(次項第五号において「加入員」という。)の退職に基因して支払われるものとする。
\item[\rensuji{3}]法第三十一条第三号に規定する政令で定める一時金(これに類する給付を含む。)は、次に掲げる一時金とする。
\begin{description}
\item[一]特定退職金共済団体が行う退職金共済に関する制度に基づいて支給される一時金で、当該制度に係る被共済者の退職により支払われるもの
\item[二]独立行政法人勤労者退職金共済機構が中小企業退職金共済法第十条第一項(退職金)、第三十条第二項(退職金相当額の受入れ等)又は第四十三条第一項(退職金)の規定により支給するこれらの規定に規定する退職金
\item[三]独立行政法人中小企業基盤整備機構が支給する次に掲げる一時金
\begin{description}
\item[イ]法第七十五条第二項第一号(小規模企業共済等掛金控除)に規定する契約(以下この号において「小規模企業共済契約」という。)に基づいて支給される小規模企業共済法(昭和四十年法律第百二号)第九条第一項(共済金)に規定する共済金
\item[ロ]小規模企業共済法第二条第三項(定義)に規定する共済契約者で年齢六十五歳以上であるものが同法第七条第三項(契約の解除)の規定により小規模企業共済契約を解除したことにより支給される同法第十二条第一項(解約手当金)に規定する解約手当金
\item[ハ]小規模企業共済法第七条第四項の規定により小規模企業共済契約が解除されたものとみなされたことにより支給される同法第十二条第一項に規定する解約手当金
\end{description}
\item[四]法人税法附則第二十条第三項(退職年金等積立金に対する法人税の特例)に規定する適格退職年金契約に基づいて支給を受ける一時金で、その一時金が支給される基因となつた勤務をした者の退職により支払われるもの(当該契約に基づいて払い込まれた掛金又は保険料のうちに当該勤務をした者の負担した金額がある場合には、その一時金の額からその負担した金額を控除した金額に相当する部分に限る。)
\item[五]次に掲げる規定に基づいて支給を受ける一時金で加入員又は確定給付企業年金法第二十五条第一項(加入者)に規定する加入者の退職により支払われるもの(同法第三条第一項(確定給付企業年金の実施)に規定する確定給付企業年金に係る規約に基づいて拠出された掛金のうちに当該加入者の負担した金額がある場合には、その一時金の額からその負担した金額を控除した金額に相当する部分に限る。)
\begin{description}
\item[イ]平成二十五年厚生年金等改正法附則第四十二条第三項(基金中途脱退者に係る措置)、第四十三条第三項(解散基金加入員等に係る措置)、第四十六条第三項(確定給付企業年金中途脱退者に係る措置)、第四十七条第三項(終了制度加入者等に係る措置)又は第七十五条第二項(解散存続連合会の残余財産の連合会への交付)の規定
\item[ロ]平成二十五年厚生年金等改正法附則第六十三条第一項(確定給付企業年金中途脱退者等に係る措置に関する経過措置)の規定によりなおその効力を有するものとされる平成二十五年厚生年金等改正法第二条(確定給付企業年金法の一部改正)の規定による改正前の確定給付企業年金法第九十一条の二第三項(中途脱退者に係る措置)の規定
\item[ハ]平成二十五年厚生年金等改正法附則第六十三条第二項の規定によりなおその効力を有するものとされる平成二十五年厚生年金等改正法第二条の規定による改正前の確定給付企業年金法第九十一条の三第三項(終了制度加入者等に係る措置)の規定
\end{description}
\item[六]確定拠出年金法第四条第三項(承認の基準等)に規定する企業型年金規約又は同法第五十六条第三項(承認の基準等)に規定する個人型年金規約に基づいて同法第二十八条第一号(給付の種類)(同法第七十三条(企業型年金に係る規定の準用)において準用する場合を含む。)に掲げる老齢給付金として支給される一時金
\item[七]独立行政法人福祉医療機構が社会福祉施設職員等退職手当共済法(昭和三十六年法律第百五十五号)第七条(退職手当金の支給)の規定により支給する同条に規定する退職手当金
\item[八]外国の法令に基づく保険又は共済に関する制度で法第三十一条第一号及び第二号に規定する法律の規定による社会保険又は共済に関する制度に類するものに基づいて支給される一時金で、当該制度に係る被保険者又は被共済者の退職により支払われるもの
\end{description}
\end{description}
\noindent\hspace{10pt}(特定退職金共済団体の要件)
\begin{description}
\item[第七十三条]前条第三項第一号に規定する特定退職金共済団体とは、退職金共済事業を行う市町村(特別区を含む。)、商工会議所、商工会、商工会連合会、都道府県中小企業団体中央会、退職金共済事業を主たる目的とする一般社団法人又は一般財団法人その他財務大臣の指定するこれらに準ずる法人で、その行う退職金共済事業につき次に掲げる要件を備えているものとして税務署長の承認を受けたものをいう。
\begin{description}
\item[一]多数の事業主を対象として退職金共済契約(事業主が退職金共済事業を行う団体に掛金を納付し、その団体がその事業主の雇用する使用人の退職について退職給付金を支給すること(第八号イに規定する退職金に相当する額若しくは同号ハに規定する退職給付金に相当する額又は第九号に規定する引渡金額の引渡しを含む。)を約する契約をいう。以下この款において同じ。)を締結することを目的とし、かつ、加入事業主(退職金共済契約を締結した事業主をいう。以下この款において同じ。)のみがその掛金(第七号に規定する過去勤務等通算期間に対応する掛金を含む。第四号、第五号及び第十号において同じ。)を負担すること。
\item[二]被共済者(退職金共済契約に基づいて退職給付金の支給を受けるべき者をいう。以下この款において同じ。)のうちに他の特定退職金共済団体の被共済者を含まないこと。
\item[三]被共済者のうちに加入事業主である個人若しくはこれと生計を一にする親族又は加入事業主である法人の役員(法人税法第三十四条第六項(役員給与の損金不算入)に規定する使用人としての職務を有する役員を除く。)を含まないこと。
\item[四]掛金として払い込まれた金額(中小企業退職金共済法第三十一条第一項(退職金相当額の引渡し等)の規定によりその引渡しを受けた金額及び第八号ハの規定によりその引渡しを受けた金額並びにこれらの運用による利益を含む。次号において同じ。)は、加入事業主に返還しないこと。
\item[五]掛金として払い込まれた金額から退職金共済事業を行う団体の事務に要する経費として通常必要な金額を控除した残額(ヘにおいて「資産総額」という。)は、次に掲げる資産として運用し、かつ、これらの資産を担保に供し又は貸し付けないこと。
\begin{description}
\item[イ]公社債(信託会社(金融機関の信託業務の兼営等に関する法律により同法第一条第一項(兼営の認可)に規定する信託業務を営む銀行を含む。)に信託した公社債を含む。)
\item[ロ]預貯金(定期積金その他これに準ずるものを含む。)
\item[ハ]合同運用信託
\item[ニ]証券投資信託の受益権
\item[ホ]被共済者を被保険者とする生命保険の保険料その他これに類する生命共済の共済掛金(財務省令で定めるものに限る。)
\item[ヘ]加入事業主に対する貸付金で次に掲げる要件を満たすもの
\end{description}
\item[六]掛金の月額は、被共済者一人につき三万円以下であること。
\item[七]被共済者につき過去勤務期間(その者(財務省令で定める者を除く。)が被共済者となつた日の前日まで加入事業主の下で引き続き勤務した期間をいう。イにおいて同じ。)又は合併等前勤務期間(その者が、法人の合併又は事業の譲渡(それぞれ財務省令で定める合併又は事業の譲渡に限る。以下この号において同じ。)に伴い被共済者となつた者として財務省令で定める者(以下この号において「合併等被共済者」という。)である場合において、当該合併又は事業の譲渡の日の前日まで当該合併により消滅した法人若しくは当該合併後存続する法人又は当該事業の譲渡をした法人(当該合併又は事業の譲渡以外の合併又は事業の譲渡によりこれらの法人に事業が承継され、又は譲渡された法人を含む。)である事業主の下で引き続き勤務した期間をいう。イにおいて同じ。)がある場合において、これらの期間を退職給付金の額の計算の基礎に含めるときは、当該退職給付金の額の計算の基礎に含める期間(以下この号において「過去勤務等通算期間」という。)並びに当該過去勤務等通算期間に対応する掛金の額及びその払込みは、次の要件を満たすものであること。
\begin{description}
\item[イ]過去勤務等通算期間は、次に掲げる場合の区分に応じ、それぞれ次に定めるところによるものであること。
\item[ロ]過去勤務等通算期間に対応する掛金の額は、当該過去勤務等通算期間の月数を前号の掛金の月額(ハ(1)及び(3)に掲げる金額に係るものを除き、当該月額が三万円を超えるときは、三万円とする。)に乗じて得た金額と当該過去勤務等通算期間に係る運用収益として財務省令で定める金額との合計額以下とすること。
\item[ハ]過去勤務等通算期間に対応する掛金の額(次に掲げる金額があるときは、それぞれこれらの金額を控除した額)は、当該掛金の額を退職金共済契約を締結した日又は当該合併等被共済者となつた日として財務省令で定める日(以下この号において「基準日」という。)の翌日から同日以後五年を経過する日までの期間の月数(過去勤務等通算期間が五年未満であるときは当該過去勤務等通算期間の月数とし、被共済者が当該五年を経過する日前に退職をすることとされているときは当該翌日から同日以後当該退職をすることとされている日までの期間の月数とする。)で均分して、当該基準日の属する月以後毎月払い込まれること。
\end{description}
\item[八]被共済者が退職をした場合において、当該被共済者(当該退職につき退職金共済契約に基づき退職給付金の支給を受けることができる者に限る。)が次に掲げる場合に該当するときは、それぞれ次に定めるところによること。
\begin{description}
\item[イ]当該被共済者が、中小企業退職金共済法第三十条第一項(退職金相当額の受入れ等)の規定により、同項の申出をした場合
\item[ロ]当該被共済者が、中小企業退職金共済法第三十一条第一項(退職金相当額の引渡し等)の規定により独立行政法人勤労者退職金共済機構から同項に規定する退職金に相当する額の引渡しを受けて被共済者となつた者である場合
\item[ハ]他の特定退職金共済団体との間で、その退職につき退職金共済契約に基づき退職給付金の支給を受けることができる被共済者(当該退職をした者に限る。)が申し出たときは当該被共済者に係る当該退職給付金に相当する額を当該他の特定退職金共済団体に引き渡すことその他財務省令で定める事項を約する契約を締結している場合において、当該被共済者が当該退職後財務省令で定める期間内に、当該退職給付金を請求しないで当該他の特定退職金共済団体の被共済者となり、かつ、財務省令で定めるところにより申出をした場合
\item[ニ]当該被共済者が、ハに定めるところにより当該被共済者に係る特定退職金共済団体以外の特定退職金共済団体からハに規定する退職給付金に相当する額の引渡しを受けて被共済者となつた者である場合
\item[ホ]当該被共済者が、当該退職後財務省令で定める期間内に、当該退職給付金(以下この号において「引継退職給付金」という。)を請求しないで他の加入事業主(当該被共済者に係る特定退職金共済団体と退職金共済契約を締結した事業主に限る。)に係る被共済者となり、かつ、財務省令で定めるところにより申出をした場合
\end{description}
\item[九]退職金共済事業を廃止した場合において、中小企業退職金共済法第三十一条の二第一項(退職金共済事業を廃止した団体からの受入金額の受入れ等)(同条第六項において準用する場合を含む。以下この号において同じ。)に規定する事業主が、同条第一項の規定による申出をしたときは、同項に規定する廃止団体と独立行政法人勤労者退職金共済機構との間の同項の引渡しに係る契約で定めるところによつて当該事業主に係る被共済者であつた者に係る引渡金額(同項に規定する掛金の総額及び掛金に相当するものとして同項に規定する政令で定める金額並びにこれらの運用による利益の額の範囲内の金額をいう。)を独立行政法人勤労者退職金共済機構に引き渡すこと。
\item[十]掛金の額又は退職給付金の額について、加入事業主又は被共済者のうち特定の者につき不当に差別的な取扱いをしないこと。
\item[十一]退職金共済事業に関する経理は、他の経理と区分して行うこと。
\end{description}
\item[\rensuji{2}]前項に規定する一般社団法人又は一般財団法人は、一般社団法人及び一般財団法人に関する法律及び公益社団法人及び公益財団法人の認定等に関する法律の施行に伴う関係法律の整備等に関する法律(平成十八年法律第五十号)第四十条第一項(社団法人及び財団法人の存続)の規定により一般社団法人又は一般財団法人として存続するもののうち、同法第百六条第一項(移行の登記)(同法第百二十一条第一項(認定に関する規定の準用)において読み替えて準用する場合を含む。)の登記をしていないもの(同法第百三十一条第一項(認可の取消し)の規定により同法第四十五条(通常の一般社団法人又は一般財団法人への移行)の認可を取り消されたものを除く。)以外のものにあつては、次に掲げる要件を満たすものに限るものとする。
\begin{description}
\item[一]その定款に前項第十一号の退職金共済事業に関する経理に関する書類をその主たる事務所に備え置く旨並びに加入事業主及び被共済者が当該書類を閲覧できる旨の定めがあること。
\item[二]その定款に特定の個人又は団体に剰余金の分配を受ける権利を与える旨の定めがないこと。
\item[三]その定款に解散したときはその残余財産が特定の個人又は団体(国若しくは地方公共団体、公益社団法人若しくは公益財団法人、公益社団法人及び公益財団法人の認定等に関する法律(平成十八年法律第四十九号)第五条第十七号イからトまで(公益認定の基準)に掲げる法人又はその目的と類似の目的を有する他の一般社団法人若しくは一般財団法人を除く。)に帰属する旨の定めがないこと。
\item[四]前三号及び次号に掲げる要件の全てに該当していた期間において、特定の個人又は団体に剰余金の分配その他の方法(合併による資産の移転を含む。)により特別の利益を与えることを決定し、又は与えたことがないこと。
\item[五]各理事について、当該理事及び当該理事の配偶者又は三親等以内の親族その他の当該理事と財務省令で定める特殊の関係のある者である理事の合計数の理事の総数のうちに占める割合が、三分の一以下であること。
\end{description}
\item[\rensuji{3}]財務大臣は、第一項の指定をしたときは、これを告示する。
\end{description}
\noindent\hspace{10pt}(特定退職金共済団体の承認)
\begin{description}
\item[第七十四条]前条第一項の法人は、その行う退職金共済事業につき同項の承認を受けようとするときは、財務省令で定める事項を記載した申請書に退職金共済規程並びに一般社団法人及び一般財団法人にあつては定款の写しを添付し、これを当該法人の主たる事務所の所在地の所轄税務署長に提出しなければならない。
\item[\rensuji{2}]前項の退職金共済規程は、その退職金共済事業が前条第一項各号に掲げる要件に該当するかどうかを判定するために必要な事項につき規定したものでなければならない。
\item[\rensuji{3}]税務署長は、第一項の申請書の提出があつた場合において、これに添付された退職金共済規程が前条第一項各号に掲げる要件の全てに該当しているときは、その申請を承認するものとする。
\item[\rensuji{4}]税務署長は、前項の規定による承認又は却下の処分をするときは、第一項の申請書を提出した法人に対し、書面によりその旨を通知する。
\item[\rensuji{5}]前条第一項に規定する特定退職金共済団体(以下この款において「特定退職金共済団体」という。)は、第三項の規定による承認を受けた退職金共済規程のうち同条第一項各号に掲げる要件に係る事項の変更(同項第七号に規定する過去勤務期間又は合併等前勤務期間を退職給付金の額の計算の基礎に含めることとする変更を含む。以下この条及び次条第一項第一号において同じ。)をしようとするときは、その変更について第一項の税務署長の承認を受けなければならない。
\item[\rensuji{6}]第一項、第二項、第三項本文及び第四項の規定は、前項に規定する変更に係る承認について準用する。
\end{description}
\noindent\hspace{10pt}(特定退職金共済団体の承認の取消し等)
\begin{description}
\item[第七十五条]税務署長は、特定退職金共済団体につき次に掲げる事実があると認めるときは、前条第三項本文の規定による承認を取り消すことができる。
\begin{description}
\item[一]当該特定退職金共済団体の退職金共済規程のうち第七十三条第一項各号(特定退職金共済団体の要件)に掲げる要件に係る事項について前条第五項の規定による承認を受けないで変更をしたこと。
\item[二]当該特定退職金共済団体の退職金共済事業につき第七十三条第一項第一号、第四号、第五号、第十号又は第十一号に掲げる要件に反する事実があること。
\item[三]当該特定退職金共済団体の全ての被共済者につき第七十三条第一項第二号、第三号又は第六号から第八号までに掲げる要件に反する事実があること。
\end{description}
\item[\rensuji{2}]税務署長は、前項の規定による承認の取消しの処分をするときは、同項の特定退職金共済団体に対し、書面によりその旨を通知する。
\item[\rensuji{3}]特定退職金共済団体は、その行う退職金共済事業を廃止しようとするときは、その旨、その特定退職金共済団体の名称及び所在地並びに当該退職金共済事業を廃止しようとする年月日を記載した届出書を当該廃止しようとする日までに前条第一項の税務署長に提出しなければならない。
\end{description}
\noindent\hspace{10pt}(退職金共済制度等に基づく一時金で退職手当等とみなさないもの)
\begin{description}
\item[第七十六条]第七十二条第三項第一号(退職手当等とみなす一時金)に掲げる一時金は、次に掲げる給付(一時金に該当するものに限る。)を含まないものとする。
\begin{description}
\item[一]特定退職金共済団体が前条第一項の規定による承認の取消しを受け、又は同条第三項に規定する届出書を提出した場合において、その取消しを受け、又はその届出書の提出をした法人がその取消しを受けた時又は同項に規定する日以後に行う給付
\item[二]特定退職金共済団体が行う給付で、これに対応する掛金のうちに次に掲げる掛金が含まれているもの
\begin{description}
\item[イ]第七十三条第一項第一号(特定退職金共済団体の要件)に掲げる要件に反して被共済者が自ら負担した掛金
\item[ロ]第七十三条第一項第二号に掲げる要件に反して、当該特定退職金共済団体の被共済者が既に他の特定退職金共済団体の被共済者となつており、その者について、当該他の特定退職金共済団体の退職金共済契約に係る共済期間が当該特定退職金共済団体に係る共済期間と重複している場合における当該特定退職金共済団体に係る掛金
\item[ハ]第七十三条第一項第三号に掲げる要件に反して被共済者とされた者についての掛金
\item[ニ]掛金の月額が第七十三条第一項第六号に定める限度(同項第七号に規定する過去勤務等通算期間に対応する掛金の額にあつては、同号ロに定める限度)を超えて支出された場合における当該掛金
\item[ホ]第七十三条第一項第七号イに掲げる要件に反して同号に規定する過去勤務等通算期間を定め、当該過去勤務等通算期間に対応するものとして払い込んだ掛金
\item[ヘ]当該特定退職金共済団体の被共済者となつた日前の期間(当該被共済者の第七十三条第一項第七号に規定する過去勤務等通算期間を除く。)を給付の計算の基礎に含め、当該期間に対応するものとして払い込んだ掛金
\end{description}
\end{description}
\item[\rensuji{2}]第七十二条第三項第四号に規定する適格退職年金契約に基づいて支給を受ける一時金は、次に掲げる給付(一時金に該当するものに限る。)を含まないものとする。
\begin{description}
\item[一]法人税法附則第二十条第一項(退職年金等積立金に対する法人税の特例)に規定する適格退職年金契約に係る信託、生命保険又は生命共済の業務を行う信託会社(金融機関の信託業務の兼営等に関する法律により同法第一条第一項(兼営の認可)に規定する信託業務を営む銀行を含む。)、生命保険会社(保険業法第二条第三項(定義)に規定する生命保険会社及び同条第八項に規定する外国生命保険会社等をいう。)又は農業協同組合連合会(以下この項において「信託会社等」という。)が法人税法附則第二十条第三項に規定する適格退職年金契約につき法人税法施行令附則第十八条第一項(適格退職年金契約の承認の取消し)の規定による承認の取消しを受けた場合において、その信託会社等が当該契約に基づきその取消しを受けた時以後に行う給付
\item[二]前号に規定する業務を行う信託会社等が行う給付で、これに対応する掛金又は保険料のうちに法人税法施行令附則第十六条第一項第三号(適格退職年金契約の要件等)に掲げる要件に反して同項第二号に規定する受益者等とされた者に係る掛金又は保険料が含まれているもの
\end{description}
\item[\rensuji{3}]税務署長は、特定退職金共済団体の被共済者又は前項第二号に規定する受益者等のうちに第一項第二号又は前項第二号に掲げる給付を受けるべき者があると認めたときは、当該特定退職金共済団体又は同号に規定する信託会社等に対し、書面によりその旨及びその者の氏名を通知するものとする。
\item[\rensuji{4}]第一項及び第二項に規定する給付として支給される金額は、一時所得に係る収入金額とする。
\end{description}
\noindent\hspace{10pt}(退職所得の収入の時期)
\begin{description}
\item[第七十七条]居住者が一の勤務先を退職することにより二以上の法第三十条第一項(退職所得)に規定する退職手当等の支払を受ける権利を有することとなる場合には、その者の支払を受ける当該退職手当等については、これらのうち最初に支払を受けるべきものの支払を受けるべき日の属する年における収入金額として同条の規定を適用する。
\end{description}
\subsubsection*{第五款 山林所得}
\addcontentsline{toc}{subsubsection}{第五款 山林所得}
\noindent\hspace{10pt}(用語の意義)
\begin{description}
\item[第七十八条]この款において、次の各号に掲げる用語の意義は、当該各号に定めるところによる。
\begin{description}
\item[一]分収造林契約
\item[二]分収育林契約
\end{description}
\end{description}
\noindent\hspace{10pt}(分収造林契約又は分収育林契約の収益)
\begin{description}
\item[第七十八条の二]分収造林契約の当事者が当該契約に基づきその契約の目的となつた山林の造林による収益のうち当該山林の伐採又は譲渡による収益(第九十四条第一項各号(山林所得の収入金額とされる保険金等)に掲げるものを含む。次項において同じ。)を当該契約に定める一定の割合により分収する金額は、第三項に定めがあるものを除き、山林所得に係る収入金額とする。
\item[\rensuji{2}]分収育林契約の当事者が当該契約に基づきその契約の目的となつた山林の育林による収益のうち当該山林の伐採又は譲渡による収益を当該契約に定める一定の割合により分収する金額は、次項に定めがあるものを除き、山林所得に係る収入金額とする。
\item[\rensuji{3}]分収造林契約又は分収育林契約の当事者がその契約に基づき分収する金額で次の各号に掲げる金額のいずれかに該当するものは、山林所得以外の各種所得に係る収入金額とする。
\begin{description}
\item[一]分収造林契約又は分収育林契約の目的となつた山林の伐採又は譲渡前にその契約に定める一定の割合により分収する金額(第九十四条第一項各号に掲げるものを除く。)
\item[二]分収造林契約又は分収育林契約の締結の期間中引き続きその契約に係る地代、利息その他の対価(当該契約に基づく造林又は育林に係るものを除く。)に相当する金額の支払を受ける者が当該契約に定める一定の割合により分収する金額
\item[三]分収造林契約又は分収育林契約に係る権利を取得した日以後五年以内にその契約に定める一定の割合により分収する金額
\end{description}
\end{description}
\noindent\hspace{10pt}(分収造林契約又は分収育林契約に係る権利の譲渡等による所得)
\begin{description}
\item[第七十八条の三]分収造林契約又は分収育林契約に係る権利の譲渡による収入金額は、次項に定めがあるものを除き、山林所得に係る収入金額とする。
\item[\rensuji{2}]次の各号に掲げる分収造林契約又は分収育林契約の当事者の当該各号に掲げる収入金額は、事業所得又は雑所得に係る収入金額とする。
\begin{description}
\item[一]分収造林契約の当事者である土地の所有者若しくは造林者(当該土地の所有者以外の者で当該契約の目的となつた土地につき造林を行うものをいう。以下この項において同じ。)又は分収育林契約の当事者である土地の所有者若しくは育林者(当該土地の所有者以外の者で当該契約の目的となつた山林の育林を行うものをいう。以下この項において同じ。)
\item[二]分収造林契約の当事者である造林費負担者(当該契約に係る土地の所有者及び造林者以外の者でその造林に関する費用の全部又は一部を負担するものをいう。第四項において同じ。)又は分収育林契約の当事者である育林費負担者(当該契約に係る土地の所有者及び育林者以外の者でその育林に関する費用の全部又は一部を負担するものをいう。第四項において同じ。)
\end{description}
\item[\rensuji{3}]山林の所有者が当該山林につき分収育林契約を締結することにより、当該契約を締結する他の者から支払を受ける当該契約の目的となつた山林の持分の対価の額は、山林所得に係る収入金額とする。
\item[\rensuji{4}]分収造林契約又は分収育林契約の当事者が、不特定の者に対しその契約の造林費負担者又は育林費負担者として権利を取得し義務を負うこととなるための申込みを勧誘したことにより、新たに当該権利を取得し義務を負うこととなつた者から支払を受ける持分の対価の額は、山林所得に係る収入金額とする。
\end{description}
\subsubsection*{第六款 譲渡所得}
\addcontentsline{toc}{subsubsection}{第六款 譲渡所得}
\noindent\hspace{10pt}(資産の譲渡とみなされる行為)
\begin{description}
\item[第七十九条]法第三十三条第一項(譲渡所得)に規定する政令で定める行為は、建物若しくは構築物の所有を目的とする地上権若しくは賃借権(以下この条において「借地権」という。)又は地役権(特別高圧架空電線の架設、特別高圧地中電線若しくはガス事業法第二条第十二項(定義)に規定するガス事業者が供給する高圧のガスを通ずる導管の敷設、飛行場の設置、懸垂式鉄道若しくは跨こ
座式鉄道の敷設又は砂防法(明治三十年法律第二十九号)第一条(定義)に規定する砂防設備である導流堤その他財務省令で定めるこれに類するもの(第一号において「導流堤等」という。)の設置、都市計画法(昭和四十三年法律第百号)第四条第十四項(定義)に規定する公共施設の設置若しくは同法第八条第一項第四号(地域地区)の特定街区内における建築物の建築のために設定されたもので、建造物の設置を制限するものに限る。以下この条において同じ。)の設定(借地権に係る土地の転貸その他他人に当該土地を使用させる行為を含む。以下この条において同じ。)のうち、その対価として支払を受ける金額が次の各号に掲げる場合の区分に応じ当該各号に定める金額の十分の五に相当する金額を超えるものとする。
\begin{description}
\item[一]当該設定が建物若しくは構築物の全部の所有を目的とする借地権又は地役権の設定である場合(第三号に掲げる場合を除く。)
\item[二]当該設定が建物又は構築物の一部の所有を目的とする借地権の設定である場合
\item[三]当該設定が施設又は工作物(大深度地下の公共的使用に関する特別措置法(平成十二年法律第八十七号)第十六条(使用の認可の要件)の規定により使用の認可を受けた事業(以下この号において「認可事業」という。)と一体的に施行される事業として当該認可事業に係る同法第十四条第二項第二号(使用認可申請書)の事業計画書に記載されたものにより設置されるもののうち財務省令で定めるものに限る。)の全部の所有を目的とする地下について上下の範囲を定めた借地権の設定である場合
\end{description}
\item[\rensuji{2}]借地権に係る土地を他人に使用させる場合において、その土地の使用により、その使用の直前におけるその土地の利用状況に比し、その土地の所有者及びその借地権者がともにその土地の利用を制限されることとなるときは、これらの者については、これらの者が使用の対価として支払を受ける金額の合計額を前項に規定する支払を受ける金額とみなして、同項の規定を適用する。
\item[\rensuji{3}]第一項の規定の適用については、借地権又は地役権の設定の対価として支払を受ける金額が当該設定により支払を受ける地代の年額の二十倍に相当する金額以下である場合には、当該設定は、同項の行為に該当しないものと推定する。
\end{description}
\noindent\hspace{10pt}(特別の経済的な利益で借地権の設定等による対価とされるもの)
\begin{description}
\item[第八十条]前条第一項に規定する借地権又は地役権の設定(当該借地権に係る土地の転貸その他他人に当該土地を使用させる行為を含む。以下この条において同じ。)をしたことに伴い、通常の場合の金銭の貸付けの条件に比し特に有利な条件による金銭の貸付け(いずれの名義をもつてするかを問わず、これと同様の経済的性質を有する金銭の交付を含む。以下この条において同じ。)その他特別の経済的な利益を受ける場合には、当該金銭の貸付けにより通常の条件で金銭の貸付けを受けた場合に比して受ける利益その他当該特別の経済的な利益の額を前条第一項又は第二項に規定する対価の額に加算した金額をもつてこれらの規定に規定する支払を受ける金額とみなして、これらの規定を適用する。
\item[\rensuji{2}]前項の場合において、その受けた金銭の貸付けにより通常の条件で金銭の貸付けを受けた場合に比して受ける利益の額は、当該貸付けを受けた金額から、当該金額について通常の利率(当該貸付けを受けた金額につき利息を附する旨の約定がある場合には、その利息に係る利率を控除した利率)の十分の五に相当する利率による複利の方法で計算した現在価値に相当する金額(当該金銭の貸付けを受ける期間が同項の設定に係る権利の存続期間に比して著しく短い期間として約定されている場合において、長期間にわたつて地代をすえ置く旨の約定がされていることその他当該権利に係る土地の上に存する建物又は構築物の状況、地代に関する条件等に照らし、当該金銭の貸付けを受けた期間が将来更新されるものと推測するに足りる明らかな事実があるときは、借地権又は地役権の設定を受けた者が当該設定により受ける利益から判断して当該金銭の貸付けが継続されるものと合理的に推定される期間を基礎として当該方法により計算した場合の現在価値に相当する金額)を控除した金額によるものとする。
\end{description}
\noindent\hspace{10pt}(譲渡所得の基因とされないたな卸資産に準ずる資産)
\begin{description}
\item[第八十一条]法第三十三条第二項第一号(譲渡所得に含まれない所得)に規定する政令で定めるものは、次に掲げる資産とする。
\begin{description}
\item[一]不動産所得、山林所得又は雑所得を生ずべき業務に係る第三条各号(たな卸資産の範囲)に掲げる資産に準ずる資産
\item[二]減価償却資産で第百三十八条(少額の減価償却資産の取得価額の必要経費算入)の規定に該当するもの(同条に規定する取得価額が十万円未満であるもののうち、その者の業務の性質上基本的に重要なものを除く。)
\item[三]減価償却資産で第百三十九条第一項(一括償却資産の必要経費算入)の規定の適用を受けたもの(その者の業務の性質上基本的に重要なものを除く。)
\end{description}
\end{description}
\noindent\hspace{10pt}(短期譲渡所得の範囲)
\begin{description}
\item[第八十二条]法第三十三条第三項第一号(短期譲渡所得)に規定する政令で定める所得は、自己の研究の成果である特許権、実用新案権その他の工業所有権、自己の育成の成果である育成者権、自己の著作に係る著作権及び自己の探鉱により発見した鉱床に係る採掘権の譲渡による所得とする。
\end{description}
\subsubsection*{第七款 雑所得}
\addcontentsline{toc}{subsubsection}{第七款 雑所得}
\noindent\hspace{10pt}(公的年金等とされる年金)
\begin{description}
\item[第八十二条の二]法第三十五条第三項第一号(公的年金等の定義)に規定する政令で定める年金(これに類する給付を含む。)は、次に掲げる年金とする。
\begin{description}
\item[一]国民年金法等の一部を改正する法律(昭和六十年法律第三十四号)第五条(船員保険法の一部改正)の規定による改正前の船員保険法の規定に基づく年金
\item[二]厚生年金保険法附則第二十八条(指定共済組合の組合員)に規定する共済組合が支給する年金
\item[三]被用者年金制度の一元化等を図るための厚生年金保険法等の一部を改正する法律(平成二十四年法律第六十三号。以下この項において「一元化法」という。)附則第四十一条第一項(追加費用対象期間を有する者の特例等)又は第六十五条第一項(追加費用対象期間を有する者の特例等)の規定に基づく年金
\item[四]一元化法附則第三十六条第一項(改正前国共済法による職域加算額の経過措置)の規定によりなおその効力を有するものとされる同項の改正前国共済法の規定に基づく年金
\item[五]一元化法附則第三十七条第一項(改正前国共済法による給付等)の規定によりなおその効力を有するものとされる同項の改正前国共済法の規定に基づく年金
\item[六]旧令による共済組合等からの年金受給者のための特別措置法(昭和二十五年法律第二百五十六号)第三条第一項若しくは第二項(旧陸軍共済組合及び共済協会の権利義務の承継)、第四条第一項(外地関係共済組合に係る年金の支給)又は第七条の二第一項(旧共済組合員に対する年金の支給)の規定に基づく年金
\item[七]地方公務員等共済組合法の一部を改正する法律(平成二十三年法律第五十六号)附則の規定に基づく年金
\item[八]一元化法附則第六十条第一項(改正前地共済法による職域加算額の経過措置)の規定によりなおその効力を有するものとされる同項の改正前地共済法の規定に基づく年金
\item[九]一元化法附則第六十一条第一項(改正前地共済法による給付等)の規定によりなおその効力を有するものとされる同項の改正前地共済法の規定に基づく年金
\item[十]一元化法附則第七十八条第一項(改正前私学共済法による職域加算額の経過措置)の規定によりなおその効力を有するものとされる同項の改正前私学共済法の規定に基づく年金
\item[十一]一元化法附則第七十九条(改正前私学共済法による給付)の規定によりなおその効力を有するものとされる同条の改正前私学共済法の規定に基づく年金
\item[十二]厚生年金保険制度及び農林漁業団体職員共済組合制度の統合を図るための農林漁業団体職員共済組合法等を廃止する等の法律第一条(農林漁業団体職員共済組合法等の廃止)の規定による廃止前の農林漁業団体職員共済組合法(昭和三十三年法律第九十九号)の規定に基づく年金
\item[十三]旧厚生年金保険法第九章(厚生年金基金及び企業年金連合会)の規定に基づく年金
\end{description}
\item[\rensuji{2}]法第三十五条第三項第三号に規定する政令で定める年金(これに類する給付を含む。)は、次に掲げる給付とする。
\begin{description}
\item[一]第七十二条第三項第一号又は第八号(退職手当等とみなす一時金)に規定する制度に基づいて支給される年金(これに類する給付を含む。)
\item[二]中小企業退職金共済法第十二条第一項(退職金の分割支給等)に規定する分割払の方法により支給される同条第五項に規定する分割退職金
\item[三]第七十二条第三項第三号イに規定する小規模企業共済契約に基づいて小規模企業共済法第九条の三第一項(共済金の分割支給等)に規定する分割払の方法により支給される同条第五項に規定する分割共済金
\item[四]法人税法附則第二十条第三項(退職年金等積立金に対する法人税の特例)に規定する適格退職年金契約に基づいて支給を受ける退職年金(当該契約に基づいて払い込まれた掛金又は保険料のうちにその退職年金が支給される基因となつた勤務をした者の負担した金額がある場合には、その年において支給される当該退職年金の額から当該退職年金の額(その年金の支給開始の日以後に当該契約に基づいて分配を受ける剰余金の額に相当する部分の金額を除く。)に当該退職年金に係る次条第一項の規定に準じて計算した割合を乗じて計算した金額を控除した金額に相当する部分に限る。)
\item[五]第七十二条第三項第五号イからハまでに掲げる規定に基づいて支給を受ける年金(同号に規定する規約に基づいて拠出された掛金のうちにその年金が支給される確定給付企業年金法第二十五条第一項(加入者)に規定する加入者(同項に規定する加入者であつた者を含む。)の負担した金額がある場合には、その年において支給される当該年金の額から当該年金の額(その年金の支給開始の日以後に当該規約に基づいて分配を受ける剰余金の額に相当する部分の金額を除く。)に当該年金に係る次条第一項の規定に準じて計算した割合を乗じて計算した金額を控除した金額に相当する部分に限る。)
\item[六]確定拠出年金法第四条第三項(承認の基準等)に規定する企業型年金規約又は同法第五十六条第三項(承認の基準等)に規定する個人型年金規約に基づいて同法第二十八条第一号(給付の種類)(同法第七十三条(企業型年金に係る規定の準用)において準用する場合を含む。)に掲げる老齢給付金として支給される年金
\end{description}
\item[\rensuji{3}]前項第一号に掲げる給付は、第七十六条第一項各号(退職金共済制度等に基づく一時金で退職手当等とみなさないもの)に掲げる給付(年金に該当するものに限る。)を含まないものとし、前項第四号に掲げる退職年金は、第七十六条第二項各号に掲げる給付(退職年金に該当するものに限る。)を含まないものとする。
\item[\rensuji{4}]前項に規定する給付として支給される金額は、法第三十五条第三項に規定する公的年金等に係る雑所得以外の雑所得に係る収入金額とする。
\end{description}
\noindent\hspace{10pt}(確定給付企業年金の額から控除する金額)
\begin{description}
\item[第八十二条の三]法第三十五条第三項第三号(公的年金等の定義)に規定する政令で定めるところにより計算した金額は、その年において同号に規定する規約に基づいて支給される年金の額(その年金の支給開始の日以後に当該規約に基づいて分配を受ける剰余金の額に相当する部分の金額(次項において「剰余金額」という。)を除く。)に、第一号に掲げる金額のうちに第二号に掲げる金額の占める割合を乗じて計算した金額とする。
\begin{description}
\item[一]次に掲げる年金の区分に応じそれぞれ次に定める金額
\begin{description}
\item[イ]その支給開始の日において支給総額が確定している年金
\item[ロ]その支給開始の日において支給総額が確定していない年金
\end{description}
\item[二]法第三十五条第三項第三号に規定する掛金のうちその年金が支給される基因となつた同号に規定する加入者の負担した金額(当該金額に次に掲げる資産に係る当該加入者が負担した部分に相当する金額が含まれている場合には、当該金額を控除した金額)
\begin{description}
\item[イ]平成二十五年厚生年金等改正法附則第三十五条第一項(解散存続厚生年金基金の残余財産の確定給付企業年金への交付)の規定により平成二十五年厚生年金等改正法附則第三条第十一号(定義)に規定する存続厚生年金基金(ニからヘまでにおいて「存続厚生年金基金」という。)から交付された同項に規定する残余財産
\item[ロ]平成二十五年厚生年金等改正法附則第五十五条第二項(存続連合会から確定給付企業年金への年金給付等積立金等の移換)の規定により平成二十五年厚生年金等改正法附則第三条第十三号に規定する存続連合会(ハにおいて「存続連合会」という。)から移換された平成二十五年厚生年金等改正法附則第五十五条第一項に規定する年金給付等積立金等
\item[ハ]平成二十五年厚生年金等改正法附則第六十二条第二項(移換に関する経過措置)の規定によりなおその効力を有するものとされる旧厚生年金保険法第百六十五条の二第二項(連合会から確定給付企業年金への年金給付等積立金の移換)の規定により存続連合会から移換された平成二十五年厚生年金等改正法附則第六十二条第一項の規定によりなおその効力を有するものとされる旧厚生年金保険法第百六十五条第五項(連合会から基金への権利義務の移転及び年金給付等積立金の移換)に規定する年金給付等積立金
\item[ニ]平成二十五年厚生年金等改正法附則第五条第一項(存続厚生年金基金に係る改正前厚生年金保険法等の効力等)の規定によりなおその効力を有するものとされる平成二十五年厚生年金等改正法第二条(確定給付企業年金法の一部改正)の規定による改正前の確定給付企業年金法(ホ及びヘにおいて「旧効力確定給付企業年金法」という。)第百十条の二第三項(厚生年金基金の設立事業所に係る給付の支給に関する権利義務の確定給付企業年金への移転)の規定により存続厚生年金基金から権利義務が承継された同条第四項に規定する積立金
\item[ホ]旧効力確定給付企業年金法第百十一条第二項(厚生年金基金から規約型企業年金への移行)又は第百十二条第四項(厚生年金基金から基金への移行)の規定により存続厚生年金基金から権利義務が承継された平成二十五年厚生年金等改正法附則第五条第一項の規定によりなおその効力を有するものとされる旧厚生年金保険法第百三十条の二第二項(年金たる給付及び一時金たる給付に要する費用に関する契約)に規定する年金給付等積立金
\item[ヘ]旧効力確定給付企業年金法第百十五条の三第二項(厚生年金基金から確定給付企業年金への脱退一時金相当額の移換)の規定により存続厚生年金基金から移換された同条第一項に規定する脱退一時金相当額
\item[ト]旧厚生年金保険法の規定により旧厚生年金保険法第百四十九条第一項(連合会)に規定する連合会から移換された資産又は平成二十五年厚生年金等改正法第二条の規定による改正前の確定給付企業年金法の規定により平成二十五年厚生年金等改正法附則第三条第十号に規定する旧厚生年金基金から権利義務が承継され、若しくは移換された資産で、財務省令で定めるもの
\item[チ]確定拠出年金法第五十四条の四第二項(確定給付企業年金の加入者となつた者の個人別管理資産の移換)の規定により同法第二条第七項第一号ロ(定義)に規定する資産管理機関から移換された同条第十二項に規定する個人別管理資産
\item[リ]確定拠出年金法第七十四条の四第二項(確定給付企業年金の加入者となつた者の個人別管理資産の移換)の規定により同法第二条第五項に規定する連合会から移換された同条第十二項に規定する個人別管理資産
\end{description}
\end{description}
\item[\rensuji{2}]前項第一号ロに定める支給総額の見込額は、次に掲げる金額とする。
\begin{description}
\item[一]前項に規定する年金のうち次に掲げるもの(次号に該当するものを除く。)については、その支給の基礎となる規約において定められているその年額(剰余金額を除く。)に、次に掲げる年金の区分に応じそれぞれ次に定める年数を乗じて計算した金額
\begin{description}
\item[イ]有期の年金で、受給権者(その年金の支給開始の日における確定給付企業年金法第三十条第一項(裁定)に規定する受給権者をいう。以下この項において同じ。)がその期間内に死亡した場合にはその死亡後の期間につき支給を行わないもの
\item[ロ]有期の年金で、受給権者がその支給開始の日以後一定期間(以下この項において「保証期間」という。)内に死亡した場合にはその死亡後においてもその保証期間の終了の日までその支給を継続するもの
\item[ハ]終身の年金で、受給権者の生存中に限り支給するもの
\item[ニ]終身の年金で、受給権者の生存中支給するほか、受給権者が保証期間内に死亡した場合にはその死亡後においてもその保証期間の終了の日までその支給を継続するもの
\end{description}
\item[二]前号ロ又はニに掲げる年金のうち支給総額の見込額の計算の基礎となる年数が保証期間に係る年数とされるもので、受給権者に支給する年金の年額と受給権者の死亡後に支給する年金の年額とが異なるものについては、受給権者に支給する年金の年額に受給権者に係る支給開始日における余命年数を乗じて計算した金額と受給権者の死亡後に支給する年金の年額に保証期間に係る年数と当該余命年数との差に相当する年数を乗じて計算した金額との合計額
\item[三]その支給の条件が前二号に定めるところと異なる年金については、その支給の条件に応じ、その年額、受給権者(受給権者の死亡後その親族その他の者に支給する年金については、受給権者及び当該親族その他の者)に係る余命年数及び保証期間(受給権者の死亡後一定期間年金を支給する旨を定めている場合におけるその一定期間を含む。)を基礎として前二号の規定に準じて計算した金額
\end{description}
\item[\rensuji{3}]第一項に規定する割合は、小数点以下二位まで算出し、三位以下を切り上げたところによる。
\end{description}
\noindent\hspace{10pt}(勤労者財産形成基金契約に基づいて支出された信託金等の取扱い)
\begin{description}
\item[第八十二条の四]勤労者財産形成基金が、勤労者財産形成促進法第六条の三第二項(勤労者財産形成基金契約)に規定する第一種勤労者財産形成基金契約に基づいて同項第二号に規定する信託の受益者等のために支出した同項第一号に規定する信託金等又は同条第三項に規定する第二種勤労者財産形成基金契約に基づいて同項第二号に規定する勤労者について支出した同項第一号に規定する預入金等は、当該信託の受益者等又は当該勤労者に対する雑所得に係る総収入金額に含まれないものとする。
\item[\rensuji{2}]事業を営む個人が、勤労者財産形成促進法第七条の二十(拠出)の規定により前項に規定する信託金等又は預入金等の払込みに充てるために必要な金銭を支出した場合には、その支出した金額は、その支出した日の属する年分の当該事業に係る不動産所得の金額、事業所得の金額又は山林所得の金額の計算上、必要経費に算入する。
\end{description}
\subsection*{第二節 所得金額の計算の通則}
\addcontentsline{toc}{subsection}{第二節 所得金額の計算の通則}
\noindent\hspace{10pt}(分割対価資産の一部のみを分割法人の株主等に交付する場合の取扱い)
\begin{description}
\item[第八十三条]分割法人(法人税法第二条第十二号の二(定義)に規定する分割法人をいう。以下この条において同じ。)が分割により交付を受ける同法第二条第十二号の九イに規定する分割対価資産(以下この条において「分割対価資産」という。)の一部のみを当該分割法人の株主等に交付する分割(二以上の法人を分割法人とする分割で法人を設立するものを除く。)が行われた場合には、当該分割により当該株主等が交付を受けた当該分割対価資産については、分割型分割(同号に規定する分割型分割をいう。次項において同じ。)が行われたものとみなして、法の規定を適用する。
\item[\rensuji{2}]二以上の法人を分割法人とする分割で法人を設立するものが行われた場合において、分割法人のうちに、当該分割により交付を受けた分割対価資産の全部又は一部をその株主等に交付した法人があるときは、当該法人を分割法人とする分割型分割が行われたものとみなして、法の規定を適用する。
\item[\rensuji{3}]前二項の規定の適用がある場合には、前二項の株主等が交付を受けた分割対価資産に係る前二項の分割型分割により分割承継法人(法人税法第二条第十二号の三に規定する分割承継法人をいう。以下この項において同じ。)に移転した分割法人の資産及び負債の金額は、前二項の分割により分割承継法人に移転した当該分割法人の資産及び負債の金額のうち法人税法施行令第百二十三条の七(株式等を分割法人と分割法人の株主等とに交付する分割における移転資産等の按あん
分)の規定により算定された当該分割型分割に係る資産及び負債の金額とする。
\end{description}
\noindent\hspace{10pt}(合併等により交付する株式に一に満たない端数がある場合の所得計算)
\begin{description}
\item[第八十三条の二]合併に係る合併法人が当該合併により当該合併に係る被合併法人の株主等に交付すべき第百十二条第一項(合併により取得した株式等の取得価額)に規定する合併親法人株式(以下この項において「合併親法人株式」という。)の数(出資にあつては、金額。以下第三項までにおいて同じ。)に一に満たない端数が生ずる場合において、当該端数に応じて金銭が交付されるときは、当該端数に相当する部分は、当該合併親法人株式に含まれるものとして、当該株主等の各年分の事業所得の金額、譲渡所得の金額又は雑所得の金額を計算する。
\item[\rensuji{2}]分割型分割に係る分割法人が当該分割型分割によりその株主等に交付すべき当該分割型分割に係る分割承継法人の株式(出資を含む。)又は第百十三条第一項(分割型分割により取得した株式等の取得価額)に規定する分割承継親法人株式(以下この項において「分割承継法人株式等」という。)の数に一に満たない端数が生ずる場合において、当該端数に応じて金銭が交付されるときは、当該端数に相当する部分は、当該分割承継法人株式等に含まれるものとして、当該株主等の各年分の事業所得の金額、譲渡所得の金額又は雑所得の金額を計算する。
\item[\rensuji{3}]株式分配に係る現物分配法人が当該株式分配によりその株主等に交付すべき当該株式分配に係る第百十三条の二第一項(株式分配により取得した株式等の取得価額)に規定する完全子法人株式(以下この項において「完全子法人株式」という。)の数に一に満たない端数が生ずる場合において、当該端数に応じて金銭が交付されるときは、当該端数に相当する部分は、当該完全子法人株式に含まれるものとして、当該株主等の各年分の事業所得の金額、譲渡所得の金額又は雑所得の金額を計算する。
\item[\rensuji{4}]株式交換に係る株式交換完全親法人が当該株式交換により当該株式交換に係る株式交換完全子法人の株主に交付すべき法第五十七条の四第一項(株式交換等に係る譲渡所得等の特例)に規定する政令で定める関係がある法人の株式(以下この項において「株式交換完全支配親法人株式」という。)の数に一に満たない端数が生ずる場合において、当該端数に応じて金銭が交付されるときは、当該端数に相当する部分は、当該株式交換完全支配親法人株式に含まれるものとして、当該株主の各年分の事業所得の金額、譲渡所得の金額又は雑所得の金額を計算する。
\item[\rensuji{5}]この条において、次の各号に掲げる用語の意義は、当該各号に定めるところによる。
\begin{description}
\item[一]合併法人
\item[二]被合併法人
\item[三]分割型分割
\item[四]分割法人
\item[五]分割承継法人
\item[六]株式分配
\item[七]現物分配法人
\item[八]株式交換完全親法人
\item[九]株式交換完全子法人
\end{description}
\end{description}
\noindent\hspace{10pt}(譲渡制限付株式の価額等)
\begin{description}
\item[第八十四条]個人が法人に対して役務の提供をした場合において、当該役務の提供の対価として譲渡制限付株式(次に掲げる要件に該当する株式(出資、投資信託及び投資法人に関する法律第二条第十四項(定義)に規定する投資口その他これらに準ずるものを含む。以下この条において同じ。)をいう。以下この項において同じ。)であつて当該役務の提供の対価として当該個人に生ずる債権の給付と引換えに当該個人に交付されるものその他当該個人に給付されることに伴つて当該債権が消滅する場合の当該譲渡制限付株式(以下この項において「特定譲渡制限付株式」という。)が当該個人に交付されたとき(合併又は前条第五項第三号に規定する分割型分割に際し当該合併又は分割型分割に係る同項第二号に規定する被合併法人又は同項第四号に規定する分割法人の当該特定譲渡制限付株式を有する者に対し交付される当該合併又は分割型分割に係る同項第一号に規定する合併法人又は同項第五号に規定する分割承継法人の譲渡制限付株式その他の財務省令で定める譲渡制限付株式(以下この項において「承継譲渡制限付株式」という。)が当該個人に交付されたときを含む。)における当該特定譲渡制限付株式又は承継譲渡制限付株式に係る法第三十六条第二項(収入金額)の価額は、当該特定譲渡制限付株式又は承継譲渡制限付株式の譲渡(担保権の設定その他の処分を含む。第一号において同じ。)についての制限が解除された日における価額とする。
\begin{description}
\item[一]譲渡についての制限がされており、かつ、当該譲渡についての制限に係る期間(次号において「譲渡制限期間」という。)が設けられていること。
\item[二]当該個人から役務の提供を受ける法人又はその株式を発行し、若しくは当該個人に交付した法人がその株式を無償で取得することとなる事由(その株式の交付を受けた当該個人が譲渡制限期間内の所定の期間勤務を継続しないこと若しくは当該個人の勤務実績が良好でないことその他の当該個人の勤務の状況に基づく事由又はこれらの法人の業績があらかじめ定めた基準に達しないことその他のこれらの法人の業績その他の指標の状況に基づく事由に限る。)が定められていること。
\end{description}
\item[\rensuji{2}]発行法人から次の各号に掲げる権利で当該権利の譲渡についての制限その他特別の条件が付されているものを与えられた場合(株主等として与えられた場合(当該発行法人の他の株主等に損害を及ぼすおそれがないと認められる場合に限る。)を除く。)における当該権利に係る法第三十六条第二項の価額は、当該権利の行使により取得した株式のその行使の日(第五号に掲げる権利にあつては、当該権利に基づく払込み又は給付の期日(払込み又は給付の期間の定めがある場合には、当該払込み又は給付をした日))における価額から次の各号に掲げる権利の区分に応じ当該各号に定める金額を控除した金額による。
\begin{description}
\item[一]商法等の一部を改正する等の法律(平成十三年法律第七十九号)第一条(商法の一部改正)の規定による改正前の商法(明治三十二年法律第四十八号)第二百十条ノ二第二項(取締役又は使用人に譲渡するための自己株式の取得)の決議に基づき与えられた同項第三号に規定する権利
\item[二]商法等の一部を改正する法律(平成十三年法律第百二十八号。以下この号において「商法等改正法」という。)第一条(商法の一部改正)の規定による改正前の商法第二百八十条ノ十九第二項(取締役又は使用人に対する新株引受権の付与)の決議に基づき与えられた同項に規定する新株の引受権
\item[三]会社法の施行に伴う関係法律の整備等に関する法律第六十四条(商法の一部改正)の規定による改正前の商法第二百八十条ノ二十一第一項(新株予約権の有利発行の決議)の決議に基づき発行された同項に規定する新株予約権
\item[四]会社法第二百三十八条第二項(募集事項の決定)の決議(同法第二百三十九条第一項(募集事項の決定の委任)の決議による委任に基づく同項に規定する募集事項の決定及び同法第二百四十条第一項(公開会社における募集事項の決定の特則)の規定による取締役会の決議を含む。)に基づき発行された新株予約権(当該新株予約権を引き受ける者に特に有利な条件若しくは金額であることとされるもの又は役務の提供その他の行為による対価の全部若しくは一部であることとされるものに限る。)
\item[五]株式と引換えに払い込むべき額が有利な金額である場合における当該株式を取得する権利(前各号に掲げるものを除く。)
\end{description}
\end{description}
\noindent\hspace{10pt}(法人等の資産の専属的利用による経済的利益の額)
\begin{description}
\item[第八十四条の二]法人又は個人の事業の用に供する資産を専属的に利用することにより個人が受ける経済的利益の額は、その資産の利用につき通常支払うべき使用料その他その利用の対価に相当する額(その利用者がその利用の対価として支出する金額があるときは、これを控除した額)とする。
\end{description}
\noindent\hspace{10pt}(非事業用資産の減価の額の計算)
\begin{description}
\item[第八十五条]法第三十八条第二項(譲渡所得の基因となる資産の減価の額)に規定する資産の同項第二号に掲げる期間に係る減価の額は、当該資産の取得に要した金額並びに設備費及び改良費の額の合計額につき、当該資産と同種の減価償却資産に係る第百二十九条(減価償却資産の耐用年数等)に規定する耐用年数に一・五を乗じて計算した年数により第百二十条第一項第一号イ(1)(減価償却資産の償却の方法)に規定する旧定額法に準じて計算した金額に、当該資産の当該期間に係る年数を乗じて計算した金額とする。
\item[\rensuji{2}]前項の場合において、次の各号に掲げる年数に一年未満の端数があるときの処理については、当該各号に定めるところによる。
\begin{description}
\item[一]前項に規定する一・五を乗じて計算した年数
\item[二]前項に規定する期間に係る年数
\end{description}
\end{description}
\subsection*{第三節 収入金額の計算}
\addcontentsline{toc}{subsection}{第三節 収入金額の計算}
\noindent\hspace{10pt}(自家消費の場合のたな卸資産に準ずる資産の範囲)
\begin{description}
\item[第八十六条]法第三十九条(たな卸資産等の自家消費の場合の総収入金額算入)に規定する政令で定めるものは、第八十一条各号(譲渡所得の基因とされないたな卸資産に準ずる資産)に掲げる資産(山林を除く。)とする。
\end{description}
\noindent\hspace{10pt}(贈与等の場合のたな卸資産に準ずる資産の範囲)
\begin{description}
\item[第八十七条]法第四十条第一項(たな卸資産の贈与等の場合の総収入金額算入)に規定する政令で定めるものは、前条に規定する資産及び事業所得の基因となる有価証券とする。
\end{description}
\noindent\hspace{10pt}(農産物の範囲)
\begin{description}
\item[第八十八条]法第四十一条第一項(農産物の収穫の場合の総収入金額算入)に規定する政令で定める農産物は、米、麦その他の穀物、馬鈴しよ、甘しよ、たばこ、野菜、花、種苗その他のほ場作物、果樹、樹園の生産物又は温室その他特殊施設を用いて生産する園芸作物とする。
\end{description}
\noindent\hspace{10pt}(発行法人から与えられた株式を取得する権利の譲渡による収入金額)
\begin{description}
\item[第八十八条の二]法第四十一条の二(発行法人から与えられた株式を取得する権利の譲渡による収入金額)に規定する政令で定める権利は、第八十四条第二項各号(譲渡制限付株式の価額等)に掲げる権利で当該権利の行使をしたならば同項の規定の適用のあるもの(次項において「新株予約権等」という。)とする。
\item[\rensuji{2}]法第四十一条の二に規定する政令で定める者は、贈与、相続、遺贈又は譲渡により新株予約権等を取得した者で当該新株予約権等を行使できることとなるものとする。
\end{description}
\noindent\hspace{10pt}(国庫補助金等の範囲)
\begin{description}
\item[第八十九条]法第四十二条第一項(国庫補助金等の総収入金額不算入)に規定する国庫補助金等は、国又は地方公共団体の補助金又は給付金のほか、次に掲げる助成金又は補助金とする。
\begin{description}
\item[一]障害者の雇用の促進等に関する法律(昭和三十五年法律第百二十三号)第四十九条第二項(納付金関係業務)に基づく独立行政法人高齢・障害・求職者雇用支援機構の同条第一項第二号、第三号及び第五号から第七号までに規定する助成金
\item[二]福祉用具の研究開発及び普及の促進に関する法律(平成五年法律第三十八号)第七条第一号(国立研究開発法人新エネルギー・産業技術総合開発機構の業務)に基づく国立研究開発法人新エネルギー・産業技術総合開発機構の助成金
\item[三]国立研究開発法人新エネルギー・産業技術総合開発機構法(平成十四年法律第百四十五号)第十五条第三号(業務の範囲)に基づく国立研究開発法人新エネルギー・産業技術総合開発機構の助成金(外国法人、外国の政府若しくは地方公共団体に置かれる試験研究機関(試験所、研究所その他これらに類する機関をいう。以下この号において同じ。)、国際機関に置かれる試験研究機関若しくは外国の大学若しくはその附属の試験研究機関(以下この号において「外国試験研究機関等」という。)又は外国試験研究機関等の研究員と共同して行う試験研究に関する助成金を除く。)
\item[四]独立行政法人農畜産業振興機構法(平成十四年法律第百二十六号)第十条第二号(業務の範囲)に基づく独立行政法人農畜産業振興機構の補助金
\item[五]日本たばこ産業株式会社が日本たばこ産業株式会社法(昭和五十九年法律第六十九号)第九条(事業計画)の規定による認可を受けた事業計画に定めるところに従つて交付するたばこ事業法(昭和五十九年法律第六十八号)第二条第二号(定義)に規定する葉たばこの生産基盤の強化のための助成金
\end{description}
\end{description}
\noindent\hspace{10pt}(国庫補助金等に係る固定資産の償却費の計算等)
\begin{description}
\item[第九十条]法第四十二条第一項又は第二項(国庫補助金等の総収入金額不算入)の規定の適用を受けた固定資産(山林を含む。以下この条及び次条第二項において同じ。)について行うべき法第四十九条第一項(減価償却資産の償却費の計算及びその償却の方法)に規定する償却費の計算及びその固定資産の譲渡があつた場合における事業所得の金額、山林所得の金額、譲渡所得の金額又は雑所得の金額の計算については、次に定めるところによる。
\begin{description}
\item[一]法第四十二条第一項に規定する国庫補助金等により取得し、又は改良した固定資産については、その固定資産の取得に要した金額(山林については、植林費の額。次号において同じ。)又は改良費の額に相当する金額から同項の規定により総収入金額に算入されない金額に相当する金額を控除した金額をもつて取得し、又は改良したものとみなす。
\item[二]法第四十二条第二項に規定する固定資産については、その固定資産の取得に要した金額は、ないものとみなす。
\end{description}
\end{description}
\noindent\hspace{10pt}(総収入金額に算入されない条件付国庫補助金等の額の計算等)
\begin{description}
\item[第九十一条]法第四十三条第二項(条件付国庫補助金等の総収入金額不算入)に規定する政令で定める金額は、次の各号に掲げる場合の区分に応じ当該各号に掲げる金額とする。
\begin{description}
\item[一]法第四十三条第二項に規定する国庫補助金等を減価償却資産の取得に充てた場合
\begin{description}
\item[イ]当該資産の取得に要した金額
\item[ロ]当該資産の取得に要した金額から、当該金額を基礎としてその取得の日から当該国庫補助金等の返還を要しないこととなつた日までの期間に係る法第四十九条第一項(減価償却資産の償却費の計算及びその償却の方法)の規定に準じて計算した償却費の額の累積額を控除した金額
\end{description}
\item[二]法第四十三条第二項に規定する国庫補助金等を減価償却資産の改良に充てた場合
\begin{description}
\item[イ]当該資産の改良に要した金額
\item[ロ]当該資産の改良に要した金額から、当該金額を基礎としてその改良の日から当該国庫補助金等の返還を要しないこととなつた日までの期間に係る法第四十九条第一項の規定に準じて計算した償却費の額の累積額を控除した金額
\end{description}
\item[三]法第四十三条第二項に規定する国庫補助金等を減価償却資産以外の固定資産の取得若しくは改良又は山林の取得に充てた場合
\end{description}
\item[\rensuji{2}]法第四十三条第一項に規定する国庫補助金等により取得し又は改良した固定資産について行うべき法第四十九条第一項に規定する償却費の計算及びその固定資産の譲渡があつた場合における事業所得の金額、山林所得の金額、譲渡所得の金額又は雑所得の金額の計算については、当該資産は、その取得に要した金額(山林については、植林費の額)又は改良費の額に相当する金額から当該国庫補助金等の額のうち法第四十三条第二項に規定する返還を要しないことが確定した部分に相当する金額を控除した金額をもつて取得し又は改良したものとみなし、当該確定した部分に相当する金額から前項第一号又は第二号に掲げる金額を控除した金額に相当する金額は、同項第一号ロ又は第二号ロに規定する期間に係る当該償却費として各年分の不動産所得の金額、事業所得の金額、山林所得の金額又は雑所得の金額の計算上必要経費に算入されなかつたものとみなす。
\end{description}
\noindent\hspace{10pt}(資産の移転等に含まれない行為)
\begin{description}
\item[第九十二条]法第四十四条(移転等の支出に充てるための交付金の総収入金額不算入)に規定する政令で定める行為は、第百八十一条(資本的支出)に規定する支出に係る行為とする。
\end{description}
\noindent\hspace{10pt}(収用に類するやむを得ない事由)
\begin{description}
\item[第九十三条]法第四十四条(移転等の支出に充てるための交付金の総収入金額不算入)に規定する政令で定めるやむを得ない事由は、租税特別措置法第三十三条第一項各号(収用等に伴い代替資産を取得した場合の課税の特例)に規定する収用、買取り、換地処分、権利変換、買収若しくは権利の消滅、同条第三項第一号に規定する土地収用法等の規定に基づく使用、同項第二号に規定する事由に基づく同号に規定する資産の取壊し若しくは除去若しくは同項第三号に規定する事由に基づく同号に規定する資産の除却又はマンションの建替え等の円滑化に関する法律(平成十四年法律第七十八号)第百四十九条(権利消滅期日における権利の帰属等)の規定による同法第百五十三条(補償金)に規定する権利の消滅とする。
\end{description}
\noindent\hspace{10pt}(減額された外国所得税額のうち総収入金額に算入しないもの)
\begin{description}
\item[第九十三条の二]法第四十四条の三(減額された外国所得税額の総収入金額不算入等)に規定する政令で定める金額は、同条に規定する外国所得税の額が減額された金額のうちその減額されることとなつた日の属する年において第二百二十六条第一項(外国所得税が減額された場合の特例)の規定による同項に規定する納付控除対象外国所得税額からの控除又は同条第三項の規定による同項に規定する控除限度超過額からの控除に充てられることとなる部分の金額に相当する金額とする。
\end{description}
\noindent\hspace{10pt}(事業所得の収入金額とされる保険金等)
\begin{description}
\item[第九十四条]不動産所得、事業所得、山林所得又は雑所得を生ずべき業務を行なう居住者が受ける次に掲げるもので、その業務の遂行により生ずべきこれらの所得に係る収入金額に代わる性質を有するものは、これらの所得に係る収入金額とする。
\begin{description}
\item[一]当該業務に係るたな卸資産(第八十一条各号(譲渡所得の基因とされないたな卸資産に準ずる資産)に掲げる資産を含む。)、山林、工業所有権その他の技術に関する権利、特別の技術による生産方式若しくはこれらに準ずるもの又は著作権(出版権及び著作隣接権その他これに準ずるものを含む。)につき損失を受けたことにより取得する保険金、損害賠償金、見舞金その他これらに類するもの(山林につき法第五十一条第三項(山林損失の必要経費算入)の規定に該当する損失を受けたことにより取得するものについては、その損失の金額をこえる場合におけるそのこえる金額に相当する部分に限る。)
\item[二]当該業務の全部又は一部の休止、転換又は廃止その他の事由により当該業務の収益の補償として取得する補償金その他これに類するもの
\end{description}
\item[\rensuji{2}]第七十九条第一項(資産の譲渡とみなされる行為)の規定に該当する同項の行為に係る対価で法第三十三条第二項第一号(譲渡所得)の規定により譲渡所得の収入金額に含まれないものは、事業所得又は雑所得に係る収入金額とし、当該対価につき第百七十四条から第百七十七条まで(借地権の設定をした場合の譲渡所得に係る取得費等)の規定に準じて計算した金額は、当該事業所得又は雑所得に係る必要経費に算入する。
\end{description}
\noindent\hspace{10pt}(譲渡所得の収入金額とされる補償金等)
\begin{description}
\item[第九十五条]契約(契約が成立しない場合に法令によりこれに代わる効果を認められる行政処分その他の行為を含む。)に基づき、又は資産の消滅(価値の減少を含む。以下この条において同じ。)を伴う事業でその消滅に対する補償を約して行なうものの遂行により譲渡所得の基因となるべき資産が消滅をしたこと(借地権の設定その他当該資産について物権を設定し又は債権が成立することにより価値が減少したことを除く。)に伴い、その消滅につき一時に受ける補償金その他これに類するものの額は、譲渡所得に係る収入金額とする。
\end{description}
\subsection*{第四節 必要経費等の計算}
\addcontentsline{toc}{subsection}{第四節 必要経費等の計算}
\subsubsection*{第一款 必要経費に算入されないもの}
\addcontentsline{toc}{subsubsection}{第一款 必要経費に算入されないもの}
\noindent\hspace{10pt}(家事関連費)
\begin{description}
\item[第九十六条]法第四十五条第一項第一号(必要経費とされない家事関連費)に規定する政令で定める経費は、次に掲げる経費以外の経費とする。
\begin{description}
\item[一]家事上の経費に関連する経費の主たる部分が不動産所得、事業所得、山林所得又は雑所得を生ずべき業務の遂行上必要であり、かつ、その必要である部分を明らかに区分することができる場合における当該部分に相当する経費
\item[二]前号に掲げるもののほか、青色申告書を提出することにつき税務署長の承認を受けている居住者に係る家事上の経費に関連する経費のうち、取引の記録等に基づいて、不動産所得、事業所得又は山林所得を生ずべき業務の遂行上直接必要であつたことが明らかにされる部分の金額に相当する経費
\end{description}
\end{description}
\noindent\hspace{10pt}(必要経費に算入される利子税の計算)
\begin{description}
\item[第九十七条]法第四十五条第一項第二号(必要経費とされない所得税)に規定する政令で定める利子税は、次の各号に掲げる利子税の区分に応じ当該各号に定める金額に相当する利子税とする。
\begin{description}
\item[一]法第四十五条第一項第二号に規定する事業を行う居住者が納付した法第百三十一条第三項(確定申告税額の延納に係る利子税)の規定による利子税
\item[二]山林所得を生ずべき事業を行う居住者が納付した法第百三十六条(延払条件付譲渡に係る所得税額の延納に係る利子税)の規定による利子税で当該事業から生じた山林所得に係るもの
\item[三]事業所得を生ずべき事業を行う居住者が納付した法第百三十七条の二第十二項(国外転出をする場合の譲渡所得等の特例の適用がある場合の納税猶予に係る利子税)の規定による利子税
\begin{description}
\item[イ]法第六十条の二第一項から第三項までの規定により行われたものとみなされた有価証券等(同条第一項に規定する有価証券等をいう。以下この項において同じ。)の譲渡による事業所得の金額、譲渡所得の金額及び雑所得の金額、未決済信用取引等(同条第二項に規定する未決済信用取引等をいう。以下この項において同じ。)の決済による事業所得の金額及び雑所得の金額並びに未決済デリバティブ取引(同条第三項に規定する未決済デリバティブ取引をいう。以下この項において同じ。)の決済による事業所得の金額及び雑所得の金額の合計額
\item[ロ]法第六十条の二第一項から第三項までの規定により行われたものとみなされた有価証券等の譲渡による事業所得の金額、未決済信用取引等の決済による事業所得の金額及び未決済デリバティブ取引の決済による事業所得の金額の合計額
\end{description}
\item[四]事業所得を生ずべき事業を行う居住者が納付した法第百三十七条の三第十四項(贈与等により非居住者に資産が移転した場合の譲渡所得等の特例の適用がある場合の納税猶予に係る利子税)の規定による利子税
\begin{description}
\item[イ]法第六十条の三第一項から第三項までの規定により行われたものとみなされた有価証券等の譲渡による事業所得の金額、譲渡所得の金額及び雑所得の金額、未決済信用取引等の決済による事業所得の金額及び雑所得の金額並びに未決済デリバティブ取引の決済による事業所得の金額及び雑所得の金額の合計額
\item[ロ]法第六十条の三第一項から第三項までの規定により行われたものとみなされた有価証券等の譲渡による事業所得の金額、未決済信用取引等の決済による事業所得の金額及び未決済デリバティブ取引の決済による事業所得の金額の合計額
\end{description}
\end{description}
\item[\rensuji{2}]前項第一号に規定する各種所得の金額の合計額並びに不動産所得の金額、事業所得の金額及び山林所得の金額の合計額は、同号に規定する年分の確定申告書に記載されたところによる。
\item[\rensuji{3}]第一項に規定する割合は、小数点以下二位まで算出し、三位以下を切り上げたところによる。
\end{description}
\noindent\hspace{10pt}(必要経費に算入されない損害賠償金の範囲)
\begin{description}
\item[第九十八条]法第四十五条第一項第七号(必要経費とされない損害賠償金)に規定する政令で定める損害賠償金(これに類するものを含む。)は、同項第一号に掲げる経費に該当する損害賠償金(これに類するものを含む。以下この条において同じ。)のほか、不動産所得、事業所得、山林所得又は雑所得を生ずべき業務に関連して、故意又は重大な過失によつて他人の権利を侵害したことにより支払う損害賠償金とする。
\end{description}
\subsubsection*{第二款 棚卸資産の評価}
\addcontentsline{toc}{subsubsection}{第二款 棚卸資産の評価}
\subsubsubsection*{第一目 棚卸資産の評価の方法}
{第一目 棚卸資産の評価の方法}
\noindent\hspace{10pt}(棚卸資産の評価の方法)
\begin{description}
\item[第九十九条]法第四十七条第一項(棚卸資産の売上原価等の計算及びその評価の方法)の規定によるその年十二月三十一日(同項の居住者が年の中途において死亡し又は出国をした場合には、その死亡又は出国の時。以下この款において同じ。)において有する棚卸資産の評価額の計算上選定をすることができる同項に規定する政令で定める評価の方法は、次に掲げる方法(その年分の所得税について青色申告書を提出することにつき税務署長の承認を受けていない場合には、第一号に掲げる方法)とする。
\begin{description}
\item[一]原価法(その年十二月三十一日において有する棚卸資産(以下この項において「期末棚卸資産」という。)につき次に掲げる方法のうちいずれかの方法によつてその取得価額を算出し、その算出した取得価額をもつて当該期末棚卸資産の評価額とする方法をいう。)
\begin{description}
\item[イ]個別法(期末棚卸資産の全部について、その個々の取得価額をその取得価額とする方法をいう。)
\item[ロ]先入先出法(期末棚卸資産をその種類、品質及び型(以下この項において「種類等」という。)の異なるごとに区別し、その種類等の同じものについて、当該期末棚卸資産をその年十二月三十一日から最も近い日において取得した種類等を同じくする棚卸資産から順次成るものとみなし、そのみなされた棚卸資産の取得価額をその取得価額とする方法をいう。)
\item[ハ]総平均法(棚卸資産をその種類等の異なるごとに区別し、その種類等の同じものについて、その年一月一日において有していた種類等を同じくする棚卸資産の取得価額の総額とその年中に取得した種類等を同じくする棚卸資産の取得価額の総額との合計額をこれらの棚卸資産の総数量で除して計算した価額をその一単位当たりの取得価額とする方法をいう。)
\item[ニ]移動平均法(棚卸資産をその種類等の異なるごとに区別し、その種類等の同じものについて、当初の一単位当たりの取得価額が、再び種類等を同じくする棚卸資産を取得した場合にはその取得の時において有する当該棚卸資産とその取得した棚卸資産との数量及び取得価額を基礎として算出した平均単価によつて改定されたものとみなし、以後種類等を同じくする棚卸資産を取得する都度同様の方法により一単位当たりの取得価額が改定されたものとみなし、その年十二月三十一日から最も近い日において改定されたものとみなされた一単位当たりの取得価額をその一単位当たりの取得価額とする方法をいう。)
\item[ホ]最終仕入原価法(期末棚卸資産をその種類等の異なるごとに区別し、その種類等の同じものについて、その年十二月三十一日から最も近い日において取得したものの一単位当たりの取得価額をその一単位当たりの取得価額とする方法をいう。)
\item[ヘ]売価還元法(期末棚卸資産をその種類等又は通常の差益の率(棚卸資産の通常の販売価額のうちに当該通常の販売価額から当該棚卸資産を取得するために通常要する価額を控除した金額の占める割合をいう。以下この項において同じ。)の異なるごとに区別し、その種類等又は通常の差益の率の同じものについて、その年十二月三十一日における種類等又は通常の差益の率を同じくする棚卸資産の通常の販売価額の総額に原価の率(当該通常の販売価額の総額とその年中に販売した当該棚卸資産の対価の総額との合計額のうちにその年一月一日における当該棚卸資産の取得価額の総額とその年中に取得した当該棚卸資産の取得価額の総額との合計額の占める割合をいう。)を乗じて計算した金額をその取得価額とする方法をいう。)
\end{description}
\item[二]低価法(期末棚卸資産をその種類等(前号ヘに掲げる売価還元法により算出した取得価額による原価法により計算した価額を基礎とするものにあつては、種類等又は通常の差益の率。以下この号において同じ。)の異なるごとに区別し、その種類等の同じものについて、前号に掲げる方法のうちいずれかの方法により算出した取得価額による原価法により評価した価額とその年十二月三十一日における価額とのうちいずれか低い価額をもつてその評価額とする方法をいう。)
\end{description}
\item[\rensuji{2}]前項第一号イに掲げる個別法により算出した取得価額による原価法(当該原価法により評価した価額を基礎とする同項第二号に掲げる低価法を含む。)は、棚卸資産のうち通常一の取引によつて大量に取得され、かつ、規格に応じて価格が定められているものについては、同項の規定にかかわらず、選定することができない。
\end{description}
\noindent\hspace{10pt}(棚卸資産の特別な評価の方法)
\begin{description}
\item[第九十九条の二]居住者は、その有する棚卸資産の評価額を前条第一項に規定する評価の方法に代え当該評価の方法以外の評価の方法により計算することについて納税地の所轄税務署長の承認を受けた場合には、当該資産のその承認を受けた日の属する年分以後の各年分の評価額の計算については、その承認を受けた評価の方法を選定することができる。
\item[\rensuji{2}]前項の承認を受けようとする居住者は、その採用しようとする評価の方法の内容、その方法を採用しようとする理由、その方法により評価額の計算をしようとする次条第一項に規定する事業の種類及び資産の区分その他財務省令で定める事項を記載した申請書を納税地の所轄税務署長に提出しなければならない。
\item[\rensuji{3}]税務署長は、前項の申請書の提出があつた場合には、遅滞なく、これを審査し、その申請に係る評価の方法並びに次条第一項に規定する事業の種類及び資産の区分を承認し、又はその申請に係る評価の方法によつてはその居住者の各年分の事業所得の金額の計算が適正に行われ難いと認めるときは、その申請を却下する。
\item[\rensuji{4}]税務署長は、第一項の承認をした後、その承認に係る評価の方法によりその承認に係る棚卸資産の評価額の計算をすることを不適当とする特別の事情が生じたと認める場合には、その承認を取り消すことができる。
\item[\rensuji{5}]税務署長は、前二項の処分をするときは、その処分に係る居住者に対し、書面によりその旨を通知する。
\item[\rensuji{6}]第四項の処分があつた場合には、その処分のあつた日の属する年分以後の各年分の事業所得の金額を計算する場合のその処分に係る棚卸資産の評価額の計算についてその処分の効果が生ずるものとする。
\item[\rensuji{7}]居住者は、第四項の処分を受けた場合には、その処分を受けた日の属する年分の所得税に係る確定申告期限までに、その処分に係る棚卸資産につき、次条第一項に規定する事業の種類及び資産の区分ごとに、前条第一項に規定する評価の方法のうちそのよるべき方法を書面により納税地の所轄税務署長に届け出なければならない。
\end{description}
\noindent\hspace{10pt}(棚卸資産の評価の方法の選定)
\begin{description}
\item[第百条]第九十九条第一項(棚卸資産の評価の方法)に規定する棚卸資産の評価の方法は、居住者の営む事業の種類ごとに、かつ、商品又は製品(副産物及び作業くずを除く。)、半製品、仕掛品(半成工事を含む。)、主要原材料及び補助原材料その他の棚卸資産の区分ごとに選定しなければならない。
\item[\rensuji{2}]居住者は、次の各号に掲げる者の区分に応じ当該各号に掲げる日の属する年分の所得税に係る確定申告期限までに、棚卸資産につき、前項に規定する事業の種類及び資産の区分ごとに、第九十九条第一項に規定する評価の方法のうちそのよるべき方法を書面により納税地の所轄税務署長に届け出なければならない。
\begin{description}
\item[一]新たに事業所得を生ずべき事業を開始した居住者
\item[二]前号の事業を開始した後新たに他の種類の事業を開始し又は事業の種類を変更した居住者
\end{description}
\end{description}
\noindent\hspace{10pt}(棚卸資産の評価の方法の変更手続)
\begin{description}
\item[第百一条]居住者は、棚卸資産につき選定した評価の方法(その評価の方法を届け出なかつた者がよるべきこととされている次条第一項に規定する評価の方法を含む。)を変更しようとするときは、納税地の所轄税務署長の承認を受けなければならない。
\item[\rensuji{2}]前項の承認を受けようとする居住者は、その新たな評価の方法を採用しようとする年の三月十五日までに、その旨、変更しようとする理由その他財務省令で定める事項を記載した申請書を納税地の所轄税務署長に提出しなければならない。
\item[\rensuji{3}]税務署長は、前項の申請書の提出があつた場合において、その申請書を提出した居住者が現によつている評価の方法を採用してから相当期間を経過していないとき、又は変更しようとする評価の方法によつてはその者の各年分の事業所得の金額の計算が適正に行われ難いと認めるときは、その申請を却下することができる。
\item[\rensuji{4}]税務署長は、第二項の申請書の提出があつた場合において、その申請につき承認又は却下の処分をするときは、その申請をした居住者に対し、書面によりその旨を通知する。
\item[\rensuji{5}]第二項の申請書の提出があつた場合において、その年十二月三十一日までにその申請につき承認又は却下の処分がなかつたときは、同日においてその承認があつたものとみなす。
\end{description}
\noindent\hspace{10pt}(棚卸資産の法定評価方法)
\begin{description}
\item[第百二条]法第四十七条第一項(棚卸資産の売上原価等の計算及びその評価の方法)に規定する評価の方法を選定しなかつた場合又は選定した方法により評価しなかつた場合における政令で定める方法は、第九十九条第一項第一号ホ(棚卸資産の評価の方法)に掲げる最終仕入原価法により算出した取得価額による原価法とする。
\item[\rensuji{2}]税務署長は、居住者が棚卸資産につき選定した評価の方法(評価の方法を届け出なかつた居住者がよるべきこととされている前項に規定する評価の方法を含む。)により評価しなかつた場合において、その居住者が行つた評価の方法が第九十九条第一項に規定する評価の方法のうちいずれかの方法に該当し、かつ、その行つた評価の方法によつてもその居住者の各年分の事業所得の金額の計算を適正に行うことができると認めるときは、その行つた評価の方法により計算した各年分の事業所得の金額を基礎として更正又は決定をすることができる。
\end{description}
\subsubsubsection*{第二目 棚卸資産の取得価額}
{第二目 棚卸資産の取得価額}
\noindent\hspace{10pt}(棚卸資産の取得価額)
\begin{description}
\item[第百三条]第九十九条第一項(棚卸資産の評価の方法)又は第九十九条の二第一項(棚卸資産の特別な評価の方法)の規定による棚卸資産の評価額の計算の基礎となる棚卸資産の取得価額は、別段の定めがあるものを除き、次の各号に掲げる資産の区分に応じ当該各号に掲げる金額とする。
\begin{description}
\item[一]購入した棚卸資産
\begin{description}
\item[イ]当該資産の購入の代価(引取運賃、荷役費、運送保険料、購入手数料、関税(関税法(昭和二十九年法律第六十一号)第二条第一項第四号の二(定義)に規定する附帯税を除く。)その他当該資産の購入のために要した費用がある場合には、その費用の額を加算した金額)
\item[ロ]当該資産を消費し又は販売の用に供するために直接要した費用の額
\end{description}
\item[二]自己の製造、採掘、採取、栽培、養殖その他これらに準ずる行為(以下この条において「製造等」という。)に係る棚卸資産
\begin{description}
\item[イ]当該資産の製造等のために要した原材料費、労務費及び経費の額
\item[ロ]当該資産を消費し又は販売の用に供するために直接要した費用の額
\end{description}
\item[三]前二号に規定する方法以外の方法により取得した棚卸資産
\begin{description}
\item[イ]その取得の時における当該資産の取得のために通常要する価額
\item[ロ]当該資産を消費し又は販売の用に供するために直接要した費用の額
\end{description}
\end{description}
\item[\rensuji{2}]次の各号に掲げる棚卸資産の前項に規定する取得価額は、当該各号に掲げる金額とする。
\begin{description}
\item[一]贈与、相続又は遺贈により取得した棚卸資産(法第四十条第一項第一号(棚卸資産の贈与等の場合の総収入金額算入)に掲げる贈与又は遺贈により取得したものを除く。)
\item[二]法第四十条第一項第二号に掲げる譲渡により取得した棚卸資産
\end{description}
\item[\rensuji{3}]法第四十一条第二項(農産物の収穫の場合の総収入金額算入)の規定により取得したものとみなされる同項に規定する農産物の第一項に規定する取得価額は、同条第二項に規定する収穫価額に当該農産物を消費し又は販売の用に供するために直接要した費用の額を加算した金額とする。
\end{description}
\noindent\hspace{10pt}(棚卸資産の取得価額の特例)
\begin{description}
\item[第百四条]居住者の有する棚卸資産につき次に掲げる事実が生じた場合には、その事実の生じた日の属する年以後の各年における当該資産の第九十九条第一項(棚卸資産の評価の方法)又は第九十九条の二第一項(棚卸資産の特別な評価の方法)の規定による評価額の計算については、その年十二月三十一日における当該資産の価額をもつて、前条第一項に規定する取得価額とすることができる。
\begin{description}
\item[一]当該資産が災害により著しく損傷したこと。
\item[二]当該資産が著しく陳腐化したこと。
\item[三]前二号に準ずる特別の事実
\end{description}
\end{description}
\subsubsection*{第三款 有価証券の評価}
\addcontentsline{toc}{subsubsection}{第三款 有価証券の評価}
\subsubsubsection*{第一目 有価証券の評価の方法}
{第一目 有価証券の評価の方法}
\noindent\hspace{10pt}(有価証券の評価の方法)
\begin{description}
\item[第百五条]法第四十八条第一項(有価証券の譲渡原価等の計算及びその評価の方法)の規定によるその年十二月三十一日(同項の居住者が年の中途において死亡し又は出国をした場合には、その死亡又は出国の時。以下この条において同じ。)において有する有価証券(以下この項において「期末有価証券」という。)の評価額の計算上選定をすることができる評価の方法は、期末有価証券につき次に掲げる方法のうちいずれかの方法によつてその取得価額を算出し、その算出した取得価額をもつて当該期末有価証券の評価額とする方法とする。
\begin{description}
\item[一]総平均法(有価証券をその種類及び銘柄(以下この項において「種類等」という。)の異なるごとに区別し、その種類等の同じものについて、その年一月一日において有していた種類等を同じくする有価証券の取得価額の総額とその年中に取得した種類等を同じくする有価証券の取得価額の総額との合計額をこれらの有価証券の総数で除して計算した価額をその一単位当たりの取得価額とする方法をいう。)
\item[二]移動平均法(有価証券をその種類等の異なるごとに区別し、その種類等の同じものについて、当初の一単位当たりの取得価額が、種類等を同じくする有価証券を再び取得した場合にはその取得の時において有する当該有価証券とその取得した有価証券との数及び取得価額を基礎として算出した平均単価によつて改定されたものとみなし、以後種類等を同じくする有価証券を取得する都度同様の方法により一単位当たりの取得価額が改定されたものとみなし、その年十二月三十一日から最も近い日において改定されたものとみなされた一単位当たりの取得価額をその一単位当たりの取得価額とする方法をいう。)
\end{description}
\item[\rensuji{2}]居住者の有する株式(出資及び投資信託及び投資法人に関する法律第二条第十四項(定義)に規定する投資口を含む。)又は投資信託若しくは特定受益証券発行信託の受益権について、その年の中途において第百十条から第百十六条まで(株式の分割等の場合の株式等の取得価額)に規定する事実(以下この項において「事実」という。)があつた場合には、当該事実(その年中に二回以上にわたつて事実があつた場合には、その年十二月三十一日から最も近い日における事実)があつた日をその年一月一日とみなして、その年以後の各年における前項の規定による当該株式又は受益権の評価額の計算をするものとする。
\end{description}
\noindent\hspace{10pt}(有価証券の評価の方法の選定)
\begin{description}
\item[第百六条]有価証券の評価の方法は、その種類ごとに選定しなければならない。
\item[\rensuji{2}]居住者は、事業所得の基因となる有価証券を取得した場合(その取得した日の属する年の前年以前においてその有価証券と種類を同じくする有価証券で事業所得の基因となるものにつきこの項の規定による届出をすべき場合を除く。)には、同日の属する年分の所得税に係る確定申告期限までに、その有価証券と種類を同じくする有価証券につき、前条第一項に規定する評価の方法のうちそのよるべき方法を書面により納税地の所轄税務署長に届け出なければならない。
\end{description}
\noindent\hspace{10pt}(有価証券の評価の方法の変更手続)
\begin{description}
\item[第百七条]居住者は、有価証券につき選定した評価の方法(その評価の方法を届け出なかつた者がよるべきこととされている次条第一項に規定する評価の方法を含む。)を変更しようとするときは、納税地の所轄税務署長の承認を受けなければならない。
\item[\rensuji{2}]第百一条第二項から第五項まで(たな卸資産の評価の方法の変更手続)の規定は、前項の場合について準用する。
\end{description}
\noindent\hspace{10pt}(有価証券の法定評価方法)
\begin{description}
\item[第百八条]法第四十八条第一項(有価証券の譲渡原価等の計算及びその評価の方法)に規定する政令で定める方法は、第百五条第一項第一号(総平均法)に掲げる総平均法により算出した取得価額による評価の方法とする。
\item[\rensuji{2}]税務署長は、居住者が有価証券につき選定した評価の方法(その評価の方法を届け出なかつた居住者がよるべきこととされている前項に規定する評価の方法を含む。以下この項において同じ。)により評価しなかつた場合において、その居住者が行つた評価の方法がその居住者の選定した評価の方法以外の第百五条第一項に規定する評価の方法に該当し、かつ、その行つた評価の方法によつてもその居住者の各年分の事業所得の金額の計算を適正に行うことができると認めるときは、その行つた評価の方法により計算した各年分の事業所得の金額を基礎として更正又は決定をすることができる。
\end{description}
\subsubsubsection*{第二目 有価証券の取得価額}
{第二目 有価証券の取得価額}
\noindent\hspace{10pt}(有価証券の取得価額)
\begin{description}
\item[第百九条]第百五条第一項(有価証券の評価の方法)の規定による有価証券の評価額の計算の基礎となる有価証券の取得価額は、別段の定めがあるものを除き、次の各号に掲げる有価証券の区分に応じ当該各号に定める金額とする。
\begin{description}
\item[一]金銭の払込みにより取得した有価証券(第三号に該当するものを除く。)
\item[二]第八十四条第一項(譲渡制限付株式の価額等)に規定する特定譲渡制限付株式又は承継譲渡制限付株式
\item[三]発行法人から与えられた第八十四条第二項の規定に該当する場合における同項各号に掲げる権利の行使により取得した有価証券
\item[四]発行法人に対し新たな払込み又は給付を要しないで取得した当該発行法人の株式(出資及び投資口(投資信託及び投資法人に関する法律第二条第十四項に規定する投資口をいう。次条第一項において同じ。)を含む。以下この目において同じ。)又は新株予約権のうち、当該発行法人の株主等として与えられる場合(当該発行法人の他の株主等に損害を及ぼすおそれがないと認められる場合に限る。)の当該株式又は新株予約権
\item[五]購入した有価証券(第三号に該当するものを除く。)
\item[六]前各号に掲げる有価証券以外の有価証券
\end{description}
\item[\rensuji{2}]次の各号に掲げる有価証券の前項に規定する取得価額は、当該各号に定める金額とする。
\begin{description}
\item[一]贈与、相続又は遺贈により取得した有価証券(法第四十条第一項第一号(たな卸資産の贈与等の場合の総収入金額算入)に掲げる贈与又は遺贈により取得したものを除く。)
\item[二]法第四十条第一項第二号に掲げる譲渡により取得した有価証券
\end{description}
\end{description}
\noindent\hspace{10pt}(株式の分割又は併合の場合の株式等の取得価額)
\begin{description}
\item[第百十条]居住者の有する株式について、その株式(以下この項において「旧株」という。)の分割又は併合があつた場合には、その分割又は併合があつた日の属する年以後の各年における第百五条第一項(有価証券の評価の方法)の規定による分割又は併合後の所有株式(旧株を発行した法人の株式で、当該分割又は併合の直後に当該居住者が有するものをいう。以下この項において同じ。)の評価額の計算については、その計算の基礎となる分割又は併合後の所有株式の一株(出資及び投資口については、一口。以下この目において同じ。)当たりの取得価額は、旧株一株の従前の取得価額に旧株の数を乗じてこれを分割又は併合後の所有株式の数で除して計算した金額とし、かつ、その分割又は併合後の所有株式のうちに旧株が含まれているときは、その旧株は、同日において取得されたものとみなす。
\item[\rensuji{2}]居住者の有する投資信託又は特定受益証券発行信託の受益権について、その受益権(以下この項において「旧受益権」という。)の分割又は併合があつた場合には、その分割又は併合があつた日の属する年以後の各年における第百五条第一項の規定による分割又は併合後の所有受益権(旧受益権に係る投資信託又は特定受益証券発行信託の受益権で、当該分割又は併合の直後に当該居住者が有するものをいう。以下この項において同じ。)の評価額の計算については、その計算の基礎となる分割又は併合後の所有受益権の一口当たりの取得価額は、旧受益権一口の従前の取得価額に旧受益権の口数を乗じてこれを分割又は併合後の所有受益権の口数で除して計算した金額とし、かつ、その分割又は併合後の所有受益権のうちに旧受益権が含まれているときは、その旧受益権は、同日において取得されたものとみなす。
\end{description}
\noindent\hspace{10pt}(株主割当てにより取得した株式の取得価額)
\begin{description}
\item[第百十一条]居住者が、その有する株式(以下この項において「旧株」という。)について、その旧株の数に応じて割り当てられた株式を取得した場合(その取得した株式(以下この項において「新株」という。)について、金銭の払込みを要する場合に限る。)には、その払込みの期日(払込みの期間の定めがある場合には、当該払込みをした日)の属する年以後の各年における第百五条第一項(有価証券の評価の方法)の規定によるこれらの株式の評価額の計算については、その計算の基礎となる旧株及び新株の一株当たりの取得価額は、旧株一株の従前の取得価額と新株一株について払い込んだ金銭の額(その金銭の払込みによる取得のために要した費用がある場合には、その費用の額を加算した金額)に旧株一株について取得した新株の数を乗じて計算した金額との合計額を、旧株一株について取得した新株の数に一を加えた数で除して計算した金額とし、かつ、その旧株は、同日において取得されたものとみなす。
\item[\rensuji{2}]居住者が、その有する株式(以下この項において「旧株」という。)について、その旧株の数に応じてその旧株を発行した法人の株式無償割当て(法人がその法人の株主等に対して新たに払込みをさせないで自己の株式の割当てをすることをいう。以下この項において同じ。)により割り当てられた株式を取得した場合(当該旧株と同一の種類の株式を取得した場合に限る。)には、その株式無償割当てがあつた日の属する年以後の各年における第百五条第一項の規定による株式無償割当て後の所有株式(旧株を発行した法人の株式で、当該株式無償割当ての直後に当該居住者が有するものをいう。以下この項において同じ。)の評価額の計算については、その計算の基礎となる株式無償割当て後の所有株式の一株当たりの取得価額は、旧株一株の従前の取得価額に旧株の数を乗じてこれを株式無償割当て後の所有株式の数で除して計算した金額とし、かつ、その株式無償割当て後の所有株式のうちに旧株が含まれているときは、その旧株は、同日において取得されたものとみなす。
\end{description}
\noindent\hspace{10pt}(合併により取得した株式等の取得価額)
\begin{description}
\item[第百十二条]居住者が、その有する株式(以下この項において「旧株」という。)について、その旧株を発行した法人の合併(法人課税信託に係る信託の併合を含むものとし、当該合併に係る第六十一条第六項第五号(所有株式に対応する資本金等の額又は連結個別資本金等の額の計算方法等)に規定する被合併法人(次項において「被合併法人」という。)の株主等に当該合併に係る同条第六項第十号に規定する合併法人(以下この項及び次項において「合併法人」という。)の株式又は合併法人との間に当該合併法人の発行済株式若しくは出資(自己が有する自己の株式を除く。次条第一項及び第四項並びに第百十三条の二第三項(株式分配により取得した株式等の取得価額)において「発行済株式等」という。)の全部を保有する関係として財務省令で定める関係がある法人(以下この項において「合併親法人」という。)の株式のいずれか一方の株式以外の資産(当該株主等に対する株式に係る剰余金の配当、利益の配当又は剰余金の分配として交付がされた金銭その他の資産及び合併に反対する当該株主等に対するその買取請求に基づく対価として交付がされる金銭その他の資産を除く。)が交付されなかつたものに限る。)により合併法人からその合併法人の株式又は合併親法人の株式を取得した場合には、その合併のあつた日の属する年以後の各年における第百五条第一項(有価証券の評価の方法)の規定による合併法人の株式又は合併親法人の株式の評価額の計算については、その計算の基礎となるその取得した合併法人の株式(以下この項において「合併法人株式」という。)又は合併親法人の株式(以下この項において「合併親法人株式」という。)の一株当たりの取得価額は、旧株一株の従前の取得価額(法第二十五条第一項第一号(合併の場合のみなし配当)の規定により剰余金の配当、利益の配当、剰余金の分配若しくは金銭の分配として交付を受けたものとみなされる金額又はその合併法人株式若しくは合併親法人株式の取得のために要した費用の額がある場合には、当該交付を受けたものとみなされる金額及び費用の額のうち旧株一株に対応する部分の金額を加算した金額)を旧株一株について取得した合併法人株式又は合併親法人株式の数で除して計算した金額とする。
\item[\rensuji{2}]居住者の有する株式(以下この項において「所有株式」という。)について、その所有株式を発行した法人を合併法人とする合併(法人課税信託に係る信託の併合を含むものとし、法人税法施行令第四条の三第二項第一号(適格組織再編成における株式の保有関係等)に規定する無対価合併に該当するもので同項第二号ロに掲げる関係があるものに限る。以下この項において「無対価合併」という。)が行われた場合には、その無対価合併のあつた日の属する年以後の各年における第百五条第一項の規定による所有株式の評価額の計算については、その計算の基礎となる所有株式一株当たりの取得価額は、所有株式一株の従前の取得価額に、旧株(当該無対価合併に係る被合併法人の株式でその居住者が当該無対価合併の直前に有していたものをいう。以下この項において同じ。)一株の従前の取得価額(法第二十五条第一項第一号の規定により剰余金の配当、利益の配当又は剰余金の分配として交付を受けたものとみなされる金額がある場合には、当該交付を受けたものとみなされる金額のうち旧株一株に対応する部分の金額を加算した金額)にその旧株の数を乗じてこれをその所有株式の数で除して計算した金額を加算した金額とし、かつ、その所有株式は、同日において取得されたものとみなす。
\item[\rensuji{3}]居住者が、その有する投資信託又は特定受益証券発行信託(以下この項において「投資信託等」という。)の受益権(以下この項において「旧受益権」という。)について、その旧受益権に係る投資信託等の信託の併合(当該信託の併合に係る従前の投資信託等の受益者に当該併合に係る新たな信託である投資信託等(以下この項において「併合投資信託等」という。)の受益権以外の資産(信託の併合に反対する当該受益者に対するその買取請求に基づく対価として交付がされる金銭その他の資産を除く。)が交付されなかつたものに限る。)により併合投資信託等からその併合投資信託等の受益権を取得した場合には、その信託の併合のあつた日の属する年以後の各年における第百五条第一項の規定による併合投資信託等の受益権の評価額の計算については、その計算の基礎となるその取得した併合投資信託等の受益権の一口当たりの取得価額は、旧受益権一口の従前の取得価額(その併合投資信託等の受益権の取得のために要した費用の額がある場合には、当該費用の額のうち旧受益権一口に対応する部分の金額を加算した金額)を旧受益権一口について取得した併合投資信託等の受益権の口数で除して計算した金額とする。
\end{description}
\noindent\hspace{10pt}(分割型分割により取得した株式等の取得価額)
\begin{description}
\item[第百十三条]居住者が、その有する株式(以下この項において「所有株式」という。)について、その所有株式を発行した法人の法第二十四条第一項(配当所得)に規定する分割型分割(法人税法第二条第十二号の九イ(定義)に規定する分割対価資産として当該分割型分割に係る第六十一条第六項第三号(所有株式に対応する資本金等の額又は連結個別資本金等の額の計算方法等)に規定する分割承継法人(以下第四項までにおいて「分割承継法人」という。)の株式又は分割承継法人との間に当該分割承継法人の発行済株式等の全部を保有する関係として財務省令で定める関係がある法人(以下第四項までにおいて「分割承継親法人」という。)の株式のいずれか一方の株式以外の資産が交付されなかつたものに限る。以下この項において同じ。)によりその分割承継法人の株式又は分割承継親法人の株式を取得した場合には、その分割型分割のあつた日の属する年以後の各年における第百五条第一項(有価証券の評価の方法)の規定による分割承継法人の株式又は分割承継親法人の株式の評価額の計算については、その計算の基礎となるその取得した分割承継法人の株式(以下この項において「分割承継法人株式」という。)又は分割承継親法人の株式(以下この項において「分割承継親法人株式」という。)の一株当たりの取得価額は、所有株式一株の従前の取得価額に当該分割型分割に係る第六十一条第二項第二号に規定する割合を乗じて計算した金額を所有株式一株について取得した分割承継法人株式又は分割承継親法人株式の数で除して計算した金額(法第二十五条第一項第二号(分割型分割の場合のみなし配当)の規定により剰余金の配当若しくは利益の配当として交付を受けたものとみなされる金額又はその分割承継法人株式若しくは分割承継親法人株式の取得のために要した費用の額がある場合には、当該交付を受けたものとみなされる金額及び費用の額のうち分割承継法人株式又は分割承継親法人株式一株に対応する部分の金額を加算した金額)とする。
\item[\rensuji{2}]居住者の有する株式(以下この項において「所有株式」という。)について、その所有株式を発行した法人を分割承継法人とする法第二十四条第一項に規定する分割型分割(法人税法施行令第四条の三第六項第一号イ(適格組織再編成における株式の保有関係等)に規定する無対価分割に該当するもので同項第二号イ(2)に掲げる関係があるものに限る。以下この項及び次項において「無対価分割型分割」という。)が行われた場合には、その無対価分割型分割のあつた日の属する年以後の各年における第百五条第一項の規定による所有株式の評価額の計算については、その計算の基礎となる所有株式一株当たりの取得価額は、所有株式一株の従前の取得価額に、旧株(当該無対価分割型分割に係る第六十一条第六項第六号に規定する分割法人(次項及び第四項において「分割法人」という。)の株式でその居住者が当該無対価分割型分割の直前に有していたものをいう。以下この項において同じ。)一株の従前の取得価額に当該無対価分割型分割に係る第六十一条第二項第二号に規定する割合を乗じて計算した金額にその旧株の数を乗じてこれをその所有株式の数で除して計算した金額(法第二十五条第一項第二号の規定により剰余金の配当又は利益の配当として交付を受けたものとみなされる金額がある場合には、当該交付を受けたものとみなされる金額のうち所有株式一株に対応する部分の金額を加算した金額)を加算した金額とし、かつ、その所有株式は、同日において取得されたものとみなす。
\item[\rensuji{3}]居住者の有する株式(以下この項において「所有株式」という。)を発行した法人の法第二十四条第一項に規定する分割型分割によりその居住者が分割承継法人の株式、分割承継親法人の株式その他の資産の交付を受けた場合又は所有株式を発行した法人を分割法人とする無対価分割型分割が行われた場合には、その分割型分割又は無対価分割型分割のあつた日の属する年以後の各年における第百五条第一項の規定による所有株式の評価額の計算については、その計算の基礎となる所有株式一株当たりの取得価額は、所有株式一株の従前の取得価額から所有株式一株の従前の取得価額に当該分割型分割又は無対価分割型分割に係る第六十一条第二項第二号に規定する割合を乗じて計算した金額を控除した金額とし、かつ、その所有株式は、同日において取得されたものとみなす。
\item[\rensuji{4}]第一項に規定する分割型分割に係る分割承継法人の株式又は分割承継親法人の株式が当該分割型分割に係る分割法人の発行済株式等の総数又は総額のうちに占める当該分割法人の各株主等の有する当該分割法人の株式の数又は金額の割合に応じて交付されない場合には、当該分割型分割は、同項に規定する分割型分割に該当しないものとする。
\item[\rensuji{5}]第三項に規定する所有株式を発行した法人は、法第二十四条第一項に規定する分割型分割を行つた場合には、当該所有株式を有していた個人に対し、当該分割型分割に係る第三項に規定する割合を通知しなければならない。
\item[\rensuji{6}]居住者が、その有する特定受益証券発行信託の受益権(以下この項において「旧受益権」という。)について、その旧受益権に係る特定受益証券発行信託の信託の分割(当該信託の分割に係る分割信託(信託の分割によりその信託財産の一部を受託者を同一とする他の信託又は新たな信託の信託財産として移転する信託をいう。以下この項及び第八項において同じ。)の受益者に当該信託の分割に係る承継信託(信託の分割により受託者を同一とする他の信託からその信託財産の一部の移転を受ける信託をいう。以下第八項までにおいて同じ。)の受益権以外の資産(信託の分割に反対する当該受益者に対する信託法第百三条第六項(受益権取得請求)に規定する受益権取得請求に基づく対価として交付される金銭その他の資産を除く。)が交付されなかつたものに限る。以下この項において同じ。)によりその承継信託の受益権を取得した場合には、その信託の分割のあつた日の属する年以後の各年における第百五条第一項の規定による承継信託の受益権の評価額の計算については、その計算の基礎となるその取得した承継信託の受益権(以下この項において「承継信託受益権」という。)の一口当たりの取得価額は、旧受益権一口の従前の取得価額に第一号に掲げる金額のうちに第二号に掲げる金額の占める割合を乗じて計算した金額を旧受益権一口について取得した承継信託受益権の口数で除して計算した金額(その承継信託受益権の取得のために要した費用の額がある場合には、当該費用の額のうち承継信託受益権一口に対応する部分の金額を加算した金額)とする。
\begin{description}
\item[一]当該信託の分割に係る分割信託の当該信託の分割前に終了した計算期間のうち最も新しいものの終了の時の資産の価額として当該分割信託の貸借対照表に記載された金額の合計額からその時の負債の価額として当該貸借対照表に記載された金額の合計額を控除した金額
\item[二]当該信託の分割に係る承継信託が当該信託の分割により移転を受けた資産の価額として当該承継信託の帳簿に記載された金額の合計額から当該信託の分割により移転を受けた負債の価額として当該帳簿に記載された金額の合計額を控除した金額(当該金額が前号に掲げる金額を超える場合には、同号に掲げる金額)
\end{description}
\item[\rensuji{7}]居住者が、その有する特定受益証券発行信託の受益権(以下この項において「旧受益権」という。)に係る特定受益証券発行信託の信託の分割により承継信託の受益権その他の資産の交付を受けた場合には、その信託の分割のあつた日の属する年以後の各年における第百五条第一項の規定による旧受益権の評価額の計算については、その計算の基礎となる旧受益権一口当たりの取得価額は、旧受益権一口の従前の取得価額から旧受益権一口の従前の取得価額に当該信託の分割に係る前項に規定する割合を乗じて計算した金額を控除した金額とし、かつ、その旧受益権は、同日において取得されたものとみなす。
\item[\rensuji{8}]第六項に規定する信託の分割に係る承継信託の受益権が当該信託の分割に係る分割信託の受益者の有する当該分割信託の受益権の口数又は価額の割合に応じて交付されない場合には、当該信託の分割は、同項に規定する信託の分割に該当しないものとする。
\item[\rensuji{9}]第七項に規定する旧受益権に係る特定受益証券発行信託の受託者は、信託の分割を行つた場合には、当該旧受益権を有していた個人に対し、当該信託の分割に係る同項に規定する割合を通知しなければならない。
\end{description}
\noindent\hspace{10pt}(株式分配により取得した株式等の取得価額)
\begin{description}
\item[第百十三条の二]居住者が、その有する株式(以下この項において「所有株式」という。)について、その所有株式を発行した法人の行つた法第二十四条第一項(配当所得)に規定する株式分配(法人税法第二条第十二号の十五の二(定義)に規定する完全子法人(以下第三項までにおいて「完全子法人」という。)の株式以外の資産が交付されなかつたものに限る。以下この項において同じ。)によりその完全子法人の株式を取得した場合には、その株式分配のあつた日の属する年以後の各年における第百五条第一項(有価証券の評価の方法)の規定による完全子法人の株式の評価額の計算については、その計算の基礎となるその取得した完全子法人の株式(以下この項において「完全子法人株式」という。)の一株当たりの取得価額は、所有株式一株の従前の取得価額に当該株式分配に係る第六十一条第二項第三号(所有株式に対応する資本金等の額又は連結個別資本金等の額の計算方法等)に規定する割合を乗じて計算した金額を所有株式一株について取得した完全子法人株式の数で除して計算した金額(法第二十五条第一項第三号(株式分配の場合のみなし配当)の規定により剰余金の配当若しくは利益の配当として交付を受けたものとみなされる金額又はその完全子法人株式の取得のために要した費用の額がある場合には、当該交付を受けたものとみなされる金額及び費用の額のうち完全子法人株式一株に対応する部分の金額を加算した金額)とする。
\item[\rensuji{2}]居住者の有する株式(以下この項において「所有株式」という。)を発行した法人の行つた法第二十四条第一項に規定する株式分配によりその居住者が完全子法人の株式その他の資産の交付を受けた場合には、その株式分配のあつた日の属する年以後の各年における第百五条第一項の規定による所有株式の評価額の計算については、その計算の基礎となる所有株式一株当たりの取得価額は、所有株式一株の従前の取得価額から所有株式一株の従前の取得価額に当該株式分配に係る第六十一条第二項第三号に規定する割合を乗じて計算した金額を控除した金額とし、かつ、その所有株式は、同日において取得されたものとみなす。
\item[\rensuji{3}]第一項に規定する株式分配に係る完全子法人の株式が当該株式分配に係る第六十一条第六項第九号に規定する現物分配法人の発行済株式等の総数又は総額のうちに占める当該現物分配法人の各株主等の有する当該現物分配法人の株式の数又は金額の割合に応じて交付されない場合には、当該株式分配は、第一項に規定する株式分配に該当しないものとする。
\item[\rensuji{4}]第二項に規定する所有株式を発行した法人は、法第二十四条第一項に規定する株式分配を行つた場合には、当該所有株式を有していた個人に対し、当該株式分配に係る第二項に規定する割合を通知しなければならない。
\end{description}
\noindent\hspace{10pt}(資本の払戻し等があつた場合の株式等の取得価額)
\begin{description}
\item[第百十四条]居住者が、その有する株式(以下この項において「旧株」という。)を発行した法人の資本の払戻し(法第二十五条第一項第四号(配当等とみなす金額)に規定する資本の払戻しをいう。)又は解散による残余財産の分配(以下この項において「払戻し等」という。)として金銭その他の資産を取得した場合には、その払戻し等のあつた日の属する年以後の各年における第百五条第一項(有価証券の評価の方法)の規定による旧株の評価額の計算については、その計算の基礎となる旧株一株当たりの取得価額は、旧株一株の従前の取得価額から旧株一株の従前の取得価額に当該払戻し等に係る第六十一条第二項第四号(所有株式に対応する資本金等の額又は連結個別資本金等の額の計算方法等)に規定する割合(当該払戻し等が法第二十四条第一項(配当所得)に規定する出資等減少分配である場合には、当該出資等減少分配に係る第六十一条第二項第五号に規定する割合。第五項において「払戻し等割合」という。)を乗じて計算した金額を控除した金額とし、かつ、その旧株は、同日において取得されたものとみなす。
\item[\rensuji{2}]居住者が、その有する法人の出資(口数の定めがないものに限る。以下この項において「所有出資」という。)につき当該法人の出資の払戻し(以下この項において「払戻し」という。)として金銭その他の資産を取得した場合には、その払戻しのあつた日の属する年以後の各年における第百五条第一項の規定による所有出資の評価額の計算については、その計算の基礎となる所有出資一単位当たりの取得価額は、所有出資一単位の従前の取得価額から所有出資一単位の従前の取得価額に当該払戻しの直前の当該所有出資の金額のうちに当該払戻しに係る出資の金額の占める割合を乗じて計算した金額を控除した金額とし、かつ、当該払戻し後の所有出資は、同日において取得されたものとみなす。
\item[\rensuji{3}]居住者が、その有するオープン型の証券投資信託の受益権(以下この項において「旧受益権」という。)につきその収益の分配を受けた場合(当該オープン型の証券投資信託の終了又は当該オープン型の証券投資信託の一部の解約により支払を受ける場合を除くものとし、その収益の分配のうちに第二十七条(オープン型の証券投資信託の収益の分配のうち非課税とされるもの)に規定する特別分配金が含まれている場合に限る。)には、その収益の分配のあつた日の属する年以後の各年における第百五条第一項の規定による旧受益権の評価額の計算については、その計算の基礎となる旧受益権一口当たりの取得価額は、旧受益権一口の従前の取得価額にその収益の分配の直前においてその居住者の有する旧受益権の数を乗じて計算した金額から当該特別分配金として分配される金額を控除した金額を当該旧受益権の数で除して計算した金額とし、かつ、その旧受益権は、同日において取得されたものとみなす。
\item[\rensuji{4}]居住者が、その有する投資信託又は特定受益証券発行信託の受益権(以下この項において「旧受益権」という。)の一部につき当該旧受益権に係る投資信託又は特定受益証券発行信託の一部の解約をした場合には、その一部の解約のあつた日の属する年以後の各年における第百五条第一項の規定による旧受益権の評価額の計算については、その計算の基礎となる旧受益権一口当たりの取得価額は、旧受益権一口の従前の取得価額とし、かつ、その旧受益権は、同日において取得されたものとみなす。
\item[\rensuji{5}]第一項に規定する旧株を発行した法人は、同項に規定する払戻し等を行つた場合には、当該払戻し等を受けた個人に対し、当該払戻し等に係る払戻し等割合を通知しなければならない。
\end{description}
\noindent\hspace{10pt}(組織変更があつた場合の株式等の取得価額)
\begin{description}
\item[第百十五条]居住者が、その有する株式(以下この条において「旧株」という。)を発行した法人の組織変更(当該組織変更をした法人(以下この条において「組織変更法人」という。)の株主等に当該組織変更法人の株式のみが交付されたものに限る。)により組織変更法人の株式(以下この条において「新株」という。)を取得した場合には、その組織変更のあつた日の属する年以後の各年における第百五条第一項(有価証券の評価の方法)の規定による新株の評価額の計算については、その計算の基礎となるその取得した新株一単位当たりの取得価額は、旧株一単位の従前の取得価額(その新株の取得のために要した費用の額がある場合には、当該費用の額のうち旧株一単位に対応する部分の金額を加算した金額)に旧株の数を乗じてこれを取得した新株の数で除して計算した金額とする。
\end{description}
\noindent\hspace{10pt}(合併等があつた場合の新株予約権等の取得価額)
\begin{description}
\item[第百十六条]居住者が、その有する新株予約権又は新株予約権付社債(以下この条において「旧新株予約権等」という。)を発行した法人を被合併法人(法人税法第二条第十一号(定義)に規定する被合併法人をいう。)、分割法人(同条第十二号の二に規定する分割法人をいう。)、株式交換完全子法人(同条第十二号の六に規定する株式交換完全子法人をいう。)又は株式移転完全子法人(同条第十二号の六の五に規定する株式移転完全子法人をいう。)とする合併、分割、株式交換又は株式移転(以下この条において「合併等」という。)により当該旧新株予約権等に代えて当該合併等に係る合併法人(同法第二条第十二号に規定する合併法人をいう。)、分割承継法人(同条第十二号の三に規定する分割承継法人をいう。)、株式交換完全親法人(同条第十二号の六の三に規定する株式交換完全親法人をいう。)又は株式移転完全親法人(同条第十二号の六の六に規定する株式移転完全親法人をいう。)の新株予約権又は新株予約権付社債(以下この条において「合併法人等新株予約権等」という。)のみの交付を受けた場合には、その合併等のあつた日の属する年以後の各年における第百五条第一項(有価証券の評価の方法)の規定による合併法人等新株予約権等の評価額の計算については、その計算の基礎となるその取得した合併法人等新株予約権等一単位当たりの取得価額は、旧新株予約権等一単位の従前の取得価額(その合併法人等新株予約権等の取得のために要した費用の額がある場合には、当該費用の額のうち旧新株予約権等一単位に対応する部分の金額を加算した金額)を旧新株予約権等一単位について取得した合併法人等新株予約権等の数で除して計算した金額とする。
\end{description}
\noindent\hspace{10pt}(旧株一株の従前の取得価額等)
\begin{description}
\item[第百十七条]居住者の有する株式又は投資信託若しくは特定受益証券発行信託の受益権について、その年の中途において第百十条から前条までに規定する事実(以下この条において「事実」という。)があつた場合には、これらの規定の適用については、その年一月一日(同日から当該事実があつた日までの間に他の事実があつた場合には、当該事実の直前の他の事実があつた日)から当該事実があつた日までの期間を基礎として、当該事実があつた日において有するこれらの規定に規定する旧株、旧受益権、所有株式、所有出資又は旧新株予約権等につきその者の採用している評価の方法により計算した当該旧株、旧受益権、所有株式、所有出資又は旧新株予約権等の評価額に相当する金額をもつて第百十条から前条までに規定する旧株一株、旧受益権一口、所有株式一株、所有出資一単位、旧株一単位又は旧新株予約権等一単位の従前の取得価額とする。
\end{description}
\subsubsubsection*{第三目 譲渡所得の基因となる有価証券の取得費等}
{第三目 譲渡所得の基因となる有価証券の取得費等}
\noindent\hspace{10pt}(譲渡所得の基因となる有価証券の取得費等)
\begin{description}
\item[第百十八条]居住者が法第四十八条第三項(譲渡所得の基因となる有価証券の取得費等の計算)に規定する二回以上にわたつて取得した同一銘柄の有価証券で雑所得又は譲渡所得の基因となるものを譲渡した場合には、その譲渡につき法第三十七条第一項(必要経費)の規定によりその者のその譲渡の日の属する年分の雑所得の金額の計算上必要経費に算入する金額又は法第三十八条第一項(譲渡所得の金額の計算上控除する取得費)の規定によりその者の当該年分の譲渡所得の金額の計算上取得費に算入する金額は、当該有価証券を最初に取得した時(その後既に当該有価証券の譲渡をしている場合には、直前の譲渡の時。以下この項において同じ。)から当該譲渡の時までの期間を基礎として、当該最初に取得した時において有していた当該有価証券及び当該期間内に取得した当該有価証券につき第百五条第一項第一号(総平均法)に掲げる総平均法に準ずる方法によつて算出した一単位当たりの金額により計算した金額とする。
\item[\rensuji{2}]第百九条から前条までの規定は、前項に規定する所得の基因となる有価証券について準用する。
\end{description}
\noindent\hspace{10pt}(信用取引等による株式又は公社債の取得価額)
\begin{description}
\item[第百十九条]居住者が金融商品取引法第百五十六条の二十四第一項(免許及び免許の申請)に規定する信用取引若しくは発行日取引(有価証券が発行される前にその有価証券の売買を行う取引であつて財務省令で定める取引をいう。)又は同法第二十八条第八項第三号イ(通則)に掲げる取引の方法による株式又は公社債の売買を行い、かつ、これらの取引による株式又は公社債の売付けと買付けとにより当該取引の決済を行つた場合には、当該売付けに係る株式又は公社債の取得に要した経費としてその者のその年分の事業所得の金額又は雑所得の金額の計算上必要経費に算入する金額は、第百五条から前条までの規定にかかわらず、これらの取引において当該買付けに係る株式又は公社債を取得するために要した金額とする。
\end{description}
\subsubsection*{第四款 減価償却資産の償却}
\addcontentsline{toc}{subsubsection}{第四款 減価償却資産の償却}
\subsubsubsection*{第一目 減価償却資産の償却の方法}
{第一目 減価償却資産の償却の方法}
\noindent\hspace{10pt}(減価償却資産の償却の方法)
\begin{description}
\item[第百二十条]平成十九年三月三十一日以前に取得された減価償却資産(第六号に掲げる減価償却資産にあつては、当該減価償却資産についての同号に規定する改正前リース取引に係る契約が平成二十年三月三十一日までに締結されたもの)の償却費(法第四十九条第一項(減価償却資産の償却費の計算及びその償却の方法)の規定による減価償却資産の償却費をいう。以下この款において同じ。)の額の計算上選定をすることができる同項に規定する政令で定める償却の方法は、次の各号に掲げる資産の区分に応じ当該各号に定める方法とする。
\begin{description}
\item[一]建物(第三号に掲げるものを除く。)
\begin{description}
\item[イ]平成十年三月三十一日以前に取得された建物
\item[ロ]イに掲げる建物以外の建物
\end{description}
\item[二]第六条第一号(減価償却資産の範囲)に掲げる建物の附属設備及び同条第二号から第七号までに掲げる減価償却資産(次号及び第六号に掲げるものを除く。)
\begin{description}
\item[イ]旧定額法
\item[ロ]旧定率法
\end{description}
\item[三]鉱業用減価償却資産(第五号及び第六号に掲げるものを除く。)
\begin{description}
\item[イ]旧定額法
\item[ロ]旧定率法
\item[ハ]旧生産高比例法(当該鉱業用減価償却資産の取得価額からその残存価額を控除した金額を当該資産の耐用年数(当該資産の属する鉱区の採掘予定年数がその耐用年数より短い場合には、当該鉱区の採掘予定年数)の期間内における当該資産の属する鉱区の採掘予定数量で除して計算した一定単位当たりの金額に各年における当該鉱区の採掘数量を乗じて計算した金額をその年分の償却費として償却する方法をいう。以下この目及び第三目において同じ。)
\end{description}
\item[四]第六条第八号に掲げる無形固定資産(次号に掲げる鉱業権を除く。)及び同条第九号に掲げる生物
\item[五]第六条第八号イに掲げる鉱業権
\begin{description}
\item[イ]旧定額法
\item[ロ]旧生産高比例法
\end{description}
\item[六]国外リース資産(所得税法施行令の一部を改正する政令(平成十九年政令第八十二号)による改正前の所得税法施行令第百八十四条の二第一項(リース取引に係る各種所得の金額の計算)に規定するリース取引(同項又は同条第二項の規定により資産の賃貸借取引以外の取引とされるものを除く。以下この号において「改正前リース取引」という。)の目的とされている減価償却資産で非居住者又は外国法人に対して賃貸されているもの(これらの者の専ら国内において行う事業の用に供されるものを除く。)をいう。以下この項及び次項において同じ。)
\end{description}
\item[\rensuji{2}]前項第三号に規定する鉱業用減価償却資産とは、鉱業経営上直接必要な減価償却資産で鉱業の廃止により著しくその価値を減ずるものをいい、同項第六号に規定する見積残存価額とは、国外リース資産をその賃貸借の終了の時において譲渡するとした場合に見込まれるその譲渡対価の額に相当する金額をいう。
\item[\rensuji{3}]第一項第六号の月数は、暦に従つて計算し、一月に満たない端数を生じたときは、これを一月とする。
\end{description}
\begin{description}
\item[第百二十条の二]平成十九年四月一日以後に取得された減価償却資産(第六号に掲げる減価償却資産にあつては、当該減価償却資産についての所有権移転外リース取引に係る契約が平成二十年四月一日以後に締結されたもの)の償却費の額の計算上選定をすることができる法第四十九条第一項(減価償却資産の償却費の計算及びその償却の方法)に規定する政令で定める償却の方法は、次の各号に掲げる資産の区分に応じ当該各号に定める方法とする。
\begin{description}
\item[一]第六条第一号及び第二号(減価償却資産の範囲)に掲げる減価償却資産(第三号及び第六号に掲げるものを除く。)
\begin{description}
\item[イ]平成二十八年三月三十一日以前に取得された減価償却資産(建物を除く。)
\item[ロ]イに掲げる減価償却資産以外の減価償却資産
\end{description}
\item[二]第六条第三号から第七号までに掲げる減価償却資産(次号及び第六号に掲げるものを除く。)
\begin{description}
\item[イ]定額法
\item[ロ]定率法
\end{description}
\item[三]鉱業用減価償却資産(第五号及び第六号に掲げるものを除く。)
\begin{description}
\item[イ]平成二十八年四月一日以後に取得された第六条第一号及び第二号に掲げる減価償却資産
\item[ロ]イに掲げる減価償却資産以外の減価償却資産
\end{description}
\item[四]第六条第八号に掲げる無形固定資産(次号及び第六号に掲げるものを除く。)及び同条第九号に掲げる生物
\item[五]第六条第八号イに掲げる鉱業権
\begin{description}
\item[イ]定額法
\item[ロ]生産高比例法
\end{description}
\item[六]リース資産
\end{description}
\item[\rensuji{2}]前項及びこの項において、次の各号に掲げる用語の意義は、当該各号に定めるところによる。
\begin{description}
\item[一]償却保証額
\item[二]改定取得価額
\begin{description}
\item[イ]減価償却資産の前項第一号イ(2)に規定する取得価額に同号イ(2)に規定する耐用年数に応じた償却率を乗じて計算した金額(以下この号において「調整前償却額」という。)が償却保証額に満たない場合(その年の前年における調整前償却額が償却保証額以上である場合に限る。)
\item[ロ]連続する二以上の年において減価償却資産の調整前償却額がいずれも償却保証額に満たない場合
\end{description}
\item[三]鉱業用減価償却資産
\item[四]リース資産
\item[五]所有権移転外リース取引
\begin{description}
\item[イ]リース期間終了の時又はリース期間の中途において、当該リース取引に係る契約において定められている当該リース取引の目的とされている資産(以下この号において「目的資産」という。)が無償又は名目的な対価の額で当該リース取引に係る賃借人に譲渡されるものであること。
\item[ロ]当該リース取引に係る賃借人に対し、リース期間終了の時又はリース期間の中途において目的資産を著しく有利な価額で買い取る権利が与えられているものであること。
\item[ハ]目的資産の種類、用途、設置の状況等に照らし、当該目的資産がその使用可能期間中当該リース取引に係る賃借人によつてのみ使用されると見込まれるものであること又は当該目的資産の識別が困難であると認められるものであること。
\item[ニ]リース期間が目的資産の第百二十九条(減価償却資産の耐用年数、償却率等)に規定する財務省令で定める耐用年数に比して相当短いもの(当該リース取引に係る賃借人の所得税の負担を著しく軽減することになると認められるものに限る。)であること。
\end{description}
\item[六]残価保証額
\item[七]リース期間
\end{description}
\item[\rensuji{3}]第一項第六号の月数は、暦に従つて計算し、一月に満たない端数を生じたときは、これを一月とする。
\end{description}
\noindent\hspace{10pt}(減価償却資産の特別な償却の方法)
\begin{description}
\item[第百二十条の三]居住者は、その有する第六条第一号から第八号まで(減価償却資産の範囲)に掲げる減価償却資産(次条又は第百二十二条(特別な償却率による償却の方法)の規定の適用を受けるもの並びに第百二十条第一項第一号ロ及び第六号(減価償却資産の償却の方法)並びに前条第一項第一号ロ及び第六号に掲げる減価償却資産を除く。)の償却費の額を当該資産の区分に応じて定められている第百二十条第一項第一号から第五号まで又は前条第一項第一号から第五号までに定める償却の方法に代え当該償却の方法以外の償却の方法(同項第三号イに掲げる減価償却資産(第三項において「鉱業用建築物」という。)にあつては、定率法その他これに準ずる方法を除く。以下この項において同じ。)により計算することについて納税地の所轄税務署長の承認を受けた場合には、当該資産のその承認を受けた日の属する年分以後の各年分の償却費の額の計算については、その承認を受けた償却の方法を選定することができる。
\item[\rensuji{2}]前項の承認を受けようとする居住者は、その採用しようとする償却の方法の内容、その方法を採用しようとする理由、その方法により償却費の額の計算をしようとする資産の種類(償却の方法の選定の単位を設備の種類とされているものについては、設備の種類とし、二以上の事業所又は船舶を有する居住者で事業所又は船舶ごとに償却の方法を選定しようとする場合にあつては、事業所又は船舶ごとのこれらの種類とする。次項において同じ。)その他財務省令で定める事項を記載した申請書を納税地の所轄税務署長に提出しなければならない。
\item[\rensuji{3}]税務署長は、前項の申請書の提出があつた場合には、遅滞なく、これを審査し、その申請に係る償却の方法及び資産の種類を承認し、又はその申請に係る償却の方法によつてはその居住者の各年分の不動産所得の金額、事業所得の金額、山林所得の金額又は雑所得の金額の計算が適正に行われ難いと認めるとき(その申請に係る資産の種類が鉱業用建築物である場合に当該償却の方法が定率法その他これに準ずる方法であると認めるときを含む。)は、その申請を却下する。
\item[\rensuji{4}]税務署長は、第一項の承認をした後、その承認に係る償却の方法によりその承認に係る減価償却資産の償却費の額の計算をすることを不適当とする特別の事由が生じたと認める場合には、その承認を取り消すことができる。
\item[\rensuji{5}]税務署長は、前二項の処分をするときは、その処分に係る居住者に対し、書面によりその旨を通知する。
\item[\rensuji{6}]第四項の処分があつた場合には、その処分のあつた日の属する年分以後の各年分の不動産所得の金額、事業所得の金額、山林所得の金額又は雑所得の金額を計算する場合のその処分に係る減価償却資産の償却費の額の計算についてその処分の効果が生ずるものとする。
\item[\rensuji{7}]居住者は、第四項の処分を受けた場合には、その処分を受けた日の属する年分の所得税に係る確定申告期限までに、その処分に係る減価償却資産につき、第百二十三条第一項(減価償却資産の償却の方法の選定)に規定する区分(二以上の事業所又は船舶を有する居住者で事業所又は船舶ごとに償却の方法を選定しようとする場合にあつては、事業所又は船舶ごとの当該区分)ごとに、第百二十条第一項又は前条第一項に規定する償却の方法のうちそのよるべき方法を書面により納税地の所轄税務署長に届け出なければならない。
\end{description}
\noindent\hspace{10pt}(取替資産に係る償却の方法の特例)
\begin{description}
\item[第百二十一条]取替資産の償却費の額の計算については、納税地の所轄税務署長の承認を受けた場合には、その採用している第百二十条第一項第二号又は第百二十条の二第一項第一号若しくは第二号(減価償却資産の償却の方法)に定める償却の方法に代えて、取替法を選定することができる。
\item[\rensuji{2}]前項に規定する取替法とは、次に掲げる金額の合計額を各年分の償却費として償却する方法をいう。
\begin{description}
\item[一]当該取替資産につきその取得価額(その年以前の各年に係る次号に掲げる新たな資産の取得価額に相当する金額を除くものとし、当該資産が昭和二十七年十二月三十一日以前に取得された資産である場合には、当該資産に係る法第六十一条第三項(昭和二十七年十二月三十一日以前に取得した資産の取得費等)に規定する昭和二十八年一月一日における価額として政令で定めるところにより計算した金額とする。)の百分の五十に達するまで旧定額法、旧定率法、定額法又は定率法のうちその採用している方法により計算した金額
\item[二]当該取替資産が使用に耐えなくなつたためその年において種類及び品質を同じくするこれに代わる新たな資産と取り替えた場合におけるその新たな資産の取得価額
\end{description}
\item[\rensuji{3}]前二項に規定する取替資産とは、事業所得を生ずべき事業の用に供される軌条、枕木その他多量に同一の目的のために使用される減価償却資産で、毎年使用に耐えなくなつたこれらの資産の一部がほぼ同数量ずつ取り替えられるもののうち財務省令で定めるものをいう。
\item[\rensuji{4}]第一項の承認を受けようとする居住者は、第二項に規定する取替法(次項及び第百三十二条第一項(年の中途で業務の用に供した減価償却資産等の償却費の特例)において「取替法」という。)を採用しようとする年の三月十五日までに、第一項の規定の適用を受けようとする減価償却資産の種類及び名称、その所在する場所その他財務省令で定める事項を記載した申請書を納税地の所轄税務署長に提出しなければならない。
\item[\rensuji{5}]税務署長は、前項の申請書の提出があつた場合において、その申請に係る減価償却資産の償却費の計算を取替法によつて行う場合にはその居住者の各年分の事業所得の金額の計算が適正に行われ難いと認めるときは、その申請を却下することができる。
\item[\rensuji{6}]税務署長は、第四項の申請書の提出があつた場合において、その申請につき承認又は却下の処分をするときは、その申請をした居住者に対し、書面によりその旨を通知する。
\item[\rensuji{7}]第四項の申請書の提出があつた場合において、同項に規定する年の十二月三十一日(その申請書を提出した居住者がその年の中途において死亡し又は出国をした場合には、その死亡又は出国の時)までにその申請につき承認又は却下の処分がなかつたときは、その日又は時においてその承認があつたものとみなす。
\end{description}
\noindent\hspace{10pt}(リース賃貸資産の償却の方法の特例)
\begin{description}
\item[第百二十一条の二]リース賃貸資産(第百二十条第一項第六号(減価償却資産の償却の方法)に規定する改正前リース取引の目的とされている減価償却資産(同号に規定する国外リース資産を除く。)をいう。以下この条において同じ。)については、その採用している同項又は第百二十条の二第一項(減価償却資産の償却の方法)に規定する償却の方法に代えて、旧リース期間定額法(当該リース賃貸資産の改定取得価額を改定リース期間の月数で除して計算した金額にその年における当該改定リース期間の月数を乗じて計算した金額を各年分の償却費として償却する方法をいう。)を選定することができる。
\item[\rensuji{2}]前項の規定の適用を受けようとする居住者は、同項に規定する旧リース期間定額法を採用しようとする年分の所得税に係る確定申告期限までに、同項の規定の適用を受けようとするリース賃貸資産の第百二十条の三第二項(減価償却資産の特別な償却の方法)に規定する資産の種類その他財務省令で定める事項を記載した届出書を納税地の所轄税務署長に提出しなければならない。
\item[\rensuji{3}]第一項に規定する改定取得価額とは、同項の規定の適用を受けるリース賃貸資産の当該適用を受ける最初の年の一月一日(当該リース賃貸資産が同日後に賃貸の用に供したものである場合には、当該賃貸の用に供した日)における取得価額(既に償却費としてその年の前年分以前の各年分の不動産所得の金額、事業所得の金額、山林所得の金額又は雑所得の金額の計算上必要経費に算入された金額がある場合には、当該金額を控除した金額)から残価保証額(当該リース賃貸資産の同項に規定する改正前リース取引に係る契約において定められている当該リース賃貸資産の賃貸借の期間(以下この項において「リース期間」という。)の終了の時に当該リース賃貸資産の処分価額が当該改正前リース取引に係る契約において定められている保証額に満たない場合にその満たない部分の金額を当該改正前リース取引に係る賃借人その他の者がその賃貸人に支払うこととされている場合における当該保証額をいい、当該保証額の定めがない場合には零とする。)を控除した金額をいい、第一項に規定する改定リース期間とは、同項の規定の適用を受けるリース賃貸資産のリース期間(当該リース賃貸資産が他の者から移転を受けたもの(法第六十条第一項各号(贈与等により取得した資産の取得費等)に掲げる事由により移転を受けた第百二十六条第二項(減価償却資産の取得価額)に規定する減価償却資産を除く。)である場合には、当該移転の日以後の期間に限る。)のうち第一項の規定の適用を受ける最初の年の一月一日以後の期間(当該リース賃貸資産が同日以後に賃貸の用に供したものである場合には、当該リース期間)をいう。
\item[\rensuji{4}]第一項の月数は、暦に従つて計算し、一月に満たない端数を生じたときは、これを一月とする。
\end{description}
\noindent\hspace{10pt}(特別な償却率による償却の方法)
\begin{description}
\item[第百二十二条]減価償却資産(第百二十条の二第一項第六号(減価償却資産の償却の方法)に掲げるリース資産を除く。)のうち、漁網、活字に常用されている金属その他財務省令で定めるものの償却費の額の計算については、その採用している第百二十条第一項(減価償却資産の償却の方法)又は第百二十条の二第一項に規定する償却の方法に代えて、当該資産の取得価額に当該資産につき納税地の所轄国税局長の認定を受けた償却率を乗じて計算した金額を各年分の償却費の額として償却する方法を選定することができる。
\item[\rensuji{2}]前項の認定を受けようとする居住者は、同項の規定の適用を受けようとする減価償却資産の種類及び名称、その所在する場所その他財務省令で定める事項を記載した申請書に当該認定に係る償却率の算定の基礎となるべき事項を記載した書類を添付し、納税地の所轄税務署長を経由して、これを納税地の所轄国税局長に提出しなければならない。
\item[\rensuji{3}]国税局長は、前項の申請書の提出があつた場合には、遅滞なく、これを審査し、その申請に係る減価償却資産の償却率を認定するものとする。
\item[\rensuji{4}]国税局長は、第一項の認定をした後、その認定に係る償却率により同項の減価償却資産の償却費の額の計算をすることを不適当とする特別の事由が生じたと認める場合には、その償却率を変更することができる。
\item[\rensuji{5}]国税局長は、前二項の処分をするときは、その認定に係る居住者に対し、書面によりその旨を通知する。
\item[\rensuji{6}]第三項又は第四項の処分があつた場合には、その処分のあつた日の属する年分以後の各年分の不動産所得の金額、事業所得の金額又は雑所得の金額を計算する場合のその処分に係る減価償却資産の償却費の額の計算についてその処分の効果が生ずるものとする。
\end{description}
\noindent\hspace{10pt}(減価償却資産の償却の方法の選定)
\begin{description}
\item[第百二十三条]第百二十条第一項又は第百二十条の二第一項(減価償却資産の償却の方法)に規定する減価償却資産の償却の方法は、第百二十条第一項各号又は第百二十条の二第一項各号に掲げる減価償却資産ごとに、かつ、第百二十条第一項第一号イ、第二号、第三号及び第五号並びに第百二十条の二第一項第一号イ、第二号、第三号イ、同号ロ及び第五号に掲げる減価償却資産については設備の種類その他の財務省令で定める区分ごとに選定しなければならない。
\item[\rensuji{2}]居住者は、次の各号に掲げる者の区分に応じ当該各号に定める日の属する年分の所得税に係る確定申告期限までに、その有する減価償却資産と同一の区分(前項に規定する区分をいい、二以上の事業所又は船舶を有する居住者で事業所又は船舶ごとに償却の方法を選定しようとする場合にあつては、事業所又は船舶ごとの当該区分をいう。)に属する減価償却資産につき、当該区分ごとに、第百二十条第一項又は第百二十条の二第一項に規定する償却の方法のうちそのよるべき方法を書面により納税地の所轄税務署長に届け出なければならない。
\begin{description}
\item[一]新たに不動産所得、事業所得、山林所得又は雑所得を生ずべき業務を開始した居住者
\item[二]前号の業務を開始した後既にそのよるべき償却の方法を選定している減価償却資産(その償却の方法を届け出なかつたことにより第百二十五条(減価償却資産の法定償却方法)に規定する償却の方法によるべきこととされているものを含む。)以外の減価償却資産を取得した居住者
\item[三]新たに事業所を設けた居住者で、当該事業所に属する減価償却資産につき当該減価償却資産と同一の区分(前項に規定する区分をいう。)に属する資産について既に選定している償却の方法と異なる償却の方法を選定しようとするもの又は既に事業所ごとに異なる償却の方法を選定しているもの
\item[四]新たに船舶を取得した居住者で、当該船舶につき当該船舶以外の船舶について既に選定している償却の方法と異なる償却の方法を選定しようとするもの又は既に船舶ごとに異なる償却の方法を選定しているもの
\end{description}
\item[\rensuji{3}]平成十九年三月三十一日以前に取得された減価償却資産(以下この項において「旧償却方法適用資産」という。)につき既にそのよるべき償却の方法として旧定額法、旧定率法又は旧生産高比例法を選定している場合(その償却の方法を届け出なかつたことにより第百二十五条に規定する償却の方法によるべきこととされている場合を含むものとし、二以上の事業所又は船舶を有する場合で既に事業所又は船舶ごとに異なる償却の方法を選定している場合を除く。)において、同年四月一日以後に取得された減価償却資産(以下この項において「新償却方法適用資産」という。)で、同年三月三十一日以前に取得されるとしたならば当該旧償却方法適用資産と同一の区分(第一項に規定する区分をいう。)に属するものにつき前項の規定による届出をしていないときは、当該新償却方法適用資産については、当該旧償却方法適用資産につき選定した次の各号に掲げる償却の方法の区分に応じ当該各号に定める償却の方法(第百二十条の二第一項第三号イに掲げる減価償却資産に該当する新償却方法適用資産にあつては、当該旧償却方法適用資産につき選定した第一号又は第三号に掲げる償却の方法の区分に応じそれぞれ第一号又は第三号に定める償却の方法)を選定したものとみなす。
\begin{description}
\item[一]旧定額法
\item[二]旧定率法
\item[三]旧生産高比例法
\end{description}
\item[\rensuji{4}]第百二十条の二第一項第三号に掲げる減価償却資産のうち平成二十八年三月三十一日以前に取得されたもの(以下この項において「旧選定対象資産」という。)につき既にそのよるべき償却の方法として定額法を選定している場合(二以上の事業所又は船舶を有する場合で既に事業所又は船舶ごとに異なる償却の方法を選定している場合を除く。)において、同号イに掲げる減価償却資産(以下この項において「新選定対象資産」という。)で、同日以前に取得されるとしたならば当該旧選定対象資産と同一の区分(第一項に規定する区分をいう。以下この項において同じ。)に属するものにつき第二項の規定による届出をしていないときは、当該新選定対象資産については、定額法を選定したものとみなす。
\item[\rensuji{5}]第二項ただし書に規定する減価償却資産については、居住者が当該資産を取得した日において第百二十条第一項第一号ロ、第四号若しくは第六号又は第百二十条の二第一項第一号ロ、第四号若しくは第六号に定める償却の方法を選定したものとみなす。
\end{description}
\noindent\hspace{10pt}(減価償却資産の償却の方法の変更手続)
\begin{description}
\item[第百二十四条]居住者は、減価償却資産につき選定した償却の方法(その償却の方法を届け出なかつた者がよるべきこととされている次条に規定する償却の方法を含む。)を変更しようとするとき(二以上の事業所又は船舶を有する居住者で事業所又は船舶ごとに償却の方法を選定していないものが事業所又は船舶ごとに償却の方法を選定しようとするときを含む。)は、納税地の所轄税務署長の承認を受けなければならない。
\item[\rensuji{2}]前項の承認を受けようとする居住者は、その新たな償却の方法を採用しようとする年の三月十五日までに、その旨、変更しようとする理由その他財務省令で定める事項を記載した申請書を納税地の所轄税務署長に提出しなければならない。
\item[\rensuji{3}]税務署長は、前項の申請書の提出があつた場合において、その申請書を提出した居住者が現によつている償却の方法を採用してから相当期間を経過していないとき、又は変更しようとする償却の方法によつてはその者の各年分の不動産所得の金額、事業所得の金額、山林所得の金額又は雑所得の金額の計算が適正に行われ難いと認めるときは、その申請を却下することができる。
\item[\rensuji{4}]税務署長は、第二項の申請書の提出があつた場合において、その申請につき承認又は却下の処分をするときは、その申請をした居住者に対し、書面によりその旨を通知する。
\item[\rensuji{5}]第二項の申請書の提出があつた場合において、同項に規定する年の十二月三十一日(その申請書を提出した居住者がその年の中途において死亡し又は出国をした場合には、その死亡又は出国の時)までにその申請につき承認又は却下の処分がなかつたときは、その日又は時においてその承認があつたものとみなす。
\end{description}
\noindent\hspace{10pt}(減価償却資産の法定償却方法)
\begin{description}
\item[第百二十五条]法第四十九条第一項(減価償却資産の償却費の計算及びその償却の方法)に規定する償却の方法を選定しなかつた場合における政令で定める方法は、次の各号に掲げる資産の区分に応じ当該各号に定める方法とする。
\begin{description}
\item[一]平成十九年三月三十一日以前に取得された減価償却資産
\begin{description}
\item[イ]第百二十条第一項第一号イ及び同項第二号(減価償却資産の償却の方法)に掲げる減価償却資産
\item[ロ]第百二十条第一項第三号及び第五号に掲げる減価償却資産
\end{description}
\item[二]平成十九年四月一日以後に取得された減価償却資産
\begin{description}
\item[イ]第百二十条の二第一項第一号イ及び第二号(減価償却資産の償却の方法)に掲げる減価償却資産
\item[ロ]第百二十条の二第一項第三号及び第五号に掲げる減価償却資産
\end{description}
\end{description}
\end{description}
\subsubsubsection*{第二目 減価償却資産の取得価額等}
{第二目 減価償却資産の取得価額等}
\noindent\hspace{10pt}(減価償却資産の取得価額)
\begin{description}
\item[第百二十六条]減価償却資産の第百二十条から第百二十二条まで(減価償却資産の償却の方法)に規定する取得価額は、別段の定めがあるものを除き、次の各号に掲げる資産の区分に応じ当該各号に掲げる金額とする。
\begin{description}
\item[一]購入した減価償却資産
\begin{description}
\item[イ]当該資産の購入の代価(引取運賃、荷役費、運送保険料、購入手数料、関税(関税法第二条第一項第四号の二(定義)に規定する附帯税を除く。)その他当該資産の購入のために要した費用がある場合には、その費用の額を加算した金額)
\item[ロ]当該資産を業務の用に供するために直接要した費用の額
\end{description}
\item[二]自己の建設、製作又は製造(以下この条において「建設等」という。)に係る減価償却資産
\begin{description}
\item[イ]当該資産の建設等のために要した原材料費、労務費及び経費の額
\item[ロ]当該資産を業務の用に供するために直接要した費用の額
\end{description}
\item[三]自己が成育させた第六条第九号イ(生物)に掲げる生物(以下この号において「牛馬等」という。)
\begin{description}
\item[イ]成育させるために取得した牛馬等に係る第一号イ若しくは第五号イに掲げる金額又は種付費及び出産費の額並びに当該取得した牛馬等の成育のために要した飼料費、労務費及び経費の額
\item[ロ]成育させた牛馬等を業務の用に供するために直接要した費用の額
\end{description}
\item[四]自己が成熟させた第六条第九号ロ及びハに掲げる生物(以下この号において「果樹等」という。)
\begin{description}
\item[イ]成熟させるために取得した果樹等に係る第一号イ若しくは次号イに掲げる金額又は種苗費の額並びに当該取得した果樹等の成熟のために要した肥料費、労務費及び経費の額
\item[ロ]成熟させた果樹等を業務の用に供するために直接要した費用の額
\end{description}
\item[五]前各号に規定する方法以外の方法により取得した減価償却資産
\begin{description}
\item[イ]その取得の時における当該資産の取得のために通常要する価額
\item[ロ]当該資産を業務の用に供するために直接要した費用の額
\end{description}
\end{description}
\item[\rensuji{2}]法第六十条第一項各号(贈与等により取得した資産の取得費等)に掲げる事由により取得した減価償却資産(法第四十条第一項第一号(たな卸資産の贈与等の場合の総収入金額算入)の規定の適用があつたものを除く。)の前項に規定する取得価額は、当該減価償却資産を取得した者が引き続き所有していたものとみなした場合における当該減価償却資産のこの条及び次条第二項の規定による取得価額に相当する金額とする。
\end{description}
\noindent\hspace{10pt}(資本的支出の取得価額の特例)
\begin{description}
\item[第百二十七条]居住者が有する減価償却資産(次条の規定に該当するものを除く。以下この条において同じ。)について支出する金額のうちに第百八十一条(資本的支出)の規定によりその支出する日の属する年分の不動産所得の金額、事業所得の金額、山林所得の金額又は雑所得の金額の計算上必要経費に算入されなかつた金額がある場合には、当該金額を前条第一項の規定による取得価額として、その有する減価償却資産と種類及び耐用年数を同じくする減価償却資産を新たに取得したものとする。
\item[\rensuji{2}]前項に規定する場合において、同項に規定する居住者が有する減価償却資産についてそのよるべき償却の方法として第百二十条第一項(減価償却資産の償却の方法)に規定する償却の方法を採用しているときは、前項の規定にかかわらず、同項の支出した金額を当該減価償却資産の前条の規定による取得価額に加算することができる。
\item[\rensuji{3}]第一項に規定する場合において、同項に規定する居住者が有する減価償却資産がリース資産(第百二十条の二第二項第四号(減価償却資産の償却の方法)に規定するリース資産をいう。以下この項において同じ。)であるときは、第一項の規定により新たに取得したものとされる減価償却資産は、リース資産に該当するものとする。
\item[\rensuji{4}]居住者のその年の前年分の所得税につき第一項に規定する必要経費に算入されなかつた金額がある場合において、同項に規定する居住者が有する減価償却資産(平成二十四年三月三十一日以前に取得された資産を除く。以下この項において「旧減価償却資産」という。)及び第一項の規定により新たに取得したものとされた減価償却資産(以下この条において「追加償却資産」という。)についてそのよるべき償却の方法として定率法を採用しているときは、同項の規定にかかわらず、その年の一月一日において、同日における旧減価償却資産の前条の規定による取得価額(既に償却費としてその年の前年分以前の各年分の不動産所得の金額、事業所得の金額、山林所得の金額又は雑所得の金額の計算上必要経費に算入された金額がある場合には、当該金額を控除した金額。以下この条において「取得価額等」という。)と追加償却資産の取得価額等との合計額を前条第一項の規定による取得価額とする一の減価償却資産を、新たに取得したものとすることができる。
\item[\rensuji{5}]居住者のその年の前年分の所得税につき第一項に規定する必要経費に算入されなかつた金額がある場合において、当該金額に係る追加償却資産について、そのよるべき償却の方法として定率法を採用し、かつ、前項の規定の適用を受けないときは、第一項及び前項の規定にかかわらず、その年の一月一日において、当該適用を受けない追加償却資産のうち種類及び耐用年数を同じくするものの同日における取得価額等の合計額を前条第一項の規定による取得価額とする一の減価償却資産を、新たに取得したものとすることができる。
\end{description}
\noindent\hspace{10pt}(昭和二十七年十二月三十一日以前に取得した非事業用資産で業務の用に供されたものの取得価額)
\begin{description}
\item[第百二十八条]昭和二十七年十二月三十一日以前から引き続き所有していた家屋その他使用又は期間の経過により減価する資産で不動産所得、事業所得、山林所得又は雑所得を生ずべき業務の用に供していないものを当該業務の用に供した場合には、当該資産の第百二十六条第一項(減価償却資産の取得価額)に規定する取得価額は、当該資産に係る法第六十一条第三項(昭和二十七年十二月三十一日以前に取得した資産の取得費等)に規定する政令で定めるところにより計算した金額と当該資産につき昭和二十八年一月一日から当該業務の用に供された日までの間に支出された設備費及び改良費の額との合計額とする。
\item[\rensuji{2}]前条第一項、第二項、第四項及び第五項の規定は、前項に規定する資産を同項の業務の用に供した後において当該資産につき支出する金額のうちに同条第一項に規定する必要経費に算入されなかつた金額がある場合について準用する。
\end{description}
\noindent\hspace{10pt}(減価償却資産の耐用年数、償却率等)
\begin{description}
\item[第百二十九条]減価償却資産の第百二十条第一項第一号及び第三号並びに第百二十条の二第一項第一号及び第三号(減価償却資産の償却の方法)に規定する耐用年数、第百二十条第一項第一号及び第百二十条の二第一項第一号に規定する耐用年数に応じた償却率、同号に規定する耐用年数に応じた改定償却率、同条第二項第一号に規定する耐用年数に応じた保証率並びに第百二十条第一項第一号及び第三号に規定する残存価額については、財務省令で定めるところによる。
\end{description}
\noindent\hspace{10pt}(耐用年数の短縮)
\begin{description}
\item[第百三十条]青色申告書を提出する居住者は、その有する減価償却資産が次に掲げる事由のいずれかに該当する場合において、その該当する減価償却資産の使用可能期間のうちいまだ経過していない期間(以下この項から第四項までにおいて「未経過使用可能期間」という。)を基礎としてその償却費の額を計算することについて納税地の所轄国税局長の承認を受けたときは、当該資産のその承認を受けた日の属する年分以後の各年分の償却費の額の計算については、その承認に係る未経過使用可能期間をもつて前条に規定する財務省令で定める耐用年数(以下この項において「法定耐用年数」という。)とみなす。
\begin{description}
\item[一]当該資産の材質又は製作方法がこれと種類及び構造を同じくする他の減価償却資産の通常の材質又は製作方法と著しく異なることにより、その使用可能期間が法定耐用年数に比して著しく短いこと。
\item[二]当該資産の存する地盤が隆起し、又は沈下したことにより、その使用可能期間が法定耐用年数に比して著しく短いこととなつたこと。
\item[三]当該資産が陳腐化したことにより、その使用可能期間が法定耐用年数に比して著しく短いこととなつたこと。
\item[四]当該資産がその使用される場所の状況に基因して著しく腐食したことにより、その使用可能期間が法定耐用年数に比して著しく短いこととなつたこと。
\item[五]当該資産が通常の修理又は手入れをしなかつたことに基因して著しく損耗したことにより、その使用可能期間が法定耐用年数に比して著しく短いこととなつたこと。
\item[六]前各号に掲げる事由以外の事由で財務省令で定めるものにより、当該資産の使用可能期間が法定耐用年数に比して著しく短いこと又は短いこととなつたこと。
\end{description}
\item[\rensuji{2}]前項の承認を受けようとする居住者は、同項の規定の適用を受けようとする減価償却資産の種類及び名称、その所在する場所、その使用可能期間、その未経過使用可能期間その他財務省令で定める事項を記載した申請書に当該資産が前項各号に掲げる事由のいずれかに該当することを証する書類を添付し、納税地の所轄税務署長を経由して、これを納税地の所轄国税局長に提出しなければならない。
\item[\rensuji{3}]国税局長は、前項の申請書の提出があつた場合には、遅滞なく、これを審査し、その申請に係る減価償却資産の使用可能期間及び未経過使用可能期間を認め、若しくはその使用可能期間及び未経過使用可能期間を定めて第一項の承認をし、又はその申請を却下する。
\item[\rensuji{4}]国税局長は、第一項の承認をした後、その承認に係る未経過使用可能期間により同項の減価償却資産の償却費の額の計算をすることを不適当とする特別の事由が生じたと認める場合には、その承認を取り消し、又はその承認に係る使用可能期間及び未経過使用可能期間を伸長することができる。
\item[\rensuji{5}]国税局長は、前二項の処分をするときは、その処分に係る居住者に対し、書面によりその旨を通知する。
\item[\rensuji{6}]第三項の承認の処分又は第四項の処分があつた場合には、その処分のあつた日の属する年分以後の各年分の不動産所得の金額、事業所得の金額、山林所得の金額又は雑所得の金額を計算する場合のその処分に係る減価償却資産の償却費の額の計算についてその処分の効果が生ずるものとする。
\item[\rensuji{7}]青色申告書を提出する居住者が、その有する第一項の承認に係る減価償却資産の一部についてこれに代わる新たな資産(以下この項において「更新資産」という。)と取り替えた場合その他の財務省令で定める場合において、当該更新資産を取得した日の属する年分の所得税に係る確定申告期限までに、当該更新資産の名称、その所在する場所その他財務省令で定める事項を記載した届出書を納税地の所轄税務署長を経由して納税地の所轄国税局長に提出したときは、当該届出書をもつて第二項の申請書とみなし、当該届出書の提出をもつて同日の属する年の十二月三十一日(その者が年の中途において死亡し又は出国をした場合には、その死亡又は出国の時。次項において同じ。)において第一項の承認があつたものとみなす。
\item[\rensuji{8}]青色申告書を提出する居住者が、その有する第一項の承認(同項第一号に掲げる事由による承認その他財務省令で定める事由による承認に限る。)に係る減価償却資産と材質又は製作方法を同じくする減価償却資産(当該財務省令で定める事由による承認の場合には、財務省令で定める減価償却資産)を取得した場合において、その取得した日の属する年分の所得税に係る確定申告期限までに、その取得した減価償却資産の名称、その所在する場所その他財務省令で定める事項を記載した届出書を納税地の所轄税務署長を経由して納税地の所轄国税局長に提出したときは、当該届出書をもつて第二項の申請書とみなし、当該届出書の提出をもつて同日の属する年の十二月三十一日において第一項の承認があつたものとみなす。
\item[\rensuji{9}]青色申告書を提出する居住者が、その有する減価償却資産につき第一項の承認を受けた場合には、当該資産の第百二十条第一項第一号イ(1)若しくは第三号ハ又は第百二十条の二第一項第一号イ(1)若しくは第三号イ(2)若しくは第二項第一号(減価償却資産の償却の方法)に規定する取得価額には、当該資産の償却費として当該資産につきその承認を受けた日の属する年の前年分以前の各年分の不動産所得の金額、事業所得の金額、山林所得の金額又は雑所得の金額の計算上必要経費に算入された金額の累積額を含まないものとする。
\item[\rensuji{10}]第百三十四条第二項(減価償却資産の償却累積額による償却費の特例)の規定は、第一項の承認に係る減価償却資産(そのよるべき償却の方法として定率法を採用しているものに限る。)につきその承認を受けた日の属する年分において同項の規定を適用しないで計算した第百二十条の二第二項第二号イに規定する調整前償却額が前項の規定を適用しないで計算した同条第二項第一号に規定する償却保証額に満たない場合について準用する。
\item[\rensuji{11}]第一項の承認を受けた居住者が青色申告書の提出の承認を取り消され、又は青色申告書による申告をやめる旨の届出書の提出をした場合には、同項の承認は、その青色申告書の提出の承認の取消しの基因となつた事実のあつた日の属する年又はそのやめた年の一月一日においてその効力を失うものとする。
\end{description}
\subsubsubsection*{第三目 減価償却資産の償却費の計算}
{第三目 減価償却資産の償却費の計算}
\noindent\hspace{10pt}(減価償却資産の償却費の計算)
\begin{description}
\item[第百三十一条]居住者の有する減価償却資産につきその償却費としてその者の各年分の不動産所得の金額、事業所得の金額、山林所得の金額又は雑所得の金額の計算上必要経費に算入する金額は、当該資産につきその者が採用している償却の方法に基づいて計算した金額とする。
\end{description}
\noindent\hspace{10pt}(年の中途で業務の用に供した減価償却資産等の償却費の特例)
\begin{description}
\item[第百三十二条]居住者の有する減価償却資産(第百二十条第一項第六号及び第百二十条の二第一項第六号(減価償却資産の償却の方法)に掲げる減価償却資産を除く。)が次の各号に掲げる場合に該当することとなつたときは、当該資産の償却費としてその該当することとなつた日の属する年分の不動産所得の金額、事業所得の金額、山林所得の金額又は雑所得の金額の計算上必要経費に算入する金額は、前条の規定にかかわらず、当該各号に定める金額とする。
\begin{description}
\item[一]当該資産が年の中途において不動産所得、事業所得、山林所得又は雑所得を生ずべき業務の用に供された場合(次号に掲げる場合に該当する場合を除く。)
\begin{description}
\item[イ]そのよるべき償却の方法として旧定額法、旧定率法、定額法、定率法又は取替法を採用している減価償却資産(取替法を採用しているものについては、第百二十一条第二項第二号(取替資産に係る償却の方法の特例)に規定する新たな資産に該当するものを除く。次号イ及び第三号イにおいて同じ。)
\item[ロ]そのよるべき償却の方法として旧生産高比例法又は生産高比例法を採用している減価償却資産
\item[ハ]そのよるべき償却の方法として第百二十条の三第一項(減価償却資産の特別な償却の方法)に規定する納税地の所轄税務署長の承認を受けた償却の方法を採用している減価償却資産
\end{description}
\item[二]当該資産が年の中途において前号に規定する業務の用以外の用に供された場合
\begin{description}
\item[イ]そのよるべき償却の方法として旧定額法、旧定率法、定額法、定率法又は取替法を採用している減価償却資産
\item[ロ]そのよるべき償却の方法として旧生産高比例法又は生産高比例法を採用している減価償却資産
\item[ハ]そのよるべき償却の方法として第百二十条の三第一項に規定する納税地の所轄税務署長の承認を受けた償却の方法を採用している減価償却資産
\end{description}
\item[三]当該資産を有する居住者が年の中途において死亡し又は出国をする場合(前二号に掲げる場合に該当する場合を除く。)
\begin{description}
\item[イ]そのよるべき償却の方法として旧定額法、旧定率法、定額法、定率法又は取替法を採用している減価償却資産
\item[ロ]そのよるべき償却の方法として旧生産高比例法又は生産高比例法を採用している減価償却資産
\item[ハ]そのよるべき償却の方法として第百二十条の三第一項に規定する納税地の所轄税務署長の承認を受けた償却の方法を採用している減価償却資産
\end{description}
\end{description}
\item[\rensuji{2}]前項各号の月数は、暦に従つて計算し、一月に満たない端数を生じたときは、これを一月とする。
\end{description}
\noindent\hspace{10pt}(通常の使用時間を超えて使用される機械及び装置の償却費の特例)
\begin{description}
\item[第百三十三条]青色申告書を提出する居住者が、その有する機械及び装置(そのよるべき償却の方法として旧定額法、旧定率法、定額法又は定率法を採用しているものに限る。)の使用時間がその者の行う不動産所得、事業所得又は山林所得を生ずべき業務の通常の経済事情における当該機械及び装置の平均的な使用時間を超える場合において、当該機械及び装置の当該年分の償却費の額と当該償却費の額に当該機械及び装置の当該平均的な使用時間を超えて使用することによる損耗の程度に応ずるものとして財務省令で定めるところにより計算した増加償却割合を乗じて計算した金額との合計額をもつて当該機械及び装置の当該年分の償却費の額としようとする旨その他財務省令で定める事項を記載した書類を、当該年分の所得税に係る確定申告期限までに、納税地の所轄税務署長に提出し、かつ、当該平均的な使用時間を超えて使用したことを証する書類を保存しているときは、当該機械及び装置の償却費として当該年分の不動産所得の金額、事業所得の金額又は山林所得の金額の計算上必要経費に算入する金額は、前二条の規定にかかわらず、当該合計額とする。
\end{description}
\noindent\hspace{10pt}(減価償却資産の償却累積額による償却費の特例)
\begin{description}
\item[第百三十四条]居住者の有する次の各号に掲げる減価償却資産の償却費としてその者のその年の前年分以前の各年分の不動産所得の金額、事業所得の金額、山林所得の金額又は雑所得の金額の計算上必要経費に算入された金額の累積額と当該減価償却資産につき当該各号に規定する償却の方法により計算したその年分の償却費の額に相当する金額との合計額が当該各号に掲げる減価償却資産の区分に応じ当該各号に定める金額を超える場合には、当該減価償却資産については、第百三十一条から前条までの規定にかかわらず、当該償却費の額に相当する金額からその超える部分の金額を控除した金額をもつてその年分の償却費の額とする。
\begin{description}
\item[一]平成十九年三月三十一日以前に取得されたもの(ニ及びホに掲げる減価償却資産にあつては、当該減価償却資産についての第百二十条第一項第六号(減価償却資産の償却の方法)に規定する改正前リース取引に係る契約が平成二十年三月三十一日までに締結されたもの)で、そのよるべき償却の方法として旧定額法、旧定率法、旧生産高比例法、旧国外リース期間定額法、第百二十条の三第一項(減価償却資産の特別な償却の方法)に規定する償却の方法又は第百二十一条の二第一項(リース賃貸資産の償却の方法の特例)に規定する旧リース期間定額法を採用しているもの
\begin{description}
\item[イ]第六条第一号から第七号まで(減価償却資産の範囲)に掲げる減価償却資産(坑道並びにニ及びホに掲げる減価償却資産を除く。)
\item[ロ]坑道及び第六条第八号に掲げる無形固定資産(ホに掲げる減価償却資産を除く。)
\item[ハ]第六条第九号に掲げる生物(ホに掲げる減価償却資産を除く。)
\item[ニ]第百二十条第一項第六号に掲げる減価償却資産
\item[ホ]第百二十一条の二第一項の規定の適用を受けている同項に規定するリース賃貸資産
\end{description}
\item[二]平成十九年四月一日以後に取得されたもの(ハに掲げる減価償却資産にあつては、当該減価償却資産についての第百二十条の二第二項第五号(減価償却資産の償却の方法)に規定する所有権移転外リース取引に係る契約が平成二十年四月一日以後に締結されたもの)で、そのよるべき償却の方法として定額法、定率法、生産高比例法、リース期間定額法又は第百二十条の三第一項に規定する償却の方法を採用しているもの
\begin{description}
\item[イ]第六条第一号から第七号まで及び第九号に掲げる減価償却資産(坑道及びハに掲げる減価償却資産を除く。)
\item[ロ]坑道及び第六条第八号に掲げる無形固定資産
\item[ハ]第百二十条の二第一項第六号に掲げる減価償却資産
\end{description}
\end{description}
\item[\rensuji{2}]居住者の有する前項第一号イ又はハに掲げる減価償却資産(そのよるべき償却の方法として同号に規定する償却の方法を採用しているものに限る。)の償却費としてその者のその年の前年分以前の各年分の不動産所得の金額、事業所得の金額、山林所得の金額又は雑所得の金額の計算上必要経費に算入された金額の累積額が当該減価償却資産の同号イ又はハに定める金額に達している場合には、当該減価償却資産については、第百三十一条から前条まで及び同項の規定にかかわらず、当該減価償却資産の取得価額から同号イ又はハに定める金額及び一円を控除した金額を五で除して計算した金額(当該計算した金額と当該減価償却資産の償却費としてその者のその年の前年分以前の各年分の不動産所得の金額、事業所得の金額、山林所得の金額又は雑所得の金額の計算上必要経費に算入された金額の累積額との合計額が当該減価償却資産の取得価額から一円を控除した金額を超える場合には、その超える部分の金額を控除した金額)をもつてその年分の償却費の額とする。
\item[\rensuji{3}]第百三十二条(年の中途で業務の用に供した減価償却資産等の償却費の特例)の規定は、前項の規定の適用を受ける減価償却資産について準用する。
\end{description}
\noindent\hspace{10pt}(堅牢な建物等の償却費の特例)
\begin{description}
\item[第百三十四条の二]居住者の有する次に掲げる減価償却資産(前条第一項第一号の規定の適用を受けるものに限る。)のうち、その償却費としてその年の前年分以前の各年分の不動産所得の金額、事業所得の金額、山林所得の金額又は雑所得の金額の計算上必要経費に算入された金額の累積額がその取得価額の百分の九十五に相当する金額に達したものが、なおその者のこれらの所得を生ずべき業務の用に供されている場合には、第百三十一条から前条までの規定にかかわらず、当該資産がなお当該業務の用に供されている間に限り、当該資産の取得価額の百分の五に相当する金額から一円を控除した金額を当該資産の第百二十九条(減価償却資産の耐用年数、償却率等)に規定する財務省令で定める耐用年数の十分の三に相当する年数で除して計算した金額は、当該資産の償却費としてその者のその年分以後の各年分の不動産所得の金額、事業所得の金額、山林所得の金額又は雑所得の金額の計算上、必要経費に算入することができる。
\begin{description}
\item[一]鉄骨鉄筋コンクリート造、鉄筋コンクリート造、れんが造、石造又はブロック造の建物
\item[二]鉄骨鉄筋コンクリート造、鉄筋コンクリート造、コンクリート造、れんが造、石造又は土造の構築物又は装置
\end{description}
\item[\rensuji{2}]前項の規定により耐用年数の十分の三に相当する年数を計算する場合において、一年未満の端数を生じたときは、これを一年とする。
\item[\rensuji{3}]第百三十二条(年の中途で業務の用に供した減価償却資産等の償却費の特例)の規定は、第一項の規定の適用を受ける減価償却資産について準用する。
\end{description}
\noindent\hspace{10pt}(非事業用資産を業務の用に供した場合の償却費の計算の特例)
\begin{description}
\item[第百三十五条]居住者がその有する家屋その他使用又は期間の経過により減価する資産で不動産所得、事業所得、山林所得又は雑所得を生ずべき業務の用に供していないものを当該業務の用に供した場合(次条の規定に該当する場合を除く。)には、当該業務の用に供した後における当該資産の償却費の額は、当該業務の用に供した日に当該資産の譲渡があつたものとみなして法第三十八条第二項(譲渡所得の金額の計算上控除する取得費)の規定を適用した場合に当該資産の取得費とされる金額に相当する金額を同日における当該資産の償却後の価額として計算するものとし、当該資産の第百二十六条(減価償却資産の取得価額)及び第百二十七条第二項(資本的支出の取得価額の特例)の規定に準じて計算した取得価額と当該償却後の価額との差額に相当する金額は、第百三十四条(減価償却資産の償却累積額による償却費の特例)及び前条の規定の適用については、当該資産の償却費としてその者の各年分の不動産所得の金額、事業所得の金額、山林所得の金額又は雑所得の金額の計算上必要経費に算入された金額とみなすものとする。
\end{description}
\noindent\hspace{10pt}(昭和二十七年十二月三十一日以前に取得した非事業用資産を業務の用に供した場合の償却費の計算の特例)
\begin{description}
\item[第百三十六条]居住者が昭和二十七年十二月三十一日以前から引き続き所有していた前条に規定する資産を同条の業務の用に供した場合には、当該業務の用に供した後における当該資産の償却費の額は、当該業務の用に供した日に当該資産の譲渡があつたものとみなして法第六十一条第三項(昭和二十七年十二月三十一日以前に取得した資産の取得費等)の規定を適用した場合に当該資産の取得費とされる金額に相当する金額を同日における当該資産の償却後の価額として計算するものとし、当該資産の第百二十八条(昭和二十七年十二月三十一日以前に取得した非事業用資産で業務の用に供されたものの取得価額)の規定による取得価額と当該償却後の価額との差額に相当する金額は、第百三十四条(減価償却資産の償却累積額による償却費の特例)及び第百三十四条の二(堅牢な建物等の償却費の特例)の規定の適用については、当該資産の償却費としてその者の各年分の不動産所得の金額、事業所得の金額、山林所得の金額又は雑所得の金額の計算上必要経費に算入された金額とみなすものとする。
\end{description}
\subsubsubsection*{第四目 減価償却資産の償却費の計算の細目}
{第四目 減価償却資産の償却費の計算の細目}
\begin{description}
\item[第百三十六条の二]前三目(減価償却資産の償却の方法等)に定めるもののほか、減価償却資産の償却費の計算に関する細目は、財務省令で定める。
\end{description}
\subsubsection*{第五款 繰延資産の償却}
\addcontentsline{toc}{subsubsection}{第五款 繰延資産の償却}
\noindent\hspace{10pt}(繰延資産の償却費の計算)
\begin{description}
\item[第百三十七条]法第五十条第一項(繰延資産の償却費の計算及びその償却の方法)に規定する政令で定めるところにより計算した金額は、次の各号に掲げる繰延資産の区分に応じ当該各号に定める金額とする。
\begin{description}
\item[一]第七条第一項第一号又は第二号(繰延資産の範囲)に掲げる繰延資産
\item[二]第七条第一項第三号に掲げる繰延資産
\end{description}
\item[\rensuji{2}]前項の月数は、暦に従つて計算し、一月に満たない端数を生じたときは、これを一月とする。
\item[\rensuji{3}]居住者が、第一項第一号に掲げる繰延資産につきその年分の不動産所得の金額、事業所得の金額、山林所得の金額又は雑所得の金額の計算上必要経費に算入すべき金額として、当該繰延資産の額の範囲内の金額をその年分の確定申告書に記載した場合には、同号に掲げる金額は、同号の規定にかかわらず、当該金額として記載された金額とする。
\end{description}
\subsubsection*{第六款 少額の減価償却資産等の取得価額の必要経費算入}
\addcontentsline{toc}{subsubsection}{第六款 少額の減価償却資産等の取得価額の必要経費算入}
\noindent\hspace{10pt}(少額の減価償却資産の取得価額の必要経費算入)
\begin{description}
\item[第百三十八条]居住者が不動産所得、事業所得、山林所得又は雑所得を生ずべき業務の用に供した減価償却資産(第百二十条第一項第六号及び第百二十条の二第一項第六号(減価償却資産の償却の方法)に掲げるものを除く。)で、第百八十一条第一号(資本的支出)に規定する使用可能期間が一年未満であるもの又は取得価額(第百二十六条第一項各号若しくは第二項(減価償却資産の取得価額)の規定により計算した価額をいう。次条第一項において同じ。)が十万円未満であるものについては、第四款(減価償却資産の償却)の規定にかかわらず、その取得価額に相当する金額を、その者のその業務の用に供した年分の不動産所得の金額、事業所得の金額、山林所得の金額又は雑所得の金額の計算上、必要経費に算入する。
\end{description}
\noindent\hspace{10pt}(一括償却資産の必要経費算入)
\begin{description}
\item[第百三十九条]居住者が不動産所得、事業所得、山林所得又は雑所得を生ずべき業務の用に供した減価償却資産で取得価額が二十万円未満であるもの(第百二十条第一項第六号及び第百二十条の二第一項第六号(減価償却資産の償却の方法)に掲げるもの並びに前条の規定の適用があるものを除く。)については、その居住者が当該減価償却資産の全部又は特定の一部を一括し、その一括した減価償却資産(以下この条において「一括償却資産」という。)の取得価額の合計額をその業務の用に供した年以後三年間の各年の費用の額とする方法を選択したときは、第四款(減価償却資産の償却)の規定にかかわらず、当該一括償却資産につき当該各年分の不動産所得の金額、事業所得の金額、山林所得の金額又は雑所得の金額の計算上必要経費に算入する金額は、当該一括償却資産の取得価額の合計額(以下この条において「一括償却対象額」という。)を三で除して計算した金額とする。
\item[\rensuji{2}]前項の規定は、一括償却資産を業務の用に供した日の属する年分の確定申告書に一括償却対象額を記載した書類を添付し、かつ、その計算に関する書類を保存している場合に限り、適用する。
\item[\rensuji{3}]居住者は、その年において一括償却対象額につき必要経費に算入した金額がある場合には、その年分の確定申告書に、第一項の規定により必要経費に算入される金額の計算に関する明細書を添付しなければならない。
\end{description}
\noindent\hspace{10pt}(繰延資産となる費用のうち少額のものの必要経費算入)
\begin{description}
\item[第百三十九条の二]居住者が支出する第七条第一項第三号(繰延資産の範囲)に掲げる費用のうちその支出する金額が二十万円未満であるものについては、前款(繰延資産の償却)の規定にかかわらず、その支出する金額に相当する金額を、その者のその支出する日の属する年分の不動産所得の金額、事業所得の金額、山林所得の金額又は雑所得の金額の計算上、必要経費に算入する。
\end{description}
\subsubsection*{第七款 資産損失}
\addcontentsline{toc}{subsubsection}{第七款 資産損失}
\noindent\hspace{10pt}(固定資産に準ずる資産の範囲)
\begin{description}
\item[第百四十条]法第五十一条第一項(資産損失の必要経費算入)に規定する政令で定める資産は、不動産所得、事業所得又は山林所得を生ずべき事業に係る繰延資産のうちまだ必要経費に算入されていない部分とする。
\end{description}
\noindent\hspace{10pt}(必要経費に算入される損失の生ずる事由)
\begin{description}
\item[第百四十一条]法第五十一条第二項(資産損失の必要経費算入)に規定する政令で定める事由は、次に掲げる事由で不動産所得、事業所得又は山林所得を生ずべき事業の遂行上生じたものとする。
\begin{description}
\item[一]販売した商品の返戻又は値引き(これらに類する行為を含む。)により収入金額が減少することとなつたこと。
\item[二]保証債務の履行に伴う求償権の全部又は一部を行使することができないこととなつたこと。
\item[三]不動産所得の金額、事業所得の金額若しくは山林所得の金額の計算の基礎となつた事実のうちに含まれていた無効な行為により生じた経済的成果がその行為の無効であることに基因して失われ、又はその事実のうちに含まれていた取り消すことのできる行為が取り消されたこと。
\end{description}
\end{description}
\noindent\hspace{10pt}(必要経費に算入される資産損失の金額)
\begin{description}
\item[第百四十二条]次の各号に掲げる資産について生じた法第五十一条第一項、第三項又は第四項(資産損失の必要経費算入)に規定する損失の金額の計算の基礎となるその資産の価額は、当該各号に掲げる金額とする。
\begin{description}
\item[一]固定資産
\item[二]山林
\item[三]繰延資産
\end{description}
\end{description}
\noindent\hspace{10pt}(昭和二十七年十二月三十一日以前に取得した資産の損失の金額の特例)
\begin{description}
\item[第百四十三条]次の各号に掲げる資産について生じた法第五十一条第一項、第三項又は第四項(資産損失の必要経費算入)に規定する損失の金額の計算の基礎となるその資産の価額は、前条第一号及び第二号の規定にかかわらず、当該各号に掲げる金額とする。
\begin{description}
\item[一]昭和二十七年十二月三十一日以前から引き続き所有していた固定資産
\item[二]昭和二十七年十二月三十一日以前から引き続き所有していた山林
\end{description}
\end{description}
\subsubsection*{第八款 引当金}
\addcontentsline{toc}{subsubsection}{第八款 引当金}
\subsubsubsection*{第一目 貸倒引当金}
{第一目 貸倒引当金}
\noindent\hspace{10pt}(個別評価貸金等に係る貸倒引当金勘定への繰入限度額)
\begin{description}
\item[第百四十四条]法第五十二条第一項(貸倒引当金)に規定する政令で定める事実は、次の各号に掲げる事実とし、同項に規定する政令で定めるところにより計算した金額は、当該各号に掲げる事実の区分に応じ当該各号に定める金額とする。
\begin{description}
\item[一]法第五十二条第一項の居住者がその年十二月三十一日(その者が年の中途において死亡した場合には、その死亡の時。以下この項において同じ。)において有する貸金等(同条第一項に規定する貸金等をいう。以下この条において同じ。)につき、当該貸金等に係る債務者について生じた次に掲げる事由に基づいてその弁済を猶予され、又は賦払により弁済されること
\begin{description}
\item[イ]更生計画認可の決定
\item[ロ]再生計画認可の決定
\item[ハ]特別清算に係る協定の認可の決定
\item[ニ]イからハまでに掲げる事由に準ずるものとして財務省令で定める事由
\end{description}
\item[二]法第五十二条第一項の居住者がその年十二月三十一日において有する貸金等に係る債務者につき、債務超過の状態が相当期間継続し、かつ、その営む事業に好転の見通しがないこと、災害、経済事情の急変等により多大な損害が生じたことその他の事由により、当該貸金等の一部の金額につきその取立て等の見込みがないと認められること(当該貸金等につき前号に掲げる事実が生じている場合を除く。)
\item[三]法第五十二条第一項の居住者がその年十二月三十一日において有する貸金等に係る債務者につき次に掲げる事由が生じていること(当該貸金等につき、第一号に掲げる事実が生じている場合及び前号に掲げる事実が生じていることにより同項の規定の適用を受けた場合を除く。)
\begin{description}
\item[イ]更生手続開始の申立て
\item[ロ]再生手続開始の申立て
\item[ハ]破産手続開始の申立て
\item[ニ]特別清算開始の申立て
\item[ホ]イからニまでに掲げる事由に準ずるものとして財務省令で定める事由
\end{description}
\item[四]法第五十二条第一項の居住者がその年十二月三十一日において有する貸金等に係る債務者である外国の政府、中央銀行又は地方公共団体の長期にわたる債務の履行遅滞によりその貸金等の経済的な価値が著しく減少し、かつ、その弁済を受けることが著しく困難であると認められること
\end{description}
\item[\rensuji{2}]居住者の有する貸金等について前項各号に掲げる事実が生じている場合においても、当該事実が生じていることを証する書類その他の財務省令で定める書類の保存がされていないときは、当該貸金等に係る同項の規定の適用については、当該事実は、生じていないものとみなす。
\item[\rensuji{3}]税務署長は、前項の書類の保存がない場合においても、その書類の保存がなかつたことについてやむを得ない事情があると認めるときは、その書類の保存のなかつた貸金等に係る金額につき同項の規定を適用しないことができる。
\end{description}
\noindent\hspace{10pt}(一括評価貸金に係る貸倒引当金勘定への繰入限度額)
\begin{description}
\item[第百四十五条]法第五十二条第二項(貸倒引当金)に規定する政令で定めるところにより計算した金額は、同項の居住者のその年十二月三十一日(その者が年の中途において死亡した場合には、その死亡の時)において有する一括評価貸金(同項に規定する一括評価貸金をいう。以下この条において同じ。)の帳簿価額(当該一括評価貸金のうち当該居住者が当該一括評価貸金に係る債務者から受け入れた金額があるためその全部又は一部が実質的に債権とみられないものにあつては、その債権とみられない部分の金額に相当する金額を控除した残額。次項において同じ。)の合計額に、その者の営む事業所得を生ずべき事業のうち主たるものが次の各号に掲げる事業のいずれに該当するかに応じ当該各号に定める割合を乗じて計算した金額とする。
\begin{description}
\item[一]金融業以外の事業
\item[二]金融業
\end{description}
\item[\rensuji{2}]前項の一括評価貸金の帳簿価額の計算については、同項の居住者で平成二十七年一月一日以後引き続き事業所得を生ずべき事業を営んでいるものは、同項の規定にかかわらず、その年十二月三十一日(その者が年の中途において死亡した場合には、その死亡の時)における一括評価貸金の額に、平成二十七年及び平成二十八年の各年の十二月三十一日における一括評価貸金の額の合計額のうちに当該各年の十二月三十一日における同項に規定する債権とみられない部分の金額の合計額の占める割合(当該割合に小数点以下三位未満の端数があるときは、これを切り捨てる。)を乗じて計算した金額をもつて、同項に規定する債権とみられない部分の金額に相当する金額とすることができる。
\end{description}
\noindent\hspace{10pt}(貸倒引当金勘定への繰入れが認められない場合)
\begin{description}
\item[第百四十六条]法第五十二条第二項(貸倒引当金)に規定する政令で定める場合は、同項の居住者が死亡した場合において、その相続人のうちに、その居住者の同項に規定する事業を承継した者でその死亡の日の属する年分の所得税につき青色申告書を提出することについて税務署長の承認を受けているもの(当該所得税につき法第百四十四条(青色申告の承認の申請)の申請書を提出したものを含む。)がないときとする。
\end{description}
\noindent\hspace{10pt}(死亡の場合の貸倒引当金勘定の金額の処理)
\begin{description}
\item[第百四十七条]法第五十二条第一項又は第二項(貸倒引当金)の居住者が死亡した場合において、これらの規定によりその居住者の死亡の日の属する年分の事業所得の金額の計算上必要経費に算入された貸倒引当金勘定の金額があるときは、当該貸倒引当金勘定の金額は、次の各号に掲げる貸倒引当金勘定の金額の区分に応じ、当該各号に定める相続人の当該年分の事業所得の金額の計算上、総収入金額に算入する。
\begin{description}
\item[一]法第五十二条第一項の規定によりその年分の必要経費に算入された貸倒引当金勘定の金額
\item[二]法第五十二条第二項の規定によりその年分の必要経費に算入された貸倒引当金勘定の金額
\end{description}
\end{description}
\begin{description}
\item[第百四十八条]削除
\end{description}
\begin{description}
\item[第百四十九条]削除
\end{description}
\begin{description}
\item[第百五十条]削除
\end{description}
\begin{description}
\item[第百五十一条]削除
\end{description}
\begin{description}
\item[第百五十二条]削除
\end{description}
\subsubsubsection*{第二目 退職給与引当金}
{第二目 退職給与引当金}
\noindent\hspace{10pt}(退職給与規程の範囲)
\begin{description}
\item[第百五十三条]法第五十四条第一項(退職給与引当金)に規定する政令で定める退職給与規程は、次に掲げる規程とする。
\begin{description}
\item[一]労働協約により定められる退職給与の支給に関する規程
\item[二]労働基準法第八十九条(就業規則の作成及び届出の義務)又は船員法第九十七条第二項(就業規則の作成及び届出)の規定により行政官庁に届け出られた就業規則により定められる退職給与の支給に関する規程
\item[三]労働基準法第八十九条又は船員法第九十七条の規定の適用を受けない居住者がその作成した退職給与の支給に関する規程をあらかじめ納税地の所轄税務署長に届け出た場合における当該規程
\end{description}
\end{description}
\noindent\hspace{10pt}(退職給与引当金勘定への繰入限度額)
\begin{description}
\item[第百五十四条]法第五十四条第一項(退職給与引当金)に規定する政令で定めるところにより計算した金額は、次に掲げる金額のうちいずれか少ない金額とする。
\begin{description}
\item[一]イに掲げる金額からロに掲げる金額を控除した金額
\begin{description}
\item[イ]その年十二月三十一日(法第五十四条第一項の居住者が年の中途において死亡した場合には、その死亡の時。以下この条において同じ。)において在職する使用人の全員が同日において自己の都合により退職するものと仮定した場合に各使用人につき同日現在において定められている法第五十四条第一項に規定する退職給与規程(同一の使用人につき前条第一号に掲げる規程と同条第二号又は第三号に掲げる規程とが共に適用されることとなつている場合には、同条第一号に掲げる規程。以下第百五十八条までにおいて「退職給与規程」という。)により計算される退職給与の額の合計額(以下この条において「期末退職給与の要支給額」という。)
\item[ロ]イに規定する使用人のうちその年の前年十二月三十一日から引き続き在職している者の全員が同日において自己の都合により退職するものと仮定した場合に各使用人につき同日現在において定められている退職給与規程(同日において退職給与規程が定められていない場合には、その後最初に定められた退職給与規程)により計算される退職給与の額の合計額
\end{description}
\item[二]累積限度額(期末退職給与の要支給額の百分の二十に相当する金額をいう。次条第一項において同じ。)から、その年十二月三十一日におけるその年の前年から繰り越された法第五十四条第二項に規定する退職給与引当金勘定の金額(その年における相続(包括遺贈を含む。)によつて第百五十七条第二項(死亡の場合の退職給与引当金勘定の金額の処理)の規定により当該居住者が有するものとみなされた退職給与引当金勘定の金額がある場合には、当該退職給与引当金勘定の金額を含む。)を控除した金額
\end{description}
\item[\rensuji{2}]前項の場合において、その年十二月三十一日において前条第一号に掲げる規程を定めていない居住者(第百五十八条第一項又は第二項(退職給与規程に関する書類の提出)の規定により提出する書類(同項の規定による書類の提出が二回以上あつた場合には、最近の時期において提出した当該書類)に、労働基準法第九十条第一項(作成の手続)若しくは船員法第九十八条(就業規則の作成の手続)の意見を記載した書面及び労働基準法第百六条第一項(法令等の周知義務)の労働者への周知若しくは船員法第百十三条第一項(就業規則等の掲示等)の掲示若しくは備置きを行つた事実の詳細を記載した書面で前条第二号に掲げる規程に係るもの又は財務省令で定めるこれらの書面に準ずる書面で同条第三号に掲げる規程に係るものを添付して税務署長に提出した居住者を除く。)については、前項第一号に掲げる金額が同日において在職する使用人(日日雇い入れられる者、臨時に期間を定めて雇い入れられる者その他の者で退職給与の支給の対象とならないものを除く。)に係る給料、賃金、賞与及びこれらの性質を有する給与でその年分の事業所得の金額の計算上必要経費に算入されるものの総額の百分の六に相当する金額を超えるときは、同号の金額は、当該給与の総額の百分の六に相当する金額とする。
\end{description}
\noindent\hspace{10pt}(退職給与引当金勘定の金額の取崩し)
\begin{description}
\item[第百五十五条]法第五十四条第二項(退職給与引当金)に規定する退職給与引当金勘定の金額(以下この条において「退職給与引当金勘定の金額」という。)を有する居住者は、次の各号に掲げる場合に該当することとなつたときは、次項の規定に該当する場合を除き、当該各号に定める退職給与引当金勘定の金額を取り崩さなければならない。
\begin{description}
\item[一]使用人が退職した場合において、その使用人がその年の前年十二月三十一日において自己の都合により退職するものと仮定した場合に同日現在において定められている退職給与規程により退職給与の支給を受けるべきとき。
\item[二]その年十二月三十一日(法第五十四条第一項の居住者が年の中途において死亡した場合には、その死亡の時)における退職給与引当金勘定の金額が累積限度額を超えるに至つた場合
\item[三]正当の理由がないのに退職給与規程に基づく退職給与を支給しない事実があつた場合
\item[四]第百五十三条各号(退職給与規程の範囲)に掲げる規程のすべてが存在しないこととなつた場合
\item[五]明らかに所得税を免れる目的で退職給与規程を改正したと認められる事実があつた場合
\item[六]事業所得を生ずべき事業の全部を譲渡し又は廃止した場合
\item[七]退職給与引当金勘定の金額を第一号及び第二号に掲げる場合以外の場合に取り崩した場合
\end{description}
\item[\rensuji{2}]退職給与引当金勘定の金額を有する居住者が青色申告書の提出の承認を取り消され、又は青色申告書による申告をやめる旨の届出書の提出をした場合には、その取消しの基因となつた事実のあつた日又はその届出書の提出をした日(その届出書の提出をした日がその申告をやめた年の翌年である場合には、そのやめた年の十二月三十一日)の属する年並びにその翌年及び翌翌年において、それぞれ、これらの日における退職給与引当金勘定の金額の三分の一に相当する金額を取りくずさなければならない。
\end{description}
\noindent\hspace{10pt}(退職金共済契約等を締結している場合の繰入限度額の特例等)
\begin{description}
\item[第百五十六条]居住者が、独立行政法人勤労者退職金共済機構若しくは第七十四条第五項(特定退職金共済団体の承認)に規定する特定退職金共済団体が行う退職金共済に関する制度に該当する退職金共済契約その他これに類する契約(以下この条において「退職金共済契約等」という。)若しくは法人税法附則第二十条第三項(退職年金等積立金に対する法人税の特例)に規定する適格退職年金契約(以下この条において「適格退職年金契約」という。)その他これに類する契約(以下この条において「適格退職年金契約等」という。)を締結している場合、平成二十五年厚生年金等改正法附則第三条第十二号(定義)に規定する厚生年金基金(以下この条において「厚生年金基金」という。)を設立している場合又は確定給付企業年金法第二条第一項(定義)に規定する確定給付企業年金(以下この条において「確定給付企業年金」という。)若しくは確定拠出年金法第二条第二項(定義)に規定する企業型年金(以下この条において「確定拠出企業型年金」という。)を実施している場合における前二条の規定の適用については、次に定めるところによる。
\begin{description}
\item[一]退職給与規程において使用人に支給する退職給与のうちに退職金共済契約等若しくは適格退職年金契約等に基づく給付金又は確定給付企業年金法第三条第一項(確定給付企業年金の実施)に規定する確定給付企業年金に係る規約(以下この条において「確定給付企業年金規約」という。)に基づく給付金を含む旨を定めている場合には、当該使用人に係る第百五十四条第一項第一号イ又はロ(退職給与引当金勘定への繰入限度額)に規定する退職給与の額は、当該使用人が自己の都合により退職するものと仮定した場合に当該退職給与規程により計算される退職給与の額のうち当該退職金共済契約等又は適格退職年金契約等に基づく給付金及び当該確定給付企業年金規約に基づく給付金以外の給与(以下この条において「事業主の支給する退職給与」という。)の額による。
\item[二]次に掲げる場合には、その年十二月三十一日(その居住者が年の中途において死亡した場合には、その死亡の時。以下この条において同じ。)において在職する使用人に係る第百五十四条第一項第一号ロに規定する退職給与の額は、当該使用人につき同日における退職給与規程がその年の前年十二月三十一日において適用されるものとした場合に当該使用人につき支給すべきこととなる事業主の支給する退職給与の額による。
\begin{description}
\item[イ]退職給与規程の改正、退職金共済契約等若しくは適格退職年金契約等の変更又は確定給付企業年金規約の変更により、その年十二月三十一日において在職する使用人のうちその年の前年十二月三十一日から引き続き在職しているものに対する退職給与について、同日においては退職給与として支給されることとなつていた金額の全部又は一部がその年十二月三十一日においては退職金共済契約等若しくは適格退職年金契約等に基づく給付金、厚生年金基金からの給付金又は確定給付企業年金規約に基づく給付金として支給されることとなつた場合
\item[ロ]確定拠出企業型年金の実施又は確定拠出年金法第四条第三項(承認の基準等)に規定する企業型年金規約の変更により、退職給与規程を改正し、その年十二月三十一日において在職する使用人のうちその年の前年十二月三十一日から引き続き在職しているものに対する退職給与について、同日においては退職給与として支給されることとなつていた金額の全部又は一部に相当する金額がその年十二月三十一日においては同法第五十四条第一項(他の制度の資産の移換)の企業型年金の資産管理機関に払い込まれている場合
\end{description}
\item[三]適格退職年金契約を締結している居住者、厚生年金基金を設立している居住者又は確定給付企業年金若しくは確定拠出企業型年金を実施している居住者で、その年以前の各年において前号イ又はロに掲げる場合に該当することとなつたことに伴い、その該当することとなつた日の属する年においてこの号の規定を適用しないで計算した場合における前条第一項第二号に定める金額(以下この号において「調整前累積限度超過額」という。)が生ずることとなつたものについては、その調整前累積限度超過額が最初に生ずることとなつた年からその年十二月三十一日におけるその年の前年から繰り越された法第五十四条第二項(退職給与引当金)に規定する退職給与引当金勘定の金額(その年における相続(包括遺贈を含む。)によつて次条第二項の規定により当該居住者が有するものとみなされた退職給与引当金勘定の金額がある場合には、当該退職給与引当金勘定の金額を含む。イにおいて「繰越退職給与引当金勘定の金額」という。)が同日におけるこの号の規定を適用しないで計算した前条第一項第二号に規定する累積限度額(以下この号において「調整前累積限度額」という。)以下となる最初の年の前年までの各年の同項第二号に規定する累積限度額は、イ又はロに掲げる金額のうちいずれか少ない金額とする。
\begin{description}
\item[イ]その年十二月三十一日における繰越退職給与引当金勘定の金額
\item[ロ]その年の調整前累積限度額に、調整前累積限度超過額を七で除してこれに七から前号イ又はロに掲げる場合に該当することとなつた日の属する年の翌年一月一日からその年十二月三十一日までの年数に相当する数(その数が七を超えるときは、七。以下この号において「経過期間の年数」という。)を控除した数を乗じて計算した金額(その該当することとなつた日の属する年の翌年からその年までの間に支出した法人税法施行令第百五十六条の二第四号(用語の意義)に規定する過去勤務掛金額その他財務省令で定める金額の合計額が、調整前累積限度超過額に経過期間の年数を乗じて七で除して計算した金額を超える場合には、その超える部分の金額に相当する金額を控除した残額)を加算した金額(その該当することとなつた日の属する年については、当該年の調整前累積限度額と調整前累積限度超過額との合計額)
\end{description}
\end{description}
\end{description}
\noindent\hspace{10pt}(死亡の場合の退職給与引当金勘定の金額の処理)
\begin{description}
\item[第百五十七条]法第五十四条第二項(退職給与引当金)に規定する退職給与引当金勘定の金額(以下この条において「退職給与引当金勘定の金額」という。)を有する居住者が死亡した場合には、その死亡の時における退職給与引当金勘定の金額のうち次に掲げる金額は、その者のその死亡の日の属する年分の事業所得の金額の計算上、総収入金額に算入する。
\begin{description}
\item[一]その居住者の相続人のうちに、居住者の事業所得を生ずべき事業を承継してその居住者の使用人を引き続き雇用している者でその居住者の死亡の日の属する年分の所得税につき青色申告書を提出することについて税務署長の承認を受けているもの(当該所得税につき法第百四十四条(青色申告の承認の申請)の申請書を提出したものを含む。)がない場合には、当該退職給与引当金勘定の金額の全額
\item[二]その居住者の相続人のうちに前号に規定する者がある場合には、当該退職給与引当金勘定の金額から、当該金額にその居住者の死亡の時における第百五十四条第一項(退職給与引当金勘定への繰入限度額)に規定する期末退職給与の要支給額のうちにその相続人が引き続き雇用する前号の使用人に係る当該期末退職給与の要支給額の占める割合を乗じて計算した金額を控除した金額
\end{description}
\item[\rensuji{2}]退職給与引当金勘定の金額を有する居住者が死亡した場合において、前項第二号に規定する場合に該当するときは、その死亡の時における退職給与引当金勘定の金額のうち同号に掲げる金額以外の部分の金額は、前三条及び前項の規定の適用については、その居住者の相続人が当該死亡の時において有する退職給与引当金勘定の金額とみなす。
\item[\rensuji{3}]前項の規定の適用を受けた相続人が同項の居住者の死亡の日の属する年分の所得税につき法第百四十四条の申請書を提出した者である場合において、その申請が却下されたときは、当該相続人は、その却下の日における同項の退職給与引当金勘定の金額をとりくずさなければならない。
\item[\rensuji{4}]相続(包括遺贈を含む。以下この条において同じ。)により被相続人の事業所得を生ずべき事業を承継した居住者でその相続の日の属する年分の所得税につき青色申告書を提出することについて税務署長の承認を受けているもの(当該所得税につき法第百四十四条の申請書を提出したもののうち前項の規定に該当しないものを含む。)が、その年において、被相続人の使用人で引き続き在職するもののうち被相続人から退職給与の支給を受けなかつた者の退職による退職給与に充てるため退職給与引当金勘定に繰り入れた金額については、当該被相続人の死亡の日を第百五十四条第一項第一号ロ(退職給与引当金勘定への繰入限度額)に規定する前年十二月三十一日とみなし、かつ、被相続人がその死亡の日において退職給与規程を定めていた者である場合には当該退職給与規程を当該前年十二月三十一日現在において定められている退職給与規程とみなして、同号の金額を計算する。
\item[\rensuji{5}]前項に規定する居住者が、その相続の日の属する年において、その被相続人(その死亡の日において第二項の規定により当該居住者が有するものとみなされる退職給与引当金勘定の金額があるものに限る。)の使用人で引き続き在職するもののうち当該被相続人から退職給与の支給を受けなかつた者の退職につき第百五十五条第一項第一号(退職給与引当金勘定の金額の取崩し)の規定により取り崩すべき退職給与引当金勘定の金額の計算については、同日を同号に規定する前年十二月三十一日とみなし、かつ、当該被相続人がその死亡の日において定めていた退職給与規程を当該前年十二月三十一日現在において定められている退職給与規程とみなして同号の退職給与の額を計算するものとする。
\end{description}
\noindent\hspace{10pt}(退職給与規程に関する書類の提出)
\begin{description}
\item[第百五十八条]新たに法第五十四条第一項(退職給与引当金)の規定の適用を受けようとする居住者は、その年の前年十二月三十一日における退職給与規程(同日において退職給与規程が定められていない場合には、その後最初に定められた退職給与規程)及びその年十二月三十一日(その者が年の中途で死亡した場合には、その死亡の時)までに退職給与規程が改正された場合にはその改正後のすべての退職給与規程の写しを、その年分の所得税に係る確定申告期限までに、納税地の所轄税務署長に提出しなければならない。
\item[\rensuji{2}]法第五十四条第一項の規定の適用を受けた居住者でその後引き続いて同項の規定の適用を受けようとするものは、退職給与規程若しくは労働協約のうち退職給与の支給に関する事項について異動を生じたとき、又は新たに退職給与の支給に関する労働協約を結んだときは、すみやかに、その旨及び異動後の退職給与規程若しくは労働協約のうち退職給与の支給に関する事項又は新たに結ばれた労働協約の退職給与の支給に関する事項を記載した書類を納税地の所轄税務署長に提出しなければならない。
\end{description}
\noindent\hspace{10pt}(労働協約が失効した場合の処理)
\begin{description}
\item[第百五十九条]退職給与の支給に関する労働協約の効力が消滅した後新たな退職給与の支給に関する労働協約が結ばれていない場合には、その効力の消滅した後六月は、当該従前の労働協約がなお有効に存続するものとみなして、法第五十四条(退職給与引当金)及び第百五十三条から前条までの規定を適用する。
\end{description}
\begin{description}
\item[第百六十条]削除
\end{description}
\begin{description}
\item[第百六十一条]削除
\end{description}
\begin{description}
\item[第百六十二条]削除
\end{description}
\begin{description}
\item[第百六十三条]削除
\end{description}
\subsubsection*{第九款 専従者控除}
\addcontentsline{toc}{subsubsection}{第九款 専従者控除}
\noindent\hspace{10pt}(青色事業専従者給与の判定基準等)
\begin{description}
\item[第百六十四条]法第五十七条第一項(事業に専従する親族がある場合の必要経費の特例等)に規定する政令で定める状況は、次に掲げる状況とする。
\begin{description}
\item[一]法第五十七条第一項に規定する青色事業専従者の労務に従事した期間、労務の性質及びその提供の程度
\item[二]その事業に従事する他の使用人が支払を受ける給与の状況及びその事業と同種の事業でその規模が類似するものに従事する者が支払を受ける給与の状況
\item[三]その事業の種類及び規模並びにその収益の状況
\end{description}
\item[\rensuji{2}]法第五十七条第二項に規定する書類を提出した居住者は、当該書類に記載した事項を変更する場合には、財務省令で定めるところにより、その旨その他必要な事項を記載した書類を納税地の所轄税務署長に提出しなければならない。
\end{description}
\noindent\hspace{10pt}(親族が事業に専ら従事するかどうかの判定)
\begin{description}
\item[第百六十五条]法第五十七条第一項又は第三項(事業に専従する親族がある場合の必要経費の特例等)に規定する居住者と生計を一にする配偶者その他の親族が専らその居住者の営むこれらの規定に規定する事業に従事するかどうかの判定は、当該事業に専ら従事する期間がその年を通じて六月をこえるかどうかによる。
\begin{description}
\item[一]当該事業が年の中途における開業、廃業、休業又はその居住者の死亡、当該事業が季節営業であることその他の理由によりその年中を通じて営まれなかつたこと。
\item[二]当該事業に従事する者の死亡、長期にわたる病気、婚姻その他相当の理由によりその年中を通じてその居住者と生計を一にする親族として当該事業に従事することができなかつたこと。
\end{description}
\item[\rensuji{2}]前項の場合において、同項に規定する親族につき次の各号の一に該当する者である期間があるときは、当該期間は、同項に規定する事業に専ら従事する期間に含まれないものとする。
\begin{description}
\item[一]学校教育法第一条(学校の範囲)、第百二十四条(専修学校)又は第百三十四条第一項(各種学校)の学校の学生又は生徒である者(夜間において授業を受ける者で昼間を主とする当該事業に従事するもの、昼間において授業を受ける者で夜間を主とする当該事業に従事するもの、同法第百二十四条又は同項の学校の生徒で常時修学しないものその他当該事業に専ら従事することが妨げられないと認められる者を除く。)
\item[二]他に職業を有する者(その職業に従事する時間が短い者その他当該事業に専ら従事することが妨げられないと認められる者を除く。)
\item[三]老衰その他心身の障害により事業に従事する能力が著しく阻害されている者
\end{description}
\end{description}
\noindent\hspace{10pt}(事業専従者控除の限度額の計算)
\begin{description}
\item[第百六十六条]法第五十七条第三項第二号(事業に専従する親族がある場合の必要経費の特例等)に規定する山林所得の金額は、法第三十二条第三項(山林所得の金額)に規定する残額とする。
\item[\rensuji{2}]居住者が不動産所得、事業所得又は山林所得のうち二以上の所得を生ずべき事業(法第五十七条第三項に規定する事業専従者の従事する事業に限る。)を営む場合における同項第二号の規定の適用については、当該事業に係る同号に規定する不動産所得の金額、事業所得の金額又は山林所得の金額の合計額及び当該事業に従事するすべての当該事業専従者の数を基礎として同号に掲げる金額を計算するものとする。
\end{description}
\noindent\hspace{10pt}(二以上の事業に従事した場合の事業専従者給与等の必要経費算入額の計算)
\begin{description}
\item[第百六十七条]居住者が不動産所得、事業所得又は山林所得のうち二以上の所得を生ずべき事業を営み、かつ、同一の法第五十七条第一項又は第三項(事業に専従する親族がある場合の必要経費の特例等)に規定する青色事業専従者又は事業専従者が当該二以上の所得を生ずべき事業に従事する場合における当該事業に係る不動産所得の金額、事業所得の金額又は山林所得の金額の計算上同条第一項の規定により必要経費に算入される金額(以下この条において「青色専従者給与額」という。)又は法第五十七条第三項の規定により必要経費とみなされる金額(以下この条において「事業専従者控除額」という。)は、次の各号に掲げる場合の区分に応じ当該各号に掲げる金額とする。
\begin{description}
\item[一]当該青色事業専従者又は事業専従者が当該二以上の所得を生ずべきそれぞれの事業に従事した分量が明らかである場合
\item[二]当該青色事業専従者又は事業専従者が当該二以上の所得を生ずべきそれぞれの事業に従事した分量が明らかでない場合
\end{description}
\end{description}
\subsubsection*{第十款 特定の損失等に充てるための負担金の必要経費算入}
\addcontentsline{toc}{subsubsection}{第十款 特定の損失等に充てるための負担金の必要経費算入}
\begin{description}
\item[第百六十七条の二]居住者が、各年において、農畜産物の価格の変動による損失、漁船が遭難した場合の救済の費用その他の特定の損失又は費用を補てんするための業務を主たる目的とする法人税法第二条第六号(定義)に規定する公益法人等又は一般社団法人若しくは一般財団法人の当該業務に係る資金のうち短期間に使用されるもので次に掲げる要件のすべてに該当するものとして国税庁長官が指定したものに充てるための負担金を支出した場合には、その支出した金額は、その支出した日の属する年分の事業所得の金額の計算上、必要経費に算入する。
\begin{description}
\item[一]当該資金に充てるために徴収される負担金の額が当該業務の内容からみて適正であること。
\item[二]当該資金の額が当該業務に必要な金額を超えることとなるときは、その負担金の徴収の停止その他必要な措置が講じられることとなつていること。
\item[三]当該資金が当該業務の目的に従つて適正な方法で管理されていること。
\end{description}
\end{description}
\subsubsection*{第十一款 給与所得者の特定支出}
\addcontentsline{toc}{subsubsection}{第十一款 給与所得者の特定支出}
\noindent\hspace{10pt}(給与所得者の特定支出の範囲)
\begin{description}
\item[第百六十七条の三]法第五十七条の二第二項第一号(給与所得者の特定支出の控除の特例)に規定する政令で定める支出は、次の各号に掲げる場合の区分に応じ当該各号に定める金額に相当する支出(航空機の利用に係るものを除く。)とする。
\begin{description}
\item[一]交通機関を利用する場合(第三号に掲げる場合に該当する場合を除く。)
\item[二]自動車その他の交通用具を使用する場合(次号に掲げる場合に該当する場合を除く。)
\item[三]交通機関を利用するほか、併せて自動車その他の交通用具を使用する場合
\end{description}
\item[\rensuji{2}]法第五十七条の二第二項第二号に規定する政令で定める支出は、同号に規定する旅行でその旅行に係る運賃、時間、距離その他の事情に照らし最も経済的かつ合理的と認められる通常の経路及び方法によるものに要する次に掲げる支出とする。
\begin{description}
\item[一]当該旅行に要する運賃及び料金(特別車両料金その他の客室の特別の設備の利用についての料金として財務省令で定めるものを除く。次項第一号及び第五項第一号において同じ。)
\item[二]当該旅行に要する自動車その他の交通用具の使用に係る燃料費及び有料の道路の料金
\item[三]前号の交通用具の修理のための支出(当該旅行に係る部分に限る。)
\end{description}
\item[\rensuji{3}]法第五十七条の二第二項第三号に規定する政令で定める支出は、転任の事実が生じた日以後一年以内にする同項に規定する転居のための自己又はその配偶者その他の親族に係る支出で次に掲げる金額に相当するものとする。
\begin{description}
\item[一]当該転居のための旅行に通常必要であると認められる運賃及び料金の額
\item[二]当該転居のために自動車を使用することにより支出する燃料費及び有料の道路の料金の額
\item[三]当該転居に伴う宿泊費の額(通常必要であると認められる額を著しく超える部分を除く。)
\item[四]当該転居のための生活の用に供する家具その他の資産の運送に要した費用(これに付随するものを含む。)の額
\end{description}
\item[\rensuji{4}]法第五十七条の二第二項第六号に規定する政令で定める場合は、配偶者と死別し、若しくは配偶者と離婚した後婚姻をしていない者又は配偶者の生死の明らかでない者で財務省令で定めるものが転任に伴い生計を一にする子で財務省令で定めるものとの別居を常況とすることとなつた場合とする。
\item[\rensuji{5}]法第五十七条の二第二項第六号に規定する政令で定める支出は、同号に規定する旅行でその旅行に係る運賃、時間、距離その他の事情に照らし最も経済的かつ合理的と認められる通常の経路及び方法によるものに要する次に掲げる支出とする。
\begin{description}
\item[一]当該旅行に要する運賃及び料金
\item[二]当該旅行に要する自動車その他の交通用具の使用に係る燃料費及び有料の道路の料金
\end{description}
\item[\rensuji{6}]法第五十七条の二第二項第七号イに規定する政令で定める図書は、次に掲げる図書であつて職務に関連するものとする。
\begin{description}
\item[一]書籍
\item[二]新聞、雑誌その他の定期刊行物
\item[三]前二号に掲げるもののほか、不特定多数の者に販売することを目的として発行される図書
\end{description}
\item[\rensuji{7}]法第五十七条の二第二項第七号イに規定する政令で定める衣服は、次に掲げる衣服であつて勤務場所において着用することが必要とされるものとする。
\begin{description}
\item[一]制服
\item[二]事務服
\item[三]作業服
\item[四]前三号に掲げるもののほか、法第五十七条の二第二項に規定する給与等の支払者により勤務場所において着用することが必要とされる衣服
\end{description}
\end{description}
\noindent\hspace{10pt}(特定支出に関する明細書の記載事項)
\begin{description}
\item[第百六十七条の四]法第五十七条の二第三項(給与所得者の特定支出の控除の特例)に規定する特定支出に関する明細書には、次に掲げる事項を記載しなければならない。
\begin{description}
\item[一]法第五十七条の二第二項各号に掲げるそれぞれの支出につきその支出の内容、相手方の氏名又は名称、年月日及び金額並びに当該支出につき同項に規定する給与等の支払者により補塡される部分があり、かつ、その補塡される部分につき所得税が課されない場合における当該補塡される部分の金額及び当該支出につき同項に規定する教育訓練給付金、母子家庭自立支援教育訓練給付金又は父子家庭自立支援教育訓練給付金が支給される部分がある場合における当該支給される部分の金額
\item[二]次に掲げる支出の区分に応じそれぞれ次に定める事項
\begin{description}
\item[イ]法第五十七条の二第二項第一号に掲げる支出
\item[ロ]法第五十七条の二第二項第二号に掲げる支出
\item[ハ]法第五十七条の二第二項第三号に掲げる支出
\item[ニ]法第五十七条の二第二項第四号に掲げる支出
\item[ホ]法第五十七条の二第二項第五号に掲げる支出
\item[ヘ]法第五十七条の二第二項第六号に掲げる支出
\item[ト]法第五十七条の二第二項第七号イに掲げる支出
\item[チ]法第五十七条の二第二項第七号ロに掲げる支出
\end{description}
\item[三]その他参考となるべき事項
\end{description}
\end{description}
\noindent\hspace{10pt}(特定支出の支出等を証する書類)
\begin{description}
\item[第百六十七条の五]法第五十七条の二第四項(給与所得者の特定支出の控除の特例)に規定する政令で定める書類は、次の各号に掲げる支出の区分に応じ当該各号に定める書類とする。
\begin{description}
\item[一]法第五十七条の二第二項第一号から第五号まで、第六号(第百六十七条の三第五項第二号(給与所得者の特定支出の範囲)に係る部分に限る。)及び第七号に掲げる支出
\item[二]法第五十七条の二第二項第六号(第百六十七条の三第五項第一号に係る部分に限る。)に掲げる支出
\begin{description}
\item[イ]航空機を利用する場合
\item[ロ]鉄道、船舶又は自動車(以下この条において「鉄道等」という。)を利用する場合(その利用に係る運賃及び料金の額が財務省令で定める金額以上である場合に限る。)
\end{description}
\end{description}
\end{description}
\subsection*{第四節の二 外貨建取引の換算}
\addcontentsline{toc}{subsection}{第四節の二 外貨建取引の換算}
\noindent\hspace{10pt}(先物外国為替契約により発生時の外国通貨の円換算額を確定させた外貨建資産・負債の換算等)
\begin{description}
\item[第百六十七条の六]不動産所得、事業所得、山林所得又は雑所得を生ずべき業務を行う居住者が、外貨建資産・負債(外貨建取引(法第五十七条の三第一項(外貨建取引の換算)に規定する外貨建取引をいう。以下この項において同じ。)によつて取得し、又は発生する資産若しくは負債をいい、同条第二項の規定の適用を受ける資産又は負債を除く。以下この項において同じ。)の取得又は発生の基因となる外貨建取引に伴つて支払い、又は受け取る外国通貨の金額の円換算額(同条第一項に規定する円換算額をいう。以下この項において同じ。)を先物外国為替契約(外貨建取引に伴つて受け取り、又は支払う外国通貨の金額の円換算額を確定させる契約として財務省令で定めるものをいう。以下この項において同じ。)により確定させ、かつ、その先物外国為替契約の締結の日においてその旨を財務省令で定めるところによりその者の当該業務に係る帳簿書類その他の財務省令で定める書類に記載した場合には、その外貨建資産・負債については、その円換算額をもつて、同条第一項の規定により換算した金額として、その者の各年分の不動産所得の金額、事業所得の金額、山林所得の金額又は雑所得の金額を計算するものとする。
\item[\rensuji{2}]外国通貨で表示された預貯金を受け入れる銀行その他の金融機関(以下この項において「金融機関」という。)を相手方とする当該預貯金に関する契約に基づき預入が行われる当該預貯金の元本に係る金銭により引き続き同一の金融機関に同一の外国通貨で行われる預貯金の預入は、法第五十七条の三第一項に規定する外貨建取引に該当しないものとする。
\end{description}
\subsection*{第五節 資産の譲渡に関する総収入金額並びに必要経費及び取得費の計算の特例}
\addcontentsline{toc}{subsection}{第五節 資産の譲渡に関する総収入金額並びに必要経費及び取得費の計算の特例}
\noindent\hspace{10pt}(株式交換等による取得株式等の取得価額の計算等)
\begin{description}
\item[第百六十七条の七]法第五十七条の四第一項(株式交換等に係る譲渡所得等の特例)に規定する政令で定める関係は、株式交換の直前に当該株式交換に係る同項に規定する株式交換完全親法人(第四項及び第五項において「株式交換完全親法人」という。)との間に当該株式交換完全親法人の同条第一項に規定する発行済株式又は出資の全部を保有する関係がある場合の当該関係とする。
\item[\rensuji{2}]法第五十七条の四第一項に規定する政令で定めるものは、法人税法施行令第四条の三第十八項第二号(適格組織再編成における株式の保有関係等)に規定する株主均等割合保有関係がある株式交換とする。
\item[\rensuji{3}]法第五十七条の四第三項第五号に規定する政令で定める新株予約権は、次に掲げる新株予約権とする。
\begin{description}
\item[一]新株予約権を引き受ける者に特に有利な条件又は金額により交付された当該新株予約権
\item[二]役務の提供その他の行為に係る対価の全部又は一部として交付された新株予約権(前号に該当するものを除く。)
\end{description}
\item[\rensuji{4}]法第五十七条の四第一項の規定の適用を受けた居住者が同項に規定する株式交換により取得をした株式交換完全親法人の株式(出資を含む。以下この項及び次項において同じ。)又は株式交換完全親法人との間に第一項に規定する関係がある法人(以下この項において「親法人」という。)の株式に係る事業所得の金額、譲渡所得の金額又は雑所得の金額の計算については、当該株式交換により当該株式交換完全親法人に譲渡をした同条第一項に規定する旧株の取得価額(当該株式交換完全親法人の株式又は親法人の株式の取得に要した費用がある場合には、当該費用の額を加算した金額)を当該取得をした当該株式交換完全親法人の株式又は親法人の株式の取得価額とする。
\item[\rensuji{5}]法第五十七条の四第一項の規定の適用を受けた居住者が同項に規定する特定無対価株式交換により同項に規定する旧株を有しないこととなつた場合における所有株式(当該特定無対価株式交換の直後にその居住者が有する当該特定無対価株式交換に係る株式交換完全親法人の株式をいう。以下この項において同じ。)に係る当該特定無対価株式交換の後の事業所得の金額、譲渡所得の金額又は雑所得の金額の計算については、当該所有株式の当該特定無対価株式交換の直前の取得価額に当該旧株の当該特定無対価株式交換の直前の取得価額を加算した金額を当該所有株式の取得価額とする。
\item[\rensuji{6}]法第五十七条の四第二項の規定の適用を受けた居住者が同項に規定する株式移転により取得をした同項に規定する株式移転完全親法人の株式に係る事業所得の金額、譲渡所得の金額又は雑所得の金額の計算については、当該株式移転により当該株式移転完全親法人に譲渡をした同項に規定する旧株の取得価額(当該株式移転完全親法人の株式の取得に要した費用がある場合には、当該費用の額を加算した金額)を当該取得をした当該株式移転完全親法人の株式の取得価額とする。
\item[\rensuji{7}]法第五十七条の四第三項の規定の適用を受けた居住者が同項各号に規定する事由により取得をした当該各号に定める株式(出資及び投資信託及び投資法人に関する法律第二条第十四項(定義)に規定する投資口を含む。以下この条において同じ。)又は新株予約権に係る事業所得の金額、譲渡所得の金額又は雑所得の金額の計算については、次の各号に掲げる当該取得をした株式又は新株予約権の区分に応じ当該各号に定める金額を当該取得をした株式又は新株予約権の取得価額とする。
\begin{description}
\item[一]法第五十七条の四第三項第一号に規定する取得請求権付株式に係る同号に定める請求権の行使による当該取得請求権付株式の取得の対価として交付を受けた当該取得をする法人の株式(同項の規定の適用を受ける場合の当該取得をする法人の株式に限る。)
\item[二]法第五十七条の四第三項第二号に規定する取得条項付株式に係る同号に定める取得事由の発生(その取得の対価として当該取得をされる同号の株主等に当該取得をする法人の株式のみが交付されたものに限る。)による当該取得条項付株式の取得の対価として交付を受けた当該取得をする法人の株式(同項の規定の適用を受ける場合の当該取得をする法人の株式に限る。)
\item[三]法第五十七条の四第三項第二号に規定する取得条項付株式に係る同号に定める取得事由の発生(その取得の対象となつた種類の株式の全てが取得され、かつ、その取得の対価として当該取得をされる同号の株主等に当該取得をする法人の株式及び新株予約権のみが交付されたものに限る。)による当該取得条項付株式の取得の対価として交付を受けた当該取得をする法人の次に掲げる株式及び新株予約権(同項の規定の適用を受ける場合の当該取得をする法人の当該株式及び新株予約権に限る。)
\begin{description}
\item[イ]当該取得をする法人の株式
\item[ロ]当該取得をする法人の新株予約権
\end{description}
\item[四]法第五十七条の四第三項第三号に規定する全部取得条項付種類株式に係る同号に定める取得決議(その取得の対価として当該取得をされる同号の株主等に当該取得をする法人の株式以外の資産(当該取得の価格の決定の申立てに基づいて交付される金銭その他の資産を除く。)が交付されなかつたものに限る。)による当該全部取得条項付種類株式の取得の対価として交付を受けた当該取得をする法人の株式(同項の規定の適用を受ける場合の当該取得をする法人の株式に限る。)
\item[五]法第五十七条の四第三項第三号に規定する全部取得条項付種類株式に係る同号に定める取得決議(その取得の対価として当該取得をされる同号の株主等に当該取得をする法人の株式及び新株予約権が交付され、かつ、これら以外の資産(当該取得の価格の決定の申立てに基づいて交付される金銭その他の資産を除く。)が交付されなかつたものに限る。)による当該全部取得条項付種類株式の取得の対価として交付を受けた当該取得をする法人の次に掲げる株式及び新株予約権(同項の規定の適用を受ける場合の当該取得をする法人の当該株式及び新株予約権に限る。)
\begin{description}
\item[イ]当該取得をする法人の株式
\item[ロ]当該取得をする法人の新株予約権
\end{description}
\item[六]法第五十七条の四第三項第四号に規定する新株予約権付社債についての社債に係る同号に定める新株予約権の行使による当該社債の取得の対価として交付を受けた当該取得をする法人の株式(同項の規定の適用を受ける場合の当該取得をする法人の株式に限る。)
\item[七]法第五十七条の四第三項第五号に規定する取得条項付新株予約権に係る同号に定める取得事由の発生による当該取得条項付新株予約権の取得の対価として交付を受けた当該取得をする法人の株式(同項の規定の適用を受ける場合の当該取得をする法人の株式に限る。)
\item[八]法第五十七条の四第三項第六号に規定する取得条項付新株予約権が付された新株予約権付社債に係る同号に定める取得事由の発生による当該新株予約権付社債の取得の対価として交付を受けた当該取得をする法人の株式(同項の規定の適用を受ける場合の当該取得をする法人の株式に限る。)
\end{description}
\item[\rensuji{8}]会社法第百六十七条第三項(効力の発生)又は第二百八十三条(一に満たない端数の処理)に規定する一株に満たない端数(これに準ずるものを含む。)に相当する部分は、法第五十七条の四第三項第一号又は第四号に規定する取得をする法人の株式に含まれるものとする。
\end{description}
\noindent\hspace{10pt}(交換による取得資産の取得価額等の計算)
\begin{description}
\item[第百六十八条]法第五十八条第一項(固定資産の交換の場合の譲渡所得の特例)の規定の適用を受けた居住者が同項に規定する取得資産(以下この条において「取得資産」という。)について行なうべき法第四十九条第一項(減価償却資産の償却費の計算及びその償却の方法)に規定する償却費の額の計算及びその者が取得資産を譲渡した場合における譲渡所得の金額の計算については、その者がその取得資産を次の各号に掲げる場合の区分に応じ当該各号に掲げる金額をもつて取得したものとみなす。
\begin{description}
\item[一]取得資産とともに交換差金等(法第五十八条第一項に規定する交換の時における取得資産の価額と譲渡資産の価額とが等しくない場合にその差額を補うために交付される金銭その他の資産をいう。以下この条において同じ。)を取得した場合
\item[二]譲渡資産とともに交換差金等を交付して取得資産を取得した場合
\item[三]取得資産を取得するために要した経費の額がある場合
\end{description}
\end{description}
\noindent\hspace{10pt}(時価による譲渡とみなす低額譲渡の範囲)
\begin{description}
\item[第百六十九条]法第五十九条第一項第二号(贈与等の場合の譲渡所得等の特例)に規定する政令で定める額は、同項に規定する山林又は譲渡所得の基因となる資産の譲渡の時における価額の二分の一に満たない金額とする。
\end{description}
\noindent\hspace{10pt}(国外転出をする場合の譲渡所得等の特例)
\begin{description}
\item[第百七十条]法第六十条の二第一項(国外転出をする場合の譲渡所得等の特例)に規定する政令で定める有価証券は、次に掲げる有価証券で法第百六十一条第一項第十二号(国内源泉所得)に掲げる所得を生ずべきものとする。
\begin{description}
\item[一]第八十四条第一項(譲渡制限付株式の価額等)に規定する特定譲渡制限付株式又は承継譲渡制限付株式で、同項に規定する譲渡についての制限が解除されていないもの
\item[二]第八十四条第二項各号に掲げる権利で当該権利の行使をしたならば同項の規定の適用のあるものを表示する有価証券
\end{description}
\item[\rensuji{2}]法第六十条の二第四項に規定する譲渡に類するものとして政令で定めるものは、租税特別措置法第三十七条の十第三項若しくは第四項(一般株式等に係る譲渡所得等の課税の特例)又は第三十七条の十一第三項若しくは第四項(上場株式等に係る譲渡所得等の課税の特例)の規定によりその額及び価額の合計額が同法第三十七条の十第一項に規定する一般株式等に係る譲渡所得等又は同法第三十七条の十一第一項に規定する上場株式等に係る譲渡所得等に係る収入金額とみなされる金銭及び金銭以外の資産の交付の基因となつた同法第三十七条の十第三項若しくは第四項各号又は第三十七条の十一第四項各号に規定する事由に基づく同法第三十七条の十第二項に規定する株式等についての当該金銭の額及び当該金銭以外の資産の価額に対応する権利の移転又は消滅とする。
\item[\rensuji{3}]法第六十条の二第五項に規定する国内に住所又は居所を有していた期間として政令で定める期間は、次に掲げる期間とする。
\begin{description}
\item[一]国内に住所又は居所を有していた期間(出入国管理及び難民認定法(昭和二十六年政令第三百十九号)別表第一(在留資格)の上欄の在留資格をもつて在留していた期間を除く。)
\item[二]法第六十条の二第一項に規定する国外転出(以下この条において「国外転出」という。)をした日の属する年分の所得税につき法第百三十七条の二第一項(国外転出をする場合の譲渡所得等の特例の適用がある場合の納税猶予)(同条第二項の規定により適用する場合を含む。以下この号において同じ。)の規定による納税の猶予を受けた個人(その相続人を含む。)に係る同日(同条第十三項の規定により同項に規定する納税猶予分の所得税額に係る納付の義務を承継した場合には、当該承継した日)から当該納税の猶予に係る期限(同条第一項、第五項、第八項又は第九項の規定その他財務省令で定める規定による期限のうち最も遅いものに限る。)までの期間(前号に掲げる期間を除く。)
\item[三]贈与、相続又は遺贈により法第百三十七条の三第一項(贈与等により非居住者に資産が移転した場合の譲渡所得等の特例の適用がある場合の納税猶予)に規定する対象資産の移転を受けた日の属する年分の所得税につき同条第一項又は第二項(これらの規定を同条第三項の規定により適用する場合を含む。以下この号において同じ。)の規定による納税の猶予を受けた個人(その相続人を含む。)に係る当該贈与の日又は相続の開始の日(同条第十五項の規定により同項に規定する納税猶予分の所得税額に係る納付の義務を承継した場合には、当該承継した日)から当該納税の猶予に係る期限(同条第一項、第二項、第六項、第九項又は第十一項の規定その他財務省令で定める規定による期限のうち最も遅いものに限る。)までの期間(前二号に掲げる期間を除く。)
\end{description}
\item[\rensuji{4}]法第六十条の二第八項に規定する政令で定める譲渡は、次に掲げる譲渡とする。
\begin{description}
\item[一]法第六十条の二第一項に規定する有価証券等(以下この条及び次条において「有価証券等」という。)の譲渡でその譲渡の時における価額より低い価額によりされるもの
\item[二]有価証券等の譲渡をすることにより法第六十条の二第八項に規定する個人(その相続人を含む。)の国外転出の日の属する年分の所得税の負担を不当に減少させる結果となると認められる場合における当該譲渡
\end{description}
\item[\rensuji{5}]法第六十条の二第八項第一号に規定する政令で定める事由は次の各号に掲げる事由とし、同項第一号に規定する政令で定めるところにより計算した金額は、同項に規定する個人が国外転出の時に有していた有価証券等(当該国外転出の時後に当該各号に掲げる事由により取得した有価証券等がある場合には、当該有価証券等)について生じた当該各号に掲げる事由により取得した有価証券等又は当該事由が生じた時前から引き続き有していた有価証券等に係る当該事由の次の各号に掲げる区分に応じ当該各号に定める金額に、同項第一号の譲渡又は限定相続等があつた有価証券等の数を乗じて計算した金額とする。
\begin{description}
\item[一]株式(出資を含む。以下この号において同じ。)を発行した法人の法第六十条の二第十一項第一号に掲げる株式交換又は株式移転
\begin{description}
\item[イ]当該株式交換により第百六十七条の七第四項(株式交換等による取得株式等の取得価額の計算等)に規定する株式交換完全親法人の株式若しくは親法人の株式(以下この号において「親法人株式等」という。)を取得した場合又は当該株式移転により同条第六項に規定する株式移転完全親法人の株式を取得した場合
\item[ロ]当該株式交換により親法人株式等を取得しなかつた場合
\end{description}
\item[二]法第六十条の二第十一項第二号に規定する取得請求権付株式、取得条項付株式、全部取得条項付種類株式、新株予約権付社債、取得条項付新株予約権又は取得条項付新株予約権が付された新株予約権付社債(以下この号において「取得請求権付株式等」という。)の同項第二号に規定する請求権の行使、取得事由の発生、取得決議又は行使(以下この号において「請求権の行使等」という。)
\item[三]株式又は投資信託若しくは特定受益証券発行信託の受益権の分割又は併合
\item[四]株式を発行した法人の第百十一条第二項(株主割当てにより取得した株式の取得価額)に規定する株式無償割当て(当該株式無償割当てにより当該株式と同一の種類の株式が割り当てられる場合の当該株式無償割当てに限る。)
\item[五]株式を発行した法人の第百十二条第一項(合併により取得した株式等の取得価額)に規定する合併
\item[六]株式を発行した法人を第百十二条第一項に規定する合併法人とする同条第二項に規定する無対価合併
\item[七]第百十二条第三項に規定する投資信託等(以下この号において「投資信託等」という。)の受益権に係る投資信託等の同項に規定する信託の併合
\item[八]株式を発行した法人の第百十三条第一項(分割型分割により取得した株式等の取得価額)に規定する分割型分割
\begin{description}
\item[イ]当該分割型分割に係る第百十三条第一項に規定する分割承継法人の株式又は同項に規定する分割承継親法人の株式
\item[ロ]当該個人が当該分割型分割の前から引き続き有している当該分割型分割に係る分割法人の株式
\end{description}
\item[九]株式を発行した法人を第百十三条第一項に規定する分割承継法人とする同条第二項に規定する無対価分割型分割
\begin{description}
\item[イ]当該無対価分割型分割に係る当該分割承継法人の株式
\item[ロ]当該個人が当該無対価分割型分割の前から引き続き有している当該無対価分割型分割に係る分割法人の株式
\end{description}
\item[十]特定受益証券発行信託の受益権に係る特定受益証券発行信託の第百十三条第六項に規定する信託の分割
\begin{description}
\item[イ]当該信託の分割に係る第百十三条第六項に規定する承継信託受益権
\item[ロ]当該個人が当該信託の分割の前から引き続き有している当該信託の分割に係る分割信託の受益権
\end{description}
\item[十の二]株式を発行した法人の第百十三条の二第一項(株式分配により取得した株式等の取得価額)に規定する株式分配
\begin{description}
\item[イ]当該株式分配に係る第百十三条の二第一項に規定する完全子法人の株式
\item[ロ]当該個人が当該株式分配の前から引き続き有している当該株式分配に係る現物分配法人の株式
\end{description}
\item[十一]株式を発行した法人の第百十四条第一項(資本の払戻し等があつた場合の株式等の取得価額)に規定する資本の払戻し又は解散による残余財産の分配
\item[十二]法人の第百十四条第二項に規定する所有出資の同項に規定する払戻し
\item[十三]オープン型の証券投資信託の受益権に係る収益の分配(当該オープン型の証券投資信託の終了又は当該オープン型の証券投資信託の一部の解約により支払われるものを除くものとし、その収益の分配のうちに第二十七条(オープン型の証券投資信託の収益の分配のうち非課税とされるもの)に規定する特別分配金が含まれているものに限る。)
\item[十四]株式を発行した法人の第百十五条(組織変更があつた場合の株式等の取得価額)に規定する組織変更
\item[十五]新株予約権(投資信託及び投資法人に関する法律第二条第十七項に規定する新投資口予約権を含む。以下この項において同じ。)又は新株予約権付社債を発行した法人を第百十六条(合併等があつた場合の新株予約権等の取得価額)に規定する被合併法人、分割法人、株式交換完全子法人又は株式移転完全子法人とする同条に規定する合併等
\item[十六]新株予約権の行使
\end{description}
\item[\rensuji{6}]前項第三号から第十五号までの規定により第百十条、第百十一条第二項、第百十二条、第百十三条第一項から第三項まで、第六項及び第七項、第百十三条の二第一項及び第二項、第百十四条第一項から第三項まで、第百十五条並びに第百十六条の規定に準じて計算する場合には、第百十条第一項中「取得価額は、旧株一株の従前の取得価額」とあるのは「第百七十条第五項第三号(国外転出をする場合の譲渡所得等の特例)に規定する国外転出時評価額(以下「国外転出時評価額」という。)は、旧株一株の従前の国外転出時評価額」と、同条第二項及び第百十一条第二項中「取得価額」とあるのは「国外転出時評価額」と、第百十二条第一項中「取得価額は、旧株一株の従前の取得価額(法第二十五条第一項第一号(合併の場合のみなし配当)の規定により剰余金の配当、利益の配当、剰余金の分配若しくは金銭の分配として交付を受けたものとみなされる金額又はその合併法人株式若しくは合併親法人株式の取得のために要した費用の額がある場合には、当該交付を受けたものとみなされる金額及び費用の額のうち旧株一株に対応する部分の金額を加算した金額)」とあるのは「国外転出時評価額は、旧株一株の従前の国外転出時評価額」と、同条第二項中「取得価額は」とあるのは「国外転出時評価額は」と、「取得価額に」とあるのは「国外転出時評価額に」と、「当該無対価合併の直前に有していた」とあるのは「法第六十条の二第一項(国外転出をする場合の譲渡所得等の特例)に規定する国外転出の時において有する」と、「取得価額(法第二十五条第一項第一号の規定により剰余金の配当、利益の配当又は剰余金の分配として交付を受けたものとみなされる金額がある場合には、当該交付を受けたものとみなされる金額のうち旧株一株に対応する部分の金額を加算した金額)」とあるのは「国外転出時評価額」と、同条第三項中「取得価額は、旧受益権一口の従前の取得価額(その併合投資信託等の受益権の取得のために要した費用の額がある場合には、当該費用の額のうち旧受益権一口に対応する部分の金額を加算した金額)」とあるのは「国外転出時評価額は、旧受益権一口の従前の国外転出時評価額」と、第百十三条第一項中「取得価額」とあるのは「国外転出時評価額」と、「金額(法第二十五条第一項第二号(分割型分割の場合のみなし配当)の規定により剰余金の配当若しくは利益の配当として交付を受けたものとみなされる金額又はその分割承継法人株式若しくは分割承継親法人株式の取得のために要した費用の額がある場合には、当該交付を受けたものとみなされる金額及び費用の額のうち分割承継法人株式又は分割承継親法人株式一株に対応する部分の金額を加算した金額)」とあるのは「金額」と、同条第二項中「取得価額」とあるのは「国外転出時評価額」と、「当該無対価分割型分割の直前に有していた」とあるのは「法第六十条の二第一項(国外転出をする場合の譲渡所得等の特例)に規定する国外転出の時において有する」と、「金額(法第二十五条第一項第二号の規定により剰余金の配当又は利益の配当として交付を受けたものとみなされる金額がある場合には、当該交付を受けたものとみなされる金額のうち所有株式一株に対応する部分の金額を加算した金額)」とあるのは「金額」と、同条第三項中「取得価額」とあるのは「国外転出時評価額」と、同条第六項中「取得価額」とあるのは「国外転出時評価額」と、「金額(その承継信託受益権の取得のために要した費用の額がある場合には、当該費用の額のうち承継信託受益権一口に対応する部分の金額を加算した金額)」とあるのは「金額」と、同条第七項中「取得価額」とあるのは「国外転出時評価額」と、第百十三条の二第一項中「取得価額」とあるのは「国外転出時評価額」と、「金額(法第二十五条第一項第三号(株式分配の場合のみなし配当)の規定により剰余金の配当若しくは利益の配当として交付を受けたものとみなされる金額又はその完全子法人株式の取得のために要した費用の額がある場合には、当該交付を受けたものとみなされる金額及び費用の額のうち完全子法人株式一株に対応する部分の金額を加算した金額)」とあるのは「金額」と、同条第二項及び第百十四条第一項から第三項までの規定中「取得価額」とあるのは「国外転出時評価額」と、第百十五条中「取得価額は、旧株一単位の従前の取得価額(その新株の取得のために要した費用の額がある場合には、当該費用の額のうち旧株一単位に対応する部分の金額を加算した金額)」とあるのは「国外転出時評価額は、旧株一単位の従前の国外転出時評価額」と、第百十六条中「取得価額は、旧新株予約権等一単位の従前の取得価額(その合併法人等新株予約権等の取得のために要した費用の額がある場合には、当該費用の額のうち旧新株予約権等一単位に対応する部分の金額を加算した金額)」とあるのは「国外転出時評価額は、旧新株予約権等一単位の従前の国外転出時評価額」と読み替えるものとする。
\item[\rensuji{7}]法第六十条の二第十一項第三号に規定する政令で定める事由は、第五項第三号から第五号まで、第七号、第八号、第十号、第十号の二及び第十四号から第十六号までに掲げる事由とする。
\item[\rensuji{8}]国外転出の日の属する年分の所得税につき法第六十条の二第一項の規定の適用を受けるべき個人(その相続人を含む。)が当該国外転出の時後に譲渡又は同条第八項に規定する限定相続等により有価証券等の移転をした場合において、その移転をした有価証券等が、その者が当該国外転出の時において有していた有価証券等に該当するかどうかの判定は、まず当該国外転出の時後に取得した同一銘柄の有価証券等(贈与、相続又は遺贈により取得した同一銘柄の有価証券等のうち、当該贈与をした者又は当該相続若しくは遺贈に係る相続人が当該贈与の日又は相続の開始の日の属する年分の所得税につき法第百三十七条の三第一項又は第二項の規定の適用を受けている場合における当該有価証券等(以下この項において「猶予適用有価証券等」という。)を除く。)の譲渡又は贈与をし、次に当該個人が当該国外転出の時に有していた有価証券等又は猶予適用有価証券等のうち先に法第六十条の二第一項又は第六十条の三第一項(贈与等により非居住者に資産が移転した場合の譲渡所得等の特例)の規定の適用があつたものから順次譲渡又は贈与をしたものとして行うものとする。
\item[\rensuji{9}]前項に規定する個人が有する有価証券等(以下この項において「従前の有価証券等」という。)について法第六十条の二第十一項各号に掲げる事由が生じた場合において、当該事由により取得した有価証券等(以下この項において「取得有価証券等」という。)が同条第十一項の規定により引き続き所有していたものとみなされるときにおける当該従前の有価証券等のうち当該取得有価証券等の取得の基因となつた部分は、当該取得有価証券等と同一銘柄の有価証券等とみなして、前項の規定を適用する。
\end{description}
\noindent\hspace{10pt}(贈与等により非居住者に資産が移転した場合の譲渡所得等の特例)
\begin{description}
\item[第百七十条の二]法第六十条の三第五項(贈与等により非居住者に資産が移転した場合の譲渡所得等の特例)に規定する国内に住所又は居所を有していた期間として政令で定める期間は、前条第三項各号に掲げる期間とする。
\item[\rensuji{2}]前条第五項の規定は、法第六十条の三第八項第一号に規定する政令で定めるところにより計算した金額について準用する。
\item[\rensuji{3}]前条第六項の規定は、前項において準用する同条第五項第三号から第十五号までの規定により第百十条(株式の分割又は併合の場合の株式等の取得価額)、第百十一条第二項(株主割当てにより取得した株式の取得価額)、第百十二条(合併により取得した株式等の取得価額)、第百十三条第一項から第三項まで、第六項及び第七項(分割型分割により取得した株式等の取得価額)、第百十三条の二第一項及び第二項(株式分配により取得した株式等の取得価額)、第百十四条第一項から第三項まで(資本の払戻し等があつた場合の株式等の取得価額)、第百十五条(組織変更があつた場合の株式等の取得価額)並びに第百十六条(合併等があつた場合の新株予約権等の取得価額)の規定に準じて計算する場合について準用する。
\item[\rensuji{4}]法第六十条の三第十項第二号の規定による納税管理人の届出をする場合において、同号の移転を受けた非居住者が二人以上あるときは、当該届出は、各非居住者が連署による一の書面で行わなければならない。
\item[\rensuji{5}]前項ただし書の方法により同項の届出をした非居住者は、遅滞なく、当該移転を受けた他の非居住者に対し、当該届出の際に提出した書面に記載した事項の要領を通知しなければならない。
\item[\rensuji{6}]前条第八項の規定は、贈与の日の属する年分の所得税につき法第六十条の三第一項から第三項までの規定の適用を受けるべき個人の受贈者又は相続の開始の日の属する年分の所得税につき同条第一項から第三項までの規定の適用を受けるべき個人の相続人が同条第一項に規定する贈与等の時後に譲渡又は同条第八項に規定する限定相続等により有価証券等の移転をした場合において、その移転をした有価証券等が、これらの者が当該贈与等により取得をした有価証券等に該当するかどうかの判定について準用する。
\item[\rensuji{7}]前項に規定する受贈者又は相続人が有する有価証券等(以下この項において「従前の有価証券等」という。)について法第六十条の二第十一項各号(国外転出をする場合の譲渡所得等の特例)に掲げる事由が生じた場合において、当該事由により取得した有価証券等(以下この項において「取得有価証券等」という。)が法第六十条の三第十二項の規定により引き続き所有していたものとみなされるときにおける当該従前の有価証券等のうち当該取得有価証券等の取得の基因となつた部分は、当該取得有価証券等と同一銘柄の有価証券等とみなして、前項において準用する前条第八項の規定を適用する。
\end{description}
\noindent\hspace{10pt}(外国転出時課税の規定の適用を受けた場合の譲渡所得等の特例)
\begin{description}
\item[第百七十条の三]法第六十条の四第一項又は第二項(外国転出時課税の規定の適用を受けた場合の譲渡所得等の特例)の規定の適用がある場合には、同条第一項に規定する収入金額に算入することとされた金額及び同条第二項に規定する利益の額に相当する金額又は損失の額に相当する金額の法第五十七条の三第一項(外貨建取引の換算)に規定する円換算額は、法第六十条の四第三項に規定する国外転出に相当する事由その他政令で定める事由が生じた時における外国為替の売買相場により換算した金額とする。
\item[\rensuji{2}]法第六十条の四第三項に規定する政令で定める事由は、次に掲げる事由とする。
\begin{description}
\item[一]国籍その他これに類するものを有しないこととなること。
\item[二]外国が締結した所得に対する租税に関する二重課税の回避のための条約の規定により当該条約の締約国若しくは締約者のうち一方の締約国若しくは締約者において法第九十五条の二第一項(国外転出をする場合の譲渡所得等の特例に係る外国税額控除の特例)に規定する外国所得税を課される者でないものとみなされることとなること又は外国居住者等の所得に対する相互主義による所得税等の非課税等に関する法律(昭和三十七年法律第百四十四号)第三条第一項各号(双方居住者の取扱い)に掲げる場合に相当する場合その他これに類する場合に該当することにより同法第二条第三号(定義)に規定する外国(同法第五条各号(相互主義)のいずれかに該当しない場合における当該外国を除く。)において法第九十五条の二第一項に規定する外国所得税を課される者でないものとみなされることとなること。
\end{description}
\end{description}
\noindent\hspace{10pt}(昭和二十七年十二月三十一日以前に取得した山林の取得費)
\begin{description}
\item[第百七十一条]法第六十一条第一項(昭和二十七年十二月三十一日以前に取得した資産の取得費等)に規定する山林の昭和二十八年一月一日における価額として政令で定めるところにより計算した金額は、同日における山林の樹種別及び樹齢別の標準的な評価額を基礎とし、これにその山林に係る地味、地域その他の事情の差異による調整を加えた価額とする。
\end{description}
\noindent\hspace{10pt}(昭和二十七年十二月三十一日以前に取得した資産の取得費)
\begin{description}
\item[第百七十二条]法第六十一条第二項又は第三項(昭和二十七年十二月三十一日以前に取得した資産の取得費)に規定する資産の昭和二十八年一月一日における価額として政令で定めるところにより計算した金額は、同日におけるその資産の現況に応じ、同日においてその資産につき相続税及び贈与税の課税標準の計算に用いるべきものとして国税庁長官が定めて公表した方法により計算した価額とする。
\item[\rensuji{2}]前項に規定する資産が資産再評価法(昭和二十五年法律第百十号)第八条第一項(個人の減価償却資産の再評価)(同法第十条第一項(非事業用資産を事業の用に供した場合の再評価)において準用する場合を含む。)又は第十六条(死亡の場合の再評価の承継)の規定により再評価を行なつているものである場合において、その資産につき前項の規定により計算した価額が当該再評価に係る同法第二条第三項(定義)に規定する再評価額に満たないときは、その資産の法第六十一条第二項又は第三項に規定する昭和二十八年一月一日における価額として政令で定めるところにより計算した金額は、前項の規定にかかわらず、当該再評価額とする。
\item[\rensuji{3}]法第六十一条第三項に規定する資産の取得に要した金額と昭和二十八年一月一日前に支出した設備費及び改良費の額との合計額を基礎として政令で定めるところにより計算した同日におけるその資産の価額は、同日においてその資産の譲渡があつたものとみなして法第三十八条第二項(譲渡所得の金額の計算上控除する取得費)の規定を適用した場合に同項の規定によりその資産の取得費とされる金額に相当する金額とする。
\end{description}
\noindent\hspace{10pt}(昭和二十七年十二月三十一日以前に取得した有価証券の取得費)
\begin{description}
\item[第百七十三条]法第六十一条第四項(昭和二十七年十二月三十一日以前に取得した資産の取得費)に規定する有価証券の昭和二十八年一月一日における価額として政令で定めるところにより計算した金額は、証券取引所(証券取引法等の一部を改正する法律(平成十八年法律第六十五号)第三条(証券取引法の一部改正)の規定による改正前の証券取引法に規定する証券取引所をいう。)において上場されている株式又は気配相場のある株式若しくは出資については、次に定めるところにより計算した金額を基礎とし、その他の株式又は出資については、その株式又は出資に係る発行法人の同日における資産の価額の合計額から負債の額の合計額を控除した金額をその発行法人の同日における発行済株式又は出資の総数又は総額で除して計算した金額を基礎としてそれぞれ計算した金額とする。
\begin{description}
\item[一]昭和二十七年十二月中における毎日の公表最終価格(金融商品取引法第百三十条(総取引高、価格等の通知等)に相当する規定により公表された最終の価格をいう。)又は最終の気配相場の価格(以下この条において「公表最終価格等」という。)の合計額を同月中の日数(公表最終価格等のない日の数を除く。)で除する。
\item[二]前号の公表最終価格等のうちにその株式又は出資に係る発行法人の資本又は出資の増加による権利落ちに係る価格が含まれている場合において、当該増加に係る株式又は出資(以下この号において「新株」という。)が昭和二十七年十二月三十一日において発行されているときは、当該権利落ち前の公表最終価格等についてはその額から当該新株の権利の価額を控除した価額を、同日において当該新株が発行されていないときは、当該権利落ち以後の公表最終価格等についてはその額に当該新株の権利の価額を加算した価額をそれぞれ基礎として前号の規定により計算する。
\end{description}
\end{description}
\noindent\hspace{10pt}(借地権等の設定をした場合の譲渡所得に係る取得費)
\begin{description}
\item[第百七十四条]第七十九条第一項(資産の譲渡とみなされる行為)に規定する借地権又は地役権(以下この条において「借地権等」という。)の設定(借地権に係る土地を他人に使用させる行為を含む。以下この条において同じ。)につき法第三十三条第一項(譲渡所得)の規定の適用がある場合において、当該設定に係る譲渡所得の金額の計算上控除する取得費は、その借地権等の設定をした土地の取得に要した金額及び改良費の額の合計額に、第一号に掲げる金額が第二号に掲げる金額のうちに占める割合を乗じて計算した金額とする。
\begin{description}
\item[一]その借地権等の設定の対価として支払を受ける金額
\item[二]前号に掲げる金額とその借地権等の設定をされている土地(以下この条において「底地」という。)としての価額(当該土地が借地権等の設定の目的である用途にのみ使用される場合において、当該底地としての価額が明らかでなく、かつ、その借地権等の設定により支払を受ける地代があるときは、その地代の年額の二十倍に相当する金額)との合計額
\end{description}
\item[\rensuji{2}]借地権等の設定をしている土地につき更に他の者に対し借地権等の設定をした場合において、前の借地権等の設定につき前項の規定によりその取得費とされた金額があるときは、当該他の者に対する借地権等の設定に係る同項の規定の適用については、当該土地に係る同項に規定する取得に要した金額及び改良費の額の合計額は、当該合計額に相当する金額から当該取得費とされた金額を控除した金額とする。
\item[\rensuji{3}]第一項の規定を適用する場合において、先に借地権等の設定があつた土地につき現に借地権等の設定がなく、かつ、同項の規定により当該先の借地権等の設定に係る譲渡所得の金額の計算上控除された取得費があるときは、当該先の借地権等(同項の使用に係る権利を含む。以下この項において同じ。)の消滅につき対価を支払つた場合を除き、第一項に規定する取得費は、同項の借地権等につき同項の規定により計算した金額から当該控除された取得費に相当する金額を控除した金額とする。
\item[\rensuji{4}]第一項の規定を適用する場合において、当該借地権等の設定に係る土地が昭和二十七年十二月三十一日以前から引き続き所有していたものであるときは、当該土地に係る同項に規定する取得に要した金額及び改良費の額の合計額は、当該土地につき第百七十二条第一項及び第二項(昭和二十七年十二月三十一日以前に取得した資産の取得費)の規定により計算した金額と昭和二十八年一月一日以後に支出した改良費の額との合計額に相当する金額とする。
\end{description}
\noindent\hspace{10pt}(借地権等の設定をした土地の底地の取得費等)
\begin{description}
\item[第百七十五条]前条第一項に規定する借地権等(以下この条において「借地権等」という。)の設定(借地権に係る土地を他人に使用させる行為を含む。以下この条において同じ。)につき法第三十三条第一項(譲渡所得)の規定の適用があつた場合において、当該設定をした土地の譲渡があつたときは、同項の規定の適用については、当該土地に係る前条第一項第二号に規定する底地(以下この条において「底地」という。)に相当する部分の譲渡があつたものとし、当該譲渡に係る譲渡所得の金額の計算上控除する取得費は、同項に規定する土地の取得に要した金額及び改良費の額の合計額から同項の規定により当該借地権等の設定に係る譲渡所得の金額の計算上控除された取得費に相当する金額を控除した金額とする。
\item[\rensuji{2}]借地権等の設定につき第八十条(特別の経済的な利益で借地権の設定等による対価とされるもの)の規定の適用を受けた者が、同条第一項の貸付けを受けた金額のうち同項の規定により当該設定の対価の額に加算された金額の全部又は一部の返済その他同項に規定する特別の経済的な利益の全部又は一部の返還をした場合において、その返還により当該借地権等に係る土地の地代の引上げ、その土地の上に存する建物又は構築物の除去その他当該土地の底地の価値の増加があつたときは、その返還をした利益の額に相当する金額は、当該設定をした土地の取得に要した金額及び改良費の額の合計額に加算する。
\end{description}
\noindent\hspace{10pt}(借地権の転貸に係る取得費)
\begin{description}
\item[第百七十六条]第七十九条第一項(資産の譲渡とみなされる行為)に規定する借地権(以下この条において「借地権」という。)に係る土地の転貸(当該土地を他人に使用させる行為を含む。以下この条において同じ。)につき法第三十三条第一項(譲渡所得)の規定の適用がある場合には、当該転貸に係る譲渡所得の金額の計算上控除する取得費は、当該転貸をした土地に係る借地権の取得に要した金額及び改良費の額の合計額に、第一号に掲げる金額が第二号に掲げる金額のうちに占める割合を乗じて計算した金額とする。
\begin{description}
\item[一]当該借地権に係る土地の転貸の対価として支払を受ける金額
\item[二]前号に掲げる金額と当該転貸直後における当該転貸をした土地に係る借地権の価額(当該転貸に係る土地が当該転貸の目的である用途にのみ使用される場合において、当該借地権の価額が明らかでなく、かつ、当該転貸により支払われる地代で当該借地権を有する者に交付するものがあるときは、その者に交付する地代の年額の二十倍に相当する金額)との合計額
\end{description}
\item[\rensuji{2}]前項の規定を適用する場合において、先に転貸をした土地につき現に当該転貸に係る権利が消滅しており、かつ、同項の規定により当該先の転貸に係る譲渡所得の金額の計算上控除された取得費があるときは、当該先の転貸に係る権利の消滅につき対価を支払つた場合を除き、同項に規定する取得費は、同項の借地権につき同項の規定により計算した金額から当該控除された取得費に相当する金額を控除した金額とする。
\item[\rensuji{3}]第一項の規定を適用する場合において、同項に規定する転貸をした土地に係る借地権が昭和二十七年十二月三十一日以前から引き続いて所有していたものであるときは、当該借地権に係る同項に規定する取得に要した金額及び改良費の額の合計額は、当該借地権につき第百七十二条第一項(昭和二十七年十二月三十一日以前に取得した資産の取得費)の規定により計算した金額と昭和二十八年一月一日以後に支出した改良費の額との合計額に相当する金額とする。
\end{description}
\noindent\hspace{10pt}(転貸をした借地権の取得費)
\begin{description}
\item[第百七十七条]前条第一項に規定する借地権(以下この条において「借地権」という。)に係る土地の同項に規定する転貸(以下この条において「転貸」という。)につき法第三十三条第一項(譲渡所得)の規定の適用があつた場合において、当該転貸をした土地に係る借地権の譲渡があつたときは、同項の規定の適用については、当該譲渡に係る譲渡所得の金額の計算上控除する取得費は、前条第一項に規定する借地権の取得に要した金額及び改良費の額の合計額から同項の規定により当該転貸に係る譲渡所得の金額の計算上控除された取得費に相当する金額を控除した金額とする。
\item[\rensuji{2}]借地権に係る土地の転貸につき第八十条(特別の経済的な利益で借地権の設定等による対価とされるもの)の規定の適用を受けた者が、同条第一項の貸付けを受けた金額のうち同項の規定により当該転貸の対価の額に加算された金額の全部又は一部の返済その他同項に規定する特別の経済的な利益の全部又は一部の返還をした場合において、その返還により当該転貸に係る使用料の引上げ、その土地の上に存する建物又は構築物の除去その他当該転貸をした土地に係る借地権の価値の増加があつたときは、その返還をした利益の額に相当する金額は、当該転貸をした土地に係る借地権の取得に要した金額及び改良費の額の合計額に加算する。
\end{description}
\noindent\hspace{10pt}(生活に通常必要でない資産の災害による損失額の計算等)
\begin{description}
\item[第百七十八条]法第六十二条第一項(生活に通常必要でない資産の災害による損失)に規定する政令で定めるものは、次に掲げる資産とする。
\begin{description}
\item[一]競走馬(その規模、収益の状況その他の事情に照らし事業と認められるものの用に供されるものを除く。)その他射こう的行為の手段となる動産
\item[二]通常自己及び自己と生計を一にする親族が居住の用に供しない家屋で主として趣味、娯楽又は保養の用に供する目的で所有するものその他主として趣味、娯楽、保養又は鑑賞の目的で所有する資産(前号又は次号に掲げる動産を除く。)
\item[三]生活の用に供する動産で第二十五条(譲渡所得について非課税とされる生活用動産の範囲)の規定に該当しないもの
\end{description}
\item[\rensuji{2}]法第六十二条第一項の規定により、同項に規定する生活に通常必要でない資産について受けた同項に規定する損失の金額をその生じた日の属する年分及びその翌年分の譲渡所得の金額の計算上控除すべき金額とみなす場合には、次に定めるところによる。
\begin{description}
\item[一]まず、当該損失の金額をその生じた日の属する年分の法第三十三条第三項第一号(譲渡所得)に掲げる所得の金額の計算上控除すべき金額とし、当該所得の金額の計算上控除しきれない損失の金額があるときは、これを当該年分の同項第二号に掲げる所得の金額の計算上控除すべき金額とする。
\item[二]前号の規定によりなお控除しきれない損失の金額があるときは、これをその生じた日の属する年の翌年分の法第三十三条第三項第一号に掲げる所得の金額の計算上控除すべき金額とし、なお控除しきれない損失の金額があるときは、これを当該翌年分の同項第二号に掲げる所得の金額の計算上控除すべき金額とする。
\end{description}
\item[\rensuji{3}]法第六十二条第一項に規定する生活に通常必要でない資産について受けた損失の金額の計算の基礎となるその資産の価額は、次の各号に掲げる資産の区分に応じ当該各号に掲げる金額とする。
\begin{description}
\item[一]法第三十八条第一項(譲渡所得の金額の計算上控除する取得費)に規定する資産(次号に掲げるものを除く。)
\item[二]法第三十八条第二項に規定する資産
\end{description}
\end{description}
\subsection*{第六節 その他の収入金額及び必要経費の計算の特例等}
\addcontentsline{toc}{subsection}{第六節 その他の収入金額及び必要経費の計算の特例等}
\subsubsection*{第一款 事業を廃止した場合等の所得計算の特例}
\addcontentsline{toc}{subsubsection}{第一款 事業を廃止した場合等の所得計算の特例}
\noindent\hspace{10pt}(事業を廃止した場合の必要経費の特例)
\begin{description}
\item[第百七十九条]法第六十三条(事業を廃止した場合の必要経費の特例)の規定により同条に規定する必要経費に算入されるべき金額を同条に規定する廃止した日の属する年分又はその前年分の不動産所得の金額、事業所得の金額又は山林所得の金額の計算上必要経費に算入する場合における当該不動産所得の金額、事業所得の金額又は山林所得の金額の計算については、次に定めるところによる。
\begin{description}
\item[一]当該必要経費に算入されるべき金額が次に掲げる金額のうちいずれか低い金額以下である場合には、当該必要経費に算入されるべき金額の全部を当該廃止した日の属する年分の不動産所得の金額、事業所得の金額又は山林所得の金額の計算上必要経費に算入する。
\begin{description}
\item[イ]当該必要経費に算入されるべき金額が生じた時の直前において確定している当該廃止した日の属する年分の総所得金額、山林所得金額及び退職所得金額の合計額
\item[ロ]イに掲げる金額の計算の基礎とされる不動産所得の金額、事業所得の金額又は山林所得の金額
\end{description}
\item[二]当該必要経費に算入されるべき金額が前号に掲げる金額のうちいずれか低い金額をこえる場合には、当該必要経費に算入されるべき金額のうち、当該いずれか低い金額に相当する部分の金額については、当該廃止した日の属する年分の不動産所得の金額、事業所得の金額又は山林所得の金額の計算上必要経費に算入し、そのこえる部分の金額に相当する金額については、次に掲げる金額のうちいずれか低い金額を限度としてその年の前年分の不動産所得の金額、事業所得の金額又は山林所得の金額の計算上必要経費に算入する。
\begin{description}
\item[イ]当該必要経費に算入されるべき金額が生じた時の直前において確定している当該前年分の総所得金額、山林所得金額及び退職所得金額の合計額
\item[ロ]イに掲げる金額の計算の基礎とされる不動産所得の金額、事業所得の金額又は山林所得の金額
\end{description}
\end{description}
\end{description}
\noindent\hspace{10pt}(資産の譲渡代金が回収不能となつた場合等の所得計算の特例)
\begin{description}
\item[第百八十条]法第六十四条第一項(資産の譲渡代金が回収不能となつた場合等の所得計算の特例)に規定する政令で定める事由は、国家公務員退職手当法(昭和二十八年法律第百八十二号)第二条の三第二項(退職手当の支払)に規定する一般の退職手当の支払を受けた者が同法第十五条第一項(退職をした者の退職手当の返納)の規定による処分を受けたことその他これに類する事由とする。
\item[\rensuji{2}]法第六十四条第一項に規定する収入金額又は総収入金額で、回収することができないこととなつたもの(同条第二項の規定により回収することができないこととなつたものとみなされるものを含む。)又は返還すべきこととなつたもの(以下この項において「回収不能額等」という。)のうち、次に掲げる金額のうちいずれか低い金額に達するまでの金額は、同条第一項に規定する各種所得の金額の計算上、なかつたものとみなす。
\begin{description}
\item[一]回収不能額等が生じた時の直前において確定している法第六十四条第一項に規定する年分の総所得金額、退職所得金額及び山林所得金額の合計額
\item[二]前号に掲げる金額の計算の基礎とされる各種所得の金額のうち当該回収不能額等に係るものから、当該回収不能額等に相当する収入金額又は総収入金額がなかつたものとした場合に計算される当該各種所得の金額を控除した残額
\end{description}
\end{description}
\subsubsection*{第二款 資本的支出}
\addcontentsline{toc}{subsubsection}{第二款 資本的支出}
\noindent\hspace{10pt}(資本的支出)
\begin{description}
\item[第百八十一条]不動産所得、事業所得、山林所得又は雑所得を生ずべき業務を行なう居住者が、修理、改良その他いずれの名義をもつてするかを問わず、その業務の用に供する固定資産について支出する金額で次に掲げる金額に該当するもの(そのいずれにも該当する場合には、いずれか多い金額)は、その者のその支出する日の属する年分の不動産所得の金額、事業所得の金額、山林所得の金額又は雑所得の金額の計算上、必要経費に算入しない。
\begin{description}
\item[一]当該支出する金額のうち、その支出により、当該資産の取得の時において当該資産につき通常の管理又は修理をするものとした場合に予測される当該資産の使用可能期間を延長させる部分に対応する金額
\item[二]当該支出する金額のうち、その支出により、当該資産の取得の時において当該資産につき通常の管理又は修理をするものとした場合に予測されるその支出の時における当該資産の価額を増加させる部分に対応する金額
\end{description}
\end{description}
\subsubsection*{第三款 借地権等の更新料を支払つた場合の必要経費算入}
\addcontentsline{toc}{subsubsection}{第三款 借地権等の更新料を支払つた場合の必要経費算入}
\noindent\hspace{10pt}(借地権等の更新料を支払つた場合の必要経費算入)
\begin{description}
\item[第百八十二条]居住者が、不動産所得、事業所得、山林所得又は雑所得を生ずべき業務の用に供する借地権(地上権若しくは土地の賃借権又はこれらの権利に係る土地の転借に係る権利をいう。)又は地役権の存続期間の更新をする場合において、その更新の対価(以下この条において「更新料」という。)の支払をしたときは、当該借地権又は地役権の取得費に、その更新の時における当該借地権又は地役権の価額のうちに当該更新料の額の占める割合を乗じて計算した金額に相当する金額は、その更新のあつた日の属する年分の不動産所得の金額、事業所得の金額、山林所得の金額又は雑所得の金額の計算上、必要経費に算入する。
\item[\rensuji{2}]前項の取得費は、同項の借地権又は地役権の取得に要した金額のほか、同項に規定する更新前に支出した改良費及び更新料の額を含むものとし、その更新前に同項の規定により必要経費に算入された金額があるときは、当該金額を控除した金額とする。
\end{description}
\subsubsection*{第四款 資産に係る控除対象外消費税額等の必要経費算入等}
\addcontentsline{toc}{subsubsection}{第四款 資産に係る控除対象外消費税額等の必要経費算入等}
\noindent\hspace{10pt}(資産に係る控除対象外消費税額等の必要経費算入)
\begin{description}
\item[第百八十二条の二]居住者の不動産所得、事業所得、山林所得又は雑所得(以下この条において「事業所得等」という。)を生ずべき業務を行う年(消費税法(昭和六十三年法律第百八号)第三十条第二項(仕入れに係る消費税額の控除)に規定する課税売上割合に準ずる割合として財務省令で定めるところにより計算した割合が百分の八十以上である年に限る。)において資産に係る控除対象外消費税額等が生じた場合には、その生じた資産に係る控除対象外消費税額等の合計額については、その年の年分の不動産所得の金額、事業所得の金額、山林所得の金額又は雑所得の金額(以下この条において「事業所得等の金額」という。)の計算上、必要経費に算入する。
\item[\rensuji{2}]居住者の事業所得等を生ずべき業務を行う年(前項に規定する年を除く。)において生じた資産に係る控除対象外消費税額等が次に掲げる場合に該当する場合には、その該当する資産に係る控除対象外消費税額等の合計額については、その年の年分の事業所得等の金額の計算上、必要経費に算入する。
\begin{description}
\item[一]棚卸資産に係るものである場合
\item[二]消費税法第五条第一項(納税義務者)に規定する特定課税仕入れに係るものである場合
\item[三]二十万円未満である場合
\end{description}
\item[\rensuji{3}]居住者の事業所得等を生ずべき業務を行う年において生じた資産に係る控除対象外消費税額等の合計額(前二項の規定により必要経費に算入される金額を除く。以下この項及び次項において「繰延消費税額等」という。)につきその年の年分の事業所得等の金額の計算上必要経費に算入する金額は、当該繰延消費税額等を六十で除しこれにその年において当該業務を行つていた期間の月数を乗じて計算した金額の二分の一に相当する金額とする。
\item[\rensuji{4}]居住者のその年の前年以前の事業所得等を生ずべき業務を行う各年において生じた繰延消費税額等につきその年の年分の事業所得等の金額の計算上必要経費に算入する金額は、当該繰延消費税額等を六十で除しこれにその年において当該業務を行つていた期間の月数を乗じて計算した金額(当該計算した金額が当該繰延消費税額等のうち既に前項及びこの項の規定により事業所得等の金額の計算上必要経費に算入された金額以外の金額を超える場合には、当該金額)とする。
\item[\rensuji{5}]第一項から第三項までに規定する資産に係る控除対象外消費税額等とは、居住者が消費税法第十九条第一項(課税期間)に規定する課税期間につき同法第三十条第一項の規定の適用を受ける場合で、当該課税期間中に行つた同法第二条第一項第九号(定義)に規定する課税資産の譲渡等につき課されるべき消費税の額及び当該消費税の額を課税標準として課されるべき地方消費税の額に相当する金額並びに同法第三十条第二項に規定する課税仕入れ等の税額及び当該課税仕入れ等の税額に係る地方消費税の額に相当する金額をこれらに係る取引の対価と区分して取り扱つたときにおける当該課税仕入れ等の税額及び当該課税仕入れ等の税額に係る地方消費税の額に相当する金額の合計額のうち、同条第一項の規定による控除をすることができない金額及び当該控除をすることができない金額に係る地方消費税の額に相当する金額の合計額でそれぞれの資産に係るものをいう。
\item[\rensuji{6}]前項に規定する課税仕入れ等の税額に係る地方消費税の額に相当する金額又は控除をすることができない金額に係る地方消費税の額に相当する金額とは、それぞれ地方消費税を税率が百分の二・二の消費税であると仮定して消費税に関する法令の規定の例により計算した場合における同法第三十条第二項に規定する課税仕入れ等の税額に相当する金額又は同条第一項の規定による控除をすることができない金額に相当する金額をいう。
\item[\rensuji{7}]第三項及び第四項の月数は、暦に従つて計算し、一月に満たない端数を生じたときは、これを一月とする。
\item[\rensuji{8}]前三項に定めるもののほか、第一項から第四項までの規定の適用に関し必要な事項は、財務省令で定める。
\item[\rensuji{9}]居住者は、その年において第一項から第三項までに規定する資産に係る控除対象外消費税額等の合計額又は同項若しくは第四項に規定する繰延消費税額等につき必要経費に算入した金額がある場合には、その年分の確定申告書に、これらの規定により必要経費に算入される金額の計算に関する明細書を添付しなければならない。
\end{description}
\noindent\hspace{10pt}(貨物割に係る延滞税等の必要経費不算入)
\begin{description}
\item[第百八十二条の三]地方税法(昭和二十五年法律第二百二十六号)第七十二条の百第二項(貨物割の賦課徴収等)に規定する貨物割に係る延滞税及び加算税並びに同法附則第九条の四第二項(譲渡割の賦課徴収の特例等)に規定する譲渡割に係る延滞税及び加算税は、それぞれ法第四十五条第一項第五号(家事関連費等の必要経費不算入等)に掲げる延滞金及び加算金に該当するものとする。
\end{description}
\subsubsection*{第五款 生命保険契約等に基づく年金等に係る所得の計算}
\addcontentsline{toc}{subsubsection}{第五款 生命保険契約等に基づく年金等に係る所得の計算}
\noindent\hspace{10pt}(生命保険契約等に基づく年金に係る雑所得の金額の計算上控除する保険料等)
\begin{description}
\item[第百八十三条]生命保険契約等に基づく年金(法第三十五条第三項(公的年金等の定義)に規定する公的年金等を除く。以下この項において同じ。)の支払を受ける居住者のその支払を受ける年分の当該年金に係る雑所得の金額の計算については、次に定めるところによる。
\begin{description}
\item[一]当該年金の支払開始の日以後に当該年金の支払の基礎となる生命保険契約等に基づき分配を受ける剰余金又は割戻しを受ける割戻金の額は、その年分の雑所得に係る総収入金額に算入する。
\item[二]その年に支払を受ける当該年金の額に、イに掲げる金額のうちにロに掲げる金額の占める割合を乗じて計算した金額は、その年分の雑所得の金額の計算上、必要経費に算入する。
\begin{description}
\item[イ]次に掲げる年金の区分に応じそれぞれ次に定める金額
\item[ロ]当該生命保険契約等に係る保険料又は掛金の総額
\end{description}
\item[三]当該生命保険契約等が年金のほか一時金を支払う内容のものである場合には、前号ロに掲げる保険料又は掛金の総額は、当該生命保険契約等に係る保険料又は掛金の総額に、同号イ(1)又は(2)に定める支払総額又は支払総額の見込額と当該一時金の額との合計額のうちに当該支払総額又は支払総額の見込額の占める割合を乗じて計算した金額とする。
\item[四]前二号に規定する割合は、小数点以下二位まで算出し、三位以下を切り上げたところによる。
\end{description}
\item[\rensuji{2}]生命保険契約等に基づく一時金(法第三十一条各号(退職手当等とみなす一時金)に掲げるものを除く。以下この項において同じ。)の支払を受ける居住者のその支払を受ける年分の当該一時金に係る一時所得の金額の計算については、次に定めるところによる。
\begin{description}
\item[一]当該一時金の支払の基礎となる生命保険契約等に基づき分配を受ける剰余金又は割戻しを受ける割戻金の額で、当該一時金とともに又は当該一時金の支払を受けた後に支払を受けるものは、その年分の一時所得に係る総収入金額に算入する。
\item[二]当該生命保険契約等に係る保険料又は掛金(第八十二条の三第一項第二号イからリまでに掲げる資産及び確定拠出年金法第五十四条第一項(他の制度の資産の移換)、第五十四条の二第一項(脱退一時金相当額等の移換)又は第七十四条の二第一項(脱退一時金相当額等の移換)の規定により移換された同法第二条第十二項(定義)に規定する個人別管理資産に充てる資産を含む。第四項において同じ。)の総額は、その年分の一時所得の金額の計算上、支出した金額に算入する。
\begin{description}
\item[イ]旧厚生年金保険法第九章(厚生年金基金及び企業年金連合会)の規定に基づく一時金(第七十二条第二項(退職手当等とみなす一時金)に規定するものを除く。)に係る同項に規定する加入員の負担した掛金
\item[ロ]確定給付企業年金法第三条第一項(確定給付企業年金の実施)に規定する確定給付企業年金に係る規約に基づいて支給を受ける一時金(法第三十一条第三号に掲げるものを除く。)の額に第八十二条の三第一項第二号イからリまでに掲げる資産に係る部分に相当する金額が含まれている場合における当該金額に係る法第三十一条第三号に規定する加入者が負担した金額
\item[ハ]第七十二条第三項第五号イからハまでに掲げる規定に基づいて支給を受ける一時金(同号に掲げるものを除く。)の額に第八十二条の三第一項第二号イからリまでに掲げる資産に係る部分に相当する金額が含まれている場合における当該金額に係る第七十二条第三項第五号に規定する加入者が負担した金額
\item[ニ]小規模企業共済法第十二条第一項(解約手当金)に規定する解約手当金(第七十二条第三項第三号ロ及びハに掲げるものを除く。)に係る同号イに規定する小規模企業共済契約に基づく掛金
\item[ホ]確定拠出年金法附則第二条の二第二項及び第三条第二項(脱退一時金)に規定する脱退一時金に係る同法第三条第三項第七号の二(規約の承認)に規定する企業型年金加入者掛金及び同法第五十五条第二項第四号(規約の承認)に規定する個人型年金加入者掛金
\end{description}
\item[三]当該生命保険契約等が一時金のほか年金を支払う内容のものである場合には、前号に規定する保険料又は掛金の総額は、当該生命保険契約等に係る保険料又は掛金の総額から、当該保険料又は掛金の総額に前項第三号に規定する割合を乗じて計算した金額を控除した金額に相当する金額とする。
\end{description}
\item[\rensuji{3}]前二項に規定する生命保険契約等とは、次に掲げる契約又は規約をいう。
\begin{description}
\item[一]生命保険契約(保険業法第二条第三項(定義)に規定する生命保険会社又は同条第八項に規定する外国生命保険会社等の締結した保険契約をいう。第三号ロ及び次条第一項において同じ。)、旧簡易生命保険契約(第三十条第一号(非課税とされる保険金、損害賠償金等)に規定する旧簡易生命保険契約をいう。)及び生命共済に係る契約
\item[二]第七十三条第一項第一号(特定退職金共済団体の要件)に規定する退職金共済契約
\item[三]退職年金に関する次に掲げる契約
\begin{description}
\item[イ]信託契約
\item[ロ]生命保険契約
\item[ハ]生命共済に係る契約
\end{description}
\item[四]確定給付企業年金法第三条第一項に規定する確定給付企業年金に係る規約
\item[五]法第七十五条第二項第一号(小規模企業共済等掛金控除)に規定する契約
\item[六]確定拠出年金法第四条第三項(承認の基準等)に規定する企業型年金規約及び同法第五十六条第三項(承認の基準等)に規定する個人型年金規約
\end{description}
\item[\rensuji{4}]第一項及び第二項に規定する保険料又は掛金の総額は、当該生命保険契約等に係る保険料又は掛金の総額から次に掲げる金額を控除して計算するものとする。
\begin{description}
\item[一]第七十五条第一項(特定退職金共済団体の承認の取消し等)の規定による承認の取消しを受けた法人又は同条第三項の規定により承認が失効をした法人に対し前項第二号に掲げる退職金共済契約に基づき支出した掛金、確定給付企業年金法第百二条第三項若しくは第六項(事業主等又は連合会に対する監督)の規定による承認の取消しを受けた当該取消しに係るこれらの規定に規定する規約型企業年金に係る規約に基づき支出した掛金又は同項の規定による解散の命令を受けた同項に規定する基金の同法第十一条第一項(基金の規約で定める事項)に規定する規約に基づき支出した掛金及び法人税法施行令附則第十八条第一項(適格退職年金契約の承認の取消し)の規定による承認の取消しを受けた第七十六条第二項第一号(退職金共済制度等に基づく一時金で退職手当等とみなさないもの)に規定する信託会社等に対し当該取消しに係る同号に規定する契約に基づき支出した掛金又は保険料のうち、これらの取消し若しくは命令を受ける前又は当該失効前に支出したものの額(次号に該当するものを除くものとし、これらの掛金又は保険料の額のうちに、法第三十一条第三号若しくは第三十五条第三項第三号若しくは第七十二条第三項第五号若しくは第八十二条の二第二項第五号(公的年金等とされる年金)に規定する加入者の負担した金額(当該金額に第八十二条の三第一項第二号イからリまでに掲げる資産に係る当該加入者が負担した部分に相当する金額が含まれている場合には、当該金額を控除した金額)又は第七十二条第三項第四号若しくは第八十二条の二第二項第四号に規定する勤務をした者の負担した金額がある場合には、これらの金額を控除した金額とする。)
\item[二]次に掲げる保険料又は掛金(第六十五条(不適格退職金共済契約等に基づく掛金の取扱い)の規定により給与所得に係る収入金額に含まれるものを除く。)の額
\begin{description}
\item[イ]第七十六条第一項第二号又は第二項第二号に掲げる給付に係る保険料又は掛金
\item[ロ]旧厚生年金保険法第九章の規定に基づく一時金(第七十二条第二項に規定するものを除く。)に係る掛金(当該掛金の額のうちに同項に規定する加入員の負担した金額がある場合には、当該金額を控除した金額に相当する部分に限る。)
\item[ハ]確定給付企業年金法第三条第一項に規定する確定給付企業年金に係る規約に基づいて支給を受ける一時金(法第三十一条第三号に掲げるものを除く。)に係る掛金(当該掛金の額のうちに同号に規定する加入者の負担した金額がある場合には、当該金額を控除した金額に相当する部分に限る。)
\item[ニ]法人税法附則第二十条第三項(退職年金等積立金に対する法人税の特例)に規定する適格退職年金契約に基づいて支給を受ける一時金(第七十二条第三項第四号に掲げるものを除く。)に係る掛金又は保険料(当該掛金又は保険料の額のうちに同号に規定する勤務をした者の負担した金額がある場合には、当該金額を控除した金額に相当する部分に限る。)
\item[ホ]第七十二条第三項第五号イからハまでに掲げる規定に基づいて支給を受ける一時金(同号に掲げるものを除く。)に係る掛金(当該掛金の額のうちに同号に規定する加入者の負担した金額がある場合には、当該金額を控除した金額に相当する部分に限る。)
\item[ヘ]確定拠出年金法附則第二条の二第二項及び第三条第二項に規定する脱退一時金に係る掛金(当該掛金の額のうちに、同法第三条第三項第七号の二に規定する企業型年金加入者掛金の額又は同法第五十五条第二項第四号に規定する個人型年金加入者掛金の額がある場合には、これらの金額を控除した金額に相当する部分に限る。)
\item[ト]中小企業退職金共済法第十六条第一項(解約手当金)に規定する解約手当金又は第七十四条第五項(特定退職金共済団体の承認)に規定する特定退職金共済団体が行うこれに類する給付に係る掛金
\end{description}
\item[三]事業を営む個人又は法人が当該個人のその事業に係る使用人又は当該法人の使用人(役員を含む。次条第三項第一号において同じ。)のために支出した当該生命保険契約等に係る保険料又は掛金で当該個人のその事業に係る不動産所得の金額、事業所得の金額若しくは山林所得の金額又は当該法人の各事業年度の所得の金額の計算上必要経費又は損金の額に算入されるもののうち、これらの使用人の給与所得に係る収入金額に含まれないものの額(前二号に掲げるものを除く。)
\item[四]当該年金の支払開始の日前又は当該一時金の支払の日前に当該生命保険契約等に基づく剰余金の分配若しくは割戻金の割戻しを受け、又は当該生命保険契約等に基づき分配を受ける剰余金若しくは割戻しを受ける割戻金をもつて当該保険料若しくは掛金の払込みに充てた場合における当該剰余金又は割戻金の額
\end{description}
\end{description}
\noindent\hspace{10pt}(損害保険契約等に基づく年金に係る雑所得の金額の計算上控除する保険料等)
\begin{description}
\item[第百八十四条]損害保険契約等(法第七十六条第六項第四号(生命保険料控除)に掲げる保険契約で生命保険契約以外のもの、法第七十七条第二項各号(地震保険料控除)に掲げる契約及び第三百二十六条第二項各号(第二号を除く。)(生命保険契約等に基づく年金に係る源泉徴収)に掲げる契約をいう。以下この項において同じ。)に基づく年金の支払を受ける居住者のその支払を受ける年分の当該年金に係る雑所得の金額の計算については、次に定めるところによる。
\begin{description}
\item[一]当該年金の支払開始の日以後に当該年金の支払の基礎となる損害保険契約等に基づき分配を受ける剰余金又は割戻しを受ける割戻金の額は、その年分の雑所得に係る総収入金額に算入する。
\item[二]その年に支払を受ける当該年金の額に、イに掲げる金額のうちにロに掲げる金額の占める割合を乗じて計算した金額は、その年分の雑所得の金額の計算上、必要経費に算入する。
\begin{description}
\item[イ]次に掲げる年金の区分に応じそれぞれ次に定める金額
\item[ロ]当該損害保険契約等に係る保険料又は掛金の総額
\end{description}
\item[三]前号に規定する割合は、小数点以下二位まで算出し、三位以下を切り上げたところによる。
\end{description}
\item[\rensuji{2}]損害保険契約等(前項に規定する損害保険契約等及び保険業法第二条第十八項(定義)に規定する少額短期保険業者の締結した同条第四項に規定する損害保険会社又は同条第九項に規定する外国損害保険会社等の締結した保険契約(第四項において「損害保険契約」という。)に類する保険契約をいう。以下この項及び次項において同じ。)に基づく満期返戻金等の支払を受ける居住者のその支払を受ける年分の当該満期返戻金等に係る一時所得の金額の計算については、次に定めるところによる。
\begin{description}
\item[一]当該満期返戻金等の支払の基礎となる損害保険契約等に基づき分配を受ける剰余金又は割戻しを受ける割戻金の額で、当該満期返戻金等とともに又は当該満期返戻金等の支払を受けた後に支払を受けるものは、その年分の一時所得に係る総収入金額に算入する。
\item[二]当該損害保険契約等に係る保険料又は掛金の総額は、その年分の一時所得の金額の計算上、支出した金額に算入する。
\end{description}
\item[\rensuji{3}]前二項に規定する保険料又は掛金の総額は、当該損害保険契約等に係る保険料又は掛金の総額から次に掲げる金額を控除して計算するものとする。
\begin{description}
\item[一]事業を営む個人又は法人が当該個人のその事業に係る使用人又は当該法人の使用人のために支出した当該損害保険契約等に係る保険料又は掛金で当該個人のその事業に係る不動産所得の金額、事業所得の金額若しくは山林所得の金額又は当該法人の各事業年度の所得の金額の計算上必要経費又は損金の額に算入されるもののうち、これらの使用人の給与所得に係る収入金額に含まれないものの額
\item[二]当該年金の支払開始の日前又は当該満期返戻金等の支払の日前に当該損害保険契約等に基づく剰余金の分配若しくは割戻金の割戻しを受け、又は当該損害保険契約等に基づき分配を受ける剰余金若しくは割戻しを受ける割戻金をもつて当該保険料若しくは掛金の払込みに充てた場合における当該剰余金又は割戻金の額
\end{description}
\item[\rensuji{4}]前二項に規定する満期返戻金等とは、次に掲げるものをいう。
\begin{description}
\item[一]第一項に規定する保険契約、法第七十七条第二項第一号に掲げる契約又は法第二百七条第三号に掲げる契約で損害保険契約に該当するもののうち保険期間の満了後満期返戻金を支払う旨の特約がされているものに基づき支払を受ける満期返戻金及び解約返戻金(第一項に規定する損害保険契約等に基づく年金として当該損害保険契約等の保険期間の満了後に支払われる満期返戻金を除く。)
\item[二]法第七十七条第二項第二号に掲げる契約又は法第二百七条第三号に掲げる契約で損害保険契約以外のもののうち建物又は動産の共済期間中の耐存を共済事故とする共済に係る契約に基づき支払を受ける共済金(当該建物又は動産の耐存中に当該期間が満了したことによるものに限る。)及び解約返戻金
\item[三]保険業法第二条第十八項に規定する少額短期保険業者の締結した損害保険契約に類する保険契約のうち返戻金を支払う旨の特約がされているものに基づき支払を受ける返戻金
\end{description}
\end{description}
\noindent\hspace{10pt}(相続等に係る生命保険契約等に基づく年金に係る雑所得の金額の計算)
\begin{description}
\item[第百八十五条]第百八十三条第三項(生命保険契約等に基づく年金に係る雑所得の金額の計算上控除する保険料等)に規定する生命保険契約等(以下この項及び次項において「生命保険契約等」という。)に基づく年金(同条第一項に規定する年金をいう。以下この条において同じ。)の支払を受ける居住者が、当該年金(当該年金に係る権利につき所得税法等の一部を改正する法律(平成二十二年法律第六号)第三条(相続税法の一部改正)の規定による改正前の相続税法(昭和二十五年法律第七十三号。次条第一項において「旧相続税法」という。)第二十四条(定期金に関する権利の評価)の規定の適用があるもの(次項において「旧相続税法対象年金」という。)に限る。)に係る保険金受取人等に該当する場合には、当該居住者のその支払を受ける年分の当該年金に係る雑所得の金額の計算については、第百八十三条第一項の規定にかかわらず、次に定めるところによる。
\begin{description}
\item[一]その年に支払を受ける確定年金(年金の支払開始の日(その日において年金の支払を受ける者が当該居住者以外の者である場合には、当該居住者が最初に年金の支払を受ける日。以下この項及び次項において「支払開始日」という。)において支払総額(年金の支払の基礎となる生命保険契約等において定められている年金の総額のうち当該居住者が支払を受ける金額をいい、支払開始日以後に当該生命保険契約等に基づき分配を受ける剰余金又は割戻しを受ける割戻金の額に相当する部分の金額を除く。以下この条において同じ。)が確定している年金をいう。以下この項及び次項において同じ。)の額(第七号の規定により総収入金額に算入される金額を除く。)のうち次に掲げる確定年金の区分に応じそれぞれ次に定める金額は、その年分の雑所得に係る総収入金額に算入する。
\begin{description}
\item[イ]残存期間年数(当該居住者に係る支払開始日におけるその残存期間に係る年数をいい、当該年数に一年未満の端数を生じたときは、これを切り上げた年数をいう。以下この条において同じ。)が十年以下の確定年金
\item[ロ]残存期間年数が十年を超え五十五年以下の確定年金
\item[ハ]残存期間年数が五十五年を超える確定年金
\end{description}
\item[二]その年に支払を受ける終身年金(その支払開始日において支払総額が確定していない年金のうち、終身の年金で契約対象者(年金の支払の基礎となる生命保険契約等においてその者の生存が支払の条件とされている者をいう。以下この項において同じ。)の生存中に限り支払われるものをいう。以下この項及び次項において同じ。)の額(第七号の規定により総収入金額に算入される金額を除く。)のうち次に掲げる終身年金の区分に応じそれぞれ次に定める金額は、その年分の雑所得に係る総収入金額に算入する。
\begin{description}
\item[イ]支払開始日余命年数(当該契約対象者についての支払開始日における別表に定める余命年数をいう。以下この条において同じ。)が十年以下の終身年金
\item[ロ]支払開始日余命年数が十年を超え五十五年以下の終身年金
\item[ハ]支払開始日余命年数が五十五年を超える終身年金
\end{description}
\item[三]その年に支払を受ける有期年金(その支払開始日において支払総額が確定していない年金のうち、有期の年金で契約対象者がその期間(以下この号及び次項第三号において「支払期間」という。)内に死亡した場合にはその死亡後の支払期間につき支払を行わないものをいう。以下この号及び次項第三号において同じ。)の額(第七号の規定により総収入金額に算入される金額を除く。)のうち当該有期年金について当該支払期間に係る年数(当該年数に一年未満の端数を生じたときは、これを切り上げた年数。以下この号及び次項第三号において「支払期間年数」という。)を残存期間年数とし、支払総額見込額(当該有期年金の契約年額に当該支払期間に係る月数を乗じてこれを十二で除して計算した金額をいう。)を支払総額とする確定年金とみなして第一号の規定の例により計算した金額は、その年分の雑所得に係る総収入金額に算入する。
\item[四]その年に支払を受ける特定終身年金(その支払開始日において支払総額が確定していない年金のうち、終身の年金で、契約対象者の生存中支払われるほか、当該契約対象者がその支払開始日以後一定期間(以下この項及び次項において「保証期間」という。)内に死亡した場合にはその死亡後においてもその保証期間の終了の日までその支払が継続されるものをいう。以下この号及び次項第四号において同じ。)の額(第七号の規定により総収入金額に算入される金額を除く。)のうち次に掲げる特定終身年金の区分に応じそれぞれ次に定める金額は、その年分の雑所得に係る総収入金額に算入する。
\begin{description}
\item[イ]ロに掲げる特定終身年金以外の特定終身年金
\item[ロ](1)に掲げる金額が(2)に掲げる金額を超える特定終身年金
\end{description}
\item[五]その年に支払を受ける特定有期年金(その支払開始日において支払総額が確定していない年金のうち、有期の年金で契約対象者が保証期間内に死亡した場合にはその死亡後においてもその保証期間の終了の日までその支払が継続されるものをいう。以下この号及び次項第五号において同じ。)の額(第七号の規定により総収入金額に算入される金額を除く。)のうち当該特定有期年金について当該有期の期間(以下この号及び次項第五号において「支払期間」という。)に係る年数(当該年数に一年未満の端数を生じたときは、これを切り上げた年数。以下この号及び次項第五号において「支払期間年数」という。)を残存期間年数とし、支払総額見込額(当該特定有期年金の契約年額に当該支払期間に係る月数を乗じてこれを十二で除して計算した金額をいう。)を支払総額とする確定年金とみなして第一号の規定の例により計算した金額は、その年分の雑所得に係る総収入金額に算入する。
\begin{description}
\item[イ]ロに掲げる特定有期年金以外の特定有期年金
\item[ロ](1)に掲げる金額が(2)に掲げる金額を超える特定有期年金
\end{description}
\item[六]その支払を受ける年金につき第一号又は第二号(前三号の規定によりその例によることとされる場合を含む。)の規定により計算した支払年金対応額がその支払を受ける年金の額以上である場合には、前各号の規定にかかわらず、これらの規定により計算した支払年金対応額は、第一号又は第二号に規定する一課税単位当たりの金額、一単位当たりの金額又は一特定単位当たりの金額の整数倍の金額に当該年金の額に係る月数を乗じてこれを十二で除して計算した金額のうち当該年金の額に満たない最も多い金額とする。
\item[七]当該年金の支払開始日以後に当該年金の支払の基礎となる生命保険契約等に基づき分配を受ける剰余金又は割戻しを受ける割戻金の額は、その年分の雑所得に係る総収入金額に算入する。
\item[八]その年に支払を受ける当該年金(当該年金の支払開始の日における当該年金の支払を受ける者(次号において「当初年金受取人」という。)が当該居住者である場合の年金に限る。)の額(第一号から第六号までの規定により総収入金額に算入される部分の金額に限る。)に、イに掲げる金額のうちにロに掲げる金額の占める割合を乗じて計算した金額は、その年分の雑所得の金額の計算上、必要経費に算入する。
\begin{description}
\item[イ]次に掲げる年金の区分に応じそれぞれ次に定める金額
\item[ロ]当該生命保険契約等に係る保険料又は掛金の総額
\end{description}
\item[九]その年において支払を受ける当該年金の当初年金受取人が当該居住者以外の者である場合におけるその年分の雑所得の金額の計算上必要経費に算入する金額は、当該年金の額(第一号から第六号までの規定により総収入金額に算入される部分の金額に限る。)に、当該当初年金受取人に係る当該年金の支払開始の日における第百八十三条第一項第二号又は前号に規定する割合を乗じて計算した金額とする。
\item[十]当該生命保険契約等が年金のほか一時金を支払う内容のものである場合には、第八号ロに掲げる保険料又は掛金の総額は、当該生命保険契約等に係る保険料又は掛金の総額に、同号イ(1)又は(2)に定める支払総額又は支払総額見込額と当該一時金の額との合計額のうちに当該支払総額又は支払総額見込額の占める割合を乗じて計算した金額とする。
\item[十一]第八号及び前号に規定する割合は、小数点以下二位まで算出し、三位以下を切り上げたところによる。
\end{description}
\item[\rensuji{2}]生命保険契約等に基づく年金の支払を受ける居住者が、当該年金(旧相続税法対象年金を除く。)に係る保険金受取人等に該当する場合には、当該居住者のその支払を受ける年分の当該年金に係る雑所得の金額の計算については、第百八十三条第一項の規定にかかわらず、次に定めるところによる。
\begin{description}
\item[一]その年に支払を受ける確定年金の額(第七号の規定により総収入金額に算入される金額を除く。)のうち次に掲げる確定年金の区分に応じそれぞれ次に定める金額は、その年分の雑所得に係る総収入金額に算入する。
\begin{description}
\item[イ]相続税評価割合が百分の五十を超える確定年金
\item[ロ]相続税評価割合が百分の五十以下の確定年金
\end{description}
\item[二]その年に支払を受ける終身年金の額(第七号の規定により総収入金額に算入される金額を除く。)のうち次に掲げる終身年金の区分に応じそれぞれ次に定める金額は、その年分の雑所得に係る総収入金額に算入する。
\begin{description}
\item[イ]相続税評価割合が百分の五十を超える終身年金
\item[ロ]相続税評価割合が百分の五十以下の終身年金
\end{description}
\item[三]その年に支払を受ける有期年金の額(第七号の規定により総収入金額に算入される金額を除く。)のうち当該有期年金について支払期間年数を残存期間年数とし、支払総額見込額(当該有期年金の契約年額に支払期間に係る月数を乗じてこれを十二で除して計算した金額をいう。)を支払総額とする確定年金とみなして第一号の規定の例により計算した金額は、その年分の雑所得に係る総収入金額に算入する。
\item[四]その年に支払を受ける特定終身年金の額(第七号の規定により総収入金額に算入される金額を除く。)のうち当該特定終身年金の支払を受ける日の次に掲げる場合の区分に応じそれぞれ次に定める金額は、その年分の雑所得に係る総収入金額に算入する。
\begin{description}
\item[イ]その支払を受ける日が保証期間内の日である場合
\item[ロ]その支払を受ける日が保証期間の終了の日後である場合
\end{description}
\item[五]その年に支払を受ける特定有期年金の額(第七号の規定により総収入金額に算入される金額を除く。)のうち当該特定有期年金について支払期間年数を残存期間年数とし、支払総額見込額(当該特定有期年金の契約年額に支払期間に係る月数を乗じてこれを十二で除して計算した金額をいう。)を支払総額とする確定年金とみなして第一号の規定の例により計算した金額は、その年分の雑所得に係る総収入金額に算入する。
\begin{description}
\item[イ]ロに掲げる特定有期年金以外の特定有期年金
\item[ロ]支払開始日余命年数が当該保証期間年数を超える特定有期年金
\end{description}
\item[六]その支払を受ける年金につき第一号又は第二号(前三号の規定によりその例によることとされる場合を含む。)の規定により計算した支払年金対応額がその支払を受ける年金の額以上である場合には、前各号の規定にかかわらず、これらの規定により計算した支払年金対応額は、第一号又は第二号に規定する一課税単位当たりの金額又は一単位当たりの金額の整数倍の金額に当該年金の額に係る月数を乗じてこれを十二で除して計算した金額のうち当該年金の額に満たない最も多い金額とする。
\item[七]当該年金の支払開始日以後に当該年金の支払の基礎となる生命保険契約等に基づき分配を受ける剰余金又は割戻しを受ける割戻金の額は、その年分の雑所得に係る総収入金額に算入する。
\end{description}
\item[\rensuji{3}]この条において、次の各号に掲げる用語の意義は、当該各号に定めるところによる。
\begin{description}
\item[一]保険金受取人等
\begin{description}
\item[イ]相続税法第三条第一項第一号(相続又は遺贈により取得したものとみなす場合)に規定する保険金受取人
\item[ロ]相続税法第三条第一項第五号に規定する定期金受取人となつた場合における当該定期金受取人
\item[ハ]相続税法第三条第一項第六号に規定する定期金に関する権利を取得した者
\item[ニ]相続税法第五条第一項(贈与により取得したものとみなす場合)(同条第二項において準用する場合を含む。)に規定する保険金受取人
\item[ホ]相続税法第六条第一項(贈与により取得したものとみなす場合)(同条第二項において準用する場合を含む。)に規定する定期金受取人
\item[ヘ]相続税法第六条第三項に規定する定期金受取人
\item[ト]相続、遺贈又は個人からの贈与により保険金受取人又は定期金受取人となつた者
\end{description}
\item[二]調整年数
\begin{description}
\item[イ]十年を超え十五年以下の場合
\item[ロ]十五年を超え二十五年以下の場合
\item[ハ]二十五年を超え三十五年以下の場合
\item[ニ]三十五年を超え五十五年以下の場合
\end{description}
\item[三]相続税評価割合
\item[四]課税割合
\begin{description}
\item[イ]相続税評価割合が百分の五十を超え百分の五十五以下の場合
\item[ロ]相続税評価割合が百分の五十五を超え百分の六十以下の場合
\item[ハ]相続税評価割合が百分の六十を超え百分の六十五以下の場合
\item[ニ]相続税評価割合が百分の六十五を超え百分の七十以下の場合
\item[ホ]相続税評価割合が百分の七十を超え百分の七十五以下の場合
\item[ヘ]相続税評価割合が百分の七十五を超え百分の八十以下の場合
\item[ト]相続税評価割合が百分の八十を超え百分の八十三以下の場合
\item[チ]相続税評価割合が百分の八十三を超え百分の八十六以下の場合
\item[リ]相続税評価割合が百分の八十六を超え百分の八十九以下の場合
\item[ヌ]相続税評価割合が百分の八十九を超え百分の九十二以下の場合
\item[ル]相続税評価割合が百分の九十二を超え百分の九十五以下の場合
\item[ヲ]相続税評価割合が百分の九十五を超え百分の九十八以下の場合
\item[ワ]相続税評価割合が百分の九十八を超える場合
\end{description}
\item[五]特定期間年数
\begin{description}
\item[イ]相続税評価割合が百分の十以下である場合
\item[ロ]相続税評価割合が百分の十を超え百分の二十以下である場合
\item[ハ]相続税評価割合が百分の二十を超え百分の三十以下である場合
\item[ニ]相続税評価割合が百分の三十を超え百分の四十以下である場合
\item[ホ]相続税評価割合が百分の四十を超え百分の五十以下である場合
\end{description}
\end{description}
\item[\rensuji{4}]第百八十三条第四項の規定は、第一項第八号ロ又は第十号に規定する保険料又は掛金の総額について準用する。
\end{description}
\noindent\hspace{10pt}(相続等に係る損害保険契約等に基づく年金に係る雑所得の金額の計算)
\begin{description}
\item[第百八十六条]第百八十四条第一項(損害保険契約等に基づく年金に係る雑所得の金額の計算上控除する保険料等)に規定する損害保険契約等(以下この条において「損害保険契約等」という。)に基づく年金の支払を受ける居住者が、当該年金(当該年金に係る権利について、旧相続税法第二十四条(定期金に関する権利の評価)の規定の適用があるもの(次項において「旧相続税法対象年金」という。)に限る。)に係る前条第三項第一号に規定する保険金受取人等に該当する場合には、当該居住者のその支払を受ける年分の当該年金に係る雑所得の金額の計算については、第百八十四条第一項の規定にかかわらず、次に定めるところによる。
\begin{description}
\item[一]その年に支払を受ける確定型年金(年金の支払開始の日(その日において年金の支払を受ける者が当該居住者以外の者である場合には、当該居住者が最初に年金の支払を受ける日。以下この条において「支払開始日」という。)において支払総額(年金の支払の基礎となる損害保険契約等において定められている年金の総額のうち当該居住者が支払を受ける金額をいい、支払開始日以後に当該損害保険契約等に基づき分配を受ける剰余金又は割戻しを受ける割戻金の額に相当する部分の金額を除く。以下この項において同じ。)が確定している年金をいう。以下この条において同じ。)の額(第四号の規定により総収入金額に算入される金額を除く。)のうち当該確定型年金について前条第一項第一号に規定する確定年金とみなして同号の規定の例により計算した金額は、その年分の雑所得に係る総収入金額に算入する。
\item[二]その年に支払を受ける特定有期型年金(その支払開始日において支払総額が確定していない年金のうち、有期の年金で契約対象者(年金の支払の基礎となる損害保険契約等においてその者の生存が支払の条件とされている者をいう。)がその支払開始日以後一定期間(以下この号において「保証期間」という。)内に死亡した場合にはその死亡した日からその保証期間の終了の日までの期間に相当する部分の金額の支払が行われるものをいう。以下この条において同じ。)の額(第四号の規定により総収入金額に算入される金額を除く。)のうち当該特定有期型年金について前条第一項第五号に規定する特定有期年金とみなして同号の規定の例により計算した金額は、その年分の雑所得に係る総収入金額に算入する。
\item[三]前条第一項第六号の規定は、前二号の規定により計算した金額に係る同項第一号イに規定する支払年金対応額がその支払を受ける年金の額以上である場合について準用する。
\item[四]当該年金の支払開始日以後に当該年金の支払の基礎となる損害保険契約等に基づき分配を受ける剰余金又は割戻しを受ける割戻金の額は、その年分の雑所得に係る総収入金額に算入する。
\item[五]その年に支払を受ける当該年金(当該年金の支払開始の日における当該年金の支払を受ける者(次号において「当初年金受取人」という。)が当該居住者である場合の年金に限る。)の額(第一号から第三号までの規定により総収入金額に算入される部分の金額に限る。)に、イに掲げる金額のうちにロに掲げる金額の占める割合を乗じて計算した金額は、その年分の雑所得の金額の計算上、必要経費に算入する。
\begin{description}
\item[イ]次に掲げる年金の区分に応じそれぞれ次に定める金額
\item[ロ]当該損害保険契約等に係る保険料又は掛金の総額
\end{description}
\item[六]その年において支払を受ける当該年金の当初年金受取人が当該居住者以外の者である場合におけるその年分の雑所得の金額の計算上必要経費に算入する金額は、当該年金の額(第一号から第三号までの規定により総収入金額に算入される部分の金額に限る。)に、当該当初年金受取人に係る当該年金の支払開始の日における第百八十四条第一項第二号又は前号に規定する割合を乗じて計算した金額とする。
\item[七]第五号に規定する割合は、小数点以下二位まで算出し、三位以下を切り上げたところによる。
\end{description}
\item[\rensuji{2}]損害保険契約等に基づく年金の支払を受ける居住者が、当該年金(旧相続税法対象年金を除く。)に係る前条第三項第一号に規定する保険金受取人等に該当する場合には、当該居住者のその支払を受ける年分の当該年金に係る雑所得の金額の計算については、第百八十四条第一項の規定にかかわらず、次に定めるところによる。
\begin{description}
\item[一]その年に支払を受ける確定型年金の額(第四号の規定により総収入金額に算入される金額を除く。)のうち当該確定型年金について前条第二項第一号の確定年金とみなして同号の規定の例により計算した金額は、その年分の雑所得に係る総収入金額に算入する。
\item[二]その年に支払を受ける特定有期型年金の額(第四号の規定により総収入金額に算入される金額を除く。)のうち当該特定有期型年金について前条第二項第五号の特定有期年金とみなして同号の規定の例により計算した金額は、その年分の雑所得に係る総収入金額に算入する。
\item[三]前条第二項第六号の規定は、前二号の規定により計算した金額に係る同項第一号イの支払年金対応額がその支払を受ける年金の額以上である場合について準用する。
\item[四]当該年金の支払開始日以後に当該年金の支払の基礎となる損害保険契約等に基づき分配を受ける剰余金又は割戻しを受ける割戻金の額は、その年分の雑所得に係る総収入金額に算入する。
\end{description}
\item[\rensuji{3}]第百八十四条第三項の規定は、第一項第五号ロに規定する保険料又は掛金の総額について準用する。
\end{description}
\begin{description}
\item[第百八十七条]削除
\end{description}
\subsection*{第七節 収入及び費用の帰属の時期の特例}
\addcontentsline{toc}{subsection}{第七節 収入及び費用の帰属の時期の特例}
\subsubsection*{第一款 リース譲渡}
\addcontentsline{toc}{subsubsection}{第一款 リース譲渡}
\noindent\hspace{10pt}(延払基準の方法)
\begin{description}
\item[第百八十八条]法第六十五条第一項(リース譲渡に係る収入及び費用の帰属時期)に規定する政令で定める延払基準の方法は、次に掲げる方法とする。
\begin{description}
\item[一]法第六十五条第一項に規定するリース譲渡(以下この款において「リース譲渡」という。)の対価の額及びその原価の額(そのリース譲渡に要した手数料の額を含む。)にそのリース譲渡に係る賦払金割合(リース譲渡の対価の額のうちに、当該対価の額に係る賦払金であつてその年においてその支払の期日が到来するものの合計額(当該賦払金につき既にその年の前年以前に支払を受けている金額がある場合には、当該金額を除くものとし、その年の翌年以後において支払の期日が到来する賦払金につきその年中に支払を受けた金額がある場合には、当該金額を含む。)の占める割合をいう。)を乗じて計算した金額をその年分の収入金額及び費用の額とする方法
\item[二]リース譲渡に係るイ及びロに掲げる金額の合計額をその年分の収入金額とし、ハに掲げる金額をその年分の費用の額とする方法
\begin{description}
\item[イ]当該リース譲渡の対価の額から利息相当額(当該リース譲渡の対価の額のうちに含まれる利息に相当する金額をいう。ロにおいて同じ。)を控除した金額(ロにおいて「元本相当額」という。)をリース資産(法第六十五条第一項に規定するリース資産をいう。)のリース期間(同項に規定するリース取引に係る契約において定められた当該リース資産の賃貸借の期間をいう。以下この号及び第三項において同じ。)の月数で除し、これにその年における当該リース期間の月数を乗じて計算した金額
\item[ロ]当該リース譲渡の利息相当額がその元本相当額のうちその支払の期日が到来していないものの金額に応じて生ずるものとした場合にその年におけるリース期間に帰せられる利息相当額
\item[ハ]当該リース譲渡の原価の額をリース期間の月数で除し、これにその年における当該リース期間の月数を乗じて計算した金額
\end{description}
\end{description}
\item[\rensuji{2}]法第六十五条第二項の対価の額のうち利息に相当する部分の金額は、リース譲渡の対価の額からその原価の額を控除した金額の百分の二十に相当する金額(次項において「利息相当額」という。)とする。
\item[\rensuji{3}]法第六十五条第二項に規定する収入金額として政令で定める金額は、第一号及び第二号に掲げる金額の合計額とし、同項に規定する費用の額として政令で定める金額は、第三号に掲げる金額とする。
\begin{description}
\item[一]リース譲渡の対価の額から利息相当額を控除した金額(次号において「元本相当額」という。)をリース期間の月数で除し、これにその年における当該リース期間の月数を乗じて計算した金額
\item[二]リース譲渡に係る賦払金の支払を、支払期間をリース期間と、支払日を当該リース譲渡に係る対価の支払の期日と、各支払日の支払額を当該リース譲渡に係る対価の各支払日の支払額と、利息の総額を利息相当額と、元本の総額を元本相当額とし、利率を当該支払期間、支払日、各支払日の支払額、利息の総額及び元本の総額を基礎とした複利法により求められる一定の率として賦払の方法により行うものとした場合にその年におけるリース期間に帰せられる利息の額に相当する金額
\item[三]リース譲渡の原価の額をリース期間の月数で除し、これにその年における当該リース期間の月数を乗じて計算した金額
\end{description}
\item[\rensuji{4}]第一項第二号及び前項の月数は、暦に従つて計算し、一月に満たない端数を生じたときは、これを一月とする。
\end{description}
\noindent\hspace{10pt}(延払基準の方法により経理しなかつた場合等の処理)
\begin{description}
\item[第百八十九条]法第六十五条第一項本文(リース譲渡に係る収入及び費用の帰属時期)の規定の適用を受ける居住者がリース譲渡に係る収入金額及び費用の額につき、そのリース譲渡の日の属する年の翌年以後のいずれかの年において同項に規定する延払基準の方法により経理しなかつた場合には、そのリース譲渡に係る収入金額及び費用の額(その経理しなかつた年の前年分以前の各年分の事業所得の金額の計算上総収入金額及び必要経費に算入されるものを除く。)は、その経理しなかつた年分の事業所得の金額の計算上、総収入金額及び必要経費に算入する。
\item[\rensuji{2}]法第六十五条第二項の規定の適用を受けている居住者がその適用を受けているリース譲渡に係る契約の解除又は他の者に対する移転をした場合には、そのリース譲渡に係る収入金額及び費用の額(その解除又は移転をした日の属する年の前年分以前の各年分の事業所得の金額の計算上総収入金額及び必要経費に算入されるものを除く。)は、その解除又は移転をした日の属する年分の事業所得の金額の計算上、総収入金額及び必要経費に算入する。
\end{description}
\begin{description}
\item[第百九十条]削除
\end{description}
\noindent\hspace{10pt}(事業の廃止、死亡等の場合のリース譲渡に係る収入及び費用の帰属時期)
\begin{description}
\item[第百九十一条]リース譲渡に係る収入金額及び費用の額につき法第六十五条第一項(リース譲渡に係る収入及び費用の帰属時期)の規定の適用を受けている居住者が次に掲げる場合に該当することとなつたときは、その該当することとなつた日の属する年以前の各年においてその者がしたリース譲渡に係る収入金額及び費用の額(当該各年分の事業所得の金額の計算上総収入金額及び必要経費に算入されるものを除く。)は、同項の規定にかかわらず、その者の同日の属する年分の事業所得の金額の計算上、総収入金額及び必要経費に算入する。
\begin{description}
\item[一]その者が死亡した場合において、当該リース譲渡に係る事業を承継した相続人がないとき。
\item[二]その者が当該リース譲渡に係る事業の全部を譲渡し、又は廃止した場合
\item[三]その者が出国をした場合
\end{description}
\item[\rensuji{2}]リース譲渡に係る収入金額及び費用の額につき法第六十五条第一項の規定の適用を受けている居住者が死亡した場合において、その者の当該リース譲渡に係る事業を承継した相続人が当該収入金額及び費用の額につき、当該死亡の日の属する年以後の各年において同項に規定する延払基準の方法(以下この条において「延払基準の方法」という。)により経理したときは、その経理した収入金額及び費用の額は、当該各年分の事業所得の金額の計算上、総収入金額及び必要経費に算入する。
\item[\rensuji{3}]前項に規定する居住者が死亡した場合において、その者の同項に規定する事業を承継した相続人が、当該死亡の日の属する年以後のいずれかの年においてその居住者のリース譲渡に係る収入金額及び費用の額につき延払基準の方法により経理しなかつたときは、その居住者のリース譲渡に係る収入金額及び費用の額(その居住者の各年分の事業所得の金額又は当該相続人のその年の前年分以前の各年分の事業所得の金額の計算上総収入金額及び必要経費に算入されるものを除く。)は、その該当することとなつた年分の事業所得の金額の計算上、総収入金額及び必要経費に算入する。
\item[\rensuji{4}]第一項の規定は、第二項の規定の適用を受けている同項の相続人が第一項各号に掲げる場合に該当することとなつた場合について準用する。
\item[\rensuji{5}]リース譲渡に係る収入金額及び費用の額につき法第六十五条第二項の規定の適用を受けている居住者が第一項各号に掲げる場合に該当することとなつたときは、その該当することとなつた日の属する年以前の各年においてその者がしたリース譲渡に係る収入金額及び費用の額(当該各年分の事業所得の金額の計算上総収入金額及び必要経費に算入されるものを除く。)は、同条第二項の規定にかかわらず、その者の同日の属する年分の事業所得の金額の計算上、総収入金額及び必要経費に算入する。
\item[\rensuji{6}]リース譲渡に係る収入金額及び費用の額につき法第六十五条第二項の規定の適用を受けている居住者が死亡した場合において、その者の当該リース譲渡に係る事業を承継した相続人が当該居住者から同項の規定の適用を受けているリース譲渡に係る契約の移転を受けたときは、当該死亡の日の属する年以後の各年分における当該相続人の同項の規定の適用については、当該リース譲渡に係る対価の額及び原価の額並びにリース期間(第百八十八条第一項第二号イに規定するリース期間をいう。以下この項において同じ。)は当該相続人が行つたリース譲渡に係る対価の額及び原価の額並びにリース期間と、当該居住者がした法第六十五条第三項の明細の記載は当該相続人がしたものと、それぞれみなす。
\item[\rensuji{7}]前項に規定する居住者が死亡した場合において、その者の同項に規定する事業を承継した相続人が、法第六十五条第二項の規定の適用を受けているリース譲渡に係る契約の解除又は他の者に対する移転をした場合には、そのリース譲渡に係る収入金額及び費用の額(その居住者の各年分の事業所得の金額又は当該相続人のその年の前年分以前の各年分の事業所得の金額の計算上総収入金額及び必要経費に算入されるものを除く。)は、その該当することとなつた年分の事業所得の金額の計算上、総収入金額及び必要経費に算入する。
\item[\rensuji{8}]第五項の規定は、第六項の規定の適用を受けている同項の相続人が第一項各号に掲げる場合に該当することとなつた場合について準用する。
\end{description}
\subsubsection*{第二款 工事の請負}
\addcontentsline{toc}{subsubsection}{第二款 工事の請負}
\noindent\hspace{10pt}(工事の請負)
\begin{description}
\item[第百九十二条]法第六十六条第一項(工事の請負に係る収入及び費用の帰属時期)に規定する政令で定める大規模な工事は、その請負の対価の額(その支払が外国通貨で行われるべきこととされている工事(製造及びソフトウエアの開発を含む。以下この款において同じ。)については、その工事に係る契約の時における外国為替の売買相場による円換算額とする。)が十億円以上の工事とする。
\item[\rensuji{2}]法第六十六条第一項に規定する政令で定める要件は、当該工事に係る契約において、その請負の対価の額の二分の一以上が当該工事の目的物の引渡しの期日から一年を経過する日後に支払われることが定められていないものであることとする。
\item[\rensuji{3}]法第六十六条第一項及び第二項に規定する政令で定める工事進行基準の方法は、工事の請負の対価の額及びその工事原価の額(その年十二月三十一日(年の中途において死亡した場合には、その死亡の時。次項及び第六項において同じ。)の現況によりその工事につき見積もられる工事の原価の額をいう。以下この項において同じ。)に同日におけるその工事に係る進行割合(工事原価の額のうちに工事のために既に要した原材料費、労務費その他の経費の額の合計額の占める割合その他の工事の進行の度合を示すものとして合理的と認められるものに基づいて計算した割合をいう。)を乗じて計算した金額から、それぞれその年の前年以前の各年分の収入金額とされた金額及び費用の額とされた金額を控除した金額をその年分の収入金額及び費用の額とする方法とする。
\item[\rensuji{4}]居住者の請負をした工事(当該工事に係る追加の工事を含む。)の請負の対価の額がその年十二月三十一日において確定していないときにおける法第六十六条第一項の規定の適用については、同日の現況により当該工事につき見積もられる工事の原価の額をその請負の対価の額とみなす。
\item[\rensuji{5}]居住者の請負をした工事(法第六十六条第二項本文の規定の適用を受けているものを除く。)が、請負の対価の額の引上げその他の事由によりその着手の日の属する年(以下この項において「着工の年」という。)の翌年以後の年(その工事の目的物の引渡しの日の属する年(以下この項において「引渡し年」という。)を除く。)において長期大規模工事(同条第一項に規定する長期大規模工事をいう。以下この款において同じ。)に該当することとなつた場合における同項の規定の適用については、第三項の規定にかかわらず、当該工事の請負に係る既往年分の収入金額及び費用の額(その工事の請負に係る収入金額及び費用の額につき着工の年以後の各年において同項に規定する工事進行基準の方法により当該各年分の収入金額及び費用の額を計算することとした場合に着工の年からその該当することとなつた日の属する年(以下この項において「適用開始年」という。)の前年までの各年分の収入金額及び費用の額とされる金額をいう。)は、当該適用開始年から引渡し年の前年までの各年分の当該工事の請負に係る収入金額及び費用の額に含まれないものとすることができる。
\begin{description}
\item[一]当該適用開始年以後のいずれかの年において第三項に規定する工事進行基準の方法により経理した場合
\item[二]当該適用開始年以後のいずれかの年において本文の規定の適用を受けなかつた場合
\end{description}
\item[\rensuji{6}]居住者の請負をした長期大規模工事であつて、その年の十二月三十一日において、その着手の日から六月を経過していないもの又はその第三項に規定する進行割合が百分の二十に満たないものに係る法第六十六条第一項の規定の適用については、第三項の規定にかかわらず、当該長期大規模工事の請負に係るその年分の収入金額及び費用の額は、ないものとすることができる。
\item[\rensuji{7}]法第六十六条第一項の規定を適用する場合において、同項の居住者が長期大規模工事に着手したかどうかの判定は、当該居住者がその請け負つた工事の内容を完成するために行う一連の作業のうち重要な部分の作業を開始したかどうかによるものとする。
\item[\rensuji{8}]第五項本文の規定は、同項本文の規定の適用を受けようとする年分の確定申告書に同項本文の規定の適用を受けようとする工事の名称並びにその工事の請負に係る同項本文に規定する既往年分の収入金額及び費用の額の計算に関する明細を記載した書類の添付がある場合に限り、適用する。
\item[\rensuji{9}]第四項の規定は、法第六十六条第二項本文の規定を適用する場合(第十一項の規定の適用を受ける場合を除く。)について準用する。
\item[\rensuji{10}]第七項の規定は、法第六十六条第二項本文の規定を適用する場合における同項に規定する工事に着手したかどうかの判定について準用する。
\item[\rensuji{11}]居住者の請負をした法第六十六条第二項に規定する工事のうちその請負の対価の額がその着手の日において確定していないものに係る同項の規定の適用については、当該請負の対価の額の確定の日を当該工事の着手の日とすることができる。
\end{description}
\noindent\hspace{10pt}(工事進行基準の方法による未収入金)
\begin{description}
\item[第百九十三条]居住者の請負をした工事につきその着手の日からその目的物の引渡しの日の前日までの期間内の日の属する各年分において法第六十六条第一項又は第二項本文(工事の請負に係る収入及び費用の帰属時期)の規定の適用を受けている場合には、当該工事に係る第一号に掲げる金額から第二号に掲げる金額を控除した金額を当該工事の請負に係る売掛債権等(売掛金、貸付金その他これらに準ずる金銭債権をいう。)の額として、当該各年分の事業所得の金額を計算する。
\begin{description}
\item[一]当該工事の請負に係る収入金額のうち、法第六十六条第一項又は第二項本文に規定する工事進行基準の方法によりその年の前年分以前の各年分の収入金額とされた金額及びその年の年分の収入金額とされる金額の合計額(同項ただし書に規定する経理しなかつた年の翌年分以後の年分の収入金額を除く。)
\item[二]既に当該工事の請負の対価として支払われた金額(当該対価の額でまだ支払われていない金額のうち、当該対価の支払を受ける権利の移転により当該居住者が当該対価の支払を受けない金額を含む。)
\end{description}
\item[\rensuji{2}]前項の売掛債権等につき貸倒れによる損失が生じた場合の同項の売掛債権等の額の計算その他同項の規定の適用に関し必要な事項は、財務省令で定める。
\end{description}
\noindent\hspace{10pt}(死亡の場合の工事の請負に係る収入及び費用の帰属時期)
\begin{description}
\item[第百九十四条]長期大規模工事の請負に係る収入金額及び費用の額につき法第六十六条第一項(工事の請負に係る収入及び費用の帰属時期)の規定の適用を受けている居住者が死亡したときは、その長期大規模工事の請負に係る収入金額及び費用の額のうち、その居住者のその長期大規模工事の請負に係る事業を承継した相続人の当該死亡の日の属する年からその長期大規模工事の目的物の引渡しの日の属する年の前年までの各年分の収入金額及び費用の額として同項に規定する工事進行基準の方法により計算した収入金額及び費用の額は、当該各年分の事業所得の金額の計算上、総収入金額及び必要経費に算入する。
\item[\rensuji{2}]法第六十六条第二項の工事の請負に係る収入金額及び費用の額につき同項の規定の適用を受けている居住者が死亡した場合において、その居住者のその工事の請負に係る事業を承継した相続人が当該収入金額及び費用の額につき、当該死亡の日の属する年からその工事の目的物の引渡しの日の属する年の前年までの各年において同項に規定する工事進行基準の方法により経理したときは、その経理した収入金額及び費用の額は、当該各年分の事業所得の金額の計算上、総収入金額及び必要経費に算入する。
\end{description}
\subsubsection*{第三款 小規模事業者の収入及び費用の帰属時期}
\addcontentsline{toc}{subsubsection}{第三款 小規模事業者の収入及び費用の帰属時期}
\noindent\hspace{10pt}(小規模事業者の要件)
\begin{description}
\item[第百九十五条]法第六十七条(小規模事業者の収入及び費用の帰属時期)に規定する政令で定める要件は、次の各号に掲げる要件とする。
\begin{description}
\item[一]その年の前前年分の不動産所得の金額及び事業所得の金額(法第五十七条(事業に専従する親族がある場合の必要経費の特例等)の規定を適用しないで計算した場合の金額とする。)の合計額が三百万円以下であること。
\item[二]既に法第六十七条の規定の適用を受けたことがあり、かつ、その後同条の規定の適用を受けないこととなつた者については、再び同条の規定の適用を受けることにつき財務省令で定めるところにより納税地の所轄税務署長の承認を受けた者であること。
\end{description}
\end{description}
\noindent\hspace{10pt}(小規模事業者の収入及び費用の帰属時期)
\begin{description}
\item[第百九十六条]法第六十七条(小規模事業者の収入及び費用の帰属時期)に規定する居住者で前条各号に掲げる要件に該当するもののその年分(不動産所得を生ずべき業務及び事業所得を生ずべき業務の全部を譲渡し、若しくは廃止し、又は死亡した日の属する年分を除く。)の不動産所得の金額及び事業所得の金額(山林の伐採又は譲渡に係るものを除く。)の計算上総収入金額に算入すべき金額は、法第二編第二章第二節第三款(収入金額の計算)(法第四十一条(農産物の収穫の場合の総収入金額算入)を除く。)の規定の適用を受けるものを除き、その者の選択により、これらの業務につきその年において収入した金額(金銭以外の物又は権利その他経済的な利益をもつて収入した場合には、その金銭以外の物又は権利その他経済的な利益の価額)とすることができる。
\item[\rensuji{2}]前項の規定の適用を受ける居住者のその年分の同項に規定する不動産所得の金額及び事業所得の金額の計算上必要経費に算入すべき金額は、償却費並びに法第五十一条第一項及び第四項(資産損失の必要経費算入)の規定の適用を受けるものを除き、その年においてこれらの所得の総収入金額を得るために直接支出した費用の額及びその年においてこれらの所得を生ずべき業務について支出した費用の額とする。
\item[\rensuji{3}]前二項に定めるもののほか、第一項の規定の適用を受ける居住者がその適用を受けないこととなる場合における不動産所得又は事業所得に係る総収入金額及び必要経費の特例その他前二項の規定の適用に関し必要な事項は、財務省令で定める。
\end{description}
\noindent\hspace{10pt}(収入及び費用の帰属時期の特例を受けるための手続等)
\begin{description}
\item[第百九十七条]その年分以後の各年分の所得税につき前条第一項の選択をする居住者は、その年三月十五日まで(その年一月十六日以後新たに同項に規定する業務を開始した場合には、その業務を開始した日から二月以内)に、同項の規定の適用を受けようとする旨その他財務省令で定める事項を記載した届出書を納税地の所轄税務署長に提出しなければならない。
\item[\rensuji{2}]前条第一項の規定の適用を受ける居住者は、その年分以後の各年分の所得税につき同項の規定の適用を受けることをやめようとする場合には、その年三月十五日までに、その適用を受けることをやめる旨その他財務省令で定める事項を記載した届出書を納税地の所轄税務署長に提出しなければならない。
\end{description}
\subsection*{第七節の二 リース取引}
\addcontentsline{toc}{subsection}{第七節の二 リース取引}
\noindent\hspace{10pt}(リース取引の範囲)
\begin{description}
\item[第百九十七条の二]法第六十七条の二第三項(リース取引に係る所得の金額の計算)に規定する政令で定める資産の賃貸借は、土地の賃貸借のうち、第七十九条(資産の譲渡とみなされる行為)の規定の適用のあるもの及び次に掲げる要件(これらに準ずるものを含む。)のいずれにも該当しないものとする。
\begin{description}
\item[一]当該土地の賃貸借に係る契約において定められている当該賃貸借の期間(以下この条において「賃貸借期間」という。)の終了の時又は当該賃貸借期間の中途において、当該土地が無償又は名目的な対価の額で当該賃貸借に係る賃借人に譲渡されるものであること。
\item[二]当該土地の賃貸借に係る賃借人に対し、賃貸借期間終了の時又は賃貸借期間の中途において当該土地を著しく有利な価額で買い取る権利が与えられているものであること。
\end{description}
\item[\rensuji{2}]資産の賃貸借につき、その賃貸借期間(当該資産の賃貸借に係る契約の解除をすることができないものとされている期間に限る。)において賃借人が支払う賃借料の金額の合計額がその資産の取得のために通常要する価額(当該資産を業務の用に供するために要する費用の額を含む。)のおおむね百分の九十に相当する金額を超える場合には、当該資産の賃貸借は、法第六十七条の二第三項第二号の資産の使用に伴つて生ずる費用を実質的に負担すべきこととされているものであることに該当するものとする。
\end{description}
\subsection*{第七節の三 信託に係る所得の金額の計算}
\addcontentsline{toc}{subsection}{第七節の三 信託に係る所得の金額の計算}
\begin{description}
\item[第百九十七条の三]法第六十七条の三第一項(信託に係る所得の金額の計算)に規定する政令で定める金額は、同項の法人課税信託が法人税法第二条第二十九号の二ロ(定義)に掲げる信託に該当しないこととなつた時の直前における同項に規定する受託法人の同項の信託財産に属する資産及び負債の帳簿価額に相当する金額とする。
\item[\rensuji{2}]法第六十七条の三第一項の居住者が同項の規定により資産及び負債の引継ぎを受けたものとされた場合における同項の信託財産に属する資産については、前項に規定する該当しないこととなつた時の直前における同項に規定する帳簿価額に相当する金額により取得したものとみなして、当該居住者の各年分の各種所得の金額を計算するものとする。
\item[\rensuji{3}]法第六十七条の三第一項の居住者が同項の規定により資産及び負債の引継ぎを受けたものとされた場合におけるその引継ぎにより生じた損失の額は、当該居住者の各年分の各種所得の金額の計算上、生じなかつたものとする。
\item[\rensuji{4}]法第六十七条の三第二項に規定する収益の額は、第一項に規定する資産の同項の帳簿価額の合計額が同項に規定する負債の同項の帳簿価額の合計額を超える場合におけるその超える部分の金額に相当する金額とし、前項に規定する損失の額は、当該資産の帳簿価額の合計額が当該負債の帳簿価額の合計額に満たない場合におけるその満たない部分の金額に相当する金額とする。
\item[\rensuji{5}]法第六十七条の三第三項に規定する信託に関する権利が当該信託に関する権利の全部でない場合における同項から同条第六項までの規定の適用については、次に定めるところによる。
\begin{description}
\item[一]当該信託についての受益者等(法第六十七条の三第七項に規定する受益者等をいう。以下この項において同じ。)が一である場合には、当該信託に関する権利の全部を当該受益者等が有するものとみなす。
\item[二]当該信託についての受益者等が二以上ある場合には、当該信託に関する権利の全部をそれぞれの受益者等がその有する権利の内容に応じて有するものとみなす。
\end{description}
\end{description}
\subsection*{第八節 損益通算及び損失の繰越控除}
\addcontentsline{toc}{subsection}{第八節 損益通算及び損失の繰越控除}
\noindent\hspace{10pt}(損益通算の順序)
\begin{description}
\item[第百九十八条]法第六十九条第一項(損益通算)の政令で定める順序による控除は、次に定めるところによる。
\begin{description}
\item[一]不動産所得の金額又は事業所得の金額の計算上生じた損失の金額があるときは、これをまず他の利子所得の金額、配当所得の金額、不動産所得の金額、事業所得の金額、給与所得の金額及び雑所得の金額(以下この条において「経常所得の金額」という。)から控除する。
\item[二]譲渡所得の金額の計算上生じた損失の金額があるときは、これをまず一時所得の金額から控除する。
\item[三]第一号の場合において、同号の規定による控除をしてもなお控除しきれない損失の金額があるときは、これを譲渡所得の金額及び一時所得の金額(前号の規定による控除が行なわれる場合には、同号の規定による控除後の金額)から順次控除する。
\item[四]第二号の場合において、同号の規定による控除をしてもなお控除しきれない損失の金額があるときは、これを経常所得の金額(第一号の規定による控除が行なわれる場合には、同号の規定による控除後の金額)から控除する。
\item[五]第一号又は第二号の場合において、前各号の規定による控除をしてもなお控除しきれない損失の金額があるときは、これをまず山林所得の金額から控除し、なお控除しきれない損失の金額があるときは、退職所得の金額から控除する。
\item[六]山林所得の金額の計算上生じた損失の金額があるときは、これをまず経常所得の金額(第一号又は第四号の規定による控除が行なわれる場合には、これらの規定による控除後の金額)から控除し、なお控除しきれない損失の金額があるときは、譲渡所得の金額及び一時所得の金額(第二号又は第三号の規定による控除が行なわれる場合には、これらの規定による控除後の金額)から順次控除し、なお控除しきれない損失の金額があるときは、退職所得の金額(前号の規定による控除が行なわれる場合には、同号の規定による控除後の金額)から控除する。
\end{description}
\end{description}
\noindent\hspace{10pt}(変動所得の損失等の損益通算)
\begin{description}
\item[第百九十九条]前条の場合において、不動産所得の金額、事業所得の金額又は山林所得の金額の計算上生じた損失の金額のうちに法第七十条第二項第一号(純損失の繰越控除)の変動所得の金額の計算上生じた損失の金額(以下この条において「変動所得の損失の金額」という。)、同項第二号の被災事業用資産の損失の金額(以下この条において「被災事業用資産の損失の金額」という。)又はその他の損失の金額の二以上があるときは、これらの損失の金額の控除の順序については、次に定めるところによる。
\begin{description}
\item[一]不動産所得の金額又は事業所得の金額の計算上生じた損失の金額のうちに変動所得の損失の金額、被災事業用資産の損失の金額又はその他の損失の金額の二以上があるときは、まずその他の損失の金額を控除し、次に被災事業用資産の損失の金額及び変動所得の損失の金額を順次控除する。
\item[二]山林所得の金額の計算上生じた損失の金額のうちに被災事業用資産の損失の金額とその他の損失の金額とがあるときは、まずその他の損失の金額を控除し、次に被災事業用資産の損失の金額を控除する。
\end{description}
\end{description}
\noindent\hspace{10pt}(損益通算の対象とならない損失の控除)
\begin{description}
\item[第二百条]法第六十九条第二項(損益通算の対象とならない損失)に規定する政令で定める損失の金額は、第百七十八条第一項第一号(生活に通常必要でない資産の災害による損失額の計算等)に規定する競走馬の譲渡に係る譲渡所得の金額の計算上生じた損失の金額とする。
\item[\rensuji{2}]譲渡所得の金額の計算上生じた損失の金額のうちに前項に規定する競走馬の譲渡に係る損失の金額がある場合には、当該損失の金額は、当該競走馬の保有に係る雑所得の金額から控除する。
\end{description}
\noindent\hspace{10pt}(純損失の繰越控除)
\begin{description}
\item[第二百一条]法第七十条第一項又は第二項(純損失の繰越控除)の規定による純損失の金額の控除については、次に定めるところによる。
\begin{description}
\item[一]控除する純損失の金額が前年以前三年内の二以上の年に生じたものである場合には、これらの年のうち最も古い年に生じた純損失の金額から順次控除する。
\item[二]前年以前三年内の一の年において生じた純損失の金額の控除については、次に定めるところによる。
\begin{description}
\item[イ]純損失の金額のうちに総所得金額の計算上生じた損失の部分の金額(第百九十八条第一号から第五号まで(損益通算)の規定による控除をしてもなお控除しきれない損失の金額をいう。ハにおいて同じ。)があるときは、これをまずその年分の総所得金額から控除する。
\item[ロ]純損失の金額のうちに山林所得金額の計算上生じた損失の部分の金額(第百九十八条第六号の規定による控除をしてもなお控除しきれない損失の金額をいう。ニにおいて同じ。)があるときは、これをまずその年分の山林所得金額から控除する。
\item[ハ]イの規定による控除をしてもなお控除しきれない総所得金額の計算上生じた損失の部分の金額は、その年分の山林所得金額(ロの規定による控除が行なわれる場合には、当該控除後の金額)から控除し、次に退職所得金額から控除する。
\item[ニ]ロの規定による控除をしてもなお控除しきれない山林所得金額の計算上生じた損失の部分の金額は、その年分の総所得金額(イの規定による控除が行なわれる場合には、当該控除後の金額)から控除し、次に退職所得金額(ハの規定による控除が行なわれる場合には、当該控除後の金額)から控除する。
\end{description}
\item[三]その年分の各種所得の金額の計算上生じた損失の金額があるときは、まず法第六十九条(損益通算)の規定による控除を行なつた後に法第七十条第一項又は第二項の規定による控除を行なう。
\end{description}
\end{description}
\noindent\hspace{10pt}(被災事業用資産の損失等に係る純損失の金額)
\begin{description}
\item[第二百二条]法第七十条第二項(被災事業用資産の損失等に係る純損失の繰越控除)に規定する政令で定める純損失の金額は、同項に規定するその年の前年以前三年内の各年において生じた純損失の金額のうち、同項各号に掲げる損失の金額に達するまでの金額(既に同項の規定によりその年の前年以前において控除されたものを除く。)とする。
\end{description}
\noindent\hspace{10pt}(被災事業用資産の損失に含まれる支出)
\begin{description}
\item[第二百三条]法第七十条第三項(被災事業用資産の損失の金額)に規定する政令で定める支出は、次に掲げる費用の支出とする。
\begin{description}
\item[一]災害により法第七十条第三項に規定する資産(以下この条において「事業用資産」という。)が滅失し、損壊し又はその価値が減少したことによる当該事業用資産の取壊し又は除去のための費用その他の付随費用
\item[二]災害により事業用資産が損壊し又はその価値が減少した場合その他災害により当該事業用資産を業務の用に供することが困難となつた場合において、その災害のやんだ日の翌日から一年を経過した日(大規模な災害の場合その他やむを得ない事情がある場合には、三年を経過した日)の前日までに支出する次に掲げる費用その他これらに類する費用
\begin{description}
\item[イ]災害により生じた土砂その他の障害物を除去するための費用
\item[ロ]当該事業用資産の原状回復のための修繕費
\item[ハ]当該事業用資産の損壊又はその価値の減少を防止するための費用
\end{description}
\item[三]災害により事業用資産につき現に被害が生じ、又はまさに被害が生ずるおそれがあると見込まれる場合において、当該事業用資産に係る被害の拡大又は発生を防止するため緊急に必要な措置を講ずるための費用
\end{description}
\end{description}
\noindent\hspace{10pt}(雑損失の繰越控除)
\begin{description}
\item[第二百四条]法第七十一条第一項(雑損失の繰越控除)の規定による雑損失の金額の控除については、次に定めるところによる。
\begin{description}
\item[一]控除する雑損失の金額が前年以前三年内の二以上の年に生じたものである場合には、これらの年のうち最も古い年に生じた雑損失の金額から順次控除する。
\item[二]前年以前三年内の一の年において生じた雑損失の金額で前年以前において控除されなかつた部分に相当する金額があるときは、これをその年分の総所得金額、山林所得金額又は退職所得金額から順次控除する。
\end{description}
\item[\rensuji{2}]その年の各種所得の金額の計算上生じた損失の金額がある場合又は法第七十条(純損失の繰越控除)の規定による控除が行なわれる場合には、まず、法第六十九条(損益通算)及び第七十条の規定による控除を行なつた後、法第七十一条第一項の規定による控除を行なう。
\end{description}
\section*{第二章 所得控除}
\addcontentsline{toc}{section}{第二章 所得控除}
\noindent\hspace{10pt}(雑損控除の適用を認められる親族の範囲)
\begin{description}
\item[第二百五条]法第七十二条第一項(雑損控除)に規定する政令で定める親族は、居住者の配偶者その他の親族でその年分の総所得金額、退職所得金額及び山林所得金額の合計額が四十八万円以下であるものとする。
\item[\rensuji{2}]前項に規定する親族と生計を一にする居住者が二人以上ある場合における法第七十二条第一項の規定の適用については、当該親族は、これらの居住者のうちいずれか一の居住者の親族にのみ該当するものとし、その親族がいずれの居住者の親族に該当するかについては、次に定めるところによる。
\begin{description}
\item[一]その親族が同一生計配偶者又は扶養親族に該当する場合には、その者を自己の同一生計配偶者又は扶養親族としている居住者の親族とする。
\item[二]その親族が同一生計配偶者又は扶養親族に該当しない場合には、次に定めるところによる。
\begin{description}
\item[イ]その親族が配偶者に該当する場合には、その夫又は妻である居住者の親族とする。
\item[ロ]その親族が配偶者以外の親族に該当する場合には、これらの居住者のうち総所得金額、退職所得金額及び山林所得金額の合計額が最も大きい居住者の親族とする。
\end{description}
\end{description}
\end{description}
\noindent\hspace{10pt}(雑損控除の対象となる雑損失の範囲等)
\begin{description}
\item[第二百六条]法第七十二条第一項(雑損控除)に規定する政令で定めるやむを得ない支出は、次に掲げる支出とする。
\begin{description}
\item[一]災害により法第七十二条第一項に規定する資産(以下この項において「住宅家財等」という。)が滅失し、損壊し又はその価値が減少したことによる当該住宅家財等の取壊し又は除去のための支出その他の付随する支出
\item[二]災害により住宅家財等が損壊し又はその価値が減少した場合その他災害により当該住宅家財等を使用することが困難となつた場合において、その災害のやんだ日の翌日から一年を経過した日(大規模な災害の場合その他やむを得ない事情がある場合には、三年を経過した日)の前日までにした次に掲げる支出その他これらに類する支出
\begin{description}
\item[イ]災害により生じた土砂その他の障害物を除去するための支出
\item[ロ]当該住宅家財等の原状回復のための支出(当該災害により生じた当該住宅家財等の第三項に規定する損失の金額に相当する部分の支出を除く。第四号において同じ。)
\item[ハ]当該住宅家財等の損壊又はその価値の減少を防止するための支出
\end{description}
\item[三]災害により住宅家財等につき現に被害が生じ、又はまさに被害が生ずるおそれがあると見込まれる場合において、当該住宅家財等に係る被害の拡大又は発生を防止するため緊急に必要な措置を講ずるための支出
\item[四]盗難又は横領による損失が生じた住宅家財等の原状回復のための支出その他これに類する支出
\end{description}
\item[\rensuji{2}]法第七十二条第一項第一号に規定する政令で定める金額は、その年においてした前項第一号から第三号までに掲げる支出の金額(保険金、損害賠償金その他これらに類するものにより補塡される部分の金額を除く。)とする。
\item[\rensuji{3}]法第七十二条第一項の規定を適用する場合には、同項に規定する資産について受けた損失の金額は、当該損失を生じた時の直前におけるその資産の価額(その資産が法第三十八条第二項(譲渡所得の金額の計算上控除する取得費)に規定する資産である場合には、当該価額又は当該損失の生じた日にその資産の譲渡があつたものとみなして同項の規定(その資産が昭和二十七年十二月三十一日以前から引き続き所有していたものである場合には、法第六十一条第三項(昭和二十七年十二月三十一日以前に取得した資産の取得費等)の規定)を適用した場合にその資産の取得費とされる金額に相当する金額)を基礎として計算するものとする。
\end{description}
\noindent\hspace{10pt}(医療費の範囲)
\begin{description}
\item[第二百七条]法第七十三条第二項(医療費の範囲)に規定する政令で定める対価は、次に掲げるものの対価のうち、その病状その他財務省令で定める状況に応じて一般的に支出される水準を著しく超えない部分の金額とする。
\begin{description}
\item[一]医師又は歯科医師による診療又は治療
\item[二]治療又は療養に必要な医薬品の購入
\item[三]病院、診療所(これに準ずるものとして財務省令で定めるものを含む。)又は助産所へ収容されるための人的役務の提供
\item[四]あん摩マツサージ指圧師、はり師、きゆう師等に関する法律(昭和二十二年法律第二百十七号)第三条の二(名簿)に規定する施術者(同法第十二条の二第一項(医業類似行為を業とすることができる者)の規定に該当する者を含む。)又は柔道整復師法(昭和四十五年法律第十九号)第二条第一項(定義)に規定する柔道整復師による施術
\item[五]保健師、看護師又は准看護師による療養上の世話
\item[六]助産師による分べんの介助
\item[七]介護福祉士による社会福祉士及び介護福祉士法(昭和六十二年法律第三十号)第二条第二項(定義)に規定する喀痰かくたん
吸引等又は同法附則第三条第一項(認定特定行為業務従事者に係る特例)に規定する認定特定行為業務従事者による同項に規定する特定行為
\end{description}
\end{description}
\noindent\hspace{10pt}(社会保険料の範囲)
\begin{description}
\item[第二百八条]法第七十四条第二項(社会保険料の意義)に規定する政令で定めるものは、次に掲げるものとする。
\begin{description}
\item[一]労働者災害補償保険法第四章の二(特別加入)の規定により労働者災害補償保険の保険給付を受けることができることとされた者に係る労働保険の保険料の徴収等に関する法律(昭和四十四年法律第八十四号)の規定による保険料
\item[二]地方公共団体の職員が条例の規定により組織する団体(以下この号において「互助会」という。)の行う職員の相互扶助に関する制度で次に掲げる要件を備えているものとして財務省令で定めるところにより税務署長の承認を受けているものに基づき、その職員が負担する掛金
\begin{description}
\item[イ]当該互助会の事業が、地方公務員等共済組合法第五十三条第一項第二号から第十三号まで(短期給付の種類等)に掲げる給付(当該給付に係る同法第六十一条(療養に関する退職又は死亡後の給付)の規定による給付を含む。)に類する給付のみを行うものであること。
\item[ロ]イに規定する給付に要する費用は、主として当該職員が負担する掛金及び当該地方公共団体の補助金によつて充てられるものであること。
\item[ハ]当該互助会への加入資格のある者の全員が加入しているものであること。
\end{description}
\item[三]国家公務員共済組合法等の一部を改正する法律(昭和三十六年法律第百五十二号)附則第九条から第十一条まで(公庫等の復帰希望職員に関する経過措置)の規定による掛金
\item[四]平成二十五年厚生年金等改正法附則第五条第一項(存続厚生年金基金に係る改正前厚生年金保険法等の効力等)の規定によりなおその効力を有するものとされる旧厚生年金保険法(以下この号において「旧効力厚生年金保険法」という。)第百三十八条から第百四十一条まで(費用の負担)の規定により平成二十五年厚生年金等改正法附則第三条第十一号(定義)に規定する存続厚生年金基金の加入員として負担する掛金(旧効力厚生年金保険法第百四十条第四項(徴収金)の規定により負担する徴収金を含む。)
\end{description}
\end{description}
\noindent\hspace{10pt}(小規模企業共済等掛金控除の対象とならない共済契約)
\begin{description}
\item[第二百八条の二]法第七十五条第二項第一号(小規模企業共済等掛金控除)に規定する政令で定める共済契約は、小規模企業共済法及び中小企業事業団法の一部を改正する法律(平成七年法律第四十四号)附則第五条第一項(旧第二種共済契約に係る小規模企業共済法の規定の適用についての読替規定)の規定により読み替えられた小規模企業共済法第九条第一項各号(共済金)に掲げる事由により共済金が支給されることとなる契約とする。
\end{description}
\noindent\hspace{10pt}(新生命保険料の対象となる保険料又は掛金)
\begin{description}
\item[第二百八条の三]法第七十六条第一項(生命保険料控除)に規定する政令で定める新生命保険契約等に係る保険料又は掛金は、次に掲げる保険料又は掛金とする。
\begin{description}
\item[一]法第七十六条第五項第一号に掲げる契約の内容と同条第七項第一号に掲げる契約の内容とが一体となつて効力を有する一の保険契約のうち、同号に掲げる契約の内容を主たる内容とする保険契約として金融庁長官が財務大臣と協議して定めるもの(第二百八条の七第一号(介護医療保険料の対象となる保険料又は掛金)において「特定介護医療保険契約」という。)以外のものに係る保険料
\item[二]法第七十六条第五項第三号に掲げる契約の内容と同条第七項第二号に掲げる生命共済契約等の内容とが一体となつて効力を有する一の共済に係る契約のうち、同号に掲げる契約の内容を主たる内容とする共済に係る契約として農林水産大臣が財務大臣と協議して定めるもの(第二百八条の七第二号において「特定介護医療共済契約」という。)以外のものに係る掛金
\end{description}
\item[\rensuji{2}]金融庁長官は、前項第一号の規定により保険契約を定めたときは、これを告示する。
\item[\rensuji{3}]農林水産大臣は、第一項第二号の規定により共済に係る契約を定めたときは、これを告示する。
\end{description}
\noindent\hspace{10pt}(旧生命保険料の対象とならない保険料)
\begin{description}
\item[第二百八条の四]法第七十六条第一項(生命保険料控除)に規定する政令で定める旧生命保険契約等に係る保険料又は掛金は、次に掲げる保険料とする。
\begin{description}
\item[一]一定の偶然の事故によつて生ずることのある損害をてん補する旨の特約(法第七十六条第六項第四号に掲げる契約又は同条第一項に規定する保険金等(第二百八条の六(介護医療保険契約等に係る保険金等の支払事由の範囲)及び第二百九条(生命保険料控除の対象とならない保険契約等)において「保険金等」という。)の支払事由が身体の傷害のみに基因することとされているもの(次号において「傷害保険契約」という。)を除く。)が付されている保険契約に係る保険料のうち、当該特約に係る保険料
\item[二]法第七十六条第六項第四号に掲げる契約の内容と法第七十七条第二項第一号(地震保険料控除)に掲げる契約(傷害保険契約を除く。)の内容とが一体となつて効力を有する一の保険契約に係る保険料
\end{description}
\end{description}
\noindent\hspace{10pt}(新生命保険料等の金額から控除する剰余金等の額)
\begin{description}
\item[第二百八条の五]法第七十六条第一項第一号イ(生命保険料控除)に規定する政令で定めるところにより計算した金額は、その年において同条第五項に規定する新生命保険契約等(当該新生命保険契約等が他の保険契約(共済に係る契約を含む。以下この項において同じ。)に附帯して締結したものである場合には、当該他の保険契約及び当該他の保険契約に附帯して締結した当該新生命保険契約等以外の保険契約を含む。以下この項において同じ。)に基づき分配を受けた剰余金の額及び割戻しを受けた割戻金の額並びに当該新生命保険契約等に基づき分配を受けた剰余金又は割戻しを受けた割戻金をもつて当該新生命保険契約等に係る保険料又は掛金の払込みに充てた金額の合計額に、その年中に支払つた当該新生命保険契約等に係る保険料又は掛金の金額の合計額のうちに当該新生命保険契約等に係る同条第一項に規定する新生命保険料の金額の占める割合を乗じて計算した金額とする。
\item[\rensuji{2}]前項の規定は、法第七十六条第二項第一号に規定する政令で定めるところにより計算した金額及び同条第三項第一号イに規定する政令で定めるところにより計算した金額について準用する。
\end{description}
\noindent\hspace{10pt}(介護医療保険契約等に係る保険金等の支払事由の範囲)
\begin{description}
\item[第二百八条の六]法第七十六条第二項(生命保険料控除)に規定する政令で定める事由は、次に掲げる事由とする。
\begin{description}
\item[一]疾病にかかつたこと又は身体の傷害を受けたことを原因とする人の状態に基因して生ずる法第七十六条第二項に規定する医療費その他の費用を支払つたこと。
\item[二]疾病若しくは身体の傷害又はこれらを原因とする人の状態(法第七十六条第七項に規定する介護医療保険契約等に係る約款に、これらの事由に基因して一定額の保険金等を支払う旨の定めがある場合に限る。)
\item[三]疾病又は身体の傷害により就業することができなくなつたこと。
\end{description}
\end{description}
\noindent\hspace{10pt}(介護医療保険料の対象となる保険料又は掛金)
\begin{description}
\item[第二百八条の七]法第七十六条第二項(生命保険料控除)に規定する政令で定めるものは、次に掲げる保険料又は掛金とする。
\begin{description}
\item[一]法第七十六条第五項第一号に掲げる契約の内容と同条第七項第一号に掲げる契約の内容とが一体となつて効力を有する一の保険契約のうち、特定介護医療保険契約に係る保険料
\item[二]法第七十六条第五項第三号に掲げる契約の内容と同条第七項第二号に掲げる生命共済契約等の内容とが一体となつて効力を有する一の共済に係る契約のうち、特定介護医療共済契約に係る掛金
\end{description}
\end{description}
\noindent\hspace{10pt}(承認規定等の範囲)
\begin{description}
\item[第二百八条の八]法第七十六条第五項(生命保険料控除)に規定する確定給付企業年金法第三条第一項第一号(確定給付企業年金の実施)その他政令で定める規定は、同法第六条第一項(規約の変更等)(同法第七十九条第一項若しくは第二項(実施事業所に係る給付の支給に関する権利義務の他の確定給付企業年金への移転)、第八十一条第二項(基金から規約型企業年金への移行)又は附則第二十五条第一項(適格退職年金契約に係る権利義務の確定給付企業年金への移転)の規定、平成二十五年厚生年金等改正法附則第五条第一項(存続厚生年金基金に係る改正前厚生年金保険法等の効力等)の規定によりなおその効力を有するものとされる平成二十五年厚生年金等改正法第二条(確定給付企業年金法の一部改正)の規定による改正前の確定給付企業年金法(次項において「旧効力確定給付企業年金法」という。)第百七条第一項(実施事業所に係る給付の支給に関する権利義務の厚生年金基金への移転)、第百十条の二第三項(厚生年金基金の設立事業所に係る給付の支給に関する権利義務の確定給付企業年金への移転)又は第百十一条第二項(厚生年金基金から規約型企業年金への移行)の規定その他財務省令で定める規定に規定する権利義務の移転又は承継に伴う確定給付企業年金法第三条第一項に規定する確定給付企業年金に係る規約(次項において「規約」という。)の変更について承認を受ける場合に限る。)、第七十四条第四項(規約型企業年金の統合)及び第七十五条第二項(規約型企業年金の分割)の規定とする。
\item[\rensuji{2}]法第七十六条第五項に規定する確定給付企業年金法第三条第一項第二号その他政令で定める規定は、同法第十六条第一項(基金の規約の変更等)(同法第七十六条第四項(基金の合併)、第七十七条第五項(基金の分割)、第七十九条第一項若しくは第二項、第八十条第二項(規約型企業年金から基金への移行)又は附則第二十五条第一項の規定、旧効力確定給付企業年金法第百七条第一項又は第百十条の二第三項の規定その他財務省令で定める規定に規定する権利義務の移転又は承継に伴う規約の変更について認可を受ける場合に限る。)、第七十六条第一項及び第七十七条第一項の規定、旧効力確定給付企業年金法第百十二条第一項(厚生年金基金から基金への移行)の規定その他財務省令で定める規定とする。
\end{description}
\noindent\hspace{10pt}(生命保険料控除の対象とならない保険契約等)
\begin{description}
\item[第二百九条]法第七十六条第五項第一号(生命保険料控除)に規定する政令で定める保険契約は、保険期間が五年に満たない保険業法第二条第三項(定義)に規定する生命保険会社又は同条第八項に規定する外国生命保険会社等の締結した保険契約のうち、被保険者が保険期間の満了の日に生存している場合に限り保険金等を支払う定めのあるもの又は被保険者が保険期間の満了の日に生存している場合及び当該期間中に災害、感染症の予防及び感染症の患者に対する医療に関する法律(平成十年法律第百十四号)第六条第二項若しくは第三項(感染症の定義)に規定する一類感染症若しくは二類感染症その他これらに類する特別の事由により死亡した場合に限り保険金等を支払う定めのあるものとする。
\item[\rensuji{2}]法第七十六条第五項第三号に規定する政令で定める生命共済に係る契約は、共済期間が五年に満たない生命共済に係る契約のうち、被共済者が共済期間の満了の日に生存している場合に限り保険金等を支払う定めのあるもの又は被共済者が共済期間の満了の日に生存している場合及び当該期間中に災害、前項に規定する感染症その他これらに類する特別の事由により死亡した場合に限り保険金等を支払う定めのあるものとする。
\item[\rensuji{3}]法第七十六条第六項第四号に規定する政令で定めるものは、外国への旅行のために住居を出発した後、住居に帰着するまでの期間(次項において「海外旅行期間」という。)内に発生した疾病又は身体の傷害その他これらに類する事由に基因して保険金等が支払われる保険契約とする。
\item[\rensuji{4}]法第七十六条第七項第二号に規定する政令で定めるものは、海外旅行期間内に発生した疾病又は身体の傷害その他これらに類する事由に基因して保険金等が支払われる同条第五項第三号に規定する生命共済契約等とする。
\end{description}
\noindent\hspace{10pt}(生命共済契約等の範囲)
\begin{description}
\item[第二百十条]法第七十六条第五項第三号(生命保険料控除)に規定する共済に係る契約に類する政令で定める共済に係る契約は、次に掲げる契約とする。
\begin{description}
\item[一]農業協同組合法第十条第一項第十号(共済に関する施設)の事業を行う農業協同組合連合会の締結した生命共済に係る契約
\item[二]水産業協同組合法第十一条第一項第十一号(漁業協同組合の組合員の共済に関する事業)若しくは第九十三条第一項第六号の二(水産加工業協同組合の組合員の共済に関する事業)の事業を行う漁業協同組合若しくは水産加工業協同組合又は共済水産業協同組合連合会の締結した生命共済に係る契約(漁業協同組合又は水産加工業協同組合の締結した契約にあつては、財務省令で定める要件を備えているものに限る。)
\item[三]消費生活協同組合法(昭和二十三年法律第二百号)第十条第一項第四号(組合員の生活の共済を図る事業)の事業を行う消費生活協同組合連合会の締結した生命共済に係る契約
\item[四]中小企業等協同組合法第九条の二第七項(事業協同組合及び事業協同小組合)に規定する共済事業を行う同項に規定する特定共済組合、同法第九条の九第一項第三号(協同組合連合会)に掲げる事業を行う協同組合連合会又は同条第四項に規定する特定共済組合連合会の締結した生命共済に係る契約
\item[五]法律の規定に基づく共済に関する事業を行う法人の締結した生命共済に係る契約でその事業及び契約の内容が前各号に掲げるものに準ずるものとして財務大臣の指定するもの
\end{description}
\end{description}
\noindent\hspace{10pt}(退職年金に関する契約の範囲)
\begin{description}
\item[第二百十条の二]法第七十六条第五項第四号(生命保険料控除)に規定する退職年金に関する契約で政令で定めるものは、法人税法附則第二十条第三項(退職年金等積立金に対する法人税の特例)に規定する適格退職年金契約とする。
\end{description}
\noindent\hspace{10pt}(年金給付契約の対象となる契約の範囲)
\begin{description}
\item[第二百十一条]法第七十六条第八項(生命保険料控除)に規定する年金を給付する定めのある契約で政令で定めるものは、次に掲げる契約とする。
\begin{description}
\item[一]法第七十六条第五項第一号に掲げる契約で年金の給付を目的とするもの(退職年金の給付を目的とするものを除く。)のうち、当該契約の内容(同条第三項に規定する特約が付されている契約又は他の保険契約に附帯して締結した契約にあつては、当該特約又は他の保険契約の内容を除く。)が次に掲げる要件を満たすもの
\begin{description}
\item[イ]当該契約に基づく年金以外の金銭の支払(剰余金の分配及び解約返戻金の支払を除く。)は、当該契約で定める被保険者が死亡し、又は重度の障害に該当することとなつた場合に限り行うものであること。
\item[ロ]当該契約で定める被保険者が死亡し、又は重度の障害に該当することとなつた場合に支払う金銭の額は、当該契約の締結の日以後の期間又は支払保険料の総額に応じて逓増的に定められていること。
\item[ハ]当該契約に基づく年金の支払は、当該年金の支払期間を通じて年一回以上定期に行うものであり、かつ、当該契約に基づき支払うべき年金(年金の支払開始日から一定の期間内に年金受取人が死亡してもなお年金を支払う旨の定めのある契約にあつては、当該一定の期間内に支払うべき年金とする。)の一部を一括して支払う旨の定めがないこと。
\item[ニ]当該契約に基づく剰余金の金銭による分配(当該分配を受ける剰余金をもつて当該契約に係る保険料の払込みに充てられる部分を除く。)は、年金の支払開始日前において行わないもの又は当該剰余金の分配をする日の属する年において払い込むべき当該保険料の金額の範囲内の額とするものであること。
\end{description}
\item[二]法第七十六条第五項第二号に規定する旧簡易生命保険契約で年金の給付を目的とするもの(退職年金の給付を目的とするものを除く。)のうち、当該契約の内容(同条第三項に規定する特約が付されている契約にあつては、当該特約の内容を除く。)が前号イからニまでに掲げる要件を満たすもの
\item[三]第二百十条第一号及び第二号(生命共済契約等の範囲)に掲げる生命共済に係る契約(法第七十六条第五項第三号に規定する農業協同組合の締結した生命共済に係る契約を含む。)で年金の給付を目的とするもの(退職年金の給付を目的とするものを除く。次号において同じ。)のうち、当該契約の内容(法第七十六条第三項に規定する特約が付されている契約又は他の生命共済に係る契約に附帯して締結した契約にあつては、当該特約又は他の生命共済に係る契約の内容を除く。次号ロにおいて同じ。)が第一号イからニまでに掲げる要件に相当する要件その他の財務省令で定める要件を満たすもの
\item[四]第二百十条第三号及び第五号に掲げる生命共済に係る契約で年金の給付を目的とするもののうち、次に掲げる要件を満たすものとして財務大臣の指定するもの
\begin{description}
\item[イ]当該年金の給付を目的とする生命共済に関する事業に関し、適正に経理の区分が行われていること及び当該事業の継続が確実であると見込まれること並びに当該契約に係る掛金の安定運用が確保されていること。
\item[ロ]当該契約に係る年金の額及び掛金の額が適正な保険数理に基づいて定められており、かつ、当該契約の内容が第一号イからニまでに掲げる要件に相当する要件を満たしていること。
\end{description}
\end{description}
\end{description}
\noindent\hspace{10pt}(生命保険料控除の対象となる年金給付契約の要件)
\begin{description}
\item[第二百十二条]法第七十六条第八項第三号(生命保険料控除)に規定する政令で定める要件は、前条各号に掲げる契約に基づく同項第一号に定める個人に対する年金の支払を次のいずれかとするものであることとする。
\begin{description}
\item[一]当該年金の受取人の年齢が六十歳に達した日の属する年の一月一日以後の日(六十歳に達した日が同年の一月一日から六月三十日までの間である場合にあつては、同年の前年七月一日以後の日)で当該契約で定める日以後十年以上の期間にわたつて定期に行うものであること。
\item[二]当該年金の受取人が生存している期間にわたつて定期に行うものであること。
\item[三]第一号に定める年金の支払のほか、当該契約に係る被保険者又は被共済者の重度の障害を原因として年金の支払を開始し、かつ、当該年金の支払開始日以後十年以上の期間にわたつて、又はその者が生存している期間にわたつて定期に行うものであること。
\end{description}
\end{description}
\noindent\hspace{10pt}(地震保険料控除の対象とならない保険料又は掛金)
\begin{description}
\item[第二百十三条]法第七十七条第一項(地震保険料控除)に規定する政令で定める保険料又は掛金は、同項に規定する損害保険契約等に係る地震等損害部分の保険料又は掛金のうち、次に掲げる保険料又は掛金とする。
\begin{description}
\item[一]法第七十七条第一項に規定する地震等損害(次号において「地震等損害」という。)により臨時に生ずる費用、同項に規定する資産(同号において「家屋等」という。)の取壊し又は除去に係る費用その他これらに類する費用に対して支払われる保険金又は共済金に係る保険料又は掛金
\item[二]一の法第七十七条第一項に規定する損害保険契約等(当該損害保険契約等においてイに掲げる額が地震保険に関する法律施行令(昭和四十一年政令第百六十四号)第二条(保険金額の限度額)に規定する金額以上とされているものを除く。)においてイに掲げる額のロに掲げる額に対する割合が百分の二十未満とされている場合における当該損害保険契約等に係る地震等損害部分の保険料又は掛金(前号に掲げるものを除く。)
\begin{description}
\item[イ]地震等損害により家屋等について生じた損失の額をてん補する保険金又は共済金の額(当該保険金又は共済金の額の定めがない場合にあつては、当該地震等損害により支払われることとされている保険金又は共済金の限度額)
\item[ロ]火災(地震若しくは噴火又はこれらによる津波を直接又は間接の原因とするものを除く。)による損害により家屋等について生じた損失の額をてん補する保険金又は共済金の額(当該保険金又は共済金の額の定めがない場合にあつては、当該火災による損害により支払われることとされている保険金又は共済金の限度額)
\end{description}
\end{description}
\end{description}
\noindent\hspace{10pt}(地震保険料控除の対象となる共済に係る契約の範囲)
\begin{description}
\item[第二百十四条]法第七十七条第二項第二号(地震保険料控除)に規定する政令で定める共済に係る契約は、次に掲げる契約とする。
\begin{description}
\item[一]農業協同組合法第十条第一項第十号(共済に関する施設)の事業を行う農業協同組合連合会の締結した建物更生共済又は火災共済に係る契約
\item[二]農業保険法(昭和二十二年法律第百八十五号)第九十七条第一項第六号(共済事業の種類)又は第百六十三条第二項(共済金を交付する事業)の事業を行う農業共済組合又は農業共済組合連合会の締結した火災共済その他建物を共済の目的とする共済に係る契約
\item[三]水産業協同組合法第十一条第一項第十一号(漁業協同組合の組合員の共済に関する事業)若しくは第九十三条第一項第六号の二(水産加工業協同組合の組合員の共済に関する事業)の事業を行う漁業協同組合若しくは水産加工業協同組合又は共済水産業協同組合連合会の締結した建物若しくは動産の共済期間中の耐存を共済事故とする共済又は火災共済に係る契約(漁業協同組合又は水産加工業協同組合の締結した契約にあつては、財務省令で定める要件を備えているものに限る。)
\item[四]中小企業等協同組合法第九条の九第三項(協同組合連合会)に規定する火災等共済組合の締結した火災共済に係る契約
\item[五]消費生活協同組合法第十条第一項第四号(組合員の生活の共済を図る事業)の事業を行う消費生活協同組合連合会の締結した火災共済又は自然災害共済に係る契約
\item[六]法律の規定に基づく共済に関する事業を行う法人の締結した火災共済又は自然災害共済に係る契約でその事業及び契約の内容が前各号に掲げるものに準ずるものとして財務大臣の指定するもの
\end{description}
\end{description}
\noindent\hspace{10pt}(法人の設立のための寄附金の要件)
\begin{description}
\item[第二百十五条]法第七十八条第二項第二号(寄附金控除)に規定する政令で定める寄附金は、同号に規定する法人の設立に関する許可又は認可があることが確実であると認められる場合においてされる寄附金とする。
\end{description}
\noindent\hspace{10pt}(指定寄附金の指定についての審査事項等)
\begin{description}
\item[第二百十六条]法第七十八条第二項第二号(寄附金控除)の財務大臣の指定は、次に掲げる事項を審査して行うものとする。
\begin{description}
\item[一]寄附金を募集しようとする法人又は団体の行う事業の内容及び寄附金の使途
\item[二]寄附金の募集の目的及び目標額並びにその募集の区域及び対象
\item[三]寄附金の募集期間
\item[四]募集した寄附金の管理の方法
\item[五]寄附金の募集に要する経費
\item[六]その他当該指定のために必要な事項
\end{description}
\item[\rensuji{2}]財務大臣は、前項の指定をしたときは、これを告示する。
\end{description}
\noindent\hspace{10pt}(公益の増進に著しく寄与する法人の範囲)
\begin{description}
\item[第二百十七条]法第七十八条第二項第三号(寄附金控除)に規定する政令で定める法人は、次に掲げる法人とする。
\begin{description}
\item[一]独立行政法人
\item[一の二]地方独立行政法人法(平成十五年法律第百十八号)第二条第一項(定義)に規定する地方独立行政法人で同法第二十一条第一号又は第三号から第六号まで(業務の範囲)に掲げる業務(同条第三号に掲げる業務にあつては同号チに掲げる事業の経営に、同条第六号に掲げる業務にあつては地方独立行政法人法施行令(平成十五年政令第四百八十六号)第六条第一号又は第三号(公共的な施設の範囲)に掲げる施設の設置及び管理に、それぞれ限るものとする。)を主たる目的とするもの
\item[二]自動車安全運転センター、日本司法支援センター、日本私立学校振興・共済事業団及び日本赤十字社
\item[三]公益社団法人及び公益財団法人
\item[四]私立学校法(昭和二十四年法律第二百七十号)第三条(定義)に規定する学校法人で学校(学校教育法第一条(定義)に規定する学校及び就学前の子どもに関する教育、保育等の総合的な提供の推進に関する法律(平成十八年法律第七十七号)第二条第七項(定義)に規定する幼保連携型認定こども園をいう。以下この号において同じ。)の設置若しくは学校及び専修学校(学校教育法第百二十四条(専修学校)に規定する専修学校で財務省令で定めるものをいう。以下この号において同じ。)若しくは各種学校(学校教育法第百三十四条第一項(各種学校)に規定する各種学校で財務省令で定めるものをいう。以下この号において同じ。)の設置を主たる目的とするもの又は私立学校法第六十四条第四項(私立専修学校等)の規定により設立された法人で専修学校若しくは各種学校の設置を主たる目的とするもの
\item[五]社会福祉法人
\item[六]更生保護法人
\end{description}
\end{description}
\noindent\hspace{10pt}(特定公益信託の要件等)
\begin{description}
\item[第二百十七条の二]法第七十八条第三項(特定公益信託)に規定する政令で定める要件は、次に掲げる事項が信託行為において明らかであり、かつ、受託者が信託会社(金融機関の信託業務の兼営等に関する法律により同法第一条第一項(兼営の認可)に規定する信託業務を営む同項に規定する金融機関を含む。)であることとする。
\begin{description}
\item[一]当該公益信託の終了(信託の併合による終了を除く。次号において同じ。)の場合において、その信託財産が国若しくは地方公共団体に帰属し、又は当該公益信託が類似の目的のための公益信託として継続するものであること。
\item[二]当該公益信託は、合意による終了ができないものであること。
\item[三]当該公益信託の受託者がその信託財産として受け入れる資産は、金銭に限られるものであること。
\item[四]当該公益信託の信託財産の運用は、次に掲げる方法に限られるものであること。
\begin{description}
\item[イ]預金又は貯金
\item[ロ]国債、地方債、特別の法律により法人の発行する債券又は貸付信託の受益権の取得
\item[ハ]イ又はロに準ずるものとして財務省令で定める方法
\end{description}
\item[五]当該公益信託につき信託管理人が指定されるものであること。
\item[六]当該公益信託の受託者がその信託財産の処分を行う場合には、当該受託者は、当該公益信託の目的に関し学識経験を有する者の意見を聴かなければならないものであること。
\item[七]当該公益信託の信託管理人及び前号に規定する学識経験を有する者に対してその信託財産から支払われる報酬の額は、その任務の遂行のために通常必要な費用の額を超えないものであること。
\item[八]当該公益信託の受託者がその信託財産から受ける報酬の額は、当該公益信託の信託事務の処理に要する経費として通常必要な額を超えないものであること。
\end{description}
\item[\rensuji{2}]法第七十八条第三項に規定する政令で定めるところにより証明がされた公益信託は、同項に定める要件を満たす公益信託であることにつき当該公益信託に係る主務大臣(当該公益信託が次項第二号に掲げるものを目的とする公益信託である場合を除き、公益信託ニ関スル法律(大正十一年法律第六十二号)第十一条(主務官庁の権限に属する事務の処理)その他の法令の規定により当該公益信託に係る主務官庁の権限に属する事務を行うこととされた都道府県の知事その他の執行機関を含む。以下この条において同じ。)の証明を受けたものとする。
\item[\rensuji{3}]法第七十八条第三項に規定する政令で定める特定公益信託は、次に掲げるものの一又は二以上のものをその目的とする同項に規定する特定公益信託で、その目的に関し相当と認められる業績が持続できることにつき当該特定公益信託に係る主務大臣の認定を受けたもの(その認定を受けた日の翌日から五年を経過していないものに限る。)とする。
\begin{description}
\item[一]科学技術(自然科学に係るものに限る。)に関する試験研究を行う者に対する助成金の支給
\item[二]人文科学の諸領域について、優れた研究を行う者に対する助成金の支給
\item[三]学校教育法第一条(定義)に規定する学校における教育に対する助成
\item[四]学生又は生徒に対する学資の支給又は貸与
\item[五]芸術の普及向上に関する業務(助成金の支給に限る。)を行うこと。
\item[六]文化財保護法(昭和二十五年法律第二百十四号)第二条第一項(定義)に規定する文化財の保存及び活用に関する業務(助成金の支給に限る。)を行うこと。
\item[七]開発途上にある海外の地域に対する経済協力(技術協力を含む。)に資する資金の贈与
\item[八]自然環境の保全のため野生動植物の保護繁殖に関する業務を行うことを主たる目的とする法人で当該業務に関し国又は地方公共団体の委託を受けているもの(これに準ずるものとして財務省令で定めるものを含む。)に対する助成金の支給
\item[九]すぐれた自然環境の保全のためその自然環境の保存及び活用に関する業務(助成金の支給に限る。)を行うこと。
\item[十]国土の緑化事業の推進(助成金の支給に限る。)
\item[十一]社会福祉を目的とする事業に対する助成
\item[十二]就学前の子どもに関する教育、保育等の総合的な提供の推進に関する法律第二条第七項(定義)に規定する幼保連携型認定こども園における教育及び保育に対する助成
\end{description}
\item[\rensuji{4}]当該公益信託に係る主務大臣は、第二項の証明又は前項の認定をしようとするとき(当該証明がされた公益信託の第一項各号に掲げる事項に関する信託の変更を当該公益信託の主務官庁が命じ、又は許可するときを含む。)は、財務大臣に協議しなければならない。
\item[\rensuji{5}]第二項又は第三項の規定により都道府県が処理することとされている事務は、地方自治法第二条第九項第一号(法定受託事務)に規定する第一号法定受託事務とする。
\end{description}
\noindent\hspace{10pt}(二以上の居住者がある場合の同一生計配偶者の所属)
\begin{description}
\item[第二百十八条]法第八十五条第四項(扶養親族等の判定の時期等)の場合において、同項に規定する配偶者が同項に規定する同一生計配偶者又は扶養親族のいずれに該当するかは、同項に規定する居住者の提出するその年分の法第百十二条第一項(予定納税額の減額の承認の申請手続)に規定する申請書、確定申告書又は法第百九十四条第一項若しくは第二項(給与所得者の扶養控除等申告書)、第百九十五条第一項若しくは第二項(従たる給与についての扶養控除等申告書)、第百九十五条の二第一項(給与所得者の配偶者控除等申告書)若しくは第二百三条の五第一項(公的年金等の受給者の扶養親族等申告書)の規定による申告書(同条第二項の規定により提出した同条第一項の申告書を含む。以下この条において「申告書等」という。)に記載されたところによる。
\item[\rensuji{2}]前項の場合において、同項の居住者が同一人をそれぞれ自己の同一生計配偶者又は扶養親族として申告書等に記載したとき、その他同項の規定により同一生計配偶者又は扶養親族のいずれに該当するかを定められないときは、その夫又は妻である居住者の同一生計配偶者とする。
\end{description}
\noindent\hspace{10pt}(二以上の居住者がある場合の扶養親族の所属)
\begin{description}
\item[第二百十九条]法第八十五条第五項(扶養親族等の判定の時期等)の場合において、同項に規定する二以上の居住者の扶養親族に該当する者をいずれの居住者の扶養親族とするかは、これらの居住者の提出するその年分の前条第一項に規定する申告書等(法第百九十五条の二第一項(給与所得者の配偶者控除等申告書)の規定による申告書を除く。以下この条において「申告書等」という。)に記載されたところによる。
\item[\rensuji{2}]前項の場合において、二以上の居住者が同一人をそれぞれ自己の扶養親族として申告書等に記載したとき、その他同項の規定によりいずれの居住者の扶養親族とするかを定められないときは、次に定めるところによる。
\begin{description}
\item[一]その年において既に一の居住者が申告書等の記載によりその扶養親族としている場合には、当該親族は、当該居住者の扶養親族とする。
\item[二]前号の規定によつてもいずれの居住者の扶養親族とするかが定められない扶養親族は、居住者のうち総所得金額、退職所得金額及び山林所得金額の合計額又は当該親族がいずれの居住者の扶養親族とするかを判定すべき時における当該合計額の見積額が最も大きい居住者の扶養親族とする。
\end{description}
\end{description}
\noindent\hspace{10pt}(居住者が再婚した場合における同一生計配偶者等の特例)
\begin{description}
\item[第二百二十条]法第八十五条第六項(扶養親族等の判定の時期等)の場合において、同項の居住者の同一生計配偶者又は法第八十三条の二第一項(配偶者特別控除)に規定する生計を一にする配偶者に該当するものは、その死亡した配偶者又は再婚した配偶者のうち一人に限るものとする。
\item[\rensuji{2}]前項の居住者の死亡した配偶者又は再婚した配偶者のうちこれらの配偶者と生計を一にする他の居住者の扶養親族にも該当するものは、同項の居住者がこれらの配偶者のうちの一人を同項の規定により同一生計配偶者としたときは、その同一生計配偶者とされた者以外の者は当該他の居住者の扶養親族には該当しないものとし、同項の居住者がこれらの配偶者のいずれをも同一生計配偶者としないときは、これらの配偶者のうちの一人に限り、当該他の居住者の扶養親族に該当するものとする。
\item[\rensuji{3}]前項の場合において、第二百十八条第一項(二以上の居住者がある場合の同一生計配偶者の所属)の規定により、前項の配偶者の死亡の日までに提出された同条第一項に規定する申告書等(その年において当該申告書等を提出すべき期限が到来していないときは、その前年分の所得税につき最後に提出した当該申告書等)の記載に従つて当該死亡した配偶者が当該他の居住者の扶養親族とされていた場合には、当該死亡した配偶者は、当該他の居住者の扶養親族に該当するものとし、第一項の再婚した配偶者は、前項の規定にかかわらず、第一項の居住者の同一生計配偶者又はこれらの居住者以外の生計を一にする居住者の扶養親族に該当するものとする。
\end{description}
\section*{第三章 税額控除}
\addcontentsline{toc}{section}{第三章 税額控除}
\noindent\hspace{10pt}(分配時調整外国税相当額)
\begin{description}
\item[第二百二十条の二]法第九十三条第一項(分配時調整外国税相当額控除)に規定する政令で定める金額は、居住者が支払を受ける法第百七十六条第三項(信託財産に係る利子等の課税の特例)に規定する集団投資信託の収益の分配に係る次に掲げる金額の合計額とする。
\begin{description}
\item[一]法第百七十六条第三項の規定により当該収益の分配に係る所得税の額から控除された外国所得税(第三百条第一項(信託財産に係る利子等の課税の特例)に規定する外国所得税をいう。次号において同じ。)の額に、当該収益の分配(法第百八十一条(源泉徴収義務)又は第二百十二条(源泉徴収義務)の規定により所得税を徴収されるべきこととなる部分に限る。以下この号において同じ。)の額の総額のうちに当該居住者が支払を受ける収益の分配の額の占める割合を乗じて計算した金額(当該金額がその支払を受ける収益の分配につき法第百八十一条の規定により徴収された又は徴収されるべき所得税の額を超える場合には、当該所得税の額)
\item[二]法第百八十条の二第三項(信託財産に係る利子等の課税の特例)の規定により当該収益の分配に係る所得税の額から控除された外国所得税の額に、当該収益の分配(法第百八十一条又は第二百十二条の規定により所得税を徴収されるべきこととなる部分に限る。以下この号において同じ。)の額の総額のうちに当該居住者が支払を受ける収益の分配の額の占める割合を乗じて計算した金額(当該金額がその支払を受ける収益の分配につき法第百八十一条の規定により徴収された又は徴収されるべき所得税の額を超える場合には、当該所得税の額)
\end{description}
\end{description}
\noindent\hspace{10pt}(外国所得税の範囲)
\begin{description}
\item[第二百二十一条]法第九十五条第一項(外国税額控除)に規定する外国の法令により課される所得税に相当する税で政令で定めるものは、外国の法令に基づき外国又はその地方公共団体により個人の所得を課税標準として課される税(以下この章において「外国所得税」という。)とする。
\item[\rensuji{2}]外国又はその地方公共団体により課される次に掲げる税は、外国所得税に含まれるものとする。
\begin{description}
\item[一]超過所得税その他個人の所得の特定の部分を課税標準として課される税
\item[二]個人の所得又はその特定の部分を課税標準として課される税の附加税
\item[三]個人の所得を課税標準として課される税と同一の税目に属する税で、個人の特定の所得につき、徴税上の便宜のため、所得に代えて収入金額その他これに準ずるものを課税標準として課されるもの
\item[四]個人の特定の所得につき、所得を課税標準とする税に代え、個人の収入金額その他これに準ずるものを課税標準として課される税
\end{description}
\item[\rensuji{3}]外国又はその地方公共団体により課される次に掲げる税は、外国所得税に含まれないものとする。
\begin{description}
\item[一]税を納付する者が、当該税の納付後、任意にその金額の全部又は一部の還付を請求することができる税
\item[二]税の納付が猶予される期間を、その税の納付をすることとなる者が任意に定めることができる税
\item[三]複数の税率の中から税の納付をすることとなる者と外国若しくはその地方公共団体又はこれらの者により税率の合意をする権限を付与された者との合意により税率が決定された税(当該複数の税率のうち最も低い税率(当該最も低い税率が当該合意がないものとした場合に適用されるべき税率を上回る場合には当該適用されるべき税率)を上回る部分に限る。)
\item[四]外国所得税に附帯して課される附帯税に相当する税その他これに類する税
\end{description}
\end{description}
\noindent\hspace{10pt}(国外所得金額)
\begin{description}
\item[第二百二十一条の二]法第九十五条第一項(外国税額控除)に規定する政令で定める金額は、居住者の各年分の次に掲げる国外源泉所得(同項に規定する国外源泉所得をいう。以下この章において同じ。)に係る所得の金額の合計額(当該合計額が零を下回る場合には、零)とする。
\begin{description}
\item[一]法第九十五条第四項第一号に掲げる国外源泉所得
\item[二]法第九十五条第四項第二号から第十七号までに掲げる国外源泉所得(同項第二号から第十四号まで、第十六号及び第十七号に掲げる国外源泉所得にあつては、同項第一号に掲げる国外源泉所得に該当するものを除く。)
\end{description}
\end{description}
\noindent\hspace{10pt}(国外事業所等帰属所得に係る所得の金額の計算)
\begin{description}
\item[第二百二十一条の三]居住者の各年分の前条第一号に掲げる国外源泉所得(以下第二百二十一条の五(特定の内部取引に係る国外事業所等帰属所得に係る所得の金額の計算)までにおいて「国外事業所等帰属所得」という。)に係る所得の金額は、居住者のその年の国外事業所等(法第九十五条第四項第一号(外国税額控除)に規定する国外事業所等をいう。以下第二百二十一条の五までにおいて同じ。)を通じて行う事業に係る所得のみについて所得税を課するものとした場合に課税標準となるべき金額とする。
\item[\rensuji{2}]居住者の各年分の国外事業所等帰属所得に係る所得の金額の計算上その年分の課税標準となるべき金額は、別段の定めがあるものを除き、居住者の国外事業所等を通じて行う事業につき、居住者の各年分の所得の金額の計算に関する所得税に関する法令の規定に準じて計算した場合にその年分の総所得金額、退職所得金額及び山林所得金額となる金額とする。
\item[\rensuji{3}]居住者の各年分の国外事業所等帰属所得に係る所得の金額につき、前項の規定により法第三十七条(必要経費)の規定に準じて計算する場合には、同条第一項に規定する販売費、一般管理費その他同項に規定する所得を生ずべき業務について生じた費用及び同条第二項に規定する山林の植林費、取得に要した費用、管理費、伐採費その他その山林の育成又は譲渡に要した費用のうち内部取引(法第九十五条第四項第一号に規定する内部取引をいう。以下この条、次条第二項及び第二百二十一条の五において同じ。)に係るものについては、債務の確定しないものを含むものとする。
\item[\rensuji{4}]居住者の各年分の国外事業所等帰属所得に係る所得の金額につき、第二項の規定により法第五十二条(貸倒引当金)の規定に準じて計算する場合には、同条第一項及び第二項に規定する金銭債権には、当該居住者の国外事業所等と事業場等(法第九十五条第四項第一号に規定する事業場等をいう。次項、次条第二項及び第二百二十一条の五において同じ。)との間の内部取引に係る金銭債権に相当するものは、含まれないものとする。
\item[\rensuji{5}]居住者の国外事業所等と事業場等との間で当該国外事業所等における資産の購入その他資産の取得に相当する内部取引がある場合には、その内部取引の時にその内部取引に係る資産を取得したものとして、第二項の規定により準じて計算することとされる居住者の各年分の所得の金額の計算に関する所得税に関する法令の規定を適用する。
\item[\rensuji{6}]第一項の規定を適用する場合において、居住者のその年分の不動産所得の金額、事業所得の金額又は雑所得の金額(事業所得の金額及び雑所得の金額のうち山林の伐採又は譲渡に係るものを除く。)の計算上必要経費に算入された金額のうちに法第三十七条第一項に規定する販売費、一般管理費その他の費用で国外事業所等帰属所得に係る所得を生ずべき業務とそれ以外の業務の双方に関連して生じたものの額(以下この項及び次項において「共通費用の額」という。)があるときは、当該共通費用の額は、これらの業務に係る収入金額、資産の価額、使用人の数その他の基準のうちこれらの業務の内容及び費用の性質に照らして合理的と認められる基準により国外事業所等帰属所得に係る所得の金額の計算上の必要経費として配分するものとする。
\item[\rensuji{7}]前項の規定による共通費用の額の配分を行つた居住者は、当該配分の計算の基礎となる事項を記載した書類その他の財務省令で定める書類を作成しなければならない。
\item[\rensuji{8}]法第九十五条第一項から第三項までの規定の適用を受ける居住者は、確定申告書、修正申告書又は更正請求書にその年分の国外事業所等帰属所得に係る所得の金額の計算に関する明細を記載した書類を添付しなければならない。
\end{description}
\noindent\hspace{10pt}(国外事業所等に帰せられるべき純資産に対応する負債の利子)
\begin{description}
\item[第二百二十一条の四]居住者の各年の国外事業所等を通じて行う事業に係る負債の利子(手形の割引料その他経済的な性質が利子に準ずるものを含む。次項において同じ。)の額のうち、当該国外事業所等に係る純資産の額(その年分の当該国外事業所等に係る資産の帳簿価額の平均的な残高として合理的な方法により計算した金額からその年分の当該国外事業所等に係る負債の帳簿価額の平均的な残高として合理的な方法により計算した金額を控除した残額をいう。)が当該国外事業所等に帰せられるべき純資産の額に満たない場合におけるその満たない金額に対応する部分の金額は、その居住者のその年分の国外事業所等帰属所得に係る所得の金額の計算上、必要経費に算入しない。
\item[\rensuji{2}]前項に規定する負債の利子の額は、次に掲げる金額の合計額とする。
\begin{description}
\item[一]国外事業所等を通じて行う事業に係る負債の利子の額(次号及び第三号に掲げる金額を除く。)
\item[二]内部取引において居住者の国外事業所等から当該居住者の事業場等に対して支払う利子に該当することとなるものの金額
\item[三]前条第六項に規定する共通費用の額のうち同項の規定により国外事業所等帰属所得に係る所得の金額の計算上の必要経費として配分した金額に含まれる負債の利子の額
\end{description}
\item[\rensuji{3}]第一項に規定する国外事業所等に帰せられるべき純資産の額は、次に掲げるいずれかの方法により計算した金額とする。
\begin{description}
\item[一]資本配賦法(居住者のイに掲げる金額からロに掲げる金額を控除した残額に、ハに掲げる金額のニに掲げる金額に対する割合を乗じて計算した金額をもつて国外事業所等に帰せられるべき純資産の額とする方法をいう。)
\begin{description}
\item[イ]当該居住者のその年の総資産の帳簿価額の平均的な残高として合理的な方法により計算した金額
\item[ロ]当該居住者のその年の総負債の帳簿価額の平均的な残高として合理的な方法により計算した金額
\item[ハ]当該居住者のその年十二月三十一日(その者がその年の中途において死亡し又は出国をした場合には、その死亡又は出国の時。以下この項、次項及び第六項において同じ。)における当該国外事業所等に帰せられる資産の額について、取引の相手方の契約不履行その他の財務省令で定める理由により発生し得る危険(以下この項及び次項において「発生し得る危険」という。)を勘案して計算した金額
\item[ニ]当該居住者のその年十二月三十一日における総資産の額について、発生し得る危険を勘案して計算した金額
\end{description}
\item[二]同業個人比準法(居住者のその年十二月三十一日における国外事業所等に帰せられる資産の額について発生し得る危険を勘案して計算した金額に、イに掲げる金額のロに掲げる金額に対する割合を乗じて計算した金額をもつて国外事業所等に帰せられるべき純資産の額とする方法をいう。)
\begin{description}
\item[イ]比較対象者(当該居住者の国外事業所等を通じて行う主たる事業と同種の事業を国外事業所等所在地国(当該国外事業所等が所在する国又は地域をいう。以下この号及び第六項第二号において同じ。)において行う個人(当該個人が国外事業所等所在地国に住所又は居所を有する個人以外の個人である場合には、当該国外事業所等所在地国の国外事業所等を通じて当該同種の事業を行うものに限る。)で、その同種の事業に係る事業規模その他の状況が類似するものをいう。以下この号及び同項第二号において同じ。)のその年の前年以前三年内の各年のうちいずれかの年(当該比較対象者の純資産の額の総資産の額に対する割合が当該同種の事業を行う個人の当該割合に比して著しく高い場合として財務省令で定める場合に該当する年を除く。以下この号及び同項第二号において「比較対象年」という。)の十二月三十一日において貸借対照表に計上されている当該比較対象者の純資産の額(当該比較対象者が国外事業所等所在地国に住所又は居所を有する個人以外の個人である場合には、当該個人の国外事業所等(当該国外事業所等所在地国に所在するものに限る。)に係る純資産の額)
\item[ロ]比較対象者の比較対象年の十二月三十一日における総資産の額(当該比較対象者が国外事業所等所在地国に住所又は居所を有する個人以外の個人である場合には、当該個人の国外事業所等(当該国外事業所等所在地国に所在するものに限る。)に係る資産の額)について、発生し得る危険を勘案して計算した金額
\end{description}
\end{description}
\item[\rensuji{4}]前項第一号ハ若しくはニに掲げる金額又は同項第二号に規定する居住者のその年十二月三十一日における国外事業所等に帰せられる資産の額について発生し得る危険を勘案して計算した金額(以下この項及び次項において「危険勘案資産額」という。)に関し、居住者の行う事業の特性、規模その他の事情により、その年分以後の各年分の確定申告期限までに当該危険勘案資産額を計算することが困難な常況にあると認められる場合には、その年七月一日から十二月三十一日までの間の一定の日における前項第一号ハ若しくは同項第二号に規定する居住者の国外事業所等に帰せられる資産の額又は同項第一号ニに規定する居住者の総資産の額について発生し得る危険を勘案して計算した金額をもつて当該危険勘案資産額とすることができる。
\item[\rensuji{5}]前項の規定は、同項の規定の適用を受けようとする最初の年の翌年三月十五日までに、納税地の所轄税務署長に対し、同項に規定する確定申告期限までに危険勘案資産額を計算することが困難である理由、同項に規定する一定の日その他の財務省令で定める事項を記載した届出書を提出した場合に限り、適用する。
\item[\rensuji{6}]第三項各号に規定する居住者は、同項の規定にかかわらず、同項第一号に定める方法は第一号に掲げる方法とし、同項第二号に定める方法は第二号に掲げる方法とすることができる。
\begin{description}
\item[一]資本配賦簡便法(第三項第一号イに掲げる金額から同号ロに掲げる金額を控除した残額に、イに掲げる金額のロに掲げる金額に対する割合を乗じて計算する方法をいう。)
\begin{description}
\item[イ]当該居住者のその年十二月三十一日における当該国外事業所等に帰せられる資産の帳簿価額
\item[ロ]当該居住者のその年十二月三十一日において貸借対照表に計上されている総資産の帳簿価額
\end{description}
\item[二]簿価資産資本比率比準法(当該居住者のその年の国外事業所等に帰せられる資産の帳簿価額の平均的な残高として合理的な方法により計算した金額に、イに掲げる金額のロに掲げる金額に対する割合を乗じて計算する方法をいう。)
\begin{description}
\item[イ]比較対象者の比較対象年の十二月三十一日において貸借対照表に計上されている純資産の額(当該比較対象者が国外事業所等所在地国に住所又は居所を有する個人以外の個人である場合には、当該個人の国外事業所等(当該国外事業所等所在地国に所在するものに限る。)に係る純資産の額)
\item[ロ]比較対象者の比較対象年の十二月三十一日において貸借対照表に計上されている総資産の額(当該比較対象者が国外事業所等所在地国に住所又は居所を有する個人以外の個人である場合には、当該個人の国外事業所等(当該国外事業所等所在地国に所在するものに限る。)に係る資産の額)
\end{description}
\end{description}
\item[\rensuji{7}]その年の前年分の国外事業所等に帰せられるべき純資産の額(第一項に規定する国外事業所等に帰せられるべき純資産の額をいう。以下この項において同じ。)を資本配賦法等(第三項第一号又は前項第一号に掲げる方法をいう。以下この項において同じ。)により計算した居住者がその年分の当該国外事業所等に帰せられるべき純資産の額を計算する場合には、当該居住者の当該国外事業所等を通じて行う事業の種類の変更その他これに類する事情がある場合に限り同業個人比準法等(第三項第二号又は前項第二号に掲げる方法をいう。以下この項において同じ。)により計算することができるものとし、その年の前年分の国外事業所等に帰せられるべき純資産の額を同業個人比準法等により計算した居住者がその年分の当該国外事業所等に帰せられるべき純資産の額を計算する場合には、当該居住者の当該国外事業所等を通じて行う事業の種類の変更その他これに類する事情がある場合に限り資本配賦法等により計算することができるものとする。
\item[\rensuji{8}]第一項に規定する満たない金額に対応する部分の金額は、同項に規定する負債の利子の額に、同項に規定する国外事業所等に帰せられるべき純資産の額から第一号に掲げる金額を控除した残額(当該残額が第二号に掲げる金額を超える場合には、同号に掲げる金額)の第二号に掲げる金額に対する割合を乗じて計算した金額とする。
\begin{description}
\item[一]当該居住者のその年分の当該国外事業所等に係る第一項に規定する純資産の額
\item[二]当該居住者のその年分の当該国外事業所等に帰せられる負債(第一項に規定する利子の支払の基因となるものに限る。)の帳簿価額の平均的な残高として合理的な方法により計算した金額
\end{description}
\item[\rensuji{9}]第一項及び第三項第一号の帳簿価額は、当該居住者がその会計帳簿に記載した資産又は負債の金額によるものとする。
\item[\rensuji{10}]第一項の規定は、確定申告書、修正申告書又は更正請求書に同項の規定により必要経費に算入されない金額及びその計算に関する明細を記載した書類の添付があり、かつ、国外事業所等に帰せられるべき純資産の額の計算の基礎となる事項を記載した書類その他の財務省令で定める書類の保存がある場合に限り、適用する。
\item[\rensuji{11}]税務署長は、第一項の規定により必要経費に算入されない金額の全部又は一部につき前項の書類の保存がない場合においても、当該書類の保存がなかつたことについてやむを得ない事情があると認めるときは、当該書類の提出があつた場合に限り、第一項の規定を適用することができる。
\end{description}
\noindent\hspace{10pt}(特定の内部取引に係る国外事業所等帰属所得に係る所得の金額の計算)
\begin{description}
\item[第二百二十一条の五]居住者の国外事業所等と事業場等との間で資産(法第九十五条第四項第三号又は第五号(外国税額控除)に掲げる国外源泉所得を生ずべき資産に限る。以下この条において同じ。)の当該国外事業所等による取得又は譲渡に相当する内部取引があつた場合には、当該内部取引は当該資産の内部取引の直前の価額に相当する金額により行われたものとして、当該居住者の各年分の国外事業所等帰属所得に係る所得の金額を計算する。
\item[\rensuji{2}]前項に規定する直前の価額に相当する金額とは、居住者の国外事業所等と事業場等との間の内部取引が次の各号に掲げる内部取引のいずれに該当するかに応じ、当該各号に定める金額とする。
\begin{description}
\item[一]国外事業所等による資産の取得に相当する内部取引
\item[二]国外事業所等による資産の譲渡に相当する内部取引
\end{description}
\item[\rensuji{3}]第一項の規定の適用がある場合の居住者の国外事業所等と事業場等との間の内部取引(当該国外事業所等による資産の取得に相当する内部取引に限る。以下この項において同じ。)に係る当該資産の当該国外事業所等における取得価額は、前項第一号に定める金額(当該内部取引による取得のために要した費用がある場合には、その費用の額を加算した金額)とする。
\end{description}
\noindent\hspace{10pt}(その他の国外源泉所得に係る所得の金額の計算)
\begin{description}
\item[第二百二十一条の六]第二百二十一条の二第二号(国外所得金額)に掲げる国外源泉所得に係る所得の金額は、同号に掲げる国外源泉所得に係る所得のみについて各年分の所得税を課するものとした場合に課税標準となるべきその年分の総所得金額、退職所得金額及び山林所得金額の合計額に相当する金額とする。
\item[\rensuji{2}]居住者のその年分の不動産所得の金額、事業所得の金額又は雑所得の金額(事業所得の金額及び雑所得の金額のうち山林の伐採又は譲渡に係るものを除く。)の計算上必要経費に算入された金額のうちに法第三十七条第一項(必要経費)に規定する販売費、一般管理費その他の費用で第二百二十一条の二第二号に掲げる所得を生ずべき業務とそれ以外の業務の双方に関連して生じたものの額(以下この項及び次項において「共通費用の額」という。)があるときは、当該共通費用の額は、これらの業務に係る収入金額、資産の価額、使用人の数その他の基準のうちこれらの業務の内容及び費用の性質に照らして合理的と認められる基準により同号に掲げる国外源泉所得に係る所得の金額の計算上の必要経費として配分するものとする。
\item[\rensuji{3}]前項の規定による共通費用の額の配分を行つた居住者は、当該配分の計算の基礎となる事項を記載した書類その他の財務省令で定める書類を作成しなければならない。
\item[\rensuji{4}]法第九十五条第一項から第三項まで(外国税額控除)の規定の適用を受ける居住者は、確定申告書、修正申告書又は更正請求書にその年分の第二百二十一条の二第二号に掲げる国外源泉所得に係る所得の金額の計算に関する明細を記載した書類を添付しなければならない。
\end{description}
\noindent\hspace{10pt}(控除限度額の計算)
\begin{description}
\item[第二百二十二条]法第九十五条第一項(外国税額控除)に規定する政令で定めるところにより計算した金額は、同項の居住者のその年分の所得税の額(同条の規定を適用しないで計算した場合の所得税の額とし、附帯税の額を除く。)に、その年分の所得総額のうちにその年分の調整国外所得金額の占める割合を乗じて計算した金額とする。
\item[\rensuji{2}]前項に規定するその年分の所得総額は、法第七十条第一項若しくは第二項(純損失の繰越控除)又は第七十一条(雑損失の繰越控除)の規定を適用しないで計算した場合のその年分の総所得金額、退職所得金額及び山林所得金額の合計額(次項において「その年分の所得総額」という。)とする。
\item[\rensuji{3}]第一項に規定するその年分の調整国外所得金額とは、法第七十条第一項若しくは第二項又は第七十一条の規定を適用しないで計算した場合のその年分の法第九十五条第一項に規定する国外所得金額(非永住者については、当該国外所得金額のうち、国内において支払われ、又は国外から送金された国外源泉所得に係る部分に限る。以下この項において同じ。)をいう。
\end{description}
\noindent\hspace{10pt}(外国税額控除の対象とならない外国所得税の額)
\begin{description}
\item[第二百二十二条の二]法第九十五条第一項(外国税額控除)に規定する政令で定める取引は、次に掲げる取引とする。
\begin{description}
\item[一]居住者が、当該居住者が金銭の借入れをしている者又は預入を受けている者と特殊の関係のある者に対し、その借り入れられ、又は預入を受けた金銭の額に相当する額の金銭の貸付けをする取引(当該貸付けに係る利率その他の条件が、その借入れ又は預入に係る利率その他の条件に比し、特に有利な条件であると認められる場合に限る。)
\item[二]貸付債権その他これに類する債権を譲り受けた居住者が、当該債権に係る債務者(当該居住者に対し当該債権を譲渡した者(以下この号において「譲渡者」という。)と特殊の関係のある者に限る。)から当該債権に係る利子の支払を受ける取引(当該居住者が、譲渡者に対し、当該債権から生ずる利子の額のうち譲渡者が当該債権を所有していた期間に対応する部分の金額を支払う場合において、その支払う金額が、次に掲げる額の合計額に相当する額であるときに限る。)
\begin{description}
\item[イ]当該債権から生ずる利子の額から当該債務者が住所又は本店若しくは主たる事務所を有する国又は地域において当該居住者が当該利子につき納付した外国所得税の額を控除した額のうち、譲渡者が当該債権を所有していた期間に対応する部分の額
\item[ロ]当該利子に係る外国所得税の額(我が国が租税条約(法第二条第一項第八号の四ただし書(定義)に規定する条約をいう。以下この号及び第四項において同じ。)を締結している条約相手国等(租税条約の我が国以外の締約国又は締約者をいう。以下この号及び同項第四号において同じ。)の法律又は当該租税条約の規定により軽減され、又は免除された当該条約相手国等の租税の額で当該租税条約の規定により当該居住者が納付したものとみなされるものの額を含む。)のうち、譲渡者が当該債権を所有していた期間に対応する部分の額の全部又は一部に相当する額
\end{description}
\end{description}
\item[\rensuji{2}]前項に規定する特殊の関係のある者とは、次に掲げる者をいう。
\begin{description}
\item[一]法人税法施行令第四条(同族関係者の範囲)に規定する個人又は法人
\item[二]次に掲げる事実その他これに類する事実が存在することにより二の者のいずれか一方の者が他方の者の事業の方針の全部又は一部につき実質的に決定できる関係にある者
\begin{description}
\item[イ]当該他方の者の役員の二分の一以上又は代表する権限を有する役員が、当該一方の者の役員若しくは使用人を兼務している者又は当該一方の者の役員若しくは使用人であつた者であること。
\item[ロ]当該他方の者がその事業活動の相当部分を当該一方の者との取引に依存して行つていること。
\item[ハ]当該他方の者がその事業活動に必要とされる資金の相当部分を当該一方の者からの借入れにより、又は当該一方の者の保証を受けて調達していること。
\end{description}
\item[三]その者の前項に規定する居住者に対する債務の弁済につき、同項第一号に規定する居住者が金銭の借入れをしている者若しくは預入を受けている者が保証をしている者又は同項第二号に規定する譲渡者が保証をしている者
\end{description}
\item[\rensuji{3}]法第九十五条第一項に規定する居住者の所得税に関する法令の規定により所得税が課されないこととなる金額を課税標準として外国所得税に関する法令により課されるものとして政令で定める外国所得税の額は、次に掲げる外国所得税の額とする。
\begin{description}
\item[一]法第二十五条第一項各号(配当等とみなす金額)に掲げる事由により交付を受ける金銭の額及び金銭以外の資産の価額に対して課される外国所得税の額(当該交付の基因となつた同項に規定する法人の株式又は出資の取得価額を超える部分の金額に対して課される部分を除く。)
\item[二]法第九十五条第四項第一号に規定する国外事業所等から同号に規定する事業場等への支払につき当該国外事業所等の所在する国又は地域において当該支払に係る金額を課税標準として課される外国所得税の額
\item[三]租税特別措置法第九条の八(非課税口座内の少額上場株式等に係る配当所得の非課税)に規定する非課税口座内上場株式等の配当等又は同法第九条の九第一項(未成年者口座内の少額上場株式等に係る配当所得の非課税)に規定する未成年者口座内上場株式等の配当等に対して課される外国所得税の額
\end{description}
\item[\rensuji{4}]法第九十五条第一項に規定するその他政令で定める外国所得税の額は、次に掲げる外国所得税の額とする。
\begin{description}
\item[一]居住者がその年以前の年において非居住者であつた期間内に生じた所得に対して課される外国所得税の額
\item[二]外国法人から受ける租税特別措置法第四十条の五第一項(居住者の外国関係会社に係る所得の課税の特例)に規定する剰余金の配当等の額(同項又は同条第二項の規定の適用を受けるものに限る。)を課税標準として課される外国所得税の額(居住者の次に掲げる場合の区分に応じそれぞれ次に定める外国法人から受ける同条第一項に規定する剰余金の配当等の額の計算の基礎となつた当該外国法人の所得のうち当該居住者に帰せられるものとして計算される金額を課税標準として当該居住者に対して課される外国所得税の額を含む。)
\begin{description}
\item[イ]租税特別措置法第四十条の五第一項各号に掲げる金額を有する場合
\item[ロ]租税特別措置法第四十条の五第二項第二号に掲げる金額を有する場合
\end{description}
\item[三]外国法人から受ける租税特別措置法第四十条の八第一項(特殊関係株主等である居住者に係る外国関係法人に係る所得の課税の特例)に規定する剰余金の配当等の額(同項又は同条第二項の規定の適用を受けるものに限る。)を課税標準として課される外国所得税の額(居住者の次に掲げる場合の区分に応じそれぞれ次に定める外国法人から受ける同条第一項に規定する剰余金の配当等の額の計算の基礎となつた当該外国法人の所得のうち当該居住者に帰せられるものとして計算される金額を課税標準として当該居住者に対して課される外国所得税の額を含む。)
\begin{description}
\item[イ]租税特別措置法第四十条の八第一項各号に掲げる金額を有する場合
\item[ロ]租税特別措置法第四十条の八第二項第二号に掲げる金額を有する場合
\end{description}
\item[四]我が国が租税条約を締結している条約相手国等又は外国(外国居住者等の所得に対する相互主義による所得税等の非課税等に関する法律第二条第三号(定義)に規定する外国をいい、同法第五条各号(相互主義)のいずれかに該当しない場合における当該外国を除く。以下この号において同じ。)において課される外国所得税の額のうち、当該租税条約の規定(当該外国所得税の軽減又は免除に関する規定に限る。)により当該条約相手国等において課することができることとされる額を超える部分に相当する金額若しくは免除することとされる額に相当する金額又は当該外国において、同条第一号に規定する所得税等の非課税等に関する規定により当該外国に係る同法第二条第三号に規定する外国居住者等の同法第五条第一号に規定する対象国内源泉所得に対して所得税を軽減し、若しくは課さないこととされる条件と同等の条件により軽減することとされる部分に相当する金額若しくは免除することとされる額に相当する金額
\item[五]居住者の所得に対して課される外国所得税の額で租税条約の規定において法第九十五条第一項から第三項までの規定による控除をされるべき金額の計算に当たつて考慮しないものとされるもの
\end{description}
\end{description}
\noindent\hspace{10pt}(地方税控除限度額)
\begin{description}
\item[第二百二十三条]法第九十五条第二項(外国税額控除)に規定する地方税控除限度額として政令で定める金額は、地方税法施行令(昭和二十五年政令第二百四十五号)第七条の十九第三項(道府県民税からの外国所得税額の控除)の規定による限度額と同令第四十八条の九の二第四項(市町村民税からの外国所得税額の控除)の規定による限度額との合計額とする。
\end{description}
\noindent\hspace{10pt}(繰越控除限度額等)
\begin{description}
\item[第二百二十四条]法第九十五条第二項(外国税額控除)に規定するその年に繰り越される部分として政令で定める金額は、その年の前年以前三年内の各年(次項及び次条第一項において「前三年以内の各年」という。)の国税の控除余裕額又は地方税の控除余裕額を、最も古い年のものから順次に、かつ、同一年のものについては国税の控除余裕額及び地方税の控除余裕額の順に、その年の控除限度超過額に充てるものとした場合に当該控除限度超過額に充てられることとなる当該国税の控除余裕額の合計額に相当する金額とする。
\item[\rensuji{2}]前三年以内の各年のうちいずれかの年において納付することとなつた法第九十五条第一項に規定する控除対象外国所得税の額(以下この条及び第二百二十六条において「控除対象外国所得税の額」という。)をその納付することとなつた年の不動産所得の金額、事業所得の金額、山林所得の金額若しくは雑所得の金額の計算上必要経費に算入し、又は一時所得の金額の計算上支出した金額に算入した場合には、当該年以前の各年の国税の控除余裕額及び地方税の控除余裕額は、前項に規定する国税の控除余裕額及び地方税の控除余裕額に含まれないものとして、同項の規定を適用する。
\item[\rensuji{3}]法第九十五条第二項の規定の適用を受けることができる年後の各年に係る第一項及び次条第一項の規定の適用については、第一項の規定により当該適用を受けることができる年の控除限度超過額に充てられることとなる国税の控除余裕額及び地方税の控除余裕額並びにこれらの金額の合計額に相当する金額の当該控除限度超過額は、ないものとみなす。
\item[\rensuji{4}]前三項に規定する国税の控除余裕額とは、その年において納付することとなる控除対象外国所得税の額がその年の国税の控除限度額(法第九十五条第一項に規定する控除限度額をいう。以下この条において同じ。)に満たない場合における当該国税の控除限度額から当該控除対象外国所得税の額を控除した金額に相当する金額をいう。
\item[\rensuji{5}]第一項から第三項までに規定する地方税の控除余裕額とは、次の各号に掲げる場合の区分に応じ当該各号に掲げる金額をいう。
\begin{description}
\item[一]その年において納付することとなる控除対象外国所得税の額がその年の国税の控除限度額を超えない場合
\item[二]その年において納付することとなる控除対象外国所得税の額がその年の国税の控除限度額を超え、かつ、その超える部分の金額がその年の地方税の控除限度額に満たない場合
\end{description}
\item[\rensuji{6}]第一項及び第三項に規定する控除限度超過額とは、その年において納付することとなる控除対象外国所得税の額がその年の国税の控除限度額と地方税の控除限度額との合計額を超える場合におけるその超える部分の金額に相当する金額をいう。
\end{description}
\noindent\hspace{10pt}(繰越控除対象外国所得税額等)
\begin{description}
\item[第二百二十五条]法第九十五条第三項(外国税額控除)に規定するその年に繰り越される部分として政令で定める金額は、前三年以内の各年の控除限度超過額(前条第六項に規定する控除限度超過額をいう。以下この条において同じ。)を最も古い年のものから順次その年の国税の控除余裕額(前条第四項に規定する国税の控除余裕額をいう。以下この条において同じ。)に充てるものとした場合に当該国税の控除余裕額に充てられることとなる当該控除限度超過額の合計額に相当する金額とする。
\item[\rensuji{2}]前条第二項の規定は、前項の場合について準用する。
\item[\rensuji{3}]法第九十五条第三項の規定の適用を受けることができる年後の各年に係る第一項及び前条第一項の規定の適用については、第一項の規定により当該適用を受けることができる年の国税の控除余裕額に充てられることとなる控除限度超過額及びこれに相当する金額の当該国税の控除余裕額は、ないものとみなす。
\item[\rensuji{4}]地方税法施行令第七条の十九第二項(道府県民税からの外国所得税額の控除)の規定の適用を受けることができる年(同令第四十八条の九の二第二項(市町村民税からの外国所得税額の控除)の規定の適用をも受けることができる年を除く。)又は同令第四十八条の九の二第二項の規定の適用を受けることができる年後の各年に係る第一項及び前条第一項の規定の適用については、それぞれ、同令第七条の十九第二項又は第四十八条の九の二第二項の規定により当該適用を受けることができる年において課された外国の所得税等の額とみなされる金額に相当する控除限度超過額(当該控除限度超過額のうちに第一項の規定により当該適用を受けることができる年の国税の控除余裕額に充てられることとなるものがある場合には、当該充てられることとなる部分を除く。)及びこれに相当する金額の当該適用を受けることができる年の前条第五項に規定する地方税の控除余裕額は、ないものとみなす。
\end{description}
\noindent\hspace{10pt}(国外事業所等に帰せられるべき所得)
\begin{description}
\item[第二百二十五条の二]法第九十五条第四項第一号(外国税額控除)に規定する国外にある恒久的施設に相当するものその他の政令で定めるものは、我が国が租税条約(法第二条第一項第八号の四ただし書(定義)に規定する条約をいい、その条約の我が国以外の締約国又は締約者(以下この項において「条約相手国等」という。)内にある恒久的施設に相当するものに帰せられる所得に対して租税を課することができる旨の定めのあるものに限る。以下この項において同じ。)を締結している条約相手国等については当該租税条約の条約相手国等内にある当該租税条約に定める恒久的施設に相当するものとし、外国(外国居住者等の所得に対する相互主義による所得税等の非課税等に関する法律第二条第三号(定義)に規定する外国をいい、同法第五条各号(相互主義)のいずれかに該当しない場合における当該外国を除く。以下この項において同じ。)については当該外国にある外国居住者等の所得に対する相互主義による所得税等の非課税等に関する法律第二条第六号に規定する国内事業所等に相当するものとし、その他の国又は地域については当該国又は地域にある恒久的施設に相当するものとする。
\item[\rensuji{2}]法第九十五条第四項第一号に規定する事業場その他これに準ずるものとして政令で定めるものは、次に掲げるものとする。
\begin{description}
\item[一]法第二条第一項第八号の四イに規定する事業を行う一定の場所に相当するもの
\item[二]法第二条第一項第八号の四ロに規定する建設若しくは据付けの工事又はこれらの指揮監督の役務の提供を行う場所に相当するもの
\item[三]法第二条第一項第八号の四ハに規定する自己のために契約を締結する権限のある者に相当する者
\item[四]前三号に掲げるものに準ずるもの
\end{description}
\end{description}
\noindent\hspace{10pt}(国外にある資産の運用又は保有により生ずる所得)
\begin{description}
\item[第二百二十五条の三]次に掲げる資産の運用又は保有により生ずる所得は、法第九十五条第四項第二号(外国税額控除)に規定する国外にある資産の運用又は保有により生ずる所得とする。
\begin{description}
\item[一]外国の国債若しくは地方債若しくは外国法人の発行する債券又は外国法人の発行する金融商品取引法第二条第一項第十五号(定義)に掲げる約束手形に相当するもの
\item[二]非居住者に対する貸付金に係る債権で当該非居住者の行う業務に係るもの以外のもの
\item[三]国外にある営業所、事務所その他これらに準ずるもの又は国外において契約の締結の代理をする者を通じて締結した保険契約(保険業法第二条第六項(定義)に規定する外国保険業者、同条第三項に規定する生命保険会社、同条第四項に規定する損害保険会社又は同条第十八項に規定する少額短期保険業者の締結した保険契約をいう。)その他これに類する契約に基づく保険金の支払又は剰余金の分配(これらに準ずるものを含む。)を受ける権利
\end{description}
\end{description}
\noindent\hspace{10pt}(国外にある資産の譲渡により生ずる所得)
\begin{description}
\item[第二百二十五条の四]法第九十五条第四項第三号(外国税額控除)に規定する国外にある資産の譲渡により生ずる所得として政令で定めるものは、次に掲げる資産の譲渡(第三号に掲げる資産については、伐採又は譲渡)により生ずる所得とする。
\begin{description}
\item[一]国外にある不動産
\item[二]国外にある不動産の上に存する権利、国外における鉱業権又は国外における採石権
\item[三]国外にある山林
\item[四]外国法人の発行する株式又は外国法人の出資者の持分で、その外国法人の発行済株式又は出資の総数又は総額の一定割合以上に相当する数又は金額の株式又は出資を所有する場合にその外国法人の本店又は主たる事務所の所在する国又は地域においてその譲渡による所得に対して外国所得税が課されるもの
\item[五]不動産関連法人の株式(出資及び投資信託及び投資法人に関する法律第二条第十四項(定義)に規定する投資口を含む。次号及び次項において同じ。)
\item[六]国外にあるゴルフ場の所有又は経営に係る法人の株式を所有することがそのゴルフ場を一般の利用者に比して有利な条件で継続的に利用する権利を有する者となるための要件とされている場合における当該株式
\item[七]国外にあるゴルフ場その他の施設の利用に関する権利
\end{description}
\item[\rensuji{2}]前項第五号に規定する不動産関連法人とは、その有する資産の価額の総額のうちに次に掲げる資産の価額の合計額の占める割合が百分の五十以上である法人をいう。
\begin{description}
\item[一]国外にある土地等(土地若しくは土地の上に存する権利又は建物及びその附属設備若しくは構築物をいう。以下この項において同じ。)
\item[二]その有する資産の価額の総額のうちに国外にある土地等の価額の合計額の占める割合が百分の五十以上である法人の株式
\item[三]前号又は次号に掲げる株式を有する法人(その有する資産の価額の総額のうちに国外にある土地等並びに前号、この号及び次号に掲げる株式の価額の合計額の占める割合が百分の五十以上であるものに限る。)の株式(前号に掲げる株式に該当するものを除く。)
\item[四]前号に掲げる株式を有する法人(その有する資産の価額の総額のうちに国外にある土地等並びに前二号及びこの号に掲げる株式の価額の合計額の占める割合が百分の五十以上であるものに限る。)の株式(前二号に掲げる株式に該当するものを除く。)
\end{description}
\end{description}
\noindent\hspace{10pt}(人的役務の提供を主たる内容とする事業の範囲)
\begin{description}
\item[第二百二十五条の五]法第九十五条第四項第四号(外国税額控除)に規定する政令で定める事業は、次に掲げる事業とする。
\begin{description}
\item[一]映画若しくは演劇の俳優、音楽家その他の芸能人又は職業運動家の役務の提供を主たる内容とする事業
\item[二]弁護士、公認会計士、建築士その他の自由職業者の役務の提供を主たる内容とする事業
\item[三]科学技術、経営管理その他の分野に関する専門的知識又は特別の技能を有する者の当該知識又は技能を活用して行う役務の提供を主たる内容とする事業(機械設備の販売その他事業を行う者の主たる業務に付随して行われる場合における当該事業及び法第二条第一項第八号の四ロ(定義)に規定する建設又は据付けの工事の指揮監督の役務の提供を主たる内容とする事業を除く。)
\end{description}
\end{description}
\noindent\hspace{10pt}(国外業務に係る貸付金の利子)
\begin{description}
\item[第二百二十五条の六]法第九十五条第四項第八号(外国税額控除)に規定する債券の買戻又は売戻条件付売買取引として政令で定めるものは、債券をあらかじめ約定した期日にあらかじめ約定した価格で(あらかじめ期日及び価格を約定することに代えて、その開始以後期日及び価格の約定をすることができる場合にあつては、その開始以後約定した期日に約定した価格で)買い戻し、又は売り戻すことを約定して譲渡し、又は購入し、かつ、当該約定に基づき当該債券と同種及び同量の債券を買い戻し、又は売り戻す取引(次項において「債券現先取引」という。)とする。
\item[\rensuji{2}]法第九十五条第四項第八号に規定する差益として政令で定めるものは、国外において業務を行う者との間で行う債券現先取引で当該業務に係るものにおいて、債券を購入する際の当該購入に係る対価の額を当該債券と同種及び同量の債券を売り戻す際の当該売戻しに係る対価の額が上回る場合における当該売戻しに係る対価の額から当該購入に係る対価の額を控除した金額に相当する差益とする。
\item[\rensuji{3}]法第九十五条第四項第八号の規定の適用については、非居住者又は外国法人の業務の用に供される船舶又は航空機の購入のためにその非居住者又は外国法人に対して提供された貸付金は、同号の規定に該当する貸付金とし、居住者又は内国法人の業務の用に供される船舶又は航空機の購入のためにその居住者又は内国法人に対して提供された貸付金は、同号の規定に該当する貸付金以外の貸付金とする。
\end{description}
\noindent\hspace{10pt}(国外業務に係る使用料等)
\begin{description}
\item[第二百二十五条の七]法第九十五条第四項第九号ハ(外国税額控除)に規定する政令で定める用具は、車両及び運搬具、工具並びに器具及び備品とする。
\item[\rensuji{2}]法第九十五条第四項第九号の規定の適用については、同号ロ又はハに規定する資産で非居住者又は外国法人の業務の用に供される船舶又は航空機において使用されるものの使用料は、同号の規定に該当する使用料とし、当該資産で居住者又は内国法人の業務の用に供される船舶又は航空機において使用されるものの使用料は、同号の規定に該当する使用料以外の使用料とする。
\end{description}
\noindent\hspace{10pt}(国外に源泉がある給与又は報酬の範囲)
\begin{description}
\item[第二百二十五条の八]法第九十五条第四項第十号イ(外国税額控除)に規定する政令で定める人的役務の提供は、次に掲げる勤務その他の人的役務の提供とする。
\begin{description}
\item[一]内国法人の役員としての勤務で国外において行うもの(当該役員としての勤務を行う者が同時にその内国法人の使用人として常時勤務を行う場合の当該役員としての勤務を除く。)
\item[二]居住者又は内国法人が運航する船舶又は航空機において行う勤務その他の人的役務の提供(国外における寄航地において行われる一時的な人的役務の提供を除く。)
\end{description}
\item[\rensuji{2}]法第九十五条第四項第十号ハに規定する政令で定める人的役務の提供は、前項各号に掲げる勤務その他の人的役務の提供で当該勤務その他の人的役務の提供を行う者が非居住者であつた期間に行つたものとする。
\end{description}
\noindent\hspace{10pt}(事業の広告宣伝のための賞金)
\begin{description}
\item[第二百二十五条の九]法第九十五条第四項第十一号(外国税額控除)に規定する政令で定める賞金は、国外において事業を行う者から当該事業の広告宣伝のために賞として支払を受ける金品その他の経済的な利益(旅行その他の役務の提供を内容とするもので、金品との選択をすることができないものとされているものを除く。)とする。
\end{description}
\noindent\hspace{10pt}(年金に係る契約の範囲)
\begin{description}
\item[第二百二十五条の十]法第九十五条第四項第十二号(外国税額控除)に規定する政令で定める契約は、保険業法第二条第六項(定義)に規定する外国保険業者、同条第三項に規定する生命保険会社若しくは同条第四項に規定する損害保険会社の締結する保険契約又はこれに類する共済に係る契約であつて、年金を給付する定めのあるものとする。
\end{description}
\noindent\hspace{10pt}(匿名組合契約に準ずる契約の範囲)
\begin{description}
\item[第二百二十五条の十一]法第九十五条第四項第十四号(外国税額控除)に規定する政令で定める契約は、当事者の一方が相手方の事業のために出資をし、相手方がその事業から生ずる利益を分配することを約する契約とする。
\end{description}
\noindent\hspace{10pt}(国際運輸業所得)
\begin{description}
\item[第二百二十五条の十二]法第九十五条第四項第十五号(外国税額控除)に規定する政令で定める所得は、居住者が国内及び国外にわたつて船舶又は航空機による運送の事業を行うことにより生ずる所得のうち、船舶による運送の事業にあつては国外において乗船し又は船積みをした旅客又は貨物に係る収入金額を基準とし、航空機による運送の事業にあつてはその国外業務(国外において行う業務をいう。以下この条において同じ。)に係る収入金額又は経費、その国外業務の用に供する固定資産の価額その他その国外業務が当該運送の事業に係る所得の発生に寄与した程度を推測するに足りる要因を基準として判定したその居住者の国外業務につき生ずべき所得とする。
\end{description}
\noindent\hspace{10pt}(相手国等において租税を課することができることとされる所得)
\begin{description}
\item[第二百二十五条の十三]法第九十五条第四項第十六号(外国税額控除)に規定する政令で定めるものは、同号に規定する相手国等において外国所得税が課される所得とする。
\end{description}
\noindent\hspace{10pt}(国外に源泉がある所得)
\begin{description}
\item[第二百二十五条の十四]法第九十五条第四項第十七号(外国税額控除)に規定する政令で定める所得は、次に掲げる所得とする。
\begin{description}
\item[一]国外において行う業務又は国外にある資産に関し受ける保険金、補償金又は損害賠償金(これらに類するものを含む。)に係る所得
\item[二]国外にある資産の法人からの贈与により取得する所得
\item[三]国外において発見された埋蔵物又は国外において拾得された遺失物に係る所得
\item[四]国外において行う懸賞募集に基づいて懸賞として受ける金品その他の経済的な利益(旅行その他の役務の提供を内容とするもので、金品との選択ができないものとされているものを除く。)に係る所得
\item[五]前三号に掲げるもののほか、国外においてした行為に伴い取得する一時所得
\item[六]前各号に掲げるもののほか、国外において行う業務又は国外にある資産に関し供与を受ける経済的な利益に係る所得
\end{description}
\end{description}
\noindent\hspace{10pt}(債務の保証等に類する取引)
\begin{description}
\item[第二百二十五条の十五]法第九十五条第五項(外国税額控除)に規定する政令で定める取引は、資金の借入れその他の取引に係る債務の保証(債務を負担する行為であつて債務の保証に準ずるものを含む。)とする。
\end{description}
\noindent\hspace{10pt}(内部取引に含まれない事実の範囲等)
\begin{description}
\item[第二百二十五条の十六]法第九十五条第七項(外国税額控除)に規定する利子に準ずるものとして政令で定めるものは、手形の割引料その他経済的な性質が利子に準ずるものとする。
\item[\rensuji{2}]法第九十五条第七項に規定する政令で定める事実は、次に掲げる事実とする。
\begin{description}
\item[一]次に掲げるものの使用料の支払に相当する事実
\begin{description}
\item[イ]工業所有権その他の技術に関する権利、特別の技術による生産方式又はこれらに準ずるもの
\item[ロ]著作権(出版権及び著作隣接権その他これに準ずるものを含む。)
\item[ハ]第六条第八号イからレまで(減価償却資産の範囲)に掲げる無形固定資産(国外における同号ヲからレまでに掲げるものに相当するものを含む。)
\end{description}
\item[二]前号イからハまでに掲げるものの譲渡又は取得に相当する事実
\end{description}
\end{description}
\noindent\hspace{10pt}(外国所得税が減額された場合の特例)
\begin{description}
\item[第二百二十六条]居住者が納付することとなつた外国所得税の額につき法第九十五条第一項から第三項まで(外国税額控除)の規定の適用を受けた年の翌年以後七年内の各年において当該外国所得税の額が減額された場合には、当該居住者のその減額されることとなつた日の属する年(以下この条において「減額に係る年」という。)については、当該減額に係る年において当該居住者が納付することとなる控除対象外国所得税の額(第三項において「納付控除対象外国所得税額」という。)から減額控除対象外国所得税額に相当する金額を控除し、その控除後の金額につき法第九十五条第一項から第三項までの規定を適用する。
\item[\rensuji{2}]前項に規定する減額控除対象外国所得税額とは、居住者の減額に係る年において外国所得税の額の減額がされた金額のうち、第一号に掲げる金額から第二号に掲げる金額を控除した残額に相当する金額をいう。
\begin{description}
\item[一]当該外国所得税の額のうち居住者の法第九十五条第一項から第三項までの規定の適用を受けた年において控除対象外国所得税の額とされた部分の金額
\item[二]当該減額がされた後の当該外国所得税の額につき当該居住者の法第九十五条第一項から第三項までの規定の適用を受けた年において同条第一項の規定を適用したならば控除対象外国所得税の額とされる部分の金額
\end{description}
\item[\rensuji{3}]第一項の場合において、減額に係る年の納付控除対象外国所得税額がないとき、又は当該納付控除対象外国所得税額が前項に規定する減額控除対象外国所得税額(以下この項において「減額控除対象外国所得税額」という。)に満たないときは、減額に係る年の前年以前三年内の各年の第二百二十四条第六項(繰越控除限度額等)に規定する控除限度超過額(同条第三項又は第二百二十五条第三項若しくは第四項(繰越控除対象外国所得税額等)の規定により減額に係る年の前年以前の各年においてないものとみなされた部分の金額を除く。以下この項において「控除限度超過額」という。)から、それぞれ当該減額控除対象外国所得税額の全額又は当該減額控除対象外国所得税額のうち当該納付控除対象外国所得税額を超える部分の金額に相当する金額を控除し、その控除後の金額につき法第九十五条第三項の規定を適用する。
\end{description}
\noindent\hspace{10pt}(国外転出をする場合の譲渡所得等の特例に係る外国税額控除の特例)
\begin{description}
\item[第二百二十六条の二]法第九十五条の二第一項(国外転出をする場合の譲渡所得等の特例に係る外国税額控除の特例)(同条第二項において準用する場合を含む。以下この条において同じ。)に規定する政令で定めるところにより計算した金額は、有価証券等(法第六十条の二第一項(国外転出をする場合の譲渡所得等の特例)に規定する有価証券等をいう。第四項及び第五項において同じ。)又は法第六十条の二第二項に規定する未決済信用取引等若しくは同条第三項に規定する未決済デリバティブ取引に係る契約(以下この項及び次項において「対象資産」という。)の譲渡(同条第四項に規定する譲渡をいう。第二号及び第四項において同じ。)若しくは決済又は限定相続等(同条第八項に規定する限定相続等をいう。第四項において同じ。)による移転(以下この項において「譲渡等」という。)により生ずる所得に対して課される外国所得税(法第九十五条の二第一項に規定する外国所得税をいう。以下この項において同じ。)に関する法令の規定により当該外国所得税の課税標準の計算の基礎となる期間の所得に対して課される外国所得税の額から、当該対象資産の譲渡等により生ずる所得(法第百六十四条第一項各号(非居住者に対する課税の方法)に掲げる非居住者の区分に応じ当該各号に定める国内源泉所得に該当するものを除く。)がないものとした場合における当該期間の所得に対して課される外国所得税の額を控除した金額(次の各号に掲げる場合にあつては、当該各号に掲げる場合の区分に応じ当該各号に定める金額)とする。
\begin{description}
\item[一]当該外国所得税が当該対象資産の譲渡等(相続(限定承認に係るものに限る。)又は遺贈(包括遺贈のうち限定承認に係るものに限る。)による移転に限る。)により生ずる所得に対して課されるものである場合であつて、当該控除した金額が当該対象資産に係る法第百三十七条の二第一項(国外転出をする場合の譲渡所得等の特例の適用がある場合の納税猶予)に規定する納税猶予分の所得税額(既に同条第五項の規定の適用があつた場合には、同項の規定の適用があつた金額を除く。)を超えるとき
\item[二]当該外国所得税が当該対象資産の譲渡等(譲渡若しくは決済又は贈与による移転に限る。)により生ずる所得に対して課されるものである場合であつて、当該控除した金額が当該対象資産に係る法第百三十七条の二第五項に規定する政令で定めるところにより計算した金額を超えるとき
\end{description}
\item[\rensuji{2}]法第九十五条の二第一項の規定の適用がある場合における国外転出(法第六十条の二第一項に規定する国外転出をいう。第四項において同じ。)の日の属する年の法第九十五条第一項(外国税額控除)に規定する控除限度額の計算については、法第六十条の二第一項から第三項まで(これらの規定を同条第八項(同条第九項において準用する場合を含む。)の規定により読み替えて適用する場合を含む。)の規定により行われたものとみなされた対象資産の譲渡又は決済により生ずる所得は、第二百二十一条の二各号(国外所得金額)に掲げる国外源泉所得に該当するものとして、同条の規定を適用する。
\item[\rensuji{3}]法第六十条の二第十一項の規定は、法第九十五条の二第一項の規定の適用について準用する。
\item[\rensuji{4}]第百七十条第八項(国外転出をする場合の譲渡所得等の特例)の規定は、国外転出の日の属する年分の所得税につき法第九十五条の二第一項の規定の適用を受ける個人(その相続人を含む。)が当該国外転出の時後に有価証券等の譲渡又は限定相続等による移転をした場合において、その譲渡又は限定相続等による移転をした有価証券等が、その者が当該国外転出の時において有していた有価証券等に該当するかどうかの判定について準用する。
\item[\rensuji{5}]第百七十条第九項の規定は、前項に規定する個人が有する有価証券等(以下この項において「従前の有価証券等」という。)について第三項において準用する法第六十条の二第十一項各号に掲げる事由が生じた場合において、当該事由により取得した有価証券等(以下この項において「取得有価証券等」という。)が同条第十一項の規定により引き続き所有していたものとみなされるときにおける当該従前の有価証券等のうち当該取得有価証券等の取得の基因となつた部分について準用する。
\end{description}
\section*{第四章 税額の計算の特例}
\addcontentsline{toc}{section}{第四章 税額の計算の特例}
\begin{description}
\item[第二百二十七条]削除
\end{description}
\begin{description}
\item[第二百二十八条]削除
\end{description}
\begin{description}
\item[第二百二十九条]削除
\end{description}
\begin{description}
\item[第二百三十条]削除
\end{description}
\begin{description}
\item[第二百三十一条]削除
\end{description}
\begin{description}
\item[第二百三十二条]削除
\end{description}
\begin{description}
\item[第二百三十三条]削除
\end{description}
\begin{description}
\item[第二百三十四条]削除
\end{description}
\begin{description}
\item[第二百三十五条]削除
\end{description}
\begin{description}
\item[第二百三十六条]削除
\end{description}
\begin{description}
\item[第二百三十七条]削除
\end{description}
\begin{description}
\item[第二百三十八条]削除
\end{description}
\begin{description}
\item[第二百三十九条]削除
\end{description}
\begin{description}
\item[第二百四十条]削除
\end{description}
\begin{description}
\item[第二百四十一条]削除
\end{description}
\begin{description}
\item[第二百四十二条]削除
\end{description}
\begin{description}
\item[第二百四十三条]削除
\end{description}
\begin{description}
\item[第二百四十四条]削除
\end{description}
\begin{description}
\item[第二百四十五条]削除
\end{description}
\begin{description}
\item[第二百四十六条]削除
\end{description}
\begin{description}
\item[第二百四十七条]削除
\end{description}
\begin{description}
\item[第二百四十八条]削除
\end{description}
\begin{description}
\item[第二百四十九条]削除
\end{description}
\begin{description}
\item[第二百五十条]削除
\end{description}
\begin{description}
\item[第二百五十一条]削除
\end{description}
\begin{description}
\item[第二百五十二条]削除
\end{description}
\begin{description}
\item[第二百五十三条]削除
\end{description}
\begin{description}
\item[第二百五十四条]削除
\end{description}
\begin{description}
\item[第二百五十五条]削除
\end{description}
\begin{description}
\item[第二百五十六条]削除
\end{description}
\begin{description}
\item[第二百五十七条]削除
\end{description}
\noindent\hspace{10pt}(年の中途で非居住者が居住者となつた場合の税額の計算)
\begin{description}
\item[第二百五十八条]法第百二条(年の中途で非居住者が居住者となつた場合の税額の計算)に規定する政令で定めるところにより計算した金額は、同条に規定する居住者につき次に定める順序により計算した所得税の額とする。
\begin{description}
\item[一]その者がその年において居住者であつた期間(以下この条において「居住者期間」という。)内に生じた法第七条第一項第一号(居住者の課税所得の範囲)に掲げる所得(居住者期間のうちにその者が非永住者であつた期間がある場合には、当該所得及び当該期間内に生じた同項第二号に掲げる所得。第四項及び第五項において同じ。)及びその者がその年において非居住者であつた期間(以下この条において「非居住者期間」という。)内に生じた法第百六十四条第一項各号(非居住者に対する課税の方法)に掲げる非居住者の区分に応ずる当該各号に定める国内源泉所得に係る所得を、法第二編第二章第二節(各種所得の金額の計算)の規定に準じてそれぞれ各種所得に区分し、その各種所得ごとに所得の金額を計算する。
\item[二]前号の所得の金額(同号の規定により区分した各種所得のうちに、同種の各種所得で居住者期間内に生じたものと非居住者期間内に生じたものとがある場合には、それぞれの各種所得に係る所得の金額の合計額)を基礎とし、法第二編第二章第一節及び第三節(課税標準、損益通算及び損失の繰越控除)の規定に準じて、総所得金額、退職所得金額及び山林所得金額を計算する。
\item[三]法第二編第二章第四節(所得控除)の規定に準じ前号の総所得金額、退職所得金額又は山林所得金額から基礎控除その他の控除をして課税総所得金額、課税退職所得金額又は課税山林所得金額を計算する。
\item[四]前号の課税総所得金額、課税退職所得金額又は課税山林所得金額を基礎とし、法第二編第三章第一節(税率)の規定に準じて所得税の額を計算する。
\item[五]その者がその年において法第二編第三章第二節(税額控除)(法第百六十五条第一項(総合課税に係る所得税の課税標準、税額等の計算)の規定により同節の規定に準じて計算する場合を含む。)の規定により配当控除、分配時調整外国税相当額控除及び外国税額控除を受けることができる場合に相当する場合には、前号の所得税の額からこれらの控除を行い、控除後の所得税の額を計算する。
\item[六]その者が非居住者期間内に支払を受けるべき法第百六十四条第二項各号に掲げる非居住者の区分に応ずる当該各号に定める国内源泉所得がある場合には、当該国内源泉所得につき法第百六十九条(分離課税に係る所得税の課税標準)及び第百七十条(分離課税に係る所得税の税率)の規定を適用して所得税の額を計算し、当該所得税の額を前号の控除後の所得税の額に加算する。
\end{description}
\item[\rensuji{2}]前項第一号の規定により各種所得ごとに所得の金額を計算する場合において、給与所得、退職所得、法第三十五条第三項(公的年金等の定義)に規定する公的年金等に係る雑所得又は山林所得、譲渡所得若しくは一時所得で居住者期間内及び非居住者期間内の双方にわたつて生じたものがあるときは、これらの所得に係る法第二十八条第三項(給与所得)に規定する給与所得控除額、同条第四項若しくは法第五十七条の二第一項(給与所得者の特定支出の控除の特例)の規定による給与所得の金額、法第三十条第二項(退職所得)に規定する退職所得控除額、法第三十五条第四項に規定する公的年金等控除額又は法第三十二条第四項(山林所得)、第三十三条第四項(譲渡所得)若しくは第三十四条第三項(一時所得)に規定する特別控除額は、居住者期間内及び非居住者期間内に生じたこれらの所得をそれぞれ合算した所得につき計算する。
\item[\rensuji{3}]第一項第三号の規定により同号に規定する基礎控除その他の控除を行う場合には、これらの控除のうち次の各号に掲げるものについては、当該各号に定める金額を控除する。
\begin{description}
\item[一]雑損控除
\item[二]医療費控除
\item[三]社会保険料控除及び小規模企業共済等掛金控除
\item[四]生命保険料控除及び地震保険料控除
\end{description}
\item[\rensuji{4}]第一項第五号の規定により分配時調整外国税相当額控除を行う場合において、その者が非居住者期間内に支払を受けた法第百六十五条の五の三第一項(非居住者に係る分配時調整外国税相当額の控除)に規定する集団投資信託の収益の分配に係る同項に規定する分配時調整外国税相当額があるときは、その者の居住者期間内に生じた法第七条第一項第一号に掲げる所得の金額及び非居住者期間内に生じた法第百六十四条第一項第一号イに掲げる国内源泉所得(以下この条において「恒久的施設帰属所得」という。)に係る所得の金額について法第八十九条から第九十二条まで(税率及び配当控除)の規定により計算したその年分の所得税の額に相当する金額を限度として、その者の各年に係る分配時調整外国税相当額(法第九十三条第一項(分配時調整外国税相当額控除)に規定する分配時調整外国税相当額で居住者期間に係るもの及び法第百六十五条の五の三第一項に規定する分配時調整外国税相当額で非居住者期間に係るものの合計額をいう。)を第一項第四号の所得税の額から控除する。
\item[\rensuji{5}]第一項第五号の規定により外国税額控除を行う場合において、その者の非居住者期間内に生じた恒久的施設帰属所得があるときは、次に定めるところによる。
\begin{description}
\item[一]その者の居住者期間内に生じた法第七条第一項第一号に掲げる所得の金額及び非居住者期間内に生じた恒久的施設帰属所得に係る所得の金額について法第八十九条から第九十三条までの規定により計算したその年分の所得税の額にその年分のイに掲げる金額のうちにその年分のロに掲げる金額の占める割合を乗じて計算した金額(以下この項において「控除限度額」という。)を限度として、その者が各年において納付することとなる控除対象外国所得税合計額(法第九十五条第一項(外国税額控除)に規定する控除対象外国所得税の額で居住者期間内に生じた法第七条第一項第一号に掲げる所得につき課されるもの及び法第百六十五条の六第一項(非居住者に係る外国税額の控除)に規定する控除対象外国所得税の額で非居住者期間内に生じた恒久的施設帰属所得につき課されるものの合計額をいう。以下この項において同じ。)を第一項第四号の所得税の額から控除する。
\begin{description}
\item[イ]居住者期間内に生じた法第七条第一項第一号に掲げる所得及び非居住者期間内に生じた恒久的施設帰属所得に係る所得について、法第七十条第一項若しくは第二項(純損失の繰越控除)又は第七十一条(雑損失の繰越控除)の規定を適用しないで計算した場合のその年分の総所得金額、退職所得金額及び山林所得金額の合計額
\item[ロ]居住者期間内に生じた法第九十五条第一項に規定する国外源泉所得に係る所得について法第七十条第一項若しくは第二項又は第七十一条の規定を適用しないで計算した場合の法第九十五条第一項に規定する国外所得金額に相当する金額及び非居住者期間内に生じた法第百六十五条の六第一項に規定する国外源泉所得に係る所得について法第七十条第一項若しくは第二項又は第七十一条の規定を適用しないで計算した場合の法第百六十五条の六第一項に規定する国外所得金額に相当する金額の合計額(当該合計額がイに掲げる合計額に相当する金額を超える場合には、当該合計額に相当する金額)
\end{description}
\item[二]その者が各年において納付することとなる控除対象外国所得税合計額がその年の控除限度額と地方税控除限度額(地方税法施行令第七条の十九第三項(道府県民税からの外国所得税額の控除)の規定による限度額と同令第四十八条の九の二第四項(市町村民税からの外国所得税額の控除)の規定による限度額との合計額をいう。)との合計額を超える場合において、その年の前年以前三年内の各年(次号において「前三年以内の各年」という。)の法第百六十五条の六第一項に規定する控除限度額のうち同条第二項に規定する繰越控除限度額があるときは、当該繰越控除限度額を法第九十五条第二項に規定する繰越控除限度額とみなして、同条の規定を適用する。
\item[三]その者が各年において納付することとなる控除対象外国所得税合計額がその年の控除限度額に満たない場合において、その前三年以内の各年において納付することとなつた法第百六十五条の六第一項に規定する控除対象外国所得税の額のうち同条第三項に規定する繰越控除対象外国所得税額があるときは、当該繰越控除対象外国所得税額を法第九十五条第三項に規定する繰越控除対象外国所得税額とみなして、同条の規定を適用する。
\end{description}
\end{description}
\section*{第五章 申告、納付及び還付}
\addcontentsline{toc}{section}{第五章 申告、納付及び還付}
\subsection*{第一節 予定納税}
\addcontentsline{toc}{subsection}{第一節 予定納税}
\noindent\hspace{10pt}(予定納税基準額の計算)
\begin{description}
\item[第二百五十九条]法第百四条第一項第一号(予定納税額の納付)に規定する譲渡所得の金額、一時所得の金額、雑所得の金額又は雑所得に該当しない臨時所得の金額は、法第二編第二章第三節(損益通算及び損失の繰越控除)の規定を適用した後の金額とし、当該臨時所得は、法第九十条第一項(変動所得及び臨時所得の平均課税)の規定の適用を受けたものに限るものとする。
\item[\rensuji{2}]前年分の総所得金額のうちに法第二編第二章第三節の規定を適用して計算した後の変動所得(雑所得に該当するものに限る。)の金額又は臨時所得の金額があつた場合において、同年分の所得税につき法第九十条第一項の規定の適用を受けているときは、当該変動所得の金額又は臨時所得の金額を同年分の所得税に係る法第九十条第三項に規定する平均課税対象金額から控除した残額を同年分の当該平均課税対象金額とみなして、法第百四条第一項第一号に掲げる所得税の額を計算する。
\end{description}
\noindent\hspace{10pt}(予定納税額等の通知の所轄庁)
\begin{description}
\item[第二百六十条]法第百六条第三項(予定納税額等の通知)及び第百九条第三項(特別農業所得者に対する予定納税額等の通知)に規定する政令で定める税務署長は、これらの規定に規定する居住者の前年分の所得税につき確定申告書の提出を受け、又は当該所得税につき更正若しくは決定をした税務署長及びこれらの事実があつたことを知つている税務署長のうち、最近の納税地を所轄する税務署長とする。
\end{description}
\noindent\hspace{10pt}(申告納税見積額の計算)
\begin{description}
\item[第二百六十一条]法第百十一条第四項(予定納税額の減額の承認の申請)に規定する政令で定めるところにより計算した金額は、第一号に掲げる金額から第二号に掲げる金額を控除した金額とする。
\begin{description}
\item[一]その年分の総所得金額及び山林所得金額の見積額からその年分の法第二編第二章第四節(所得控除)に規定する控除の額の見積額を法第八十七条第二項(所得控除の順序)の規定に準じて控除した後の金額をそれぞれ課税総所得金額又は課税山林所得金額とみなして、同編第三章第一節(税率)の規定を適用して計算した場合の所得税の額から同章第二節(税額控除)の規定による控除の額を法第九十二条第二項(税額控除の順序等)の規定に準じて控除した後の所得税の額
\item[二]前号に掲げる総所得金額の計算の基礎となつた各種所得につき源泉徴収をされる所得税の額の見積額
\end{description}
\end{description}
\subsection*{第二節 確定申告及びこれに伴う納付}
\addcontentsline{toc}{subsection}{第二節 確定申告及びこれに伴う納付}
\subsubsection*{第一款 確定申告}
\addcontentsline{toc}{subsubsection}{第一款 確定申告}
\noindent\hspace{10pt}(確定申告書に関する書類等の提出又は提示)
\begin{description}
\item[第二百六十二条]法第百二十条第三項第一号(確定所得申告)(法第百二十二条第三項(還付等を受けるための申告)、第百二十三条第三項(確定損失申告)、第百二十五条第四項(年の中途で死亡した場合の確定申告)及び第百二十七条第四項(年の中途で出国をする場合の確定申告)において準用する場合を含む。)に掲げる居住者は、次に掲げる書類又は電磁的記録印刷書面(電子証明書等に記録された情報の内容を、国税庁長官の定める方法によつて出力することにより作成した書面をいう。以下この項において同じ。)を確定申告書に添付し、又は当該申告書の提出の際提示しなければならない。
\begin{description}
\item[一]確定申告書に雑損控除に関する事項を記載する場合にあつては、当該申告書に記載したその控除を受ける金額の計算の基礎となる法第七十二条第一項(雑損控除)に規定する政令で定めるやむを得ない支出をした金額につきこれを領収した者のその領収を証する書類
\item[二]確定申告書に社会保険料控除(法第七十四条第二項第五号に掲げる社会保険料に係るものに限る。)に関する事項を記載する場合にあつては、当該申告書に記載した当該社会保険料の金額を証する書類
\item[三]確定申告書に小規模企業共済等掛金控除に関する事項を記載する場合にあつては、当該申告書に記載した小規模企業共済等掛金の額を証する書類
\item[四]確定申告書に生命保険料控除に関する事項を記載する場合にあつては、当該申告書に記載したその控除を受ける金額の計算の基礎となる次に掲げる保険料の金額その他財務省令で定める事項を証する書類又は当該書類に記載すべき事項を記録した電子証明書等に係る電磁的記録印刷書面(ロに掲げる金額に係るものにあつては、当該金額が九千円を超える法第七十六条第六項に規定する旧生命保険契約等(ロにおいて「旧生命保険契約等」という。)に係るものに限る。)
\begin{description}
\item[イ]新生命保険料の金額(その年において当該新生命保険料の金額に係る法第七十六条第五項に規定する新生命保険契約等に基づく剰余金の分配若しくは割戻金の割戻しを受け、又は当該新生命保険契約等に基づき分配を受ける剰余金若しくは割戻しを受ける割戻金をもつて当該新生命保険料の払込みに充てた場合には、当該剰余金又は割戻金の額(当該新生命保険料に係る部分の金額として第二百八条の五第一項(新生命保険料等の金額から控除する剰余金等の額)の定めるところにより計算した金額に限る。)を控除した残額)
\item[ロ]旧生命保険料の金額(その年において当該旧生命保険料の金額に係る旧生命保険契約等に基づく剰余金の分配若しくは割戻金の割戻しを受け、又は当該旧生命保険契約等に基づき分配を受ける剰余金若しくは割戻しを受ける割戻金をもつて当該旧生命保険料の払込みに充てた場合には、当該剰余金又は割戻金の額(当該旧生命保険料に係る部分の金額に限る。)を控除した残額)
\item[ハ]介護医療保険料の金額(その年において当該介護医療保険料の金額に係る法第七十六条第七項に規定する介護医療保険契約等に基づく剰余金の分配若しくは割戻金の割戻しを受け、又は当該介護医療保険契約等に基づき分配を受ける剰余金若しくは割戻しを受ける割戻金をもつて当該介護医療保険料の払込みに充てた場合には、当該剰余金又は割戻金の額(当該介護医療保険料に係る部分の金額として第二百八条の五第二項において準用する同条第一項の定めるところにより計算した金額に限る。)を控除した残額)
\item[ニ]新個人年金保険料の金額(その年において当該新個人年金保険料の金額に係る法第七十六条第八項に規定する新個人年金保険契約等に基づく剰余金の分配若しくは割戻金の割戻しを受け、又は当該新個人年金保険契約等に基づき分配を受ける剰余金若しくは割戻しを受ける割戻金をもつて当該新個人年金保険料の払込みに充てた場合には、当該剰余金又は割戻金の額(当該新個人年金保険料に係る部分の金額として第二百八条の五第二項において準用する同条第一項の定めるところにより計算した金額に限る。)を控除した残額)
\item[ホ]旧個人年金保険料の金額(その年において当該旧個人年金保険料の金額に係る法第七十六条第九項に規定する旧個人年金保険契約等に基づく剰余金の分配若しくは割戻金の割戻しを受け、又は当該旧個人年金保険契約等に基づき分配を受ける剰余金若しくは割戻しを受ける割戻金をもつて当該旧個人年金保険料の払込みに充てた場合には、当該剰余金又は割戻金の額(当該旧個人年金保険料に係る部分の金額に限る。)を控除した残額)
\end{description}
\item[五]確定申告書に地震保険料控除に関する事項を記載する場合にあつては、当該申告書に記載したその控除を受ける金額の計算の基礎となる地震保険料の金額その他財務省令で定める事項を証する書類又は当該書類に記載すべき事項を記録した電子証明書等に係る電磁的記録印刷書面
\item[六]確定申告書に寄附金控除に関する事項を記載する場合にあつては、当該申告書に記載したその控除を受ける金額の計算の基礎となる法第七十八条第二項(寄附金控除)に規定する特定寄附金の明細書その他財務省令で定める書類又は当該書類に記載すべき事項を記録した電子証明書等に係る電磁的記録印刷書面
\end{description}
\item[\rensuji{2}]前項に規定する電子証明書等とは、電磁的記録(電子的方式、磁気的方式その他の人の知覚によつては認識することができない方式で作られる記録であつて、電子計算機による情報処理の用に供されるものをいう。以下この項において同じ。)でその記録された情報について電子署名(電子署名及び認証業務に関する法律(平成十二年法律第百二号)第二条第一項(定義)に規定する電子署名をいう。以下この項において同じ。)が行われているもの及び当該電子署名に係る電子証明書(電子署名を行つた者を確認するために用いられる事項が当該者に係るものであることを証明するために作成された電磁的記録であつて財務省令で定めるものをいう。)をいう。
\item[\rensuji{3}]法第百二十条第三項第二号(法第百二十二条第三項、第百二十三条第三項、第百二十五条第四項及び第百二十七条第四項において準用する場合を含む。)に掲げる居住者は、同号に規定する記載がされる親族に係る次に掲げる書類を、当該記載がされる障害者控除に係る障害者(確定申告書に控除対象配偶者又は控除対象扶養親族として記載がされる者を除く。以下この項において「国外居住障害者」という。)、当該記載がされる控除対象配偶者若しくは配偶者特別控除に係る配偶者(以下この項において「国外居住配偶者」という。)若しくは当該記載がされる控除対象扶養親族(以下この項において「国外居住扶養親族」という。)の各人別に確定申告書に添付し、又は当該申告書の提出の際提示しなければならない。
\begin{description}
\item[一]次に掲げる者の区分に応じ、次に定める旨を証する書類として財務省令で定めるもの
\begin{description}
\item[イ]国外居住障害者
\item[ロ]国外居住配偶者
\item[ハ]国外居住扶養親族
\end{description}
\item[二]当該国外居住障害者、国外居住配偶者又は国外居住扶養親族が当該居住者と生計を一にすることを明らかにする書類として財務省令で定めるもの
\end{description}
\item[\rensuji{4}]法第百二十条第三項第三号(法第百二十二条第三項、第百二十三条第三項、第百二十五条第四項及び第百二十七条第四項において準用する場合を含む。)に掲げる居住者は、法第二条第一項第三十二号ロ又はハ(定義)に掲げる者に該当する旨を証する書類として財務省令で定めるものを確定申告書に添付し、又は当該申告書の提出の際提示しなければならない。
\item[\rensuji{5}]法第百二十条第三項第四号(法第百二十二条第三項、第百二十三条第三項、第百二十五条第四項及び第百二十七条第四項において準用する場合を含む。)に掲げる居住者は、確定申告書に法第二百二十六条第一項から第三項まで及び第四項ただし書(源泉徴収票)の規定により交付される源泉徴収票を添付しなければならない。
\item[\rensuji{6}]国税庁長官は、第一項の方法を定めたときは、これを告示する。
\end{description}
\noindent\hspace{10pt}(給与所得以外の所得が少額であつても確定申告書の提出を要する場合)
\begin{description}
\item[第二百六十二条の二]法第百二十一条第一項(確定所得申告を要しない場合)に規定する政令で定める場合は、次に掲げる者がその者に係る第一号に規定する法人から、法第二十八条第一項(給与所得)に規定する給与等のほか、当該法人の事業に係る貸付金の利子又は不動産、動産、営業権その他の資産を当該事業の用に供することによる対価の支払を受ける場合とする。
\begin{description}
\item[一]法第百五十七条第一項第一号(同族会社の行為又は計算の否認)に規定する同族会社である法人の役員
\item[二]前号の役員の親族であり又はあつた者
\item[三]第一号の役員とまだ婚姻の届出をしないが事実上婚姻関係と同様の事情にあり又はあつた者
\item[四]第一号の役員から受ける金銭その他の資産によつて生計を維持している者
\end{description}
\end{description}
\noindent\hspace{10pt}(死亡の場合の確定申告の特例)
\begin{description}
\item[第二百六十三条]法第百二十四条第一項若しくは第二項(確定申告書を提出すべき者等が死亡した場合の確定申告)又は第百二十五条第一項から第三項まで(年の中途で死亡した場合の確定申告)の規定による申告書には、法第百二十条第一項各号(確定申告書の記載事項)に掲げる事項のほか、財務省令で定める事項をあわせて記載しなければならない。
\item[\rensuji{2}]前項の申告書を提出する場合において、相続人が二人以上あるときは、当該申告書は、各相続人が連署による一の書面で提出しなければならない。
\item[\rensuji{3}]前項ただし書の方法により同項に規定する申告書を提出した相続人は、遅滞なく、他の相続人に対し、当該申告書に記載した事項の要領を通知しなければならない。
\end{description}
\noindent\hspace{10pt}(各種所得につき源泉徴収をされた所得税等の額から控除する所得税の額)
\begin{description}
\item[第二百六十四条]法第百二十条第一項第五号(確定所得申告)に規定する政令で定める金額は、法第百六十一条第一項第六号(国内源泉所得)に掲げる対価につき法第二百十二条第一項(源泉徴収義務)の規定により源泉徴収をされた所得税の額のうち法第二百十五条(非居住者の人的役務の提供による給与等に係る源泉徴収の特例)の規定により徴収が行われたものとみなされる法第百六十一条第一項第十二号イ又はハに掲げる給与又は報酬に対応する部分の金額とする。
\end{description}
\subsubsection*{第二款 延払条件付譲渡に係る所得税額の延納}
\addcontentsline{toc}{subsubsection}{第二款 延払条件付譲渡に係る所得税額の延納}
\noindent\hspace{10pt}(延払条件付譲渡に係る要件)
\begin{description}
\item[第二百六十五条]法第百三十二条第三項第三号(延払条件付譲渡の要件)に規定する政令で定める要件は、当該契約において定められているその譲渡の目的物の引渡しの期日までに支払の期日の到来する賦払金の額の合計額がその譲渡の対価の額の三分の二以下となつていることとする。
\end{description}
\noindent\hspace{10pt}(延払条件付譲渡に係る税額の計算等)
\begin{description}
\item[第二百六十六条]法第百三十二条第四項(延払条件付譲渡に係る所得税額の延納)に規定する政令で定めるところにより計算した金額は、第一号に掲げる金額から第二号に掲げる金額を控除した金額とする。
\begin{description}
\item[一]法第百三十二条第一項第一号に規定する申告書に記載された法第百二十条第一項第三号(確定所得申告に係る所得税額)に掲げる所得税の額
\item[二]前号に規定する申告書に記載された法第百二十条第一項第一号に掲げる課税総所得金額、課税退職所得金額及び課税山林所得金額から、これらの金額の計算の基礎となつた譲渡所得の金額(法第三十三条第三項第二号(譲渡所得の金額)に掲げる所得に係る部分については、その金額の二分の一に相当する金額)又は山林所得の金額に、イに掲げる金額のうちにロに掲げる金額の占める割合を乗じて計算した金額を控除した金額につき法第二編第三章(税額の計算)の規定に準じて計算した所得税の額
\begin{description}
\item[イ]当該課税総所得金額又は課税山林所得金額の計算の基礎となつた譲渡所得又は山林所得に係る総収入金額
\item[ロ]法第百三十二条第四項に規定する賦払金の額の合計額
\end{description}
\end{description}
\item[\rensuji{2}]法第百三十五条第一項第二号(延払条件付譲渡に係る所得税額の延納の取消し)に規定する政令で定めるところにより計算した金額は、第一号に掲げる金額から第二号に掲げる金額を控除した金額とする。
\begin{description}
\item[一]法第百三十五条第一項第二号に規定する修正後の年税額
\item[二]法第百三十五条第一項第二号に規定する申告又は更正があつた後におけるその年分の課税総所得金額、課税退職所得金額及び課税山林所得金額から、これらの金額の計算の基礎となつた譲渡所得の金額(法第三十三条第三項第二号に掲げる所得に係る部分については、その金額の二分の一に相当する金額)又は山林所得の金額に、イに掲げる金額のうちにロに掲げる金額の占める割合を乗じて計算した金額を控除した金額につき法第二編第三章の規定に準じて計算した所得税の額
\begin{description}
\item[イ]当該課税総所得金額又は課税山林所得金額の計算の基礎となつた譲渡所得又は山林所得に係る総収入金額
\item[ロ]当該申告又は更正があつた後における法第百三十二条第四項に規定する賦払金の額の合計額
\end{description}
\end{description}
\item[\rensuji{3}]第一項第二号又は前項第二号に掲げる所得税の額を計算する場合におけるこれらの規定に定める控除については、次に定めるところによる。
\begin{description}
\item[一]その年分の譲渡所得の金額のうちに法第三十三条第三項第一号に掲げる所得に係る部分と同項第二号に掲げる所得に係る部分とがあるときは、それぞれにつき第一項第二号又は前項第二号の規定を適用して控除すべき金額を計算する。
\item[二]控除すべき譲渡所得に係る金額は、課税総所得金額、課税山林所得金額又は課税退職所得金額から順次控除する。
\item[三]控除すべき山林所得に係る金額は、課税山林所得金額、課税総所得金額又は課税退職所得金額から順次控除する。
\item[四]第一項第二号又は前項第二号に規定する割合は、小数点以下二位まで算出し、三位以下を切り上げたところによる。
\end{description}
\end{description}
\subsubsection*{第三款 納税の猶予}
\addcontentsline{toc}{subsubsection}{第三款 納税の猶予}
\noindent\hspace{10pt}(国外転出をする場合の譲渡所得等の特例の適用がある場合の納税猶予)
\begin{description}
\item[第二百六十六条の二]法第百三十七条の二第一項(国外転出をする場合の譲渡所得等の特例の適用がある場合の納税猶予)に規定する政令で定める場合は、同項に規定する国外転出(以下この条において「国外転出」という。)の日から五年を経過する日(法第百三十七条の二第二項の規定により同条第一項の規定による納税の猶予を受けている場合には、十年を経過する日)までに同項(同条第二項の規定により適用する場合を含む。第五項において同じ。)の規定による納税の猶予を受けている個人が死亡したことにより、当該国外転出の時に有していた法第六十条の二第一項(国外転出をする場合の譲渡所得等の特例)に規定する有価証券等又は締結していた同条第二項に規定する未決済信用取引等若しくは同条第三項に規定する未決済デリバティブ取引に係る契約の相続(限定承認に係るものに限る。)又は遺贈(包括遺贈のうち限定承認に係るものに限る。)による移転があつた場合とする。
\item[\rensuji{2}]法第百三十七条の二第一項に規定する納税猶予分の所得税額に百円未満の端数があるとき、又はその全額が百円未満であるときは、その端数金額又はその全額を切り捨てる。
\item[\rensuji{3}]第百七十条第二項(国外転出をする場合の譲渡所得等の特例)の規定は、法第百三十七条の二第五項に規定する譲渡に類するものとして政令で定めるものについて準用する。
\item[\rensuji{4}]法第百三十七条の二第五項に規定する政令で定めるところにより計算した金額は、第一号に掲げる金額から第二号に掲げる金額を控除した金額(当該金額が零を下回る場合には、零)とする。
\begin{description}
\item[一]法第百三十七条の二第一項に規定する納税猶予分の所得税額(既に同条第五項の規定の適用があつた場合には、同項の規定の適用があつた金額を除く。)
\item[二]当該国外転出の日の属する年分の法第百二十条第一項第三号(確定所得申告)に掲げる金額から法第百三十七条の二第一項に規定する適用資産(既に同条第五項の事由が生じたものを除く。次項において同じ。)につき法第六十条の二第一項から第三項までの規定の適用がないものとした場合における当該年分の同号に掲げる金額を控除した金額(当該金額が零を下回る場合には、零)
\end{description}
\item[\rensuji{5}]法第百三十七条の二第一項の規定による納税の猶予に係る同項に規定する満了基準日までに同条第五項の個人が国外転出の時において有していた適用資産につき同項の事由が生じた場合には、当該個人は、当該事由が生じた適用資産の種類、名称又は銘柄及び単位数並びに前項の規定による金額の計算に関する明細その他参考となるべき事項を記載した書類を、当該事由が生じた日から四月を経過する日までに、納税地の所轄税務署長に提出しなければならない。
\item[\rensuji{6}]法第百三十七条の二第九項第三号に規定する政令で定める事由は、同条第一項の規定の適用を受ける個人が国税通則法第百十七条第一項(納税管理人)に規定する納税管理人を解任し、又は当該納税管理人につき死亡、解散その他財務省令で定める事実(以下この項において「死亡等事実」という。)が生じた場合において、その解任の日から四月を経過する日又は当該個人が当該納税管理人につき死亡等事実の生じたことを知つた日から六月を経過する日までに同条第二項の規定による納税管理人の届出をしなかつたこととする。
\item[\rensuji{7}]法第百三十七条の二第十三項の規定により納付の義務を承継した同項の相続人(以下この条において「猶予承継相続人」という。)については、法第百三十七条の二第一項の規定の適用を受けた者とみなして、同条及びこの条の規定を適用する。
\item[\rensuji{8}]非居住者である猶予承継相続人は、既に国税通則法第百十七条第二項の規定による納税管理人の届出をしている場合を除き、その相続の開始があつたことを知つた日の翌日から四月以内に、同項の規定による納税管理人の届出をしなければならない。
\item[\rensuji{9}]法第百三十七条の三第十項及び第十四項(第三号に係る部分に限る。)の規定は、居住者である猶予承継相続人が国外転出をする場合について準用する。
\item[\rensuji{10}]次条第十三項及び第十四項の規定は、猶予承継相続人が法第百三十七条の二第二項の届出書、同条第六項に規定する継続適用届出書又は第五項の書類を提出する場合について準用する。
\end{description}
\noindent\hspace{10pt}(贈与等により非居住者に資産が移転した場合の譲渡所得等の特例の適用がある場合の納税猶予)
\begin{description}
\item[第二百六十六条の三]法第百三十七条の三第一項(贈与等により非居住者に資産が移転した場合の譲渡所得等の特例の適用がある場合の納税猶予)に規定する政令で定める場合は、同項に規定する贈与の日から五年を経過する日(同条第三項の規定により同条第一項の規定による納税の猶予を受けている場合には、十年を経過する日)までに当該贈与に係る非居住者である受贈者が死亡したことにより、当該贈与により移転を受けた法第六十条の三第一項(贈与等により非居住者に資産が移転した場合の譲渡所得等の特例)に規定する有価証券等(以下この条において「有価証券等」という。)又は法第六十条の三第二項に規定する未決済信用取引等(以下この条において「未決済信用取引等」という。)若しくは法第六十条の三第三項に規定する未決済デリバティブ取引(以下この条において「未決済デリバティブ取引」という。)に係る契約の相続(限定承認に係るものに限る。)又は遺贈(包括遺贈のうち限定承認に係るものに限る。)による移転があつた場合とする。
\item[\rensuji{2}]法第百三十七条の三第二項に規定する適用被相続人等の相続人は、次の各号に掲げる期限までに、それぞれ当該各号に定める相続等納税猶予分の所得税額に相当する担保を供さなければならない。
\begin{description}
\item[一]法第百三十七条の三第二項に規定する相続の開始の日の属する年分の所得税に係る同項に規定する確定申告期限
\item[二]当該相続の開始の日の属する年分の所得税に係る法第百五十一条の六第一項(遺産分割等があつた場合の修正申告の特例)の規定による修正申告書の同項に規定する提出期限
\end{description}
\item[\rensuji{3}]法第百三十七条の三第二項の規定による納税管理人の届出をする場合において、同項に規定する対象資産を取得した非居住者が二人以上あるときは、当該届出は、各非居住者が連署による一の書面で行わなければならない。
\item[\rensuji{4}]前項ただし書の方法により同項の届出をした非居住者は、遅滞なく、当該取得した他の非居住者に対し、当該届出の際に提出した書面に記載した事項の要領を通知しなければならない。
\item[\rensuji{5}]法第百三十七条の三第十項の規定は、同条第二項に規定する適用被相続人等の相続人である居住者が法第六十条の二第一項(国外転出をする場合の譲渡所得等の特例)に規定する国外転出(第十八項において「国外転出」という。)をしようとする場合について準用する。
\item[\rensuji{6}]法第百三十七条の三第二項に規定する政令で定める場合は、相続の開始の日から五年を経過する日(同条第三項の規定により同条第二項の規定による納税の猶予を受けている場合には、十年を経過する日。第九項において同じ。)までに当該相続又は遺贈(同条第二項に規定する遺贈をいう。以下この項及び第九項において同じ。)に係る非居住者である受贈者、相続人又は受遺者の全てが死亡したことにより、当該相続又は遺贈により移転を受けた有価証券等又は未決済信用取引等若しくは未決済デリバティブ取引に係る契約の全てについて相続(限定承認に係るものに限る。)又は遺贈(包括遺贈のうち限定承認に係るものに限る。)による移転があつた場合とする。
\item[\rensuji{7}]法第百三十七条の三第一項に規定する贈与納税猶予分の所得税額若しくは同条第二項に規定する相続等納税猶予分の所得税額又はこれらの金額の合計額に百円未満の端数があるとき、又はその全額が百円未満であるときは、その端数金額又はその全額を切り捨てる。
\item[\rensuji{8}]法第百三十七条の三第二項に規定する適用被相続人等の相続人は、同項に規定する相続の開始の日の属する年分の所得税につき法第百五十一条の六第一項の規定による修正申告書を提出する場合において、当該修正申告書の提出により増加した法第百三十七条の三第二項に規定する相続等納税猶予分の所得税額につき同項(同条第三項の規定により適用する場合を含む。以下この項において同じ。)の規定の適用を受けようとするときは、当該修正申告書に、同条第二項の規定の適用を受けようとする旨の記載をし、かつ、法第六十条の三第一項から第三項までの規定により行われたものとみなされた法第百三十七条の三第一項に規定する対象資産の譲渡又は決済の明細及び当該修正申告書の提出により増加した当該相続等納税猶予分の所得税額の計算に関する明細その他財務省令で定める事項を記載した書類を添付しなければならない。
\item[\rensuji{9}]法第百三十七条の三第六項に規定する政令で定める事由は、相続の開始の日から五年を経過する日までに同項の相続又は遺贈に係る非居住者である受贈者、相続人又は受遺者が死亡したことにより、当該相続又は遺贈により移転を受けた有価証券等又は未決済信用取引等若しくは未決済デリバティブ取引に係る契約の一部について相続(限定承認に係るものに限る。)又は遺贈(包括遺贈のうち限定承認に係るものに限る。)による移転があつたこととする。
\item[\rensuji{10}]法第百三十七条の三第六項に規定する政令で定めるところにより計算した金額は、第一号に掲げる金額から第二号に掲げる金額を控除した金額(当該金額が零を下回る場合には、零)とする。
\begin{description}
\item[一]法第百三十七条の三第四項に規定する納税猶予分の所得税額(既に同条第六項の規定の適用があつた場合には、同項の規定の適用があつた金額を除く。)
\item[二]当該贈与の日又は相続の開始の日(次項において「贈与等の日」という。)の属する年分の法第百二十条第一項第三号(確定所得申告)に掲げる金額から法第百三十七条の三第一項に規定する適用贈与資産又は同条第二項に規定する適用相続等資産(これらの資産について既に同条第六項の事由が生じたものを除く。第十二項において同じ。)につき法第六十条の三第一項から第三項までの規定の適用がないものとした場合における当該年分の同号に掲げる金額を控除した金額(当該金額が零を下回る場合には、零)
\end{description}
\item[\rensuji{11}]贈与等の日の属する年分の所得税につき法第百三十七条の二第一項(国外転出をする場合の譲渡所得等の特例の適用がある場合の納税猶予)の規定の適用があり、かつ、法第百三十七条の三第一項の規定の適用がある場合には、前条第四項の規定にかかわらず、前項の規定を準用する。
\item[\rensuji{12}]法第百三十七条の三第一項又は第二項(これらの規定を同条第三項の規定により適用する場合を含む。)の規定による納税の猶予に係る同条第一項に規定する贈与満了基準日又は同条第二項に規定する相続等満了基準日までに贈与、相続又は遺贈により移転を受けた適用贈与資産又は適用相続等資産について同条第六項の事由が生じた場合には、同条第七項に規定する適用贈与者等は、当該事由が生じた適用贈与資産又は適用相続等資産の種類、名称又は銘柄及び単位数並びに第十項(前項において準用する場合を含む。)の規定による金額の計算に関する明細その他参考となるべき事項を記載した書類を、当該事由が生じた日から四月を経過する日までに、納税地の所轄税務署長に提出しなければならない。
\item[\rensuji{13}]法第百三十七条の三第三項の届出書、同条第七項に規定する継続適用届出書又は前項の書類(以下この項及び次項において「継続適用届出書等」という。)を提出する場合において、同条第二項の規定の適用を受ける相続人が二人以上あるときは、当該継続適用届出書等は、各相続人が連署による一の書面で提出しなければならない。
\item[\rensuji{14}]前項ただし書の方法により継続適用届出書等を提出した同項の相続人は、遅滞なく、他の相続人に対し、当該継続適用届出書等に記載した事項の要領を通知しなければならない。
\item[\rensuji{15}]法第百三十七条の三第十一項第三号に規定する政令で定める事由は、同号の適用贈与者等が国税通則法第百十七条第一項(納税管理人)に規定する納税管理人を解任し、又は当該納税管理人につき前条第六項に規定する死亡等事実が生じた場合において、その解任の日から四月を経過する日又は当該適用贈与者等が当該納税管理人につき当該死亡等事実の生じたことを知つた日から六月を経過する日までに同法第百十七条第二項の規定による納税管理人の届出をしなかつたこととする。
\item[\rensuji{16}]法第百三十七条の三第十五項の規定により納付の義務を承継した同項に規定する適用贈与者等の相続人(以下この条において「猶予承継相続人」という。)については、法第百三十七条の三第一項の規定の適用を受けた者又は同条第二項の規定の適用を受けた相続人とみなして、同条及びこの条の規定を適用する。
\item[\rensuji{17}]非居住者である猶予承継相続人は、既に国税通則法第百十七条第二項の規定による納税管理人の届出をしている場合を除き、その相続の開始があつたことを知つた日の翌日から四月以内に、同項の規定による納税管理人の届出をしなければならない。
\item[\rensuji{18}]法第百三十七条の三第十項及び第十四項(第三号に係る部分に限る。)の規定は、居住者である猶予承継相続人が国外転出をする場合について準用する。
\item[\rensuji{19}]第十三項及び第十四項の規定は、猶予承継相続人が法第百三十七条の三第三項の届出書、同条第七項に規定する継続適用届出書又は第十二項の書類を提出する場合について準用する。
\end{description}
\subsection*{第三節 還付}
\addcontentsline{toc}{subsection}{第三節 還付}
\subsubsection*{第一款 確定申告による還付}
\addcontentsline{toc}{subsubsection}{第一款 確定申告による還付}
\noindent\hspace{10pt}(確定申告による還付)
\begin{description}
\item[第二百六十七条]法第百三十八条第一項(源泉徴収税額等の還付)又は第百三十九条第一項若しくは第二項(予納税額の還付)の規定による還付金の還付を受けようとする者は、確定申告書に次に掲げる事項を記載しなければならない。
\begin{description}
\item[一]当該還付金の支払を受けようとする銀行又は郵便局(簡易郵便局法(昭和二十四年法律第二百十三号)第二条(定義)に規定する郵便窓口業務を行う日本郵便株式会社の営業所であつて郵政民営化法(平成十七年法律第九十七号)第九十四条(定義)に規定する郵便貯金銀行を銀行法(昭和五十六年法律第五十九号)第二条第十六項(定義等)に規定する所属銀行とする同条第十四項に規定する銀行代理業の業務を行うものをいう。)の名称及び所在地
\item[二]当該還付金の額のうちにまだ納付されていない法第百三十八条第二項に規定する源泉徴収税額に相当する金額があるときは、当該金額
\item[三]その他参考となるべき事項
\end{description}
\item[\rensuji{2}]前項の規定による記載をした確定申告書を提出する場合において、その年中の各種所得につき源泉徴収をされた所得税の額があるときは、当該申告書に、当該所得税の額が源泉徴収をされた事実の説明となるべき財務省令で定める事項を記載した明細書を添附しなければならない。
\item[\rensuji{3}]第一項第二号に掲げる金額を記載した確定申告書を提出した者は、同号に規定する源泉徴収税額の納付があつた場合には、遅滞なく、その納付の日、その納付された源泉徴収税額その他必要な事項を記載した届出書を納税地の所轄税務署長に提出しなければならない。
\item[\rensuji{4}]税務署長は、第一項に規定する還付金に係る金額の記載がある確定申告書の提出があつた場合には、当該金額が過大であると認められる事由がある場合を除き、遅滞なく、法第百三十八条第一項又は第百三十九条第一項若しくは第二項の規定による還付又は充当の手続をしなければならない。
\item[\rensuji{5}]被相続人に係る第一項に規定する還付金の還付を受けようとする相続人が二人以上ある場合において、当該還付金に係る確定申告書を第二百六十三条第二項本文(相続人による確定申告書の提出)の規定により連署による一の書面で提出するときは、当該申告書には、当該還付金の額を各人別に記載しなければならない。
\end{description}
\noindent\hspace{10pt}(還付すべき所得税額の充当の順序)
\begin{description}
\item[第二百六十八条]法第百三十八条第一項(源泉徴収税額等の還付)の規定による還付金(これに係る還付加算金を含む。第三項において同じ。)を未納の国税及び滞納処分費に充当する場合には、次の各号の順序により充当するものとする。
\begin{description}
\item[一]その年分の未納の所得税で修正申告書の提出又は更正により納付すべきもの(法第百二十条第二項各号(予納税額の意義)に掲げる税額(以下この条において「予定納税額等」という。)を除く。)があるときは、当該所得税に充当する。
\item[二]前号の充当をしてもなお還付すべき金額があるときは、その他の未納の国税及び滞納処分費に充当する。
\end{description}
\item[\rensuji{2}]法第百三十九条第一項又は第二項(予納税額の還付)の規定による還付金(これに係る還付加算金を含む。次項において同じ。)を未納の国税及び滞納処分費に充当する場合には、次の各号の順序により充当するものとする。
\begin{description}
\item[一]その年分の未納の所得税で修正申告書の提出又は更正により納付すべきもの(予定納税額等を除く。)があるときは、当該所得税に充当する。
\item[二]前号の充当をしてもなお還付すべき金額がある場合において、その年分の予定納税額等で未納のものがあるときは、当該未納の予定納税額等に充当する。
\item[三]前二号の充当をしてもなお還付すべき金額があるときは、その他の未納の国税及び滞納処分費に充当する。
\end{description}
\item[\rensuji{3}]その年分の所得税に係る法第百三十八条第一項の規定による還付金と法第百三十九条第一項又は第二項の規定による還付金とがある場合において、これらの還付金をその年分の所得税で未納のものに充当するときは、次の各号に掲げる場合の区分に応じ当該各号に掲げる還付金からまず充当するものとする。
\begin{description}
\item[一]前項第一号に規定する所得税に充当する場合
\item[二]予定納税額等に充当する場合
\end{description}
\end{description}
\noindent\hspace{10pt}(予納税額に係る還付加算金の額の計算)
\begin{description}
\item[第二百六十九条]法第百三十九条第一項(予納税額の還付)の規定による還付金について還付加算金の額を計算する場合には、同項に規定する確定申告書に係る年分の前条第一項第一号に規定する予定納税額等(既に法第百三十九条第三項若しくは第百六十条第四項(更正等又は決定による予納税額の還付)の還付加算金の額の計算の基礎とされた部分の金額があり、又は法第百三十九条第一項若しくは第百六十条第一項若しくは第二項の規定による還付金をもつて充当をされる部分の金額がある場合には、これらの金額を除く。以下この条において「予定納税額等」という。)のうち次に定める順序により当該還付金の額(当該還付金をもつて前条第二項第一号又は第二号の充当をする場合には、当該充当をする還付金の額を控除した金額)に達するまで順次さかのぼつて求めた各予定納税額等を法第百三十九条第三項に規定する還付をすべき予納税額として、同項の規定を適用する。
\begin{description}
\item[一]当該予定納税額等のうち国税通則法第二条第八号(定義)に規定する法定納期限(以下この条において「法定納期限」という。)を異にするものについては、その法定納期限の遅いものを先順位とする。
\item[二]法定納期限を同じくする予定納税額等のうち確定の日を異にするものについては、その確定の日の遅いものを先順位とする。
\item[三]法定納期限及び確定の日を同じくする予定納税額等のうち納付の日を異にするものについては、その納付の日の遅いものを先順位とする。
\end{description}
\end{description}
\noindent\hspace{10pt}(予納税額に係る延滞税の還付金額の計算)
\begin{description}
\item[第二百七十条]法第百三十九条第二項(予納税額の還付)に規定する政令で定めるところにより計算した金額は、第一号に掲げる金額から第二号に掲げる金額を控除した残額とする。
\begin{description}
\item[一]法第百三十九条第一項に規定する確定申告書に係る年分の第二百六十八条第一項第一号(還付すべき所得税額の充当の順序)に規定する予定納税額等(以下この条において「予定納税額等」という。)について納付された延滞税の額の合計額(当該延滞税のうちに既に法第百三十九条第二項又は第百六十条第三項(更正等又は決定による予納税額の還付)の規定により還付されるべきこととなつたものがある場合には、その還付されるべきこととなつた延滞税の額を除く。)
\item[二]当該予定納税額等(法第百三十九条第一項又は第百六十条第一項若しくは第二項の規定による還付金をもつて充当をされる部分の金額を除く。)のうち次に定める順序により前号の確定申告書に記載された法第百二十条第一項第三号(確定所得申告)に掲げる金額(同項第五号に規定する源泉徴収税額がある場合には同号に掲げる金額とし、第二百六十八条第二項第一号の充当をされる所得税がある場合には当該所得税の額を加算した金額とする。)に達するまで順次求めた各予定納税額等につき国税に関する法律の規定により計算される延滞税の額の合計額
\begin{description}
\item[イ]当該予定納税額等のうち国税通則法第二条第八号(定義)に規定する法定納期限(以下この条において「法定納期限」という。)を異にするものについては、その法定納期限の早いものを先順位とする。
\item[ロ]法定納期限を同じくする予定納税額等のうち確定の日を異にするものについては、その確定の日の早いものを先順位とする。
\item[ハ]法定納期限及び確定の日を同じくする予定納税額等のうち納付の日を異にするものについては、その納付の日の早いものを先順位とする。
\end{description}
\end{description}
\end{description}
\subsubsection*{第二款 純損失の繰戻しによる還付}
\addcontentsline{toc}{subsubsection}{第二款 純損失の繰戻しによる還付}
\noindent\hspace{10pt}(純損失の繰戻しをする場合の計算)
\begin{description}
\item[第二百七十一条]法第百四十条第一項第二号(純損失の繰戻しによる還付の請求)又は第百四十一条第一項第二号(相続人等の純損失の繰戻しによる還付の請求)に掲げる金額を計算する場合において、純損失の金額の全部又は一部を前年分の課税総所得金額、課税退職所得金額及び課税山林所得金額から控除するときは、次に定めるところによる。
\begin{description}
\item[一]控除しようとする純損失の金額のうちに第二百一条第二号イ(純損失の繰越控除)に規定する総所得金額の計算上生じた損失の部分の金額がある場合には、これをまず前年分の課税総所得金額から控除する。
\item[二]控除しようとする純損失の金額のうちに第二百一条第二号ロに規定する山林所得金額の計算上生じた損失の部分の金額がある場合には、これをまず前年分の課税山林所得金額から控除する。
\item[三]第一号の規定による控除をしてもなお控除しきれない総所得金額の計算上生じた損失の部分の金額は、前年分の課税山林所得金額(前号の規定による控除が行なわれる場合には、当該控除後の金額)から控除し、次に課税退職所得金額から控除する。
\item[四]第二号の規定による控除をしてもなお控除しきれない山林所得金額の計算上生じた損失の部分の金額は、前年分の課税総所得金額(第一号の規定による控除が行なわれる場合には、当該控除後の金額)から控除し、次に課税退職所得金額(前号の規定による控除が行なわれる場合には、当該控除後の金額)から控除する。
\item[五]第一号又は第三号の場合において、総所得金額の計算上生じた損失の部分の金額のうちに、第百九十九条(変動所得の損失等の損益通算)に規定する変動所得の損失の金額とその他の損失の金額とがあるときは、まずその他の損失の金額を控除し、次に変動所得の損失の金額を控除する。
\item[六]第一号又は第四号の場合において、前年に法第九十条第一項(変動所得及び臨時所得の平均課税)の規定の適用があつたときは、同年分の課税総所得金額から控除しようとする純損失の金額のうち、第百九十九条に規定する変動所得の損失の金額は、まず同年分の法第九十条第三項に規定する平均課税対象金額から控除するものとし、当該変動所得以外の各種所得の金額の計算上生じた損失の部分の金額は、まず同年分の課税総所得金額のうち当該平均課税対象金額以外の部分の金額から控除するものとする。
\end{description}
\end{description}
\noindent\hspace{10pt}(事業の廃止等に準ずる事実等)
\begin{description}
\item[第二百七十二条]法第百四十条第五項(事業の全部譲渡等の場合の純損失の繰戻しによる還付の請求)に規定する政令で定める事実は、事業の全部の相当期間の休止又は重要部分の譲渡で、これらの事実が生じたことにより同項に規定する純損失の金額につき法第七十条第一項(純損失の繰越控除)の規定の適用を受けることが困難となると認められるものとする。
\item[\rensuji{2}]法第百四十条第五項又は第百四十一条第四項(相続人等による純損失の繰戻しによる還付の請求)の規定により還付を請求することができる金額は、これらの規定に規定する事実が生じた日の属する年の前前年分の課税総所得金額、課税退職所得金額及び課税山林所得金額並びにこれらにつき法第二編第三章第一節(税率)の規定を適用して計算した所得税の額並びに同日の属する年の前年において生じたこれらの条に規定する純損失の金額を基礎とし、法第百四十条第一項から第三項まで及び第百四十一条第一項から第三項まで並びに前条の規定に準じて計算した金額とする。
\end{description}
\noindent\hspace{10pt}(相続人等による還付の請求)
\begin{description}
\item[第二百七十三条]法第百四十一条第一項又は第四項(相続人等の純損失の繰戻しによる還付の請求)の規定による還付の請求をする場合において、相続人が二人以上あるときは、当該請求に係る法第百四十二条第一項(純損失の繰戻しによる還付の手続等)の規定による還付請求書は、各相続人が連署による一の書面で提出しなければならない。
\item[\rensuji{2}]前項ただし書の方法により同項の請求書を提出した相続人は、遅滞なく、他の相続人に対し、当該請求書に記載した事項の要領を通知しなければならない。
\end{description}
\section*{第六章 修正申告の特例}
\addcontentsline{toc}{section}{第六章 修正申告の特例}
\begin{description}
\item[第二百七十三条の二]法第百五十一条の六第一項第五号(遺産分割等があつた場合の修正申告の特例)に規定する政令で定める事由は、次に掲げる事由とする。
\begin{description}
\item[一]相続又は遺贈により取得した財産についての権利の帰属に関する訴えについての判決があつたこと。
\item[二]条件付の遺贈について、条件が成就したこと。
\end{description}
\end{description}
\section*{第七章 更正の請求の特例}
\addcontentsline{toc}{section}{第七章 更正の請求の特例}
\noindent\hspace{10pt}(更正の請求の特例の対象となる事実)
\begin{description}
\item[第二百七十四条]法第百五十二条(各種所得の金額に異動を生じた場合の更正の請求の特例)に規定する政令で定める事実は、次に掲げる事実とする。
\begin{description}
\item[一]確定申告書を提出し、又は決定を受けた居住者の当該申告書又は決定に係る年分の各種所得の金額(事業所得の金額並びに事業から生じた不動産所得の金額及び山林所得の金額を除く。次号において同じ。)の計算の基礎となつた事実のうちに含まれていた無効な行為により生じた経済的成果がその行為の無効であることに基因して失われたこと。
\item[二]前号に掲げる者の当該年分の各種所得の金額の計算の基礎となつた事実のうちに含まれていた取り消すことのできる行為が取り消されたこと。
\end{description}
\end{description}
\section*{第八章 更正及び決定}
\addcontentsline{toc}{section}{第八章 更正及び決定}
\noindent\hspace{10pt}(同族関係者の範囲)
\begin{description}
\item[第二百七十五条]法第百五十七条第一項(同族会社等の行為又は計算の否認等)に規定する株主等と政令で定める特殊の関係のある居住者は、次に掲げる者とする。
\begin{description}
\item[一]当該株主等の親族
\item[二]当該株主等と婚姻の届出をしていないが事実上婚姻関係と同様の事情にある者
\item[三]当該株主等の使用人
\item[四]前三号に掲げる者以外の者で当該株主等から受ける金銭その他の資産によつて生計を維持しているもの
\item[五]前三号に掲げる者と生計を一にするこれらの者の親族
\end{description}
\end{description}
\noindent\hspace{10pt}(事業の主宰者の特殊関係者の範囲)
\begin{description}
\item[第二百七十六条]法第百五十七条第一項第二号ロ(同族会社等の行為又は計算の否認等)及び第百五十八条(事業所の所得の帰属の推定)に規定する主宰者と政令で定める特殊の関係のある個人は、次に掲げる者及びこれらの者であつた者とする。
\begin{description}
\item[一]当該主宰者の親族
\item[二]当該主宰者とまだ婚姻の届出をしないが事実上婚姻関係と同様の事情にある者
\item[三]当該主宰者の使用人
\item[四]前三号に掲げる者以外の者で当該主宰者から受ける金銭その他の資産によつて生計を維持するもの
\item[五]当該主宰者の雇主
\item[六]第二号から前号までに掲げる者と生計を一にするこれらの者の親族
\end{description}
\end{description}
\noindent\hspace{10pt}(更正等又は決定による源泉徴収税額等の還付)
\begin{description}
\item[第二百七十七条]法第百五十九条第四項第二号ロ(更正等又は決定による源泉徴収税額等の還付)に規定する政令で定める理由は、国税通則法第五十八条第五項(還付加算金)に規定する政令で定める理由とする。
\item[\rensuji{2}]第二百六十八条(還付すべき所得税額の充当の順序)の規定は、法第百五十九条第一項又は第二項の規定による還付金を未納の国税及び滞納処分費に充当する場合について準用する。
\item[\rensuji{3}]法第百五十九条第一項又は第二項の規定による還付を受ける者は、その還付を受ける金額のうちに同条第三項に規定する源泉徴収税額でまだ納付されていないものがある場合において、当該源泉徴収税額の納付があつたときは、遅滞なく、その納付の日、その納付された源泉徴収税額その他必要な事項を記載した届出書を納税地の所轄税務署長に提出しなければならない。
\end{description}
\noindent\hspace{10pt}(更正等又は決定による予納税額の還付)
\begin{description}
\item[第二百七十八条]法第百六十条第三項(更正等又は決定による予納税額の還付)に規定する政令で定めるところにより計算した金額は、第一号に掲げる金額から第二号に掲げる金額を控除した残額とする。
\begin{description}
\item[一]法第百六十条第一項又は第二項の決定又は更正等があつた所得税に係る年分の法第百二十条第二項各号(予納税額の意義)に掲げる税額(次号において「予定納税額等」という。)について納付された延滞税の額の合計額(当該延滞税のうちに既に法第百三十九条第二項(予納税額の還付)又は第百六十条第三項の規定により還付されるべきこととなつたものがある場合には、その還付されるべきこととなつた延滞税の額を除く。)
\item[二]当該予定納税額等(法第百三十九条第一項又は第百六十条第一項若しくは第二項の規定による還付金をもつて充当をされる部分の金額を除く。)のうち次に定める順序により前号の決定又は更正等に係る法第百二十条第一項第三号に掲げる金額(同項第五号に規定する源泉徴収税額がある場合には同号に掲げる金額とし、第三項において準用する第二百六十八条第二項第一号(還付すべき所得税額の充当の順序)の充当をされる所得税がある場合には当該所得税の額を加算した金額とする。)に達するまで順次求めた各予定納税額等につき国税に関する法律の規定により計算される延滞税の額の合計額
\begin{description}
\item[イ]当該予定納税額等のうち国税通則法第二条第八号(定義)に規定する法定納期限(以下この号において「法定納期限」という。)を異にするものについては、その法定納期限の早いものを先順位とする。
\item[ロ]法定納期限を同じくする予定納税額等のうち確定の日を異にするものについては、その確定の日の早いものを先順位とする。
\item[ハ]法定納期限及び確定の日を同じくする予定納税額等のうち納付の日を異にするものについては、その納付の日の早いものを先順位とする。
\end{description}
\end{description}
\item[\rensuji{2}]法第百六十条第四項第二号イ(2)に規定する政令で定める理由は、国税通則法第五十八条第五項(還付加算金)に規定する政令で定める理由とする。
\item[\rensuji{3}]第二百六十八条の規定は、法第百六十条第一項から第三項までの規定による還付金を未納の国税及び滞納処分費に充当する場合について、第二百六十九条(予納税額に係る還付加算金の額の計算)の規定は、法第百六十条第一項又は第二項の規定による還付金について還付加算金の額を計算する場合についてそれぞれ準用する。
\end{description}
\part*{第三編 非居住者及び法人の納税義務}
\addcontentsline{toc}{part}{第三編 非居住者及び法人の納税義務}
\section*{第一章 国内源泉所得}
\addcontentsline{toc}{section}{第一章 国内源泉所得}
\noindent\hspace{10pt}(恒久的施設に係る内部取引の相手方である事業場等の範囲)
\begin{description}
\item[第二百七十九条]法第百六十一条第一項第一号(国内源泉所得)に規定する政令で定めるものは、次に掲げるものとする。
\begin{description}
\item[一]法第二条第一項第八号の四イ(定義)に規定する事業を行う一定の場所に相当するもの
\item[二]法第二条第一項第八号の四ロに規定する建設若しくは据付けの工事又はこれらの指揮監督の役務の提供を行う場所に相当するもの
\item[三]法第二条第一項第八号の四ハに規定する自己のために契約を締結する権限のある者に相当する者
\item[四]前三号に掲げるものに準ずるもの
\end{description}
\end{description}
\noindent\hspace{10pt}(国内にある資産の運用又は保有により生ずる所得)
\begin{description}
\item[第二百八十条]次に掲げる資産の運用又は保有により生ずる所得(法第百六十一条第一項第八号から第十六号まで(国内源泉所得)に該当するものを除く。)は、同項第二号に掲げる国内源泉所得に含まれるものとする。
\begin{description}
\item[一]公社債のうち日本国の国債若しくは地方債若しくは内国法人の発行する債券又は金融商品取引法第二条第一項第十五号(定義)に掲げる約束手形
\item[二]居住者に対する貸付金に係る債権で当該居住者の行う業務に係るもの以外のもの
\item[三]国内にある営業所、事務所その他これらに準ずるもの又は国内において契約の締結の代理をする者を通じて締結した生命保険契約(保険業法第二条第三項(定義)に規定する生命保険会社若しくは同条第八項に規定する外国生命保険会社等の締結した保険契約又は同条第十八項に規定する少額短期保険業者(以下この号において「少額短期保険業者」という。)の締結したこれに類する保険契約をいう。)、第三十条第一号(非課税とされる保険金、損害賠償金等)に規定する旧簡易生命保険契約、損害保険契約(同法第二条第四項に規定する損害保険会社若しくは同条第九項に規定する外国損害保険会社等の締結した保険契約又は少額短期保険業者の締結したこれに類する保険契約をいう。)その他これらに類する契約に基づく保険金の支払又は剰余金の分配(これらに準ずるものを含む。)を受ける権利
\end{description}
\item[\rensuji{2}]第二百八十三条第一項(国内業務に係る貸付金の利子)に規定する利子は、法第百六十一条第一項第二号に掲げる国内源泉所得に含まれないものとする。
\end{description}
\noindent\hspace{10pt}(国内にある資産の譲渡により生ずる所得)
\begin{description}
\item[第二百八十一条]法第百六十一条第一項第三号(国内源泉所得)に規定する政令で定める所得は、次に掲げる所得とする。
\begin{description}
\item[一]国内にある不動産の譲渡による所得
\item[二]国内にある不動産の上に存する権利、鉱業法(昭和二十五年法律第二百八十九号)の規定による鉱業権又は採石法(昭和二十五年法律第二百九十一号)の規定による採石権の譲渡による所得
\item[三]国内にある山林の伐採又は譲渡による所得
\item[四]内国法人の発行する株式(株主となる権利、株式の割当てを受ける権利、新株予約権及び新株予約権の割当てを受ける権利を含む。)その他内国法人の出資者の持分(会社法の施行に伴う関係法律の整備等に関する法律第二百三十条第一項(特定目的会社による特定資産の流動化に関する法律等の一部を改正する法律の一部改正に伴う経過措置等)に規定する特例旧特定目的会社の出資者の持分を除く。以下この項及び第四項において「株式等」という。)の譲渡(租税特別措置法第三十七条の十第三項若しくは第四項(一般株式等に係る譲渡所得等の課税の特例)又は第三十七条の十一第三項若しくは第四項(上場株式等に係る譲渡所得等の課税の特例)の規定によりその額及び価額の合計額が同法第三十七条の十第一項に規定する一般株式等に係る譲渡所得等又は同法第三十七条の十一第一項に規定する上場株式等に係る譲渡所得等に係る収入金額とみなされる金銭及び金銭以外の資産の交付の基因となつた同法第三十七条の十第三項(第八号及び第九号に係る部分を除く。)若しくは第四項第一号から第三号まで又は第三十七条の十一第四項第一号及び第二号に規定する事由に基づく同法第三十七条の十第二項第一号から第五号までに掲げる株式等(同項第四号に掲げる受益権にあつては、公社債投資信託以外の証券投資信託の受益権及び証券投資信託以外の投資信託で公社債等運用投資信託に該当しないものの受益権に限る。)についての当該金銭の額及び当該金銭以外の資産の価額に対応する権利の移転又は消滅を含む。以下この条において同じ。)による所得で次に掲げるもの
\begin{description}
\item[イ]同一銘柄の内国法人の株式等の買集めをし、その所有者である地位を利用して、当該株式等をその内国法人若しくはその特殊関係者に対し、又はこれらの者若しくはその依頼する者のあつせんにより譲渡をすることによる所得
\item[ロ]内国法人の特殊関係株主等である非居住者が行うその内国法人の株式等の譲渡による所得
\end{description}
\item[五]法人(不動産関連法人に限る。)の株式(出資及び投資信託及び投資法人に関する法律第二条第十四項(定義)に規定する投資口(第九項において「投資口」という。)を含む。第八項及び第十項において同じ。)の譲渡による所得
\item[六]国内にあるゴルフ場の所有又は経営に係る法人の株式又は出資を所有することがそのゴルフ場を一般の利用者に比して有利な条件で継続的に利用する権利を有する者となるための要件とされている場合における当該株式又は出資の譲渡による所得
\item[七]国内にあるゴルフ場その他の施設の利用に関する権利の譲渡による所得
\item[八]前各号に掲げるもののほか、非居住者が国内に滞在する間に行う国内にある資産の譲渡による所得
\end{description}
\item[\rensuji{2}]前項第四号イに規定する株式等の買集めとは、金融商品取引所(金融商品取引法第二条第十六項(定義)に規定する金融商品取引所をいう。第九項において同じ。)又は同条第十三項に規定する認可金融商品取引業協会がその会員(同条第十九項に規定する取引参加者を含む。)に対し特定の銘柄の株式につき価格の変動その他売買状況等に異常な動きをもたらす基因となると認められる相当数の株式の買集めがあり、又はその疑いがあるものとしてその売買内容等につき報告又は資料の提出を求めた場合における買集めその他これに類する買集めをいう。
\item[\rensuji{3}]第一項第四号イに規定する特殊関係者とは、同号イの内国法人の役員又は主要な株主等(同号イに規定する株式等の買集めをした者から当該株式等を取得することによりその内国法人の主要な株主等となることとなる者を含む。)、これらの者の親族、これらの者の支配する法人、その内国法人の主要な取引先その他その内国法人とこれらに準ずる特殊の関係のある者をいう。
\item[\rensuji{4}]第一項第四号ロに規定する特殊関係株主等とは、次に掲げる者をいう。
\begin{description}
\item[一]第一項第四号ロの内国法人の一の株主等
\item[二]前号の一の株主等と法人税法施行令第四条(同族関係者の範囲)に規定する特殊の関係その他これに準ずる関係のある者
\item[三]第一号の一の株主等が締結している組合契約(次に掲げるものを含む。)に係る組合財産である第一項第四号ロの内国法人の株式等につき、その株主等に該当することとなる者(前二号に掲げる者を除く。)
\begin{description}
\item[イ]当該一の株主等が締結している組合契約による組合(これに類するものを含む。以下この号において同じ。)が締結している組合契約
\item[ロ]イ又はハに掲げる組合契約による組合が締結している組合契約
\item[ハ]ロに掲げる組合契約による組合が締結している組合契約
\end{description}
\end{description}
\item[\rensuji{5}]前項第三号及び第十項第三号において、組合契約とは次の各号に掲げる契約をいい、組合財産とは当該各号に掲げる契約の区分に応じ当該各号に定めるものをいう。
\begin{description}
\item[一]民法第六百六十七条(組合契約)に規定する組合契約
\item[二]投資事業有限責任組合契約に関する法律(平成十年法律第九十号)第三条第一項(投資事業有限責任組合契約)に規定する投資事業有限責任組合契約
\item[三]有限責任事業組合契約に関する法律(平成十七年法律第四十号)第三条第一項(有限責任事業組合契約)に規定する有限責任事業組合契約
\item[四]外国における前三号に掲げる契約に類する契約(以下この号において「外国組合契約」という。)
\end{description}
\item[\rensuji{6}]第一項第四号ロに規定する株式等の譲渡は、次に掲げる要件を満たす場合の同号ロの非居住者の当該譲渡の日の属する年(以下この項及び第九項において「譲渡年」という。)における第二号に規定する株式又は出資の譲渡に限るものとする。
\begin{description}
\item[一]譲渡年以前三年内のいずれかの時において、第一項第四号ロの内国法人の特殊関係株主等がその内国法人の発行済株式又は出資(次号及び次項において「発行済株式等」という。)の総数又は総額の百分の二十五以上に相当する数又は金額の株式又は出資(当該特殊関係株主等が第四項第三号に掲げる者である場合には、同号の組合財産であるものに限る。次号及び次項において同じ。)を所有していたこと。
\item[二]譲渡年において、第一項第四号ロの非居住者を含む同号ロの内国法人の特殊関係株主等が最初にその内国法人の株式又は出資の譲渡をする直前のその内国法人の発行済株式等の総数又は総額の百分の五以上に相当する数又は金額の株式又は出資の譲渡をしたこと。
\end{description}
\item[\rensuji{7}]次の各号に掲げる場合のいずれかに該当するときは、第一項第四号ロの非居住者を含む同号ロの内国法人の特殊関係株主等が前項第二号に掲げる要件を満たす同号に規定する株式又は出資の譲渡をしたものとして、同項の規定を適用する。
\begin{description}
\item[一]第一項第四号ロの非居住者がその有する株式又は出資を発行した同号ロの内国法人の法人税法第二条第十二号の九(定義)に規定する分割型分割(以下この号において「分割型分割」という。)のうち次のいずれかに該当するものにより同条第十二号の三に規定する分割承継法人(以下この号において「分割承継法人」という。)の株式、第百十三条第一項(分割型分割により取得した株式等の取得価額)に規定する分割承継親法人(以下この号において「分割承継親法人」という。)の株式その他の資産の交付を受けた場合において、当該分割型分割に係る同条第三項に規定する割合に、当該内国法人の当該分割型分割の直前の発行済株式等の総数又は総額のうちに当該非居住者を含む当該内国法人の特殊関係株主等が当該分割型分割の直前に所有していた当該内国法人の株式又は出資の数又は金額の占める割合を乗じて計算した割合が百分の五以上であるとき。
\begin{description}
\item[イ]分割型分割に係る法人税法第二条第十二号の九イに規定する分割対価資産として当該分割型分割に係る分割承継法人の株式(出資を含む。以下この号において同じ。)又は分割承継親法人の株式のいずれか一方の株式以外の資産が交付される分割型分割
\item[ロ]分割型分割に係る分割承継法人の株式又は分割承継親法人の株式が当該分割型分割に係る法人税法第二条第十二号の二に規定する分割法人の発行済株式等の総数又は総額のうちに占める当該分割法人の各株主等の有する当該分割法人の株式の数又は金額の割合に応じて交付されない分割型分割
\end{description}
\item[二]第一項第四号ロの非居住者がその有する株式又は出資を発行した同号ロの内国法人の法人税法第二条第十二号の十五の二に規定する株式分配(以下この号において「株式分配」という。)のうち次のいずれかに該当するものにより同条第十二号の十五の二に規定する完全子法人(以下この号において「完全子法人」という。)の株式その他の資産の交付を受けた場合において、当該株式分配に係る第百十三条の二第二項(株式分配により取得した株式等の取得価額)に規定する割合に、当該内国法人の当該株式分配の直前の発行済株式等の総数又は総額のうちに当該非居住者を含む当該内国法人の特殊関係株主等が当該株式分配の直前に所有していた当該内国法人の株式又は出資の数又は金額の占める割合を乗じて計算した割合が百分の五以上であるとき。
\begin{description}
\item[イ]完全子法人の株式(出資を含む。ロにおいて同じ。)以外の資産が交付される株式分配
\item[ロ]株式分配に係る完全子法人の株式が当該株式分配に係る法人税法第二条第十二号の五の二に規定する現物分配法人の発行済株式等の総数又は総額のうちに占める当該現物分配法人の各株主等の有する当該現物分配法人の株式の数又は金額の割合に応じて交付されない株式分配
\end{description}
\item[三]第一項第四号ロの非居住者がその有する株式又は出資を発行した同号ロの内国法人の資本の払戻し(法第二十五条第一項第四号(配当等とみなす金額)に規定する資本の払戻しをいう。)又は解散による残余財産の分配(以下この号において「払戻し等」という。)として金銭その他の資産の交付を受けた場合において、当該払戻し等に係る第百十四条第一項(資本の払戻し等があつた場合の株式等の取得価額)に規定する払戻し等割合に、当該内国法人の当該払戻し等の直前の発行済株式等の総数又は総額のうちに当該非居住者を含む当該内国法人の特殊関係株主等が当該払戻し等の直前に所有していた当該内国法人の株式又は出資の数又は金額の占める割合を乗じて計算した割合が百分の五以上であるとき。
\end{description}
\item[\rensuji{8}]第一項第五号に規定する不動産関連法人とは、その株式の譲渡の日から起算して三百六十五日前の日から当該譲渡の直前の時までの間のいずれかの時において、その有する資産の価額の総額のうちに次に掲げる資産の価額の合計額の占める割合が百分の五十以上である法人をいう。
\begin{description}
\item[一]国内にある土地等(土地若しくは土地の上に存する権利又は建物及びその附属設備若しくは構築物をいう。以下この項において同じ。)
\item[二]その有する資産の価額の総額のうちに国内にある土地等の価額の合計額の占める割合が百分の五十以上である法人の株式
\item[三]前号又は次号に掲げる株式を有する法人(その有する資産の価額の総額のうちに国内にある土地等並びに前号、この号及び次号に掲げる株式の価額の合計額の占める割合が百分の五十以上であるものに限る。)の株式(前号に掲げる株式に該当するものを除く。)
\item[四]前号に掲げる株式を有する法人(その有する資産の価額の総額のうちに国内にある土地等並びに前二号及びこの号に掲げる株式の価額の合計額の占める割合が百分の五十以上であるものに限る。)の株式(前二号に掲げる株式に該当するものを除く。)
\end{description}
\item[\rensuji{9}]第一項第五号に規定する株式の譲渡は、次に掲げる株式(投資口を含む。以下この項において同じ。)又は出資の譲渡に限るものとする。
\begin{description}
\item[一]譲渡年の前年の十二月三十一日(以下この項において「基準日」という。)において、その株式又は出資(金融商品取引所に上場されているものその他これに類するものとして財務省令で定めるものに限る。次号において「上場株式等」という。)に係る第一項第五号の法人の特殊関係株主等が当該法人の発行済株式(投資信託及び投資法人に関する法律第二条第十二項に規定する投資法人にあつては、発行済みの投資口)又は出資(当該法人が有する自己の株式又は出資を除く。次号において「発行済株式等」という。)の総数又は総額の百分の五を超える数又は金額の株式又は出資(当該特殊関係株主等が次項第三号に掲げる者である場合には、同号の組合財産であるものに限る。)を有し、かつ、その株式又は出資の譲渡をした者が当該特殊関係株主等である場合の当該譲渡
\item[二]基準日において、その株式又は出資(上場株式等を除く。)に係る第一項第五号の法人の特殊関係株主等が当該法人の発行済株式等の総数又は総額の百分の二を超える数又は金額の株式又は出資(当該特殊関係株主等が次項第三号に掲げる者である場合には、同号の組合財産であるものに限る。)を有し、かつ、その株式又は出資の譲渡をした者が当該特殊関係株主等である場合の当該譲渡
\end{description}
\item[\rensuji{10}]前項に規定する特殊関係株主等とは、次に掲げる者をいう。
\begin{description}
\item[一]第一項第五号の法人の一の株主等
\item[二]前号の一の株主等と法人税法施行令第四条に規定する特殊の関係その他これに準ずる関係のある者
\item[三]第一号の一の株主等が締結している組合契約(次に掲げるものを含む。)に係る組合財産である第一項第五号の法人の株式につき、その株主等に該当することとなる者(前二号に掲げる者を除く。)
\begin{description}
\item[イ]当該一の株主等が締結している組合契約による組合(これに類するものを含む。以下この項において同じ。)が締結している組合契約
\item[ロ]イ又はハに掲げる組合契約による組合が締結している組合契約
\item[ハ]ロに掲げる組合契約による組合が締結している組合契約
\end{description}
\end{description}
\end{description}
\noindent\hspace{10pt}(恒久的施設を通じて行う組合事業から生ずる利益)
\begin{description}
\item[第二百八十一条の二]法第百六十一条第一項第四号(国内源泉所得)に規定する政令で定める契約は、次に掲げる契約とする。
\begin{description}
\item[一]投資事業有限責任組合契約に関する法律第三条第一項(投資事業有限責任組合契約)に規定する投資事業有限責任組合契約
\item[二]有限責任事業組合契約に関する法律第三条第一項(有限責任事業組合契約)に規定する有限責任事業組合契約
\item[三]外国における次に掲げる契約に類する契約
\begin{description}
\item[イ]民法第六百六十七条第一項(組合契約)に規定する組合契約
\item[ロ]前二号に掲げる契約
\end{description}
\end{description}
\item[\rensuji{2}]法第百六十一条第一項第四号に規定する政令で定める利益は、同号に規定する組合契約(以下この項において「組合契約」という。)に基づいて恒久的施設を通じて行う事業から生ずる収入から当該収入に係る費用(同条第一項第五号から第十六号までに掲げる国内源泉所得につき法第二百十二条第一項(源泉徴収義務)の規定により徴収された所得税を含む。)を控除したものについて当該組合契約を締結している組合員(当該組合契約を締結していた組合員並びに前項第三号に掲げる契約を締結している者及び当該契約を締結していた者を含む。)が当該組合契約に基づいて配分を受けるものとする。
\end{description}
\noindent\hspace{10pt}(国内にある土地等の譲渡による対価)
\begin{description}
\item[第二百八十一条の三]法第百六十一条第一項第五号(国内源泉所得)に規定する政令で定める対価は、土地等(国内にある土地若しくは土地の上に存する権利又は建物及びその附属設備若しくは構築物をいう。以下この条において同じ。)の譲渡による対価(その金額が一億円を超えるものを除く。)で、当該土地等を自己又はその親族の居住の用に供するために譲り受けた個人から支払われるものとする。
\end{description}
\noindent\hspace{10pt}(人的役務の提供を主たる内容とする事業の範囲)
\begin{description}
\item[第二百八十二条]法第百六十一条第一項第六号(国内源泉所得)に規定する政令で定める事業は、次に掲げる事業とする。
\begin{description}
\item[一]映画若しくは演劇の俳優、音楽家その他の芸能人又は職業運動家の役務の提供を主たる内容とする事業
\item[二]弁護士、公認会計士、建築士その他の自由職業者の役務の提供を主たる内容とする事業
\item[三]科学技術、経営管理その他の分野に関する専門的知識又は特別の技能を有する者の当該知識又は技能を活用して行う役務の提供を主たる内容とする事業(機械設備の販売その他事業を行う者の主たる業務に付随して行われる場合における当該事業及び法第二条第一項第八号の四ロ(定義)に規定する建設又は据付けの工事の指揮監督の役務の提供を主たる内容とする事業を除く。)
\end{description}
\end{description}
\noindent\hspace{10pt}(国内業務に係る貸付金の利子)
\begin{description}
\item[第二百八十三条]法第百六十一条第一項第十号(国内源泉所得)に規定する政令で定める利子は、次に掲げる債権のうち、その発生の日からその債務を履行すべき日までの期間(期間の更新その他の方法(以下この項において「期間の更新等」という。)により当該期間が実質的に延長されることが予定されているものについては、その延長された当該期間。以下この項において「履行期間」という。)が六月を超えないもの(その成立の際の履行期間が六月を超えなかつた当該債権について期間の更新等によりその履行期間が六月を超えることとなる場合のその期間の更新等が行われる前の履行期間における当該債権を含む。)の利子とする。
\begin{description}
\item[一]国内において業務を行う者に対してする資産の譲渡又は役務の提供の対価に係る債権
\item[二]前号に規定する対価の決済に関し、金融機関が国内において業務を行う者に対して有する債権
\end{description}
\item[\rensuji{2}]法第百六十一条第一項第十号の規定の適用については、居住者又は内国法人の業務の用に供される船舶又は航空機の購入のためにその居住者又は内国法人に対して提供された貸付金は、同号の規定に該当する貸付金とし、非居住者又は外国法人の業務の用に供される船舶又は航空機の購入のためにその非居住者又は外国法人に対して提供された貸付金は、同号の規定に該当する貸付金以外の貸付金とする。
\item[\rensuji{3}]法第百六十一条第一項第十号に規定する債券の買戻又は売戻条件付売買取引として政令で定めるものは、債券をあらかじめ約定した期日にあらかじめ約定した価格で(あらかじめ期日及び価格を約定することに代えて、その開始以後期日及び価格の約定をすることができる場合にあつては、その開始以後約定した期日に約定した価格で)買い戻し、又は売り戻すことを約定して譲渡し、又は購入し、かつ、当該約定に基づき当該債券と同種及び同量の債券を買い戻し、又は売り戻す取引(次項において「債券現先取引」という。)とする。
\item[\rensuji{4}]法第百六十一条第一項第十号に規定する差益として政令で定めるものは、国内において業務を行う者との間で行う債券現先取引で当該業務に係るものにおいて、債券を購入する際の当該購入に係る対価の額を当該債券と同種及び同量の債券を売り戻す際の当該売戻しに係る対価の額が上回る場合における当該売戻しに係る対価の額から当該購入に係る対価の額を控除した金額に相当する差益とする。
\end{description}
\noindent\hspace{10pt}(国内業務に係る使用料等)
\begin{description}
\item[第二百八十四条]法第百六十一条第一項第十一号ハ(国内源泉所得)に規定する政令で定める用具は、車両及び運搬具、工具並びに器具及び備品とする。
\item[\rensuji{2}]法第百六十一条第一項第十一号の規定の適用については、同号ロ又はハに規定する資産で居住者又は内国法人の業務の用に供される船舶又は航空機において使用されるものの使用料は、同号の規定に該当する使用料とし、当該資産で非居住者又は外国法人の業務の用に供される船舶又は航空機において使用されるものの使用料は、同号の規定に該当する使用料以外の使用料とする。
\end{description}
\noindent\hspace{10pt}(国内に源泉がある給与、報酬又は年金の範囲)
\begin{description}
\item[第二百八十五条]法第百六十一条第一項第十二号イ(国内源泉所得)に規定する政令で定める人的役務の提供は、次に掲げる勤務その他の人的役務の提供とする。
\begin{description}
\item[一]内国法人の役員としての勤務で国外において行うもの(当該役員としての勤務を行う者が同時にその内国法人の使用人として常時勤務を行う場合の当該役員としての勤務を除く。)
\item[二]居住者又は内国法人が運航する船舶又は航空機において行う勤務その他の人的役務の提供(国外における寄航地において行われる一時的な人的役務の提供を除く。)
\end{description}
\item[\rensuji{2}]法第百六十一条第一項第十二号ロに規定する政令で定める公的年金等は、第七十二条第三項第八号(退職手当等とみなす一時金)に規定する制度に基づいて支給される年金(これに類する給付を含む。)とする。
\item[\rensuji{3}]法第百六十一条第一項第十二号ハに規定する政令で定める人的役務の提供は、第一項各号に掲げる勤務その他の人的役務の提供で当該勤務その他の人的役務の提供を行う者が非居住者であつた期間に行つたものとする。
\end{description}
\noindent\hspace{10pt}(事業の広告宣伝のための賞金)
\begin{description}
\item[第二百八十六条]法第百六十一条第一項第十三号(国内源泉所得)に規定する政令で定める賞金は、国内において事業を行う者から当該事業の広告宣伝のために賞として支払を受ける金品その他の経済的な利益(旅行その他の役務の提供を内容とするもので、金品との選択をすることができないものとされているものを除く。)とする。
\end{description}
\noindent\hspace{10pt}(年金に係る契約の範囲)
\begin{description}
\item[第二百八十七条]法第百六十一条第一項第十四号(国内源泉所得)に規定する政令で定める契約は、第百八十三条第三項(生命保険契約等の意義)に規定する生命保険契約等又は第百八十四条第一項(損害保険年金等に係る雑所得の金額の計算上控除する保険料等)に規定する損害保険契約等であつて、年金を給付する定めのあるものとする。
\end{description}
\noindent\hspace{10pt}(匿名組合契約に準ずる契約の範囲)
\begin{description}
\item[第二百八十八条]法第百六十一条第一項第十六号(国内源泉所得)に規定する政令で定める契約は、当事者の一方が相手方の事業のために出資をし、相手方がその事業から生ずる利益を分配することを約する契約とする。
\end{description}
\noindent\hspace{10pt}(国内に源泉がある所得)
\begin{description}
\item[第二百八十九条]法第百六十一条第一項第十七号(国内源泉所得)に規定する政令で定める所得は、次に掲げる所得とする。
\begin{description}
\item[一]国内において行う業務又は国内にある資産に関し受ける保険金、補償金又は損害賠償金(これらに類するものを含む。)に係る所得
\item[二]国内にある資産の法人からの贈与により取得する所得
\item[三]国内において発見された埋蔵物又は国内において拾得された遺失物に係る所得
\item[四]国内において行う懸賞募集に基づいて懸賞として受ける金品その他の経済的な利益(旅行その他の役務の提供を内容とするもので、金品との選択ができないものとされているものを除く。)に係る所得
\item[五]前三号に掲げるもののほか、国内においてした行為に伴い取得する一時所得
\item[六]前各号に掲げるもののほか、国内において行う業務又は国内にある資産に関し供与を受ける経済的な利益に係る所得
\end{description}
\end{description}
\noindent\hspace{10pt}(債務の保証等に類する取引)
\begin{description}
\item[第二百九十条]法第百六十一条第二項(国内源泉所得)に規定する政令で定める取引は、資金の借入れその他の取引に係る債務の保証(債務を負担する行為であつて債務の保証に準ずるものを含む。)とする。
\end{description}
\noindent\hspace{10pt}(国際運輸業所得)
\begin{description}
\item[第二百九十一条]法第百六十一条第三項(国内源泉所得)に規定する政令で定める所得は、非居住者が国内及び国外にわたつて船舶又は航空機による運送の事業を行うことにより生ずる所得のうち、船舶による運送の事業にあつては国内において乗船し又は船積みをした旅客又は貨物に係る収入金額を基準とし、航空機による運送の事業にあつてはその国内業務(国内において行う業務をいう。以下この条において同じ。)に係る収入金額又は経費、その国内業務の用に供する固定資産の価額その他その国内業務が当該運送の事業に係る所得の発生に寄与した程度を推測するに足りる要因を基準として判定したその非居住者の国内業務につき生ずべき所得とする。
\end{description}
\noindent\hspace{10pt}(租税条約に異なる定めがある場合の国内源泉所得)
\begin{description}
\item[第二百九十一条の二]法第百六十二条第二項(租税条約に異なる定めがある場合の国内源泉所得)に規定する利子に準ずるものとして政令で定めるものは、手形の割引料その他経済的な性質が利子に準ずるものとする。
\item[\rensuji{2}]法第百六十二条第二項に規定する政令で定める事実は、次に掲げる事実とする。
\begin{description}
\item[一]次に掲げるものの使用料の支払に相当する事実
\begin{description}
\item[イ]工業所有権その他の技術に関する権利、特別の技術による生産方式又はこれらに準ずるもの
\item[ロ]著作権(出版権及び著作隣接権その他これに準ずるものを含む。)
\item[ハ]第六条第八号イからレまで(減価償却資産の範囲)に掲げる無形固定資産(国外における同号ヲからレまでに掲げるものに相当するものを含む。)
\end{description}
\item[二]前号イからハまでに掲げるものの譲渡又は取得に相当する事実
\end{description}
\end{description}
\section*{第二章 非居住者の納税義務}
\addcontentsline{toc}{section}{第二章 非居住者の納税義務}
\subsection*{第一節 非居住者に対する所得税の総合課税}
\addcontentsline{toc}{subsection}{第一節 非居住者に対する所得税の総合課税}
\subsubsection*{第一款 課税標準、税額等の計算}
\addcontentsline{toc}{subsubsection}{第一款 課税標準、税額等の計算}
\noindent\hspace{10pt}(恒久的施設帰属所得についての総合課税に係る所得税の課税標準等の計算)
\begin{description}
\item[第二百九十二条]非居住者の法第百六十五条第一項(総合課税に係る所得税の課税標準、税額等の計算)に規定する総合課税に係る所得税(法第百六十四条第一項第一号イ(非居住者に対する課税の方法)に掲げる国内源泉所得(次項及び第四項において「恒久的施設帰属所得」という。)に係る部分に限る。)の課税標準及び税額につき、法第百六十五条第一項の規定により次の各号に掲げる法の規定に準じて計算する場合には、当該各号に定めるところによる。
\begin{description}
\item[一]法第二十六条(不動産所得)及び第三十三条(譲渡所得)
\item[二]法第四十五条(家事関連費等の必要経費不算入等)
\item[三]法第四十七条(棚卸資産の売上原価等の計算及びその評価の方法)
\item[四]法第四十九条(減価償却資産の償却費の計算及びその償却の方法)
\item[五]法第五十条(繰延資産の償却費の計算及びその償却の方法)
\item[六]法第五十一条(資産損失の必要経費算入)
\item[七]法第五十二条(貸倒引当金)
\item[八]法第五十四条(退職給与引当金)
\item[九]法第五十八条(固定資産の交換の場合の譲渡所得の特例)
\begin{description}
\item[イ]法第五十八条第一項に規定する取得資産は、同項に規定する交換の時において国内にある固定資産に限るものとし、当該取得資産には事業場等からその交換により取得したものとされる固定資産を含むものとする。
\item[ロ]法第五十八条第一項に規定する譲渡資産は、同項に規定する交換の時において国内にある固定資産(恒久的施設を通じて行う事業に係るものに限る。)に限るものとする。
\end{description}
\item[十]法第六十二条(生活に通常必要でない資産の災害による損失)
\item[十一]法第六十五条(リース譲渡に係る収入及び費用の帰属時期)
\item[十二]法第六十七条の二(リース取引に係る所得の金額の計算)
\item[十三]法第七十二条(雑損控除)
\end{description}
\item[\rensuji{2}]非居住者の法第百六十五条第一項に規定する総合課税に係る所得税(恒久的施設帰属所得に係る部分に限る。)の課税標準及び税額につき、同項の規定により前編第一章、第二章及び第四章(居住者に係る課税標準の計算等)の規定に準じて計算する場合には、次の表の上欄に掲げる規定中同表の中欄に掲げる字句は、同表の下欄に掲げる字句にそれぞれ読み替えるものとする。
\item[\rensuji{3}]法第百六十五条第二項第二号に規定する政令で定めるところにより配分した金額は、非居住者のその年の同号に規定する販売費等及び育成費等並びに支出した金額につき、当該非居住者の恒久的施設を通じて行う事業及びそれ以外の事業に係る収入金額、資産の価額、使用人の数その他の基準のうち、これらの事業の内容及び当該費用の性質に照らして合理的と認められる基準を用いて当該非居住者の恒久的施設を通じて行う事業に配分した金額とする。
\item[\rensuji{4}]非居住者の事業場等と恒久的施設との間で当該恒久的施設における資産の購入その他資産の取得に相当する内部取引がある場合には、その内部取引の時にその内部取引に係る資産を取得したものとして、当該非居住者の恒久的施設帰属所得に係る所得の金額の計算に関する所得税に関する法令の規定を適用する。
\end{description}
\noindent\hspace{10pt}(減額された外国所得税額のうち総収入金額に算入しないもの)
\begin{description}
\item[第二百九十二条の二]法第百六十五条の二(減額された外国所得税額の総収入金額不算入等)に規定する政令で定める金額は、同条に規定する外国所得税の額が減額された金額のうちその減額されることとなつた日の属する年において第二百九十二条の十四第一項(外国所得税が減額された場合の特例)の規定による同項に規定する納付控除対象外国所得税額からの控除又は同条第三項の規定による同項に規定する控除限度超過額からの控除に充てられることとなる部分の金額に相当する金額とする。
\end{description}
\noindent\hspace{10pt}(恒久的施設に帰せられるべき純資産に対応する負債の利子の必要経費不算入)
\begin{description}
\item[第二百九十二条の三]法第百六十五条の三第一項(恒久的施設に帰せられるべき純資産に対応する負債の利子の必要経費不算入)に規定する恒久的施設に係る純資産の額として政令で定めるところにより計算した金額は、第一号に掲げる金額から第二号に掲げる金額を控除した残額とする。
\begin{description}
\item[一]当該非居住者のその年の恒久的施設に係る資産の帳簿価額の平均的な残高として合理的な方法により計算した金額
\item[二]当該非居住者のその年の恒久的施設に係る負債の帳簿価額の平均的な残高として合理的な方法により計算した金額
\end{description}
\item[\rensuji{2}]法第百六十五条の三第一項に規定する恒久的施設に帰せられるべき金額として政令で定めるところにより計算した金額(以下この条において「恒久的施設帰属資本相当額」という。)は、次に掲げるいずれかの方法により計算した金額とする。
\begin{description}
\item[一]資本配賦法(非居住者のイに掲げる金額からロに掲げる金額を控除した残額に、ハに掲げる金額のニに掲げる金額に対する割合を乗じて計算した金額をもつて恒久的施設帰属資本相当額とする方法をいう。)
\begin{description}
\item[イ]当該非居住者のその年の総資産の帳簿価額の平均的な残高として合理的な方法により計算した金額
\item[ロ]当該非居住者のその年の総負債の帳簿価額の平均的な残高として合理的な方法により計算した金額
\item[ハ]当該非居住者のその年十二月三十一日(その者がその年の中途において死亡した場合には、その死亡の時。以下第四項までにおいて同じ。)における恒久的施設に帰せられる資産の額について、取引の相手方の契約不履行その他の財務省令で定める理由により発生し得る危険(以下この項及び第四項において「発生し得る危険」という。)を勘案して計算した金額
\item[ニ]当該非居住者のその年十二月三十一日における総資産の額について、発生し得る危険を勘案して計算した金額
\end{description}
\item[二]同業個人比準法(非居住者のその年十二月三十一日における恒久的施設に帰せられる資産の額について発生し得る危険を勘案して計算した金額に、イに掲げる金額のロに掲げる金額に対する割合を乗じて計算した金額をもつて恒久的施設帰属資本相当額とする方法をいう。)
\begin{description}
\item[イ]比較対象者(当該非居住者の恒久的施設を通じて行う主たる事業と同種の事業を国内において行う個人(当該個人が非居住者である場合には、恒久的施設を通じて当該同種の事業を行うものに限る。)でその同種の事業に係る事業規模その他の状況が類似するものをいう。以下この号及び次項第二号において同じ。)のその年の前年以前三年内の各年のうちいずれかの年(当該比較対象者の純資産の額の総資産の額に対する割合が当該同種の事業を行う個人の当該割合に比して著しく低い場合として財務省令で定める場合に該当する年を除く。以下この号及び同項第二号において「比較対象年」という。)の十二月三十一日において貸借対照表に計上されている当該比較対象者の純資産の額(当該比較対象者が非居住者である場合には、当該比較対象者である非居住者の恒久的施設に係る純資産の額)
\item[ロ]比較対象者の比較対象年の十二月三十一日における総資産の額(当該比較対象者が非居住者である場合には、当該比較対象者である非居住者の恒久的施設に係る資産の額)について、発生し得る危険を勘案して計算した金額
\end{description}
\end{description}
\item[\rensuji{3}]前項各号に規定する非居住者は、同項の規定にかかわらず、同項第一号に定める方法は第一号に掲げる方法とし、同項第二号に定める方法は第二号に掲げる方法とすることができる。
\begin{description}
\item[一]資本配賦簡便法(前項第一号イに掲げる金額から同号ロに掲げる金額を控除した残額に、イに掲げる金額のロに掲げる金額に対する割合を乗じて計算する方法をいう。)
\begin{description}
\item[イ]当該非居住者のその年十二月三十一日における恒久的施設に帰せられる資産の帳簿価額
\item[ロ]当該非居住者のその年十二月三十一日において貸借対照表に計上されている総資産の帳簿価額
\end{description}
\item[二]簿価資産資本比率比準法(当該非居住者のその年の恒久的施設に帰せられる資産の帳簿価額の平均的な残高として合理的な方法により計算した金額に、イに掲げる金額のロに掲げる金額に対する割合を乗じて計算する方法をいう。)
\begin{description}
\item[イ]比較対象者の比較対象年の十二月三十一日において貸借対照表に計上されている純資産の額(当該比較対象者が非居住者である場合には、当該比較対象者である非居住者の恒久的施設に係る純資産の額)
\item[ロ]比較対象者の比較対象年の十二月三十一日において貸借対照表に計上されている総資産の額(当該比較対象者が非居住者である場合には、当該比較対象者である非居住者の恒久的施設に係る資産の額)
\end{description}
\end{description}
\item[\rensuji{4}]第二項第一号ハ若しくはニに掲げる金額又は同項第二号に規定する非居住者のその年十二月三十一日における恒久的施設に帰せられる資産の額について発生し得る危険を勘案して計算した金額(以下この項及び次項において「危険勘案資産額」という。)に関し、非居住者の行う事業の特性、規模その他の事情により、その年分以後の各年分の確定申告期限までに当該危険勘案資産額を計算することが困難な常況にあると認められる場合には、その年七月一日から十二月三十一日までの間の一定の日における第二項第一号ハ若しくは同項第二号に規定する恒久的施設に帰せられる資産の額又は同項第一号ニに規定する総資産の額について発生し得る危険を勘案して計算した金額をもつて当該危険勘案資産額とすることができる。
\item[\rensuji{5}]前項の規定は、同項の規定の適用を受けようとする最初の年の翌年三月十五日までに、納税地の所轄税務署長に対し、同項に規定する確定申告期限までに危険勘案資産額を計算することが困難である理由、同項に規定する一定の日その他の財務省令で定める事項を記載した届出書を提出した場合に限り、適用する。
\item[\rensuji{6}]その年の前年分の恒久的施設帰属資本相当額を資本配賦法等(第二項第一号又は第三項第一号に掲げる方法をいう。以下この項において同じ。)により計算した非居住者がその年分の恒久的施設帰属資本相当額を計算する場合には、次に掲げる場合に該当することにより資本配賦法等により計算することができない場合又は当該非居住者の恒久的施設を通じて行う事業の種類の変更その他これに類する事情がある場合に限り同業個人比準法等(第二項第二号又は第三項第二号に掲げる方法をいう。以下この項において同じ。)により計算することができるものとし、その年の前年分の恒久的施設帰属資本相当額を同業個人比準法等により計算した非居住者がその年分の恒久的施設帰属資本相当額を計算する場合には、当該非居住者の恒久的施設を通じて行う事業の種類の変更その他これに類する事情がある場合に限り資本配賦法等により計算することができるものとする。
\begin{description}
\item[一]第二項第一号に規定する非居住者の同号イに掲げる金額から同号ロに掲げる金額を控除する場合に控除しきれない金額がある場合
\item[二]当該非居住者の純資産の額の総資産の額に対する割合が当該非居住者の恒久的施設を通じて行う主たる事業と同種の事業を行う個人の当該割合に比して著しく低いものとして財務省令で定める場合
\end{description}
\item[\rensuji{7}]法第百六十五条の三第一項に規定する利子に準ずるものとして政令で定めるものは、手形の割引料その他経済的な性質が利子に準ずるものとする。
\item[\rensuji{8}]法第百六十五条の三第一項に規定する政令で定める金額は、次に掲げる金額の合計額とする。
\begin{description}
\item[一]恒久的施設を通じて行う事業に係る負債の利子の額(次号及び第三号に掲げる金額を除く。)
\item[二]法第百六十一条第一項第一号(国内源泉所得)に規定する内部取引において非居住者の恒久的施設から当該非居住者の同号に規定する事業場等に対して支払う利子に該当することとなるものの金額
\item[三]法第百六十五条第二項第二号(総合課税に係る所得税の課税標準、税額等の計算)に規定する恒久的施設を通じて行う事業に係るものとして政令で定めるところにより配分した金額に含まれる負債の利子の額
\end{description}
\item[\rensuji{9}]法第百六十五条の三第一項に規定するその満たない金額に対応する部分の金額として政令で定めるところにより計算した金額は、非居住者のその年の同項に規定する政令で定める金額に、当該非居住者のその年の恒久的施設帰属資本相当額から第一号に掲げる金額を控除した残額(当該残額が第二号に掲げる金額を超える場合には、同号に掲げる金額)の第二号に掲げる金額に対する割合を乗じて計算した金額とする。
\begin{description}
\item[一]当該非居住者のその年の恒久的施設に係る法第百六十五条の三第一項に規定する純資産の額として政令で定めるところにより計算した金額
\item[二]当該非居住者のその年の恒久的施設に帰せられる負債(法第百六十五条の三第一項に規定する利子の支払の基因となるものに限る。)の帳簿価額の平均的な残高として合理的な方法により計算した金額
\end{description}
\item[\rensuji{10}]第一項及び第二項第一号の帳簿価額は、当該非居住者がその会計帳簿に記載した資産又は負債の金額によるものとする。
\end{description}
\noindent\hspace{10pt}(特定の内部取引に係る恒久的施設帰属所得に係る所得の金額の計算)
\begin{description}
\item[第二百九十二条の四]法第百六十五条の五の二第一項(特定の内部取引に係る恒久的施設帰属所得に係る所得の金額の計算)に規定する政令で定める国内源泉所得は、第二百八十一条第一項第八号(国内にある資産の譲渡により生ずる所得)に掲げる所得とする。
\item[\rensuji{2}]法第百六十五条の五の二第一項に規定する政令で定める金額は、非居住者の恒久的施設と事業場等(同項に規定する事業場等をいう。次項において同じ。)との間の内部取引(同条第一項に規定する内部取引をいう。以下この条において同じ。)が次の各号に掲げる内部取引のいずれに該当するかに応じ、当該各号に定める金額とする。
\begin{description}
\item[一]恒久的施設による資産(法第百六十五条の五の二第一項に規定する資産に限る。以下この条において同じ。)の取得に相当する内部取引
\item[二]恒久的施設による資産の譲渡に相当する内部取引
\end{description}
\item[\rensuji{3}]法第百六十五条の五の二第一項の規定の適用がある場合の非居住者の恒久的施設と事業場等との間の内部取引(当該恒久的施設による資産の取得に相当する内部取引に限る。以下この項において同じ。)に係る当該資産の当該恒久的施設における取得価額は、前項第一号に定める金額(当該内部取引による取得のために要した費用がある場合には、その費用の額を加算した金額)とする。
\end{description}
\noindent\hspace{10pt}(その他の国内源泉所得についての総合課税に係る所得税の課税標準等の計算)
\begin{description}
\item[第二百九十二条の五]非居住者の法第百六十五条第一項(総合課税に係る所得税の課税標準、税額等の計算)に規定する総合課税に係る所得税(法第百六十四条第一項第一号ロ及び第二号(非居住者に対する課税の方法)に掲げる国内源泉所得(次条において「その他の国内源泉所得」という。)に係る部分に限る。)の課税標準及び税額につき、法第百六十五条第一項に規定する法の規定に準じて計算する場合には、第二百九十二条(恒久的施設帰属所得についての総合課税に係る所得税の課税標準等の計算)の規定の例による。
\end{description}
\noindent\hspace{10pt}(恒久的施設を有する非居住者の総合課税に係る所得税の課税標準の計算)
\begin{description}
\item[第二百九十二条の六]恒久的施設を有する非居住者が恒久的施設帰属所得(第二百九十二条第一項(恒久的施設帰属所得についての総合課税に係る所得税の課税標準等の計算)に規定する恒久的施設帰属所得をいう。以下この条において同じ。)及びその他の国内源泉所得を有する場合における当該非居住者の法第百六十五条第一項(総合課税に係る所得税の課税標準、税額等の計算)に規定する総合課税に係る所得税の課税標準については、恒久的施設帰属所得に係る所得及びその他の国内源泉所得に係る所得を、同項の規定により法第二編第二章第二節(各種所得の金額の計算)の規定に準じてそれぞれ各種所得に区分し、その各種所得ごとに計算した所得の金額(その区分した各種所得のうちに、同種の各種所得で恒久的施設帰属所得に係るものとその他の国内源泉所得に係るものとがある場合には、それぞれの各種所得に係る所得の金額の合計額)を基礎として、同項の規定により同章第一節及び第三節(課税標準、損益通算及び損失の繰越控除)の規定に準じて、総所得金額、退職所得金額及び山林所得金額を計算するものとする。
\end{description}
\noindent\hspace{10pt}(非居住者に係る分配時調整外国税相当額)
\begin{description}
\item[第二百九十二条の六の二]法第百六十五条の五の三第一項(非居住者に係る分配時調整外国税相当額の控除)に規定する恒久的施設を有する非居住者が支払を受ける収益の分配に対応する部分の金額として政令で定める金額は、当該非居住者が支払を受ける法第百七十六条第三項(信託財産に係る利子等の課税の特例)に規定する集団投資信託の収益の分配(法第百六十四条第一項第一号イ(非居住者に対する課税の方法)に掲げる国内源泉所得に該当するものに限る。)に係る次に掲げる金額の合計額とする。
\begin{description}
\item[一]法第百七十六条第三項の規定により当該集団投資信託の収益の分配に係る所得税の額から控除された外国所得税(第三百条第一項(信託財産に係る利子等の課税の特例)に規定する外国所得税をいう。次号において同じ。)の額に、当該収益の分配(法第百八十一条(源泉徴収義務)又は第二百十二条(源泉徴収義務)の規定により所得税を徴収されるべきこととなる部分に限る。以下この号において同じ。)の額の総額のうちに当該非居住者が支払を受ける収益の分配の額の占める割合を乗じて計算した金額(当該金額がその支払を受ける収益の分配につき法第二百十二条の規定により徴収された又は徴収されるべき所得税の額を超える場合には、当該所得税の額)
\item[二]法第百八十条の二第三項(信託財産に係る利子等の課税の特例)の規定により当該集団投資信託の収益の分配に係る所得税の額から控除された外国所得税の額に、当該収益の分配(法第百八十一条又は第二百十二条の規定により所得税を徴収されるべきこととなる部分に限る。以下この号において同じ。)の額の総額のうちに当該非居住者が支払を受ける収益の分配の額の占める割合を乗じて計算した金額(当該金額がその支払を受ける収益の分配につき法第二百十二条の規定により徴収された又は徴収されるべき所得税の額を超える場合には、当該所得税の額)
\end{description}
\item[\rensuji{2}]法第百六十五条の五の三第一項に規定する所得税の額に相当する金額として政令で定める金額は、同項の非居住者のその年分の法第百六十四条第一項第一号イに掲げる国内源泉所得に係る所得につき法第百六十五条第一項(総合課税に係る所得税の課税標準、税額等の計算)の規定により法第二編第一章から第四章まで(居住者に係る所得税の課税標準、税額等の計算)の規定に準じて計算した所得税の額(法第百六十五条の五の三及び第百六十五条の六(非居住者に係る外国税額の控除)の規定を適用しないで計算した場合の所得税の額とし、附帯税の額を除く。)とする。
\end{description}
\noindent\hspace{10pt}(国外所得金額)
\begin{description}
\item[第二百九十二条の七]法第百六十五条の六第一項(非居住者に係る外国税額の控除)に規定する政令で定める金額は、法第百六十四条第一項第一号イ(非居住者に対する課税の方法)に掲げる国内源泉所得(次項において「恒久的施設帰属所得」という。)に係る所得の金額のうち国外源泉所得(法第百六十五条の六第一項に規定する国外源泉所得をいう。次項において同じ。)に係る所得の金額とする。
\item[\rensuji{2}]前項の規定を適用する場合において、非居住者のその年分の恒久的施設帰属所得につき法第百六十五条第一項(総合課税に係る所得税の課税標準、税額等の計算)の規定により法第二編第一章及び第二章(居住者に係る所得税の課税標準の計算等)の規定に準じて計算した不動産所得の金額、事業所得の金額又は雑所得の金額(事業所得の金額及び雑所得の金額のうち山林の伐採又は譲渡に係るものを除く。)の計算上必要経費に算入された金額のうちに法第三十七条第一項(必要経費)に規定する販売費、一般管理費その他これらの所得を生ずべき業務について生じた費用で国外源泉所得に係る所得を生ずべき業務とそれ以外の恒久的施設帰属所得に係る所得を生ずべき業務の双方に関連して生じたものの額(以下この項及び次項において「共通費用の額」という。)があるときは、当該共通費用の額は、これらの業務に係る収入金額、資産の価額、使用人の数その他の基準のうちこれらの業務の内容及び費用の性質に照らして合理的と認められる基準により国外源泉所得に係る所得とそれ以外の恒久的施設帰属所得に係る所得の金額の計算上の必要経費として配分するものとする。
\item[\rensuji{3}]前項の規定による共通費用の額の配分を行つた非居住者は、当該配分の計算の基礎となる事項を記載した書類その他の財務省令で定める書類を作成しなければならない。
\item[\rensuji{4}]法第百六十五条の六第一項から第三項までの規定の適用を受ける非居住者は、確定申告書、修正申告書又は更正請求書にその年分の同条第一項に規定する国外所得金額の計算に関する明細を記載した書類を添付しなければならない。
\end{description}
\noindent\hspace{10pt}(控除限度額の計算)
\begin{description}
\item[第二百九十二条の八]法第百六十五条の六第一項(非居住者に係る外国税額の控除)に規定する政令で定めるところにより計算した金額は、同項の非居住者のその年分の法第百六十四条第一項第一号イ(非居住者に対する課税の方法)に掲げる国内源泉所得に係る所得につき法第百六十五条第一項(総合課税に係る所得税の課税標準、税額等の計算)の規定により法第二編第一章から第四章まで(居住者に係る所得税の課税標準、税額等の計算)の規定に準じて計算した所得税の額(法第百六十五条の六の規定を適用しないで計算した場合の所得税の額とし、附帯税の額を除く。)に、その年分の恒久的施設帰属所得金額のうちにその年分の調整国外所得金額の占める割合を乗じて計算した金額とする。
\item[\rensuji{2}]前項に規定するその年分の恒久的施設帰属所得金額とは、法第百六十五条第一項の規定により準じて計算する法第七十条第一項若しくは第二項(純損失の繰越控除)又は第七十一条(雑損失の繰越控除)の規定を適用しないで計算した場合のその年分の法第百六十四条第一項第一号イに掲げる国内源泉所得に係る所得の金額(次項において「その年分の恒久的施設帰属所得金額」という。)をいう。
\item[\rensuji{3}]第一項に規定するその年分の調整国外所得金額とは、法第百六十五条第一項の規定により準じて計算する法第七十条第一項若しくは第二項又は第七十一条の規定を適用しないで計算した場合のその年分の法第百六十五条の六第一項に規定する国外所得金額をいう。
\end{description}
\noindent\hspace{10pt}(外国税額控除の対象とならない外国所得税の額)
\begin{description}
\item[第二百九十二条の九]第二百二十二条の二第一項及び第二項(外国税額控除の対象とならない外国所得税の額)の規定は、法第百六十五条の六第一項(非居住者に係る外国税額の控除)に規定する政令で定める取引について準用する。
\item[\rensuji{2}]法第百六十五条の六第一項に規定する政令で定める外国所得税の額は、次に掲げる外国所得税の額とする。
\begin{description}
\item[一]非居住者が住所を有し、一定の期間を超えて居所を有し、又は国籍その他これに類するものを有することにより当該住所、居所又は国籍その他これに類するものを有する国又は地域(以下この項において「居住地国」という。)において課される外国所得税の額(当該非居住者が支払を受けるべき利子、配当その他これらに類するものの額を課税標準として源泉徴収の方法に類する方法により課される外国所得税の額で、当該居住地国の法令の規定又は法第二条第一項第八号の四ただし書(定義)に規定する条約(次号において「租税条約」という。)の規定により、当該居住地国において当該非居住者に対して課される当該外国所得税以外の外国所得税の額から控除しないこととされるものを除く。)
\item[二]非居住者の居住地国以外の国又は地域において課される外国所得税の額のうち、当該外国所得税の課税標準となる所得について我が国と当該国若しくは地域との間の租税条約の規定が適用されるとしたならば、当該租税条約における当該所得に係る外国所得税の軽減若しくは免除に関する規定の適用により当該国若しくは地域において課することができることとされる額を超える部分に相当する金額若しくは免除することとされる額に相当する金額又は当該外国所得税の課税標準となる所得を居住者の所得とした場合にその所得に対して当該外国所得税が課されるとしたならば、外国(外国居住者等の所得に対する相互主義による所得税等の非課税等に関する法律第二条第三号(定義)に規定する外国をいい、同法第五条各号(相互主義)のいずれかに該当しない場合における当該外国を除く。以下この号において同じ。)において、同条第一号に規定する所得税等の非課税等に関する規定により当該外国に係る同法第二条第三号に規定する外国居住者等の同法第五条第一号に規定する対象国内源泉所得に対して所得税を軽減し、若しくは課さないこととされる条件と同等の条件により軽減することとされる部分に相当する金額若しくは免除することとされる額に相当する金額
\end{description}
\end{description}
\noindent\hspace{10pt}(地方税控除限度額)
\begin{description}
\item[第二百九十二条の十]法第百六十五条の六第二項(非居住者に係る外国税額の控除)に規定する地方税控除限度額として政令で定める金額は、地方税法施行令第七条の十九第三項(道府県民税からの外国所得税額の控除)の規定による限度額と同令第四十八条の九の二第四項(市町村民税からの外国所得税額の控除)の規定による限度額との合計額とする。
\end{description}
\noindent\hspace{10pt}(繰越控除限度額等)
\begin{description}
\item[第二百九十二条の十一]法第百六十五条の六第二項(非居住者に係る外国税額の控除)に規定するその年に繰り越される部分として政令で定める金額は、その年の前年以前三年内の各年(次項及び次条第一項において「前三年以内の各年」という。)の国税の控除余裕額又は地方税の控除余裕額を、最も古い年のものから順に、かつ、同一年のものについては国税の控除余裕額及び地方税の控除余裕額の順に、その年の控除限度超過額に充てるものとした場合に当該控除限度超過額に充てられることとなる当該国税の控除余裕額の合計額に相当する金額とする。
\item[\rensuji{2}]前三年以内の各年のうちいずれかの年において納付することとなつた法第百六十五条の六第一項に規定する控除対象外国所得税の額(以下この条及び第二百九十二条の十四(外国所得税が減額された場合の特例)において「控除対象外国所得税の額」という。)をその納付することとなつた年の法第百六十四条第一項第一号イ(非居住者に対する課税の方法)に掲げる国内源泉所得につき法第百六十五条第一項(総合課税に係る所得税の課税標準、税額等の計算)の規定により法第二編第一章及び第二章(居住者に係る所得税の課税標準の計算等)の規定に準じて計算する不動産所得の金額、事業所得の金額、山林所得の金額若しくは雑所得の金額の計算上必要経費に算入し、又は一時所得の金額の計算上支出した金額に算入した場合には、当該年以前の各年の国税の控除余裕額及び地方税の控除余裕額は、前項に規定する国税の控除余裕額及び地方税の控除余裕額に含まれないものとして、同項の規定を適用する。
\item[\rensuji{3}]法第百六十五条の六第二項の規定の適用を受けることができる年後の各年に係る第一項及び次条第一項の規定の適用については、第一項の規定により当該適用を受けることができる年の控除限度超過額に充てられることとなる国税の控除余裕額及び地方税の控除余裕額並びにこれらの金額の合計額に相当する金額の当該控除限度超過額は、ないものとみなす。
\item[\rensuji{4}]前三項に規定する国税の控除余裕額とは、その年において納付することとなる控除対象外国所得税の額がその年の国税の控除限度額(法第百六十五条の六第一項に規定する控除限度額をいう。以下この条において同じ。)に満たない場合における当該国税の控除限度額から当該控除対象外国所得税の額を控除した金額に相当する金額をいう。
\item[\rensuji{5}]第一項から第三項までに規定する地方税の控除余裕額とは、次の各号に掲げる場合の区分に応じ当該各号に定める金額をいう。
\begin{description}
\item[一]その年において納付することとなる控除対象外国所得税の額がその年の国税の控除限度額を超えない場合
\item[二]その年において納付することとなる控除対象外国所得税の額がその年の国税の控除限度額を超え、かつ、その超える部分の金額がその年の地方税の控除限度額に満たない場合
\end{description}
\item[\rensuji{6}]第一項及び第三項に規定する控除限度超過額とは、その年において納付することとなる控除対象外国所得税の額がその年の国税の控除限度額と地方税の控除限度額との合計額を超える場合におけるその超える部分の金額に相当する金額をいう。
\end{description}
\noindent\hspace{10pt}(繰越控除対象外国所得税額等)
\begin{description}
\item[第二百九十二条の十二]法第百六十五条の六第三項(非居住者に係る外国税額の控除)に規定するその年に繰り越される部分として政令で定める金額は、前三年以内の各年の控除限度超過額(前条第六項に規定する控除限度超過額をいう。以下この条において同じ。)を最も古い年のものから順次その年の国税の控除余裕額(前条第四項に規定する控除余裕額をいう。以下この条において同じ。)に充てるものとした場合に当該国税の控除余裕額に充てられることとなる当該控除限度超過額の合計額に相当する金額とする。
\item[\rensuji{2}]前条第二項の規定は、前項の場合について準用する。
\item[\rensuji{3}]法第百六十五条の六第三項の規定の適用を受けることができる年後の各年に係る第一項及び前条第一項の規定の適用については、第一項の規定により当該適用を受けることができる年の国税の控除余裕額に充てられることとなる控除限度超過額及びこれに相当する金額の当該国税の控除余裕額は、ないものとみなす。
\item[\rensuji{4}]地方税法施行令第七条の十九第二項(道府県民税からの外国所得税額の控除)の規定の適用を受けることができる年(同令第四十八条の九の二第二項(市町村民税からの外国所得税額の控除)の規定の適用をも受けることができる年を除く。)又は同令第四十八条の九の二第二項の規定の適用を受けることができる年後の各年に係る第一項及び前条第一項の規定の適用については、それぞれ、同令第七条の十九第二項又は第四十八条の九の二第二項の規定により当該適用を受けることができる年において課された外国の所得税等の額とみなされる金額に相当する控除限度超過額(当該控除限度超過額のうちに第一項の規定により当該適用を受けることができる年の国税の控除余裕額に充てられることとなるものがある場合には、当該充てられることとなる部分を除く。)及びこれに相当する金額の当該適用を受けることができる年の前条第五項に規定する地方税の控除余裕額は、ないものとみなす。
\end{description}
\noindent\hspace{10pt}(外国税額の控除に係る国外源泉所得に関する規定の準用)
\begin{description}
\item[第二百九十二条の十三]第二百二十五条の三から第二百二十五条の七まで(国外にある資産の運用又は保有により生ずる所得等)、第二百二十五条の九から第二百二十五条の十一まで(事業の広告宣伝のための賞金等)及び第二百二十五条の十四(国外に源泉がある所得)の規定は、法第百六十五条の六第四項第一号(非居住者に係る外国税額の控除)に規定する国外にある資産の運用又は保有により生ずる所得、同項第二号に規定する国外にある資産の譲渡により生ずる所得として政令で定めるもの、同項第三号に規定する政令で定める事業、同項第七号に規定する債券の買戻又は売戻条件付売買取引として政令で定めるもの、同号に規定する差益として政令で定めるもの、同項第八号ハに規定する政令で定める用具、同項第九号に規定する政令で定める賞金、同項第十号に規定する政令で定める契約、同項第十二号に規定する政令で定める契約及び同項第十三号に規定する政令で定める所得について準用する。
\end{description}
\noindent\hspace{10pt}(外国所得税が減額された場合の特例)
\begin{description}
\item[第二百九十二条の十四]非居住者が納付することとなつた外国所得税の額につき法第百六十五条の六第一項から第三項まで(非居住者に係る外国税額の控除)の規定の適用を受けた年の翌年以後七年内の各年において当該外国所得税の額が減額された場合には、当該非居住者のその減額されることとなつた日の属する年(以下この条において「減額に係る年」という。)については、当該減額に係る年において当該非居住者が納付することとなる控除対象外国所得税の額(第三項において「納付控除対象外国所得税額」という。)から減額控除対象外国所得税額に相当する金額を控除し、その控除後の金額につき法第百六十五条の六第一項から第三項までの規定を適用する。
\item[\rensuji{2}]前項に規定する減額控除対象外国所得税額とは、非居住者の減額に係る年において外国所得税の額の減額がされた金額のうち、第一号に掲げる金額から第二号に掲げる金額を控除した残額に相当する金額をいう。
\begin{description}
\item[一]当該外国所得税の額のうち非居住者の法第百六十五条の六第一項から第三項までの規定の適用を受けた年において控除対象外国所得税の額とされた部分の金額
\item[二]当該減額がされた後の当該外国所得税の額につき当該非居住者の法第百六十五条の六第一項から第三項までの規定の適用を受けた年において同条第一項の規定を適用したならば控除対象外国所得税の額とされる部分の金額
\end{description}
\item[\rensuji{3}]第一項の場合において、減額に係る年の納付控除対象外国所得税額がないとき、又は当該納付控除対象外国所得税額が前項に規定する減額控除対象外国所得税額(以下この項において「減額控除対象外国所得税額」という。)に満たないときは、減額に係る年の前年以前三年内の各年の第二百九十二条の十一第六項(繰越控除限度額等)に規定する控除限度超過額(同条第三項又は第二百九十二条の十二第三項若しくは第四項(繰越控除対象外国所得税等)の規定により減額に係る年の前年以前の各年においてないものとみなされた部分の金額を除く。以下この項において「控除限度超過額」という。)から、それぞれ当該減額控除対象外国所得税額の全額又は当該減額控除対象外国所得税額のうち当該納付控除対象外国所得税額を超える部分の金額に相当する金額を控除し、その控除後の金額につき法第百六十五条の六第三項の規定を適用する。
\end{description}
\subsubsection*{第二款 申告、納付及び還付}
\addcontentsline{toc}{subsubsection}{第二款 申告、納付及び還付}
\noindent\hspace{10pt}(申告、納付及び還付)
\begin{description}
\item[第二百九十三条]法第百六十六条(申告、納付及び還付)において準用する法第二編第五章及び第六章(居住者に係る申告、納付及び還付)の規定の適用に係る事項については、前編第五章及び第六章(居住者に係る申告、納付及び還付)の規定を準用する。
\end{description}
\subsubsection*{第三款 更正の請求の特例}
\addcontentsline{toc}{subsubsection}{第三款 更正の請求の特例}
\noindent\hspace{10pt}(更正の請求の特例)
\begin{description}
\item[第二百九十四条]法第百六十七条(更正の請求の特例)において準用する法第二編第七章(更正の請求の特例)の規定の適用に係る事項については、前編第七章(更正の請求の特例)の規定を準用する。
\end{description}
\subsubsection*{第四款 更正及び決定}
\addcontentsline{toc}{subsubsection}{第四款 更正及び決定}
\noindent\hspace{10pt}(更正及び決定)
\begin{description}
\item[第二百九十五条]法第百六十八条(更正及び決定)において準用する法第二編第八章(更正及び決定)の規定の適用に係る事項については、前編第八章(更正及び決定)の規定を準用する。
\end{description}
\subsection*{第二節 非居住者に対する所得税の分離課税}
\addcontentsline{toc}{subsection}{第二節 非居住者に対する所得税の分離課税}
\noindent\hspace{10pt}(生命保険契約等に基づく年金等に係る課税標準)
\begin{description}
\item[第二百九十六条]法第百六十九条第五号(分離課税に係る所得税の課税標準)に規定する政令で定めるところにより計算した金額は、次の各号に掲げる場合の区分に応じ当該各号に定める金額とする。
\begin{description}
\item[一]法第百六十九条第五号に規定する契約が第二百八十七条(年金に係る契約の範囲)に規定する生命保険契約等であつて年金のみを支払う内容のものである場合
\item[二]法第百六十九条第五号に規定する契約が第二百八十七条に規定する生命保険契約等であつて年金のほか一時金を支払う内容のものである場合
\begin{description}
\item[イ]法第百六十九条第五号に規定する支払を受けるべき金額が年金の金額であるとき。
\item[ロ]法第百六十九条第五号に規定する支払を受けるべき金額が一時金の金額であるとき。
\end{description}
\item[三]法第百六十九条第五号に規定する契約が第二百八十七条に規定する損害保険契約等である場合
\end{description}
\end{description}
\noindent\hspace{10pt}(退職所得の選択課税による還付)
\begin{description}
\item[第二百九十七条]法第百七十三条第一項(退職所得の選択課税による還付)の規定による申告書を提出する場合において、同項第二号に掲げる所得税の額のうち源泉徴収をされたものがあるときは、当該申告書を提出する者は、当該申告書に、その源泉徴収をされた事実の説明となるべき財務省令で定める事項を記載した明細書を添附しなければならない。
\item[\rensuji{2}]前項の申告書を提出した者は、当該申告書の記載に係る同項に規定する所得税の額でその提出の時においてまだ納付されていなかつたものの納付があつた場合には、遅滞なく、その納付の日、その納付された所得税の額その他必要な事項を記載した届出書を納税地の所轄税務署長に提出しなければならない。
\item[\rensuji{3}]税務署長は、第一項の申告書の提出があつた場合には、当該申告書の記載に係る法第百七十三条第一項第三号に掲げる金額が過大であると認められる事由がある場合を除き、遅滞なく、同条第二項の規定による還付又は充当の手続をしなければならない。
\end{description}
\section*{第三章 法人の納税義務}
\addcontentsline{toc}{section}{第三章 法人の納税義務}
\subsection*{第一節 内国法人の納税義務}
\addcontentsline{toc}{subsection}{第一節 内国法人の納税義務}
\noindent\hspace{10pt}(内国法人に係る所得税の課税標準)
\begin{description}
\item[第二百九十八条]法第百七十四条(内国法人に係る所得税の課税標準)に規定する政令で定める金額は、同条第十号に掲げる賞金の額の百分の二十に相当する金額と六十万円との合計額とする。
\item[\rensuji{2}]法第百七十四条第四号に規定する払い込むべき掛金の額として政令で定めるものは、同号に規定する契約に基づき払い込むべき掛金の額(当該契約に基づき掛金を払い込むべきこととされている期間の中途で当該契約に基づく給付金の給付を受けた場合には、当該掛金の額から当該契約に基づき銀行に対して支払うべき利子に相当する金額を控除した金額)とする。
\item[\rensuji{3}]法第百七十四条第五号に規定する政令で定める契約は、抵当証券法(昭和六年法律第十五号)第一条第一項(証券の交付)に規定する抵当証券の販売(販売の代理又は媒介を含む。)を業として行う者と当該抵当証券の購入をした者との間で締結された当該抵当証券に記載された債権の元本及び利息の弁済の受領並びにその支払に関する事項を含む契約とする。
\item[\rensuji{4}]法第百七十四条第七号に規定する政令で定める差益は、次の各号に掲げる預貯金の区分に応じ当該各号に定める差益とする。
\begin{description}
\item[一]外国通貨で表示された預貯金でその元本及び利子をあらかじめ約定した率により本邦通貨に換算して支払うこととされているもの
\item[二]外国通貨で表示された預貯金でその元本及び利子をあらかじめ約定した率により当該外国通貨以外の外国通貨(以下この号において「他の外国通貨」という。)に換算して支払うこととされているもの
\end{description}
\item[\rensuji{5}]法第百七十四条第八号に規定する政令で定める支払方法は、同号に規定する保険契約若しくは旧簡易生命保険契約(第三十条第一号(非課税とされる保険金、損害賠償金等)に規定する旧簡易生命保険契約をいう。次項及び第七項において同じ。)又はこれらに類する共済に係る契約に係る法第百七十四条第八号に規定する保険期間等の初日から一年以内にこれらの契約に係る保険料又は掛金の総額の二分の一以上の額に相当する保険料又は掛金を支払う方法及び同日から二年以内に当該保険料又は掛金の総額の四分の三以上の額に相当する保険料又は掛金を支払う方法(これらの契約において当該保険料又は掛金の全部又は一部を前納することができることとされている場合において、その全部を前納したとき又はその一部をこれらの方法に準じて前納したときを含む。)とする。
\item[\rensuji{6}]法第百七十四条第八号に規定する政令で定める事項は、次の各号に掲げる契約の区分に応じ当該各号に定める事項とする。
\begin{description}
\item[一]生命保険契約(保険業法第二条第三項(定義)に規定する生命保険会社若しくは同条第八項に規定する外国生命保険会社等の締結した保険契約又は同条第十八項に規定する少額短期保険業者の締結したこれに類する保険契約をいう。)若しくは旧簡易生命保険契約又はこれらに類する共済に係る契約
\item[二]損害保険契約(保険業法第二条第四項に規定する損害保険会社若しくは同条第九項に規定する外国損害保険会社等の締結した保険契約又は同条第十八項に規定する少額短期保険業者の締結したこれに類する保険契約をいう。)又はこれに類する共済に係る契約
\end{description}
\item[\rensuji{7}]法第百七十四条第八号に規定する政令で定めるところにより計算した金額は、第一号に掲げる金額から第二号に掲げる金額を控除した金額とする。
\begin{description}
\item[一]法第百七十四条第八号に規定する保険契約若しくは旧簡易生命保険契約又はこれらに類する共済に係る契約に基づく満期保険金、満期返戻金若しくは満期共済金又は解約返戻金(以下この項において「満期保険金等」という。)の金額とこれらの契約に基づき分配を受ける剰余金又は割戻しを受ける割戻金の額で当該満期保険金等とともに又は当該満期保険金等の支払を受けた後に支払を受けるものとの合計額
\item[二]前号の保険契約若しくは旧簡易生命保険契約又はこれらに類する共済に係る契約に係る保険料又は掛金の総額から、これらの契約に基づく満期保険金等の支払の日前にこれらの契約に基づく剰余金の分配若しくは割戻金の割戻しを受け、又はこれらの契約に基づき分配を受ける剰余金若しくは割戻しを受ける割戻金をもつて当該保険料又は掛金の支払に充てた場合における当該剰余金又は割戻金の額を控除した金額
\end{description}
\item[\rensuji{8}]法第百七十四条第九号に規定する政令で定める契約は、第二百八十八条(匿名組合契約に準ずる契約の範囲)に規定する契約とする。
\item[\rensuji{9}]法第百七十四条第十号に規定する政令で定める賞金は、金銭で支払われる賞金とする。
\end{description}
\noindent\hspace{10pt}(内国法人に係る所得税の税率)
\begin{description}
\item[第二百九十九条]法第百七十五条第三号(内国法人に係る所得税の税率)に規定する政令で定める金額は、前条第一項に規定する金額とする。
\end{description}
\noindent\hspace{10pt}(信託財産に係る利子等の課税の特例)
\begin{description}
\item[第三百条]法第百七十六条第三項(信託財産に係る利子等の課税の特例)に規定する外国の法令により課される所得税に相当する税で政令で定めるものは、外国の法令に基づき同項の信託財産につき課される税で、法第二百十二条(源泉徴収義務)の規定による源泉徴収に係る所得税に相当するもの(以下この項、第三項及び第九項において「外国所得税」という。)のうち、当該外国所得税の課せられた収益を分配するとしたならば当該収益の分配につき法第百八十一条(源泉徴収義務)又は第二百十二条の規定により所得税を徴収されるべきこととなるものに対応する部分とする。
\item[\rensuji{2}]法第百七十六条第三項の規定により控除する所得税の額は、内国法人が同項に規定する収益の分配(当該所得税の納付をした日の属する収益の分配の計算期間に対応するものに限るものとし、当該納付に係る信託財産がその受益権を他の証券投資信託の受託者に取得させることを目的とする証券投資信託で財務省令で定めるもの(以下この項及び次項において「受託者取得目的証券投資信託」という。)に係るものである場合には、信託財産を当該受託者取得目的証券投資信託の受益権に対する投資として運用することを目的とする公社債投資信託以外の証券投資信託(次項において「受益権投資目的証券投資信託」という。)の収益の分配とする。)につき法第百八十一条又は第二百十二条の規定により所得税を徴収する際、その徴収して納付すべき所得税の額から控除するものとする。
\item[\rensuji{3}]前項の場合において、法第百七十六条第三項の規定により控除する所得税の額のうちに同項の規定により控除する外国所得税の額があるときは、まず当該外国所得税以外の当該所得税の額を集団投資信託(同項に規定する集団投資信託をいう。以下この条において同じ。)の前項に規定する収益の分配に係る所得税の額から控除し、次に当該外国所得税の額を、当該収益の分配に係る所得税の額に当該収益の分配の計算期間の末日において計算した当該収益の分配に係る集団投資信託の外貨建資産割合(集団投資信託の信託財産(当該集団投資信託が受益権投資目的証券投資信託である場合には、当該受益権投資目的証券投資信託に係る受託者取得目的証券投資信託の信託財産。以下この項において同じ。)において運用する外貨建資産(外国通貨で表示される株式、債券、その他の資産をいう。)の額が当該信託財産の総額のうちに占める割合をいう。)を乗じて計算した金額を限度として、当該集団投資信託の当該収益の分配に係る所得税の額から控除するものとする。
\item[\rensuji{4}]法第百七十六条第三項の規定の適用がある場合における第二百六十四条(各種所得につき源泉徴収をされた所得税等の額から控除する所得税の額)(第二百九十三条(申告、納付及び還付)において準用する場合を含む。)の規定の適用については、第二百六十四条中「の金額」とあるのは、「の金額及び法第百七十六条第三項(信託財産に係る利子等の課税の特例)に規定する集団投資信託の第三百条第二項(信託財産に係る利子等の課税の特例)に規定する収益の分配(法第百七十条(分離課税に係る所得税の税率)の規定の適用を受けた同条の国内源泉所得に該当するもの、租税特別措置法第三条第一項(利子所得の分離課税等)の規定の適用を受けた同項に規定する一般利子等並びに同法第八条の五第一項(確定申告を要しない配当所得等)の規定の適用を受けた同項に規定する利子等及び配当等を除く。以下この条において同じ。)に係る控除外国所得税の額(法第百七十六条第三項の規定により当該集団投資信託の第三百条第二項に規定する収益の分配に係る所得税の額から控除された同条第一項に規定する外国所得税の額に、当該集団投資信託の同条第二項に規定する収益の分配(法第百八十一条(源泉徴収義務)又は第二百十二条の規定により所得税を徴収されるべきこととなる部分に限る。以下この条において同じ。)の額の総額のうちに支払を受けた収益の分配の額の占める割合を乗じて計算した金額(当該金額がその支払を受けた当該集団投資信託の収益の分配につき法第百八十一条又は第二百十二条の規定により徴収された又は徴収されるべき所得税の額を超える場合には、当該所得税の額)をいう。)」とする。
\item[\rensuji{5}]集団投資信託を引き受けた内国法人は、当該集団投資信託の信託財産について法第百七十六条第三項に規定する所得税を課された場合には、財務省令で定めるところにより、当該所得税の額を課されたことを証する書類その他財務省令で定める書類を保存しなければならない。
\item[\rensuji{6}]集団投資信託を引き受けた内国法人(法第二百二十七条(信託の計算書)に規定する信託の受託者及び法第二百二十八条第一項(名義人受領の配当所得等の調書)に規定する利子等又は配当等の支払を受ける者に該当する者(以下第八項までにおいて「準支払者」という。)を含む。)は、個人に対して国内において当該集団投資信託の収益の分配(租税特別措置法第三条第一項(利子所得の分離課税等)の規定の適用を受けた同項に規定する一般利子等を除く。以下この項及び次項において同じ。)の支払をする場合において、その支払の確定した収益の分配に係る通知外国所得税の額があるときは、当該通知外国所得税の額その他の財務省令で定める事項を、その支払の確定した日(無記名の投資信託又は特定受益証券発行信託の受益証券に係る収益の分配に係る通知については、その支払をした日)から一月以内(準支払者が通知する場合には、四十五日以内)に、当該個人に対し、書面により通知しなければならない。
\item[\rensuji{7}]前項に規定する内国法人は、同項の書面を同一の者に対してその年中に支払つた収益の分配の額の合計額で作成する場合には、同項の規定にかかわらず、当該収益の分配に係る通知外国所得税の額その他の財務省令で定める事項を、同項に規定する支払の確定した日の属する年の翌年一月三十一日(準支払者が通知する場合には、同年二月十五日)までに、同項の個人に対し、書面により通知しなければならない。
\item[\rensuji{8}]集団投資信託を引き受けた内国法人(準支払者を含む。)は、法人に対して国内において当該集団投資信託の収益の分配の支払をする場合において、その支払の確定した収益の分配に係る通知外国所得税の額があるときは、当該通知外国所得税の額その他の財務省令で定める事項を、その支払の確定した日(無記名の投資信託又は特定受益証券発行信託の受益証券に係る収益の分配に係る通知については、その支払をした日)から一月以内(準支払者が通知する場合には、四十五日以内)に、当該法人に対し、書面により通知しなければならない。
\item[\rensuji{9}]前三項に規定する通知外国所得税の額とは、法第百七十六条第三項の規定により前三項の集団投資信託の第二項に規定する収益の分配に係る所得税の額から控除された外国所得税の額に、当該集団投資信託の同項に規定する収益の分配(法第百八十一条又は第二百十二条の規定により所得税を徴収されるべきこととなる部分に限る。以下この項において同じ。)の額の総額のうちに前三項の個人又は法人が支払を受けた収益の分配の額の占める割合を乗じて計算した金額(当該金額がその支払を受けた収益の分配につき法第百八十一条又は第二百十二条の規定により徴収された又は徴収されるべき所得税の額を超える場合には、当該所得税の額)をいう。
\item[\rensuji{10}]第六項から第八項までに規定する内国法人は、これらの規定の書面による通知に代えて、これらの規定の個人又は法人の承諾を得て、当該書面に記載すべき事項を電磁的方法(電子情報処理組織を使用する方法その他の情報通信の技術を利用する方法であつて財務省令で定めるものをいう。第十二項及び第十三項において同じ。)により提供することができる。ただし、当該個人又は法人の請求があるときは、当該個人又は法人に対し、当該書面により通知しなければならない。
\item[\rensuji{11}]前項本文の場合において、同項に規定する内国法人は、第六項から第八項までの規定による通知をしたものとみなす。
\item[\rensuji{12}]第十項に規定する内国法人は、同項本文の規定により書面に記載すべき事項を同項の個人又は法人に対し提供しようとするときは、財務省令で定めるところにより、あらかじめ、当該個人又は法人に対し、その用いる電磁的方法の種類及び内容を示し、書面又は電磁的方法による承諾を得なければならない。
\item[\rensuji{13}]前項の規定による承諾を得た同項に規定する内国法人は、同項の個人又は法人から書面又は電磁的方法により第十項本文の規定による電磁的方法による提供を受けない旨の申出があつたときは、当該個人又は法人に対し、同項の書面に記載すべき事項の提供を電磁的方法によつてしてはならない。ただし、当該個人又は法人が再び前項の規定による承諾をした場合は、この限りでない。
\item[\rensuji{14}]第六項から第八項までに規定する収益の分配の支払をするこれらの規定に規定する内国法人並びに当該収益の分配の支払を受けるこれらの規定の個人及び法人については、法第二百二十五条第二項(支払調書及び支払通知書)又は租税特別措置法第八条の四第四項から第七項まで(上場株式等に係る配当所得等の課税の特例)の規定のうち当該収益の分配に係る部分の規定の適用がある場合には、第六項から前項までの規定のうち当該適用を受けた収益の分配に係る部分の規定は、適用しない。
\end{description}
\begin{description}
\item[第三百一条]削除
\end{description}
\begin{description}
\item[第三百二条]削除
\end{description}
\begin{description}
\item[第三百三条]削除
\end{description}
\subsection*{第二節 外国法人の納税義務}
\addcontentsline{toc}{subsection}{第二節 外国法人の納税義務}
\noindent\hspace{10pt}(外国法人に係る所得税の課税標準から除かれる国内源泉所得)
\begin{description}
\item[第三百三条の二]法第百七十八条(外国法人に係る所得税の課税標準)に規定する政令で定める国内源泉所得は、次に掲げる国内源泉所得とする。
\begin{description}
\item[一]映画若しくは演劇の俳優、音楽家その他の芸能人又は職業運動家の役務の提供に係る法第百六十一条第一項第六号(国内源泉所得)に掲げる対価で不特定多数の者から支払われるもの
\item[二]外国法人が有する土地若しくは土地の上に存する権利又は家屋(以下この号において「土地家屋等」という。)に係る法第百六十一条第一項第七号に掲げる対価で、当該土地家屋等を自己又はその親族の居住の用に供するために借り受けた個人から支払われるもの
\end{description}
\end{description}
\noindent\hspace{10pt}(外国法人が課税の特例の適用を受けるための要件)
\begin{description}
\item[第三百四条]法第百八十条第一項(恒久的施設を有する外国法人の受ける国内源泉所得に係る課税の特例)に規定する政令で定める要件は、次に掲げる要件とする。
\begin{description}
\item[一]法人税法第百四十九条第一項若しくは第二項(外国普通法人となつた旨の届出)又は第百五十条第三項若しくは第四項(公益法人等又は人格のない社団等の収益事業の開始等の届出)の規定による届出書を提出していること。
\item[二]会社法第九百三十三条第一項(外国会社の登記)又は民法第三十七条第一項(外国法人の登記)の規定による登記をすべき外国法人にあつては、その登記をしていること(会社法第九百三十三条第一項の規定による登記をしている恒久的施設(法第二条第一項第八号の四イ(定義)に掲げるもの又は同号ただし書に規定する条約において恒久的施設と定められたもので同号イに掲げるものに相当するものに限る。)を有する外国法人にあつては、会社法第九百三十三条第一項第二号に規定する営業所につきその登記をしていること。)。
\item[三]法第百八十条第一項の規定の適用を受けようとする同項に規定する対象国内源泉所得が、法人税に関する法令(法第二条第一項第八号の四ただし書に規定する条約を含む。)の規定により法人税を課される所得のうちに含まれるものであること。
\item[四]偽りその他不正の行為により所得税又は法人税を免れたことがないこと。
\item[五]法第百八十条第一項の規定の適用を受けるために同項の証明書を同項に規定する対象国内源泉所得の支払者に提示する場合において、当該支払者の氏名又は名称及びその住所、事務所、事業所その他当該対象国内源泉所得の支払の場所並びにその提示した年月日を帳簿に記録することが確実であると見込まれること。
\end{description}
\end{description}
\noindent\hspace{10pt}(外国法人が課税の特例の適用を受けるための手続等)
\begin{description}
\item[第三百五条]法第百八十条第一項(恒久的施設を有する外国法人の受ける国内源泉所得に係る課税の特例)の証明書の交付を受けようとする法人は、次に掲げる事項を記載した申請書をその法人税の納税地の所轄税務署長に提出しなければならない。
\begin{description}
\item[一]その法人の名称、本店又は主たる事務所の所在地及び法人番号
\item[二]その法人の法人税法第十七条第一号(外国法人の納税地)に規定する事務所、事業所その他これらに準ずるもの(これらが二以上あるときは、そのうち主たるもの。次条第一項第一号において「納税地にある事務所等」という。)の名称及び所在地並びにその代表者その他の責任者の氏名
\item[三]前条第一号に規定する届出書を提出した年月日及び同条第二号に規定する登記をした年月日(当該登記をすることができない法人については、そのできない事情の詳細)
\item[四]前条第三号に掲げる要件に該当する事情の概要
\item[五]前条第五号の記録を確実に行う旨
\item[六]その法人が恒久的施設を通じて行う事業の内容が前条第一号の規定による届出書を提出した当時の当該事業の内容と異なつている場合には、その現在の事業の概要
\item[七]当該証明書により法第百八十条第一項の規定の適用を受けようとする同項に規定する対象国内源泉所得のうち主たるものの支払者の氏名又は名称、その住所、事務所、事業所その他当該対象国内源泉所得の支払の場所及びその支払の宛先並びに当該対象国内源泉所得の種類及び当該対象国内源泉所得の支払を受ける見込期間
\item[八]当該証明書により法第百八十条第一項の規定の適用を受けようとする国内源泉所得がその法人の同項に規定する対象国内源泉所得に該当する事情
\item[九]その他参考となるべき事項
\end{description}
\item[\rensuji{2}]前項の所轄税務署長は、同項の申請書の提出があつた場合において、当該申請書を提出した法人が前条各号に定める要件を備えていると認めるときは、同項の証明書を交付するものとする。
\item[\rensuji{3}]恒久的施設を有する外国法人から第一項の証明書の提示を受けた法第百八十条第一項に規定する対象国内源泉所得の支払者は、当該外国法人に対する当該対象国内源泉所得の支払に関する帳簿を備え、当該外国法人の名称及び同項の証明書の有効期限を記載しなければならない。
\end{description}
\noindent\hspace{10pt}(外国法人が課税の特例の要件に該当しなくなつた場合の手続等)
\begin{description}
\item[第三百六条]法第百八十条第一項(恒久的施設を有する外国法人の受ける国内源泉所得に係る課税の特例)の証明書の交付を受けている法人は、同条第二項に規定する場合には、次に掲げる事項を記載した届出書に当該証明書を添付し、これをその法人税の納税地の所轄税務署長に提出するとともに、その法人が当該証明書を提示した国内源泉所得の支払者に対しその旨を遅滞なく通知しなければならない。
\begin{description}
\item[一]その法人の納税地にある事務所等の名称及び所在地並びにその代表者その他の責任者の氏名
\item[二]第三百四条各号(外国法人が課税の特例の適用を受けるための要件)に掲げる要件に該当しないこととなり、又は恒久的施設を有しないこととなつた事情の詳細
\item[三]その法人が当該証明書を提示した国内源泉所得の支払者の氏名又は名称及びその住所、事務所、事業所その他当該国内源泉所得の支払の場所
\item[四]その他参考となるべき事項
\end{description}
\item[\rensuji{2}]前項に規定する法人は、同項の証明書に係る前条第一項の申請書に記載した同項第一号又は第二号に掲げる事項に変更があつた場合には、遅滞なく、その旨を記載した届出書を前項の所轄税務署長に提出しなければならない。
\end{description}
\noindent\hspace{10pt}(信託財産に係る利子等の課税の特例)
\begin{description}
\item[第三百六条の二]法第百八十条の二第三項(信託財産に係る利子等の課税の特例)の規定により控除する所得税の額は、外国法人が同項に規定する収益の分配(当該所得税の納付をした日の属する収益の分配の計算期間に対応するものに限るものとし、当該納付に係る信託財産がその受益権を他の証券投資信託の受託者に取得させることを目的とする証券投資信託で第三百条第二項(信託財産に係る利子等の課税の特例)に規定する財務省令で定めるもの(以下この項において「受託者取得目的証券投資信託」という。)に係るものである場合には、信託財産を当該受託者取得目的証券投資信託の受益権に対する投資として運用することを目的とする公社債投資信託以外の証券投資信託の収益の分配とする。)につき法第百八十一条(源泉徴収義務)又は第二百十二条(源泉徴収義務)の規定により所得税を徴収する際、その徴収して納付すべき所得税の額から控除するものとする。
\item[\rensuji{2}]第三百条第三項及び第四項の規定は、法第百八十条の二第三項の規定により所得税の額を控除する場合について準用する。この場合において、第三百条第四項中「第百七十六条第三項(」とあるのは「第百八十条の二第三項(」と、「第三百条第二項」とあるのは「第三百六条の二第一項」と、「(法第百七十六条第三項」とあるのは「(法第百八十条の二第三項」と、「同条第一項」とあるのは「第三百条第一項(信託財産に係る利子等の課税の特例)」と、「同条第二項」とあるのは「第三百六条の二第一項」と読み替えるものとする。
\item[\rensuji{3}]集団投資信託(法第百七十六条第三項(信託財産に係る利子等の課税の特例)に規定する集団投資信託をいう。以下この条において同じ。)を引き受けた外国法人は、当該集団投資信託の信託財産について法第百八十条の二第三項に規定する所得税を課された場合には、財務省令で定めるところにより、当該所得税の額を課されたことを証する書類その他財務省令で定める書類を保存しなければならない。
\item[\rensuji{4}]集団投資信託を引き受けた外国法人(法第二百二十七条(信託の計算書)に規定する信託の受託者及び法第二百二十八条第一項(名義人受領の配当所得等の調書)に規定する利子等又は配当等の支払を受ける者に該当する者(以下第六項までにおいて「準支払者」という。)を含む。)は、個人に対して国内において当該集団投資信託の収益の分配(租税特別措置法第三条第一項(利子所得の分離課税等)の規定の適用を受けた同項に規定する一般利子等を除く。以下この項及び次項において同じ。)の支払をする場合において、その支払の確定した収益の分配に係る通知外国所得税の額があるときは、当該通知外国所得税の額その他の財務省令で定める事項を、その支払の確定した日(無記名の投資信託又は特定受益証券発行信託の受益証券に係る収益の分配に係る通知については、その支払をした日)から一月以内(準支払者が通知する場合には、四十五日以内)に、当該個人に対し、書面により通知しなければならない。
\item[\rensuji{5}]前項に規定する外国法人は、同項の書面を同一の者に対してその年中に支払つた収益の分配の額の合計額で作成する場合には、同項の規定にかかわらず、当該収益の分配に係る通知外国所得税の額その他の財務省令で定める事項を、同項に規定する支払の確定した日の属する年の翌年一月三十一日(準支払者が通知する場合には、同年二月十五日)までに、同項の個人に対し、書面により通知しなければならない。
\item[\rensuji{6}]集団投資信託を引き受けた外国法人(準支払者を含む。)は、法人に対して国内において当該集団投資信託の収益の分配の支払をする場合において、その支払の確定した収益の分配に係る通知外国所得税の額があるときは、当該通知外国所得税の額その他の財務省令で定める事項を、その支払の確定した日(無記名の投資信託又は特定受益証券発行信託の受益証券に係る収益の分配に係る通知については、その支払をした日)から一月以内(準支払者が通知する場合には、四十五日以内)に、当該法人に対し、書面により通知しなければならない。
\item[\rensuji{7}]前三項に規定する通知外国所得税の額とは、法第百八十条の二第三項の規定により前三項の集団投資信託の第一項に規定する収益の分配に係る所得税の額から控除された第三百条第一項に規定する外国所得税の額に、当該集団投資信託の第一項に規定する収益の分配(法第百八十一条又は第二百十二条の規定により所得税を徴収されるべきこととなる部分に限る。以下この項において同じ。)の額の総額のうちに前三項の個人又は法人が支払を受けた収益の分配の額の占める割合を乗じて計算した金額(当該金額がその支払を受けた収益の分配につき法第百八十一条又は第二百十二条の規定により徴収された又は徴収されるべき所得税の額を超える場合には、当該所得税の額)をいう。
\item[\rensuji{8}]第四項から第六項までに規定する外国法人は、これらの規定の書面による通知に代えて、これらの規定の個人又は法人の承諾を得て、当該書面に記載すべき事項を電磁的方法(電子情報処理組織を使用する方法その他の情報通信の技術を利用する方法であつて財務省令で定めるものをいう。第十項及び第十一項において同じ。)により提供することができる。ただし、当該個人又は法人の請求があるときは、当該個人又は法人に対し、当該書面により通知しなければならない。
\item[\rensuji{9}]前項本文の場合において、同項に規定する外国法人は、第四項から第六項までの規定による通知をしたものとみなす。
\item[\rensuji{10}]第八項に規定する外国法人は、同項本文の規定により書面に記載すべき事項を同項の個人又は法人に対し提供しようとするときは、財務省令で定めるところにより、あらかじめ、当該個人又は法人に対し、その用いる電磁的方法の種類及び内容を示し、書面又は電磁的方法による承諾を得なければならない。
\item[\rensuji{11}]前項の規定による承諾を得た同項に規定する外国法人は、同項の個人又は法人から書面又は電磁的方法により第八項本文の規定による電磁的方法による提供を受けない旨の申出があつたときは、当該個人又は法人に対し、同項の書面に記載すべき事項の提供を電磁的方法によつてしてはならない。ただし、当該個人又は法人が再び前項の規定による承諾をした場合は、この限りでない。
\item[\rensuji{12}]第四項から第六項までに規定する収益の分配の支払をするこれらの規定に規定する外国法人並びに当該収益の分配の支払を受けるこれらの規定の個人及び法人については、法第二百二十五条第二項(支払調書及び支払通知書)又は租税特別措置法第八条の四第四項から第七項まで(上場株式等に係る配当所得等の課税の特例)の規定のうち当該収益の分配に係る部分の規定の適用がある場合には、第四項から前項までの規定のうち当該適用を受けた収益の分配に係る部分の規定は、適用しない。
\end{description}
\part*{第四編 源泉徴収}
\addcontentsline{toc}{part}{第四編 源泉徴収}
\section*{第一章 給与所得に係る源泉徴収}
\addcontentsline{toc}{section}{第一章 給与所得に係る源泉徴収}
\subsection*{第一節 源泉徴収義務及び徴収税額}
\addcontentsline{toc}{subsection}{第一節 源泉徴収義務及び徴収税額}
\begin{description}
\item[第三百七条]削除
\end{description}
\noindent\hspace{10pt}(給与等の月割額等の意義)
\begin{description}
\item[第三百八条]法第百八十五条第一項第一号又は第二号(賞与以外の給与等に係る徴収税額)に規定する給与等の月割額は、法第二十八条第一項(給与所得)に規定する給与等(以下この章において「給与等」という。)の支給すべき額をその給与等の計算期間につき定められている月の整数倍の倍数で除して計算した金額とする。
\item[\rensuji{2}]法第百八十五条第一項第一号又は第二号に規定する給与等の日割額は、給与等の支給すべき額をその給与等の計算の基礎となつた日数で除して計算した金額とする。
\end{description}
\noindent\hspace{10pt}(日払の給与等の意義)
\begin{description}
\item[第三百九条]法第百八十五条第一項第三号(賞与以外の給与等に係る徴収税額)に規定する政令で定める給与等は、日日雇い入れられる者が支払を受ける給与等(一の給与等の支払者から継続して二月をこえて支払を受ける場合におけるその二月をこえて支払を受けるものを除く。)とする。
\end{description}
\noindent\hspace{10pt}(再就職者等の給与等)
\begin{description}
\item[第三百十条]法第百八十六条第三項(賞与に係る徴収税額)に規定する政令で定める給与等は、同項に規定する他の給与等の支払者が同項に規定する居住者に対して支払うべき給与等のうちその年一月一日から当該支払者が法第百九十四条第一項(給与所得者の扶養控除等申告書)に規定する主たる給与等の支払者でなくなる日(当該支払者がその年中において当該主たる給与等の支払者でなくなる日が二以上ある場合には、最後に主たる給与等の支払者でなくなる日)までの間に支払うべきことが確定した給与等とする。
\end{description}
\subsection*{第二節 年末調整}
\addcontentsline{toc}{subsection}{第二節 年末調整}
\noindent\hspace{10pt}(再就職者等の年末調整の対象となる給与等)
\begin{description}
\item[第三百十一条]法第百九十条第一号(年末調整)に規定する政令で定める給与等は、同号に規定する他の給与等の支払者が同号に規定する居住者に対して支払うべき給与等のうちその年一月一日から当該支払者が法第百九十四条第一項(給与所得者の扶養控除等申告書)に規定する主たる給与等の支払者でなくなる日(当該支払者がその年中において当該主たる給与等の支払者でなくなる日が二以上ある場合には、最後に主たる給与等の支払者でなくなる日)までの間に支払うべきことが確定した給与等とする。
\end{description}
\noindent\hspace{10pt}(年末調整による過納額の還付の方法)
\begin{description}
\item[第三百十二条]法第百九十一条(過納額の還付)の規定により還付をする場合には、その還付をすべき金額に相当する金額は、同条に規定する給与等の支払者が法第百八十三条(源泉徴収義務)、第百九十条(年末調整)、第百九十二条(不足額の徴収)、第百九十九条(退職所得に係る源泉徴収義務)、第二百四条第一項第二号(報酬、料金等に係る源泉徴収義務)又は第二百十六条(源泉徴収に係る所得税の納期の特例)の規定により納付すべき金額から控除する。
\end{description}
\noindent\hspace{10pt}(給与等の支払者が還付できなかつた場合の処理)
\begin{description}
\item[第三百十三条]前条の規定を適用する場合において、同条に規定する給与等の支払者が次の各号のいずれかに該当することとなつたときは、当該給与等に係る所得税の法第十七条(源泉徴収に係る所得税の納税地)の規定による納税地(法第十八条第二項(納税地の指定)の規定による指定があつた場合には、その指定をされた納税地)の所轄税務署長は、法第百九十一条(過納額の還付)の規定により還付すべき金額のうちまだ還付されていない金額を同条に規定する居住者に還付する。
\begin{description}
\item[一]法第百八十三条(給与所得に係る源泉徴収義務)若しくは第百九十条(年末調整)に規定する給与等の支払者若しくは法第百九十九条(退職所得に係る源泉徴収義務)に規定する退職手当等の支払者でなくなつたこと又はこれらの規定若しくは法第百九十二条(不足額の徴収)若しくは第二百四条第一項第二号(報酬、料金等に係る源泉徴収義務)の規定により徴収して納付すべき所得税の額がなくなつたことにより法第百九十一条の規定による還付をすべき金額の全部又は一部を還付することができないこととなつた場合
\item[二]法第百九十一条の規定による還付をすべきこととなつた日の属する月の翌月一日から起算して二月を経過した後において、なお当該還付をすべき金額の全部を還付するに至らない場合
\end{description}
\item[\rensuji{2}]前項の規定の適用を受けようとする支払者は、同項各号のいずれかに該当することとなつた旨を記載した書面に、各人別の法第百九十一条の規定による還付をすべき金額及び当該金額のうちまだ還付をされていない部分の金額その他必要な事項を記載した明細書を添附して、これを同項の税務署長に提出しなければならない。
\end{description}
\begin{description}
\item[第三百十四条]削除
\end{description}
\noindent\hspace{10pt}(税引給与等の月割額の計算)
\begin{description}
\item[第三百十五条]法第百九十二条第二項第二号(不足額の徴収)に規定する月割額として政令で定めるところにより計算した金額は、その年一月からその年最後に給与等の支払を受ける日の属する月(以下この条において「給与の最終支払月」という。)の前月までの間に同号に規定する給与等の支払者から支払を受けた給与等の金額の総額から当該給与等につき法第百八十三条第一項(源泉徴収義務)の規定により徴収された又はされるべき所得税の額の合計額を控除した残額を、その年一月(その年の中途において当該支払者から給与等の支払を受けることとなつた場合には、最初に当該給与等の支払を受けた日の属する月)から給与の最終支払月の前月までの月数で除して計算した金額とする。
\end{description}
\noindent\hspace{10pt}(年末調整の不足額の徴収猶予を受けるための手続)
\begin{description}
\item[第三百十六条]法第百九十二条第二項(不足額の徴収)の税務署長の承認を受けようとする者は、次に掲げる事項を記載した申請書を、同項に規定する給与等の支払者を経由して、その年最後に給与等の支払を受ける日の前日までに、当該税務署長に提出しなければならない。
\begin{description}
\item[一]申請者の氏名及び住所(国内に住所がないときは、居所)
\item[二]当該支払者の氏名又は名称
\item[三]前条に規定する給与の最終支払月中に当該支払者から支払を受ける給与等の金額の総額から、当該給与等につき法第百八十三条第一項(源泉徴収義務)及び第百九十条(年末調整)の規定により徴収された又は徴収されるべき所得税の額を控除した残額に相当する金額
\item[四]前条に定める金額
\item[五]法第百九十条に規定する不足額及びそのうち法第百九十二条第二項の承認を受けようとする金額
\item[六]その他参考となるべき事項
\end{description}
\item[\rensuji{2}]前項の申請書の提出があつた場合において、同項第三号に掲げる金額が同項第四号に掲げる金額の十分の七に相当する金額に満たないときは、税務署長は、法第百九十二条第二項の承認をしなければならない。
\item[\rensuji{3}]税務署長は、法第百九十二条第二項の承認をする場合には、第一項の給与等の支払者を経由して、申請者に対し、書面によりその旨を通知する。
\end{description}
\subsection*{第三節 給与所得者の源泉徴収に関する申告}
\addcontentsline{toc}{subsection}{第三節 給与所得者の源泉徴収に関する申告}
\noindent\hspace{10pt}(給与所得者の扶養控除等申告書に関する書類の提出又は提示)
\begin{description}
\item[第三百十六条の二]法第百九十四条第一項又は第二項(給与所得者の扶養控除等申告書)の規定による申告書に勤労学生に該当する旨の記載をした居住者で法第二条第一項第三十二号ロ又はハ(定義)に掲げる者に該当するものは、これらの者に該当する旨を証する書類として財務省令で定めるものを当該申告書に添付し、又は当該申告書の提出の際提示しなければならない。
\item[\rensuji{2}]法第百九十四条第一項又は第二項の規定による申告書に同条第一項第七号に掲げる事項の記載をした居住者は、次の各号に掲げる国外居住親族(同条第四項に規定する国外居住親族をいう。以下この条において同じ。)の区分に応じ、当該各号に定める旨を証する書類として財務省令で定めるものを各人別に当該申告書に添付し、又は当該申告書の提出の際提示しなければならない。
\begin{description}
\item[一]法第百九十四条第一項第七号の同居特別障害者若しくはその他の特別障害者又は特別障害者以外の障害者である国外居住親族(次号及び第三号に掲げる国外居住親族を除く。)
\item[二]法第百九十四条第一項第七号に規定する源泉控除対象配偶者である国外居住親族
\item[三]法第百九十四条第一項第七号に規定する控除対象扶養親族である国外居住親族
\end{description}
\item[\rensuji{3}]法第百九十四条第五項の規定による申告書を提出する居住者は、国外居住親族が当該居住者と生計を一にすることを明らかにする書類として財務省令で定めるものを各人別に当該申告書に添付し、又は当該申告書の提出の際提示しなければならない。
\end{description}
\noindent\hspace{10pt}(従たる給与についての扶養控除等申告書の提出ができる場合の判定)
\begin{description}
\item[第三百十七条]法第百九十五条第一項(従たる給与についての扶養控除等申告書)に規定する政令で定めるところにより計算した金額は、第一号に掲げる金額から第二号に掲げる金額を控除した金額とする。
\begin{description}
\item[一]その年中に主たる給与等の支払者から支払を受ける給与等の金額の見積額を法第二十八条第二項(給与所得の金額)に規定する給与等の収入金額とみなして計算した場合における同項に規定する給与所得の金額
\item[二]前号に規定する給与等の金額の見積額から控除されるべき法第七十四条第二項(社会保険料控除)に規定する社会保険料の額の見積額及び法第七十五条第二項(小規模企業共済等掛金控除)に規定する小規模企業共済等掛金の額の見積額の合計額
\end{description}
\end{description}
\noindent\hspace{10pt}(控除対象扶養親族等を従たる給与についての扶養控除等申告書に追加する場合の手続)
\begin{description}
\item[第三百十八条]法第百九十五条第一項(従たる給与についての扶養控除等申告書)の規定により従たる給与についての扶養控除等申告書を提出した居住者が、その年において提出した法第百九十四条第一項又は第二項(給与所得者の扶養控除等申告書)の規定による申告書に記載した同条第一項第六号に規定する源泉控除対象配偶者又は控除対象扶養親族を法第百九十五条第一項第三号に規定する源泉控除対象配偶者又は控除対象扶養親族としようとする場合には、当該源泉控除対象配偶者又は控除対象扶養親族について異動が生じたものとみなして法第百九十四条第二項及び第百九十五条第二項の規定を適用する。
\end{description}
\noindent\hspace{10pt}(従たる給与についての扶養控除等申告書に関する書類の提出又は提示)
\begin{description}
\item[第三百十八条の二]法第百九十五条第一項又は第二項(従たる給与についての扶養控除等申告書)の規定による申告書に同条第一項第四号に掲げる事項の記載をした居住者は、次の各号に掲げる記載がされた者の区分に応じ、当該各号に定める旨を証する書類として財務省令で定めるものを各人別に当該申告書に添付し、又は当該申告書の提出の際提示しなければならない。
\begin{description}
\item[一]法第百九十五条第一項第四号に規定する源泉控除対象配偶者で、当該申告書に非居住者である旨の記載がされた者
\item[二]法第百九十五条第一項第四号に規定する控除対象扶養親族で、当該申告書に非居住者である旨の記載がされた者
\end{description}
\end{description}
\noindent\hspace{10pt}(給与所得者の配偶者控除等申告書に関する書類の提出又は提示)
\begin{description}
\item[第三百十八条の三]法第百九十五条の二第一項(給与所得者の配偶者控除等申告書)の規定による申告書に控除対象配偶者又は同項第三号に規定する配偶者が非居住者である旨の記載をした居住者は、当該記載がされた控除対象配偶者又は配偶者についての次に掲げる書類を当該申告書に添付し、又は当該申告書の提出の際提示しなければならない。
\begin{description}
\item[一]その控除対象配偶者又は配偶者が当該居住者の配偶者に該当する旨を証する書類として財務省令で定めるもの
\item[二]その控除対象配偶者又は配偶者が当該居住者と生計を一にすることを明らかにする書類として財務省令で定めるもの
\end{description}
\end{description}
\noindent\hspace{10pt}(保険料控除申告書に関する書類等の提出又は提示)
\begin{description}
\item[第三百十九条]法第百九十六条第三項(給与所得者の保険料控除申告書)に規定する給与所得者の保険料控除申告書を提出する居住者は、次の各号に掲げる場合には、当該各号に定める書類又は電磁的記録印刷書面(第二百六十二条第一項(確定申告書に関する書類等の提出又は提示)に規定する電磁的記録印刷書面をいう。以下この条において同じ。)を当該申告書に添付し、又は当該申告書の提出の際提示しなければならない。
\begin{description}
\item[一]当該申告書に法第百九十六条第一項第二号に規定する社会保険料(法第七十四条第二項第五号(社会保険料控除)に掲げるものに限る。)の金額を記載する場合
\item[二]当該申告書に法第百九十六条第一項第二号に規定する小規模企業共済等掛金の額を記載する場合
\item[三]当該申告書に法第百九十六条第一項第三号に規定する新生命保険料の金額を記載する場合
\item[四]当該申告書に法第百九十六条第一項第三号に規定する旧生命保険料の金額を記載する場合において、当該旧生命保険料の金額に係る法第七十六条第六項に規定する旧生命保険契約等のうちに当該旧生命保険契約等に基づきその年中に支払つた当該旧生命保険料の金額(その年において当該旧生命保険契約等に基づく剰余金の分配若しくは割戻金の割戻しを受け、又は当該旧生命保険契約等に基づき分配を受ける剰余金若しくは割戻しを受ける割戻金をもつて当該旧生命保険料の払込みに充てた場合には、当該剰余金又は割戻金の額(当該旧生命保険料に係る部分の金額に限る。)を控除した残額)が九千円を超えるものがあるとき
\item[五]当該申告書に法第百九十六条第一項第三号に規定する介護医療保険料の金額を記載する場合
\item[六]当該申告書に法第百九十六条第一項第三号に規定する新個人年金保険料の金額を記載する場合
\item[七]当該申告書に法第百九十六条第一項第三号に規定する旧個人年金保険料の金額を記載する場合
\item[八]当該申告書に法第百九十六条第一項第三号に規定する地震保険料の金額を記載する場合
\end{description}
\end{description}
\noindent\hspace{10pt}(給与所得者の源泉徴収に関する申告書に記載すべき事項の電磁的方法による提供に係る承認等に関する手続)
\begin{description}
\item[第三百十九条の二]法第百九十八条第二項(給与所得者の源泉徴収に関する申告書の提出時期等の特例)に規定する給与等の支払者(以下この項、次項及び第五項において「給与等の支払者」という。)は、同条第二項に規定する所轄税務署長(以下この条において「所轄税務署長」という。)の承認を受けようとする場合には、当該給与等の支払者の氏名及び住所又は名称、所在地及び法人番号、その用いる電磁的方法(同項に規定する電磁的方法をいう。次項及び第五項において同じ。)の種類及び内容その他の財務省令で定める事項を記載した申請書を当該所轄税務署長に提出しなければならない。
\item[\rensuji{2}]所轄税務署長は、法第百九十八条第二項の承認を受けている給与等の支払者につき次の各号のいずれかに該当する事実があると認めるときは、その承認を取り消すことができる。
\begin{description}
\item[一]法第百九十八条第二項に規定する給与等の支払を受ける居住者(次号において「給与等の支払を受ける居住者」という。)が電磁的方法による同項に規定する申告書に記載すべき事項(以下この項において「記載事項」という。)の提供を適正に行うことができる措置を講じていないこと。
\item[二]法第百九十八条第二項の規定により提供を受けた記載事項について、その提供をした給与等の支払を受ける居住者を特定するための必要な措置を講じていないこと。
\item[三]法第百九十八条第二項の規定により提供を受けた記載事項について、電子計算機の映像面への表示及び書面への出力をするための必要な措置を講じていないこと。
\end{description}
\item[\rensuji{3}]所轄税務署長は、第一項の申請書の提出があつた場合において、その申請につき承認をしたとき、若しくは当該承認をしないことを決定したとき、又は前項の規定により承認を取り消したときは、その申請をした者又は当該承認を受けていた者に対し、書面によりその旨を通知するものとする。
\item[\rensuji{4}]第一項の申請書の提出があつた場合において、その申請書の提出があつた日の属する月の翌月末日までに、当該申請の承認がなかつたとき、又は当該承認をしないことの決定がなかつたときは、同日において当該申請の承認があつたものとみなす。
\item[\rensuji{5}]法第百九十八条第二項の承認を受けている給与等の支払者が、同項の規定による電磁的方法による提供を受けることをやめようとする場合には、その者は、その旨その他財務省令で定める事項を記載した届出書を所轄税務署長に提出しなければならない。
\item[\rensuji{6}]第二項の規定による承認の取消し又は前項の規定による届出書の提出があつた場合には、法第百九十八条第二項の承認は、その取消しの通知を受けた日又はその提出をした日においてその効力を失うものとする。
\end{description}
\section*{第一章の二 退職所得に係る源泉徴収}
\addcontentsline{toc}{section}{第一章の二 退職所得に係る源泉徴収}
\noindent\hspace{10pt}(特定役員退職手当等と一般退職手当等がある場合の退職所得に係る源泉徴収)
\begin{description}
\item[第三百十九条の三]法第二百一条第一項第二号ハ(徴収税額)に規定する政令で定めるところにより計算した金額は、次に掲げる金額の合計額(第二号に規定する一般退職手当等の金額が同号に規定する一般退職所得控除額に満たない場合には、その満たない部分の金額を第一号に掲げる金額から控除した残額)とする。
\begin{description}
\item[一]法第二百一条第一項第一号イに規定する特定役員退職手当等の金額から特定役員退職所得控除額を控除した残額
\item[二]法第二百一条第一項第一号イに規定する一般退職手当等の金額から一般退職所得控除額を控除した残額の二分の一に相当する金額
\end{description}
\item[\rensuji{2}]前項第一号に規定する特定役員退職所得控除額又は同項第二号に規定する一般退職所得控除額とは、法第二百一条第一項の規定による所得税を徴収すべき法第百九十九条(源泉徴収義務)に規定する退職手当等を支払うべきことが確定した時の状況における第七十一条の二第一項第一号(特定役員退職手当等と一般退職手当等がある場合の退職所得の金額の計算)に規定する特定役員退職所得控除額又は同項第二号に規定する一般退職所得控除額をいう。
\item[\rensuji{3}]第七十一条の二第五項及び第六項の規定は、第一項の規定を適用する場合について準用する。
\end{description}
\noindent\hspace{10pt}(源泉徴収の対象となる退職所得とみなされる退職一時金の範囲等)
\begin{description}
\item[第三百十九条の三の二]法第二百二条(退職所得とみなされる退職一時金に係る源泉徴収)に規定する政令で定める場合は、次の各号に掲げる場合とし、同条に規定する政令で定める金額は、当該各号に掲げる場合の区分に応じ当該各号に定める金額とする。
\begin{description}
\item[一]第七十二条第三項第四号(退職手当等とみなす一時金)に掲げる一時金の支払をする場合において、同号に規定する適格退職年金契約に基づいて払い込まれた掛金又は保険料のうちに同号に規定する勤務をした者の負担した金額があるとき
\item[二]第七十二条第三項第五号に掲げる一時金の支払をする場合において、同号に規定する規約に基づいて拠出された掛金のうちに同号に規定する加入者の負担した金額があるとき
\end{description}
\end{description}
\noindent\hspace{10pt}(退職所得の受給に関する申告書に記載すべき事項の電磁的方法による提供に係る承認等に関する手続)
\begin{description}
\item[第三百十九条の四]第三百十九条の二(給与所得者の源泉徴収に関する申告書に記載すべき事項の電磁的方法による提供に係る承認等に関する手続)の規定は、法第二百三条第四項(退職所得の受給に関する申告書)に規定する退職手当等の支払者に係る同項の承認について準用する。
\end{description}
\section*{第二章 公的年金等に係る源泉徴収}
\addcontentsline{toc}{section}{第二章 公的年金等に係る源泉徴収}
\noindent\hspace{10pt}(公的年金等の月割額)
\begin{description}
\item[第三百十九条の五]法第二百三条の三第一号イ(公的年金等に係る徴収税額)に規定する公的年金等の月割額として政令で定める金額は、同条に規定する公的年金等の金額をその公的年金等の金額に係る月数で除して計算した金額とする。
\end{description}
\noindent\hspace{10pt}(公的年金等の金額から控除する金額の調整)
\begin{description}
\item[第三百十九条の六]法第二百三条の三第二号(公的年金等に係る徴収税額)に規定する政令で定める公的年金等は、次の各号に掲げる公的年金等(法第二百三条の二(公的年金等に係る源泉徴収義務)に規定する公的年金等をいう。以下この条において同じ。)とし、法第二百三条の三第二号に規定する政令で定める金額は、当該各号に掲げる公的年金等の区分に応じ当該各号に定める金額とする。
\begin{description}
\item[一]次に掲げる公的年金等
\begin{description}
\item[イ]独立行政法人農業者年金基金法(平成十四年法律第百二十七号)第十八条第一号(給付の種類)に掲げる農業者老齢年金及び同法附則第六条第一項第一号(業務の特例)の規定により支給される農業者年金基金法の一部を改正する法律(平成十三年法律第三十九号)による改正前の農業者年金基金法(昭和四十五年法律第七十八号)第三十二条第二号(給付の種類)に掲げる農業者老齢年金
\item[ロ]国民年金法第百二十八条第一項(国民年金基金の業務)又は第百三十七条の十五第一項(国民年金基金連合会の業務)に規定する年金
\item[ハ]被用者年金制度の一元化等を図るための厚生年金保険法等の一部を改正する法律(平成二十四年法律第六十三号。以下この号及び次項第一号において「一元化法」という。)附則第三十七条第一項(改正前国共済法による給付等)の規定によりなおその効力を有するものとされる一元化法第二条(国家公務員共済組合法の一部改正)の規定による改正前の国家公務員共済組合法(ハ及びホにおいて「旧効力国共済法」という。)第七十二条第一項第一号(長期給付の種類等)に掲げる退職共済年金(旧効力国共済法附則第十二条の三(退職共済年金の特例)の規定により支給されるものその他の財務省令で定める退職共済年金を除く。)
\item[ニ]一元化法附則第六十一条第一項(改正前地共済法による給付等)の規定によりなおその効力を有するものとされる一元化法第三条(地方公務員等共済組合法の一部改正)の規定による改正前の地方公務員等共済組合法(ニにおいて「旧効力地共済法」という。)第七十四条第一号(長期給付の種類)に掲げる退職共済年金(旧効力地共済法附則第十九条(退職共済年金の特例)の規定により支給されるものその他の財務省令で定める退職共済年金を除く。)
\item[ホ]一元化法附則第七十九条(改正前私学共済法による給付)の規定によりなおその効力を有するものとされる一元化法第四条(私立学校教職員共済法の一部改正)の規定による改正前の私立学校教職員共済法(ホにおいて「旧効力私学共済法」という。)第二十条第二項第一号(給付)に掲げる退職共済年金(旧効力私学共済法第二十五条(国家公務員共済組合法の準用)において準用する旧効力国共済法附則第十二条の三の規定により支給されるものその他の財務省令で定める退職共済年金を除く。)
\end{description}
\item[二]平成二十五年厚生年金等改正法附則第五条第一項(存続厚生年金基金に係る改正前厚生年金保険法等の効力等)の規定によりなおその効力を有するものとされる旧厚生年金保険法第百三十条第一項(基金の業務)又は平成二十五年厚生年金等改正法附則第四十条第三項第一号若しくは第二号(存続連合会の業務)に規定する老齢年金給付
\end{description}
\item[\rensuji{2}]法第二百三条の三第三号に規定する政令で定める公的年金等は、次の各号に掲げる公的年金等とし、同条第三号に規定する政令で定める金額は、当該各号に掲げる公的年金等の区分に応じ当該各号に定める金額とする。
\begin{description}
\item[一]次に掲げる公的年金等(次号に掲げるものを除く。)
\begin{description}
\item[イ]国家公務員共済組合法第七十四条第一号(退職等年金給付の種類)に掲げる退職年金(次号イにおいて「退職年金」という。)及び一元化法附則第三十六条第一項(改正前国共済法による職域加算額の経過措置)の規定によりなおその効力を有するものとされる一元化法第二条の規定による改正前の国家公務員共済組合法(以下この項において「旧効力国共済法」という。)第七十七条第二項各号(退職共済年金の額)に定める金額に相当する給付(次号イにおいて「旧職域加算年金給付」という。)並びにこれらの公的年金等の支払者から支払われる厚生年金保険法第三十二条第一号(保険給付の種類)に掲げる老齢厚生年金(以下この号及び次号イにおいて「老齢厚生年金」という。)その他の財務省令で定める公的年金等
\item[ロ]地方公務員等共済組合法第七十六条第一号(退職等年金給付の種類)に掲げる退職年金(次号ロにおいて「退職年金」という。)及び一元化法附則第六十条第一項(改正前地共済法による職域加算額の経過措置)の規定によりなおその効力を有するものとされる一元化法第三条の規定による改正前の地方公務員等共済組合法(次号ロにおいて「旧効力地共済法」という。)第七十九条第一項第二号(退職共済年金の額)に掲げる金額に相当する給付(次号ロにおいて「旧職域加算年金給付」という。)並びにこれらの公的年金等の支払者から支払われる老齢厚生年金その他の財務省令で定める公的年金等
\item[ハ]私立学校教職員共済法第二十条第二項第一号(給付)に掲げる退職年金(次号ハにおいて「退職年金」という。)及び一元化法附則第七十八条第一項(改正前私学共済法による職域加算額の経過措置)の規定によりなおその効力を有するものとされる一元化法第四条の規定による改正前の私立学校教職員共済法(次号ハにおいて「旧効力私学共済法」という。)第二十五条(国家公務員共済組合法の準用)において準用する旧効力国共済法第七十七条第二項の規定により加算する同項各号に定める金額に相当する給付(次号ハにおいて「旧職域加算年金給付」という。)並びにこれらの公的年金等の支払者から支払われる老齢厚生年金その他の財務省令で定める公的年金等
\end{description}
\item[二]次に掲げる公的年金等
\begin{description}
\item[イ]国家公務員共済組合法附則第十三条第二項(支給の繰上げ)の規定により支給される退職年金(国民年金法第十五条第一号(給付の種類)に掲げる老齢基礎年金(ロ及びハにおいて「老齢基礎年金」という。)の支払を受ける者に支給されるものを除く。)及び旧効力国共済法附則第十二条の三(退職共済年金の特例)の規定により支給される旧職域加算年金給付並びにこれらの公的年金等の支払者から支払われる厚生年金保険法附則第八条の規定により支給される老齢厚生年金(ロ及びハにおいて「特例老齢厚生年金」という。)
\item[ロ]地方公務員等共済組合法附則第十九条第二項(支給の繰上げ)の規定により支給される退職年金(老齢基礎年金の支払を受ける者に支給されるものを除く。)及び旧効力地共済法附則第十九条(退職共済年金の特例)の規定により支給される旧職域加算年金給付並びにこれらの公的年金等の支払者から支払われる特例老齢厚生年金
\item[ハ]私立学校教職員共済法第二十五条(国家公務員共済組合法の準用)において準用する国家公務員共済組合法附則第十三条第二項の規定により支給される退職年金(老齢基礎年金の支払を受ける者に支給されるものを除く。)及び旧効力私学共済法第二十五条において準用する旧効力国共済法附則第十二条の三の規定により支給される旧職域加算年金給付並びにこれらの公的年金等の支払者から支払われる特例老齢厚生年金
\end{description}
\end{description}
\end{description}
\noindent\hspace{10pt}(公的年金等の月割額等の端数計算)
\begin{description}
\item[第三百十九条の七]第三百十九条の五(公的年金等の月割額)の規定により計算した金額が四円の整数倍でないときは、当該金額を超える四円の整数倍である金額のうち最も少ない金額を当該計算した金額とする。
\item[\rensuji{2}]法第二百三条の三第四号(公的年金等に係る徴収税額)に定める金額に一円未満の端数があるときは、これを一円に切り上げるものとする。
\end{description}
\noindent\hspace{10pt}(源泉徴収の対象となる確定給付企業年金の額の計算等)
\begin{description}
\item[第三百十九条の八]法第二百三条の四第二号(公的年金等から控除される社会保険料がある場合等の徴収税額の計算)に規定する政令で定めるところにより計算した金額は、同号に規定する年金の額(その年金の支給開始の日以後に同号に規定する規約に基づいて分配を受ける剰余金の額に相当する部分の金額を除く。)に当該年金に係る第八十二条の三第一項(確定給付企業年金の額から控除する金額)に規定する割合を乗じて計算した金額とする。
\item[\rensuji{2}]法第二百三条の四第三号に規定する政令で定める場合は、次の各号に掲げる場合とし、同条第三号に規定する政令で定めるところにより計算した金額は、当該各号に掲げる場合の区分に応じ当該各号に定める金額とする。
\begin{description}
\item[一]第八十二条の二第二項第四号(公的年金等とされる年金)に掲げる退職年金の支払をする場合において、同号に規定する適格退職年金契約に基づいて払い込まれた掛金又は保険料のうちに同号に規定する勤務をした者の負担した金額があるとき
\item[二]第八十二条の二第二項第五号に掲げる年金の支払をする場合において、同号に規定する規約に基づいて拠出された掛金のうちに同号に規定する加入者の負担した金額があるとき
\end{description}
\end{description}
\noindent\hspace{10pt}(公的年金等の受給者の扶養親族等申告書の提出ができない公的年金等)
\begin{description}
\item[第三百十九条の九]法第二百三条の五第一項(公的年金等の受給者の扶養親族等申告書)に規定する政令で定める公的年金等は、石炭鉱業年金基金法(昭和四十二年法律第百三十五号)第十六条第一項(坑内員に関する年金の給付)又は第十八条第一項(坑外員に関する年金の給付)の規定に基づく年金及び法第三十五条第三項第二号(公的年金等の定義)に規定する過去の勤務に基づき使用者であつた者から支給される年金(国会議員互助年金法を廃止する法律(平成十八年法律第一号)附則第七条第一項(現職国会議員の普通退職年金)に規定する普通退職年金又は同法附則第二条第一項(退職者に関する経過措置)の規定によりなおその効力を有するものとされる同法による廃止前の国会議員互助年金法(昭和三十三年法律第七十号)第九条(普通退職年金及びその年額)に規定する普通退職年金及び地方公務員の退職年金に関する条例の規定による退職を給付事由とする年金である給付を除く。)とする。
\end{description}
\noindent\hspace{10pt}(簡易な公的年金等の受給者の扶養親族等申告書の提出に係る国税庁長官の承認に関する手続)
\begin{description}
\item[第三百十九条の十]法第二百三条の五第二項(公的年金等の受給者の扶養親族等申告書)に規定する公的年金等の支払者は、同項の規定による国税庁長官の承認を受けようとする場合には、その旨及び当該承認を受けようとする事由その他財務省令で定める事項を記載した申請書を、財務省令で定める日までに、当該公的年金等に係る所得税の法第十七条(源泉徴収に係る所得税の納税地)の規定による納税地(法第十八条第二項(納税地の指定)の規定による指定があつた場合には、その指定をされた納税地)の所轄税務署長を経由して、国税庁長官に提出しなければならない。
\item[\rensuji{2}]国税庁長官は、前項の規定による申請書の提出を受けた場合には、当該申請書を提出した同項の公的年金等の支払者が当該申請書を提出した日の属する年において受理した法第二百三条の五第一項の規定による申告書(以下この項において「公的年金等の受給者の扶養親族等申告書」という。)に記載された事項について各人別の記録があり、かつ、同条第二項の規定により提出することができる公的年金等の受給者の扶養親族等申告書(第四項において「簡易な公的年金等の受給者の扶養親族等申告書」という。)に基づき法第四編第三章の二(公的年金等に係る源泉徴収)の規定による源泉徴収を行うこととすることが適当であると認めるときは当該申請を承認し、これらの事由がないと認めるときは当該申請を却下する。
\item[\rensuji{3}]国税庁長官は、前項の承認又は却下の処分をするときは、第一項の申請書を提出した同項の公的年金等の支払者に対し、書面によりその旨を通知する。
\item[\rensuji{4}]国税庁長官は、第二項の承認をした後、その承認を受けた第一項の公的年金等の支払者について簡易な公的年金等の受給者の扶養親族等申告書に基づいて法第四編第三章の二の規定による源泉徴収を行うことが適当でなくなつたと認める場合には、その承認を取り消すことができる。
\end{description}
\noindent\hspace{10pt}(公的年金等の受給者の扶養親族等申告書に関する書類の提出又は提示)
\begin{description}
\item[第三百十九条の十一]法第二百三条の五第一項(公的年金等の受給者の扶養親族等申告書)の規定による申告書に同項第六号に掲げる事項の記載をした居住者(同条第二項の規定により当該記載に代えて異動がない旨の記載をした居住者を含む。)は、次の各号に掲げる記載がされた者の区分に応じ、当該各号に定める旨を証する書類として財務省令で定めるものを各人別に当該申告書に添付し、又は当該申告書の提出の際提示しなければならない。
\begin{description}
\item[一]法第二百三条の五第一項第六号の源泉控除対象配偶者で、当該申告書に非居住者である旨の記載がされた者
\item[二]法第二百三条の五第一項第六号の控除対象扶養親族で、当該申告書に非居住者である旨の記載がされた者
\item[三]法第二百三条の五第一項第六号の同居特別障害者若しくはその他の特別障害者又は特別障害者以外の障害者で、当該申告書に非居住者である旨の記載がされた者(前二号に掲げる記載がされた者を除く。)
\end{description}
\end{description}
\noindent\hspace{10pt}(公的年金等の受給者の扶養親族等申告書に記載すべき事項の電磁的方法による提供に係る承認等に関する手続)
\begin{description}
\item[第三百十九条の十二]第三百十九条の二(給与所得者の源泉徴収に関する申告書に記載すべき事項の電磁的方法による提供に係る承認等に関する手続)の規定は、法第二百三条の五第五項(公的年金等の受給者の扶養親族等申告書)に規定する公的年金等の支払者に係る同項の承認について準用する。
\end{description}
\noindent\hspace{10pt}(源泉徴収等を要しない公的年金等の額)
\begin{description}
\item[第三百十九条の十三]法第二百三条の六(源泉徴収等を要しない公的年金等)に規定する政令で定める金額は、百八万円とする。
\end{description}
\section*{第三章 報酬、料金等に係る源泉徴収}
\addcontentsline{toc}{section}{第三章 報酬、料金等に係る源泉徴収}
\subsection*{第一節 報酬、料金、契約金又は賞金に係る源泉徴収}
\addcontentsline{toc}{subsection}{第一節 報酬、料金、契約金又は賞金に係る源泉徴収}
\noindent\hspace{10pt}(報酬、料金、契約金又は賞金に係る源泉徴収)
\begin{description}
\item[第三百二十条]法第二百四条第一項第一号(源泉徴収義務)に規定する政令で定める報酬又は料金は、テープ若しくはワイヤーの吹込み、脚本、脚色、翻訳、通訳、校正、書籍の装てい、速記、版下(写真製版用写真原板の修整を含むものとし、写真植字を除くものとする。)若しくは雑誌、広告その他の印刷物に掲載するための写真の報酬若しくは料金、技術に関する権利、特別の技術による生産方式若しくはこれらに準ずるものの使用料、技芸、スポーツその他これらに類するものの教授若しくは指導若しくは知識の教授の報酬若しくは料金又は金融商品取引法第二十八条第六項(通則)に規定する投資助言業務に係る報酬若しくは料金とする。
\item[\rensuji{2}]法第二百四条第一項第二号に規定する政令で定める者は、計理士、会計士補、企業診断員(企業経営の改善及び向上のための指導を行う者を含む。)、測量士補、建築代理士(建築代理士以外の者で建築に関する申請若しくは届出の書類を作成し、又はこれらの手続を代理することを業とするものを含む。)、不動産鑑定士補、火災損害鑑定人若しくは自動車等損害鑑定人(自動車又は建設機械に係る損害保険契約(保険業法第二条第四項(定義)に規定する損害保険会社若しくは同条第九項に規定する外国損害保険会社等の締結した保険契約又は同条第十八項に規定する少額短期保険業者の締結したこれに類する保険契約をいう。)又はこれに類する共済に係る契約の保険事故又は共済事故に関して損害額の算定又はその損害額の算定に係る調査を行うことを業とする者をいう。)又は技術士補(技術士又は技術士補以外の者で技術士の行う業務と同一の業務を行う者を含む。)とする。
\item[\rensuji{3}]法第二百四条第一項第四号に規定する政令で定める者は、プロサッカーの選手、プロテニスの選手、プロレスラー、プロゴルファー、プロボウラー、自動車のレーサー、自転車競技の選手、小型自動車競走の選手又はモーターボート競走の選手とし、同号に規定するモデルには、雑誌、広告その他の印刷物にその容姿を掲載させて報酬を受ける者を含むものとする。
\item[\rensuji{4}]法第二百四条第一項第五号に規定する政令で定める芸能は、音楽、音曲、舞踊、講談、落語、浪曲、漫談、漫才、腹話術、歌唱、奇術、曲芸又は物まねとし、同号に規定する政令で定めるものは、映画若しくは演劇の製作、振付け(剣技指導その他これに類するものを含む。)、舞台装置、照明、撮影、演奏、録音(擬音効果を含む。)、編集、美粧又は考証とする。
\item[\rensuji{5}]法第二百四条第一項第五号に規定する政令で定める芸能人は、映画若しくは演劇の俳優、映画監督若しくは舞台監督(プロジューサーを含む。)、演出家、放送演技者、音楽指揮者、楽士、舞踊家、講談師、落語家、浪曲師、漫談家、漫才家、腹話術師、歌手、奇術師、曲芸師又は物まね師とする。
\item[\rensuji{6}]法第二百四条第一項第七号に規定する政令で定める契約金は、職業野球の選手その他一定の者に専属して役務の提供をする者で、当該一定の者のために役務を提供し、又はそれ以外の者のために役務を提供しないことを約することにより一時に受ける契約金とする。
\item[\rensuji{7}]法第二百四条第一項第八号に規定する広告宣伝のための賞金で政令で定めるものは、事業の広告宣伝のために賞として支払う金品その他の経済上の利益(旅行その他役務の提供を内容とするもので、金品との選択をすることができないものとされているものを除く。)とし、同号に規定する馬主が受ける競馬の賞金で政令で定めるものは、第二百九十八条第九項(内国法人に係る所得税の課税標準)に規定する賞金とする。
\end{description}
\noindent\hspace{10pt}(金銭以外のもので支払われる賞金の価額)
\begin{description}
\item[第三百二十一条]法第二百五条第二号(報酬又は料金等に係る徴収税額)に規定する政令で定めるところにより計算した金額は、同号に規定する金銭以外のものの支払を受ける者がその受けることとなつた日において当該金銭以外のものを譲渡するものとした場合にその対価として通常受けるべき価額に相当する金額(当該金銭以外のものと金銭とのいずれかを選択することができる場合には、当該金銭の額)とする。
\end{description}
\noindent\hspace{10pt}(支払金額から控除する金額)
\begin{description}
\item[第三百二十二条]法第二百五条第二号(報酬又は料金等に係る徴収税額)に規定する政令で定める金額は、次の表の上欄に掲げる報酬又は料金の区分に応じ、同表の中欄に掲げる金額につき同表の下欄に掲げる金額とする。
\end{description}
\noindent\hspace{10pt}(報酬又は料金に係る源泉徴収の免除を受ける者の要件)
\begin{description}
\item[第三百二十三条]法第二百六条第一項(源泉徴収を要しない報酬又は料金)に規定する政令で定める要件は、同項に規定する報酬又は料金の支払を受ける居住者が当該報酬又は料金をその備え付ける帳簿に明確に記録していることのほか、次のいずれか一に該当することとする。
\begin{description}
\item[一]映画又はレコード(録音のテープ及びワイヤーを含む。)の製作を主たる事業としていること。
\item[二]自ら主催してその所有する劇場において定期的に演劇の公演を行なつていること。
\item[三]自ら主催して興行場において定期的に演劇の公演を行なうことを主たる事業としていること。
\item[四]主として自己に専属する芸能人をもつて演劇の製作及びその製作した演劇の公演を行なうことを主たる事業としていること。
\end{description}
\end{description}
\noindent\hspace{10pt}(報酬又は料金に係る源泉徴収の免除を受けるための手続)
\begin{description}
\item[第三百二十四条]法第二百六条第一項(源泉徴収を要しない報酬又は料金)の証明書の交付を受けようとする居住者は、次に掲げる事項を記載した申請書を納税地の所轄税務署長に提出しなければならない。
\begin{description}
\item[一]その者の氏名及び住所(国内に住所がないときは、居所)
\item[二]法第二百六条第一項に規定する報酬又は料金がその者の備え付ける帳簿に明確に記録されていることの事実の詳細
\item[三]その者が現に行つている事業の概要及び前条各号の要件のいずれかに該当する事情の詳細
\item[四]交付を受けようとする当該証明書の部数及び当該証明書を二部以上必要とするときは、その必要とする事情の詳細
\item[五]その他参考となるべき事項
\end{description}
\end{description}
\noindent\hspace{10pt}(源泉徴収の免除の要件に該当しなくなつた場合の手続等)
\begin{description}
\item[第三百二十五条]法第二百六条第一項(源泉徴収を要しない報酬又は料金)の証明書の交付を受けている居住者は、同条第二項の規定に該当する場合には、次に掲げる事項を記載した届出書に当該証明書を添付して、これを納税地の所轄税務署長に提出しなければならない。
\begin{description}
\item[一]その者の氏名及び住所(国内に住所がないときは、居所)
\item[二]法第二百六条第一項に規定する要件に該当しないこととなる旨
\item[三]その他参考となるべき事項
\end{description}
\item[\rensuji{2}]前項に規定する証明書の交付を受けている居住者は、その交付を受けた後、その者の氏名又は住所若しくは居所を変更した場合には、変更前の氏名及び変更後の氏名又は変更前の住所若しくは居所及び変更後の住所若しくは居所を記載した届出書にその証明書を添付し、これを納税地の所轄税務署長に提出しなければならない。
\item[\rensuji{3}]法第二百六条第三項第三号の通知をした税務署長は、遅滞なくその旨を公示するものとする。
\end{description}
\subsection*{第二節 生命保険契約等に基づく年金に係る源泉徴収}
\addcontentsline{toc}{subsection}{第二節 生命保険契約等に基づく年金に係る源泉徴収}
\noindent\hspace{10pt}(生命保険契約等に基づく年金に係る源泉徴収)
\begin{description}
\item[第三百二十六条]法第二百七条(源泉徴収義務)に規定する政令で定める年金は、確定給付企業年金法第百二条第三項又は第六項(事業主等又は連合会に対する監督)の規定による承認の取消しを受けた当該取消しに係るこれらの規定に規定する規約型企業年金に係る規約に基づきその取消しを受けた時以後に行う同法第八十九条第六項(清算)に規定する残余財産として分配される年金、同法第百二条第六項の規定による解散の命令を受けた同項に規定する基金の同法第十一条第一項(基金の規約で定める事項)に規定する規約に基づきその命令を受けた時以後に行う同法第八十九条第六項に規定する残余財産として分配される年金及び第七十六条第二項第一号(退職金共済制度等に基づく一時金で退職手当等とみなさないもの)に掲げる給付で年金として支払われるものとする。
\item[\rensuji{2}]法第二百七条第三号に規定する政令で定める契約は、次に掲げる契約とする。
\begin{description}
\item[一]保険業法第二条第四項(定義)に規定する損害保険会社若しくは同条第九項に規定する外国損害保険会社等又は同条第三項に規定する生命保険会社若しくは同条第八項に規定する外国生命保険会社等の締結した身体の傷害に基因して保険金が支払われる保険契約(法第七十七条第二項第一号(地震保険料控除)に掲げるもの及び当該外国損害保険会社等又は当該外国生命保険会社等が国外において締結したものを除く。)
\item[二]中小企業等協同組合法第九条の二第七項(事業協同組合及び事業協同小組合)に規定する共済事業(第六号において「共済事業」という。)を行う事業協同組合若しくは事業協同小組合又は協同組合連合会(同号において「事業協同組合等」という。)の締結した生命共済に係る契約(第二百十条第四号(生命共済契約等の範囲)に掲げる契約に該当するものを除く。)
\item[三]農業協同組合法第十条第一項第十号(共済に関する施設)の事業を行う農業協同組合又は農業協同組合連合会の締結した身体の傷害又は医療費の支出に関する共済に係る契約
\item[四]水産業協同組合法第十一条第一項第十一号(漁業協同組合の組合員の共済に関する事業)若しくは第九十三条第一項第六号の二(水産加工業協同組合の組合員の共済に関する事業)の事業を行う漁業協同組合若しくは水産加工業協同組合又は共済水産業協同組合連合会の締結した身体の傷害に関する共済に係る契約
\item[五]消費生活協同組合法第十条第一項第四号(組合員の生活の共済を図る事業)の事業を行う消費生活協同組合連合会の締結した身体の傷害に関する共済に係る契約
\item[六]共済事業を行う事業協同組合等の締結した身体の傷害又は医療費の支出に関する共済に係る契約
\item[七]法第七十七条第二項第二号及び第三号から前号までに掲げる契約のほか、法律の規定に基づく共済に関する事業を行う法人の締結した火災共済若しくは自然災害共済又は身体の傷害若しくは医療費の支出に関する共済に係る契約でその事業及び契約の内容がこれらの規定に掲げる契約に準ずるもの
\end{description}
\item[\rensuji{3}]法第二百八条(徴収税額)に規定する政令で定めるところにより計算した金額は、次の各号に掲げる年金の区分に応じ、当該年金の額に当該各号に定める割合を乗じて計算した金額とする。
\begin{description}
\item[一]法第七十六条第六項第一号から第四号まで(生命保険料控除)に掲げる契約のうち生命保険契約(第百八十三条第三項第一号(生命保険契約等に基づく年金に係る雑所得の金額の計算上控除する保険料等)に規定する生命保険契約をいう。次号において同じ。)、旧簡易生命保険契約(第百八十三条第三項第一号に規定する旧簡易生命保険契約をいう。)及び生命共済に係る契約に基づく年金、第一項に規定する年金又は前項第二号に掲げる生命共済に係る契約に基づく年金
\item[二]法第七十六条第六項第四号に掲げる契約で生命保険契約以外のもの、法第七十七条第二項各号に掲げる契約又は前項各号(第二号を除く。)に掲げる契約に基づく年金
\end{description}
\item[\rensuji{4}]法第二百九条第一号(源泉徴収を要しない年金)に規定する政令で定めるところにより計算した金額は、前項各号に掲げる年金の区分に応じ、当該年金の年額に当該各号に定める割合を乗じて計算した金額とする。
\item[\rensuji{5}]法第二百九条第一号に規定する政令で定める金額は、二十五万円とする。
\item[\rensuji{6}]法第二百九条第二号に規定する政令で定める契約は、次に掲げる契約とする。
\begin{description}
\item[一]法第二百七条に規定する契約に基づく年金の支払を受ける者(以下この項において「年金受取人」という。)と法第二百九条第二号に規定する保険契約者(以下この項において「保険契約者」という。)とが異なる契約(第三号に規定する団体保険に係る契約を除く。)のうち、当該契約に基づく保険金、共済金その他の給付金(以下この項において「保険金等」という。)の支払の基因となる事由(当該年金受取人に係る事由に限る。以下この項において「支払事由」という。)が生じた日以後において、当該保険金等を年金として支給することとされた契約以外のもの
\item[二]年金受取人と保険契約者とが同一である契約のうち、当該契約に基づく保険金等の支払事由が生じたことにより当該保険契約者の変更が行われたもので、当該支払事由が生じた日以後において、当該保険金等を年金として支給することとされた契約以外のもの
\item[三]団体保険(普通保険約款において、団体の代表者を保険契約者とし、当該団体に所属する者を保険法(平成二十年法律第五十六号)第二条第四号(定義)に規定する被保険者(以下この号において「被保険者」という。)とすることとなつている保険をいう。)に係る契約であつて、当該被保険者と当該契約に基づく年金受取人とが異なるもののうち、当該契約に基づく保険金等の支払事由が生じた日以後において、当該保険金等を年金として支給することとされた契約以外のもの
\end{description}
\end{description}
\subsection*{第三節 匿名組合契約等の利益の分配に係る源泉徴収}
\addcontentsline{toc}{subsection}{第三節 匿名組合契約等の利益の分配に係る源泉徴収}
\noindent\hspace{10pt}(匿名組合契約等の範囲)
\begin{description}
\item[第三百二十七条]法第二百十条(源泉徴収義務)に規定する政令で定める契約は、第二百八十八条(匿名組合契約に準ずる契約の範囲)に規定する契約とする。
\end{description}
\section*{第四章 非居住者又は法人の所得に係る源泉徴収}
\addcontentsline{toc}{section}{第四章 非居住者又は法人の所得に係る源泉徴収}
\noindent\hspace{10pt}(源泉徴収を要しない国内源泉所得)
\begin{description}
\item[第三百二十八条]法第二百十二条第一項(源泉徴収義務)に規定する政令で定める国内源泉所得は、次に掲げる国内源泉所得とする。
\begin{description}
\item[一]映画若しくは演劇の俳優、音楽家その他の芸能人又は職業運動家の役務の提供に係る法第百六十一条第一項第六号又は第十二号イ(国内源泉所得)に掲げる対価又は報酬で不特定多数の者から支払われるもの
\item[二]非居住者又は外国法人が有する土地若しくは土地の上に存する権利又は家屋(以下この号において「土地家屋等」という。)に係る法第百六十一条第一項第七号に掲げる対価で、当該土地家屋等を自己又はその親族の居住の用に供するために借り受けた個人から支払われるもの
\item[三]法第百六十九条(分離課税に係る所得税の課税標準)に規定する非居住者に対し支払われる法第百六十一条第一項第十二号イ又はハに掲げる給与又は報酬で、その者が法第百七十二条(給与等につき源泉徴収を受けない場合の申告納税等)の規定によりその支払の時までに既に納付した所得税の額の計算の基礎とされたもの
\end{description}
\end{description}
\noindent\hspace{10pt}(組合員に類する者の範囲)
\begin{description}
\item[第三百二十八条の二]法第二百十二条第五項(源泉徴収義務)に規定する組合員に類する者で政令で定めるものは、同項に規定する組合契約を締結していた組合員並びに第二百八十一条の二第一項第三号(恒久的施設を通じて行う組合事業から生ずる利益)に掲げる契約を締結している者及び当該契約を締結していた者とする。
\end{description}
\noindent\hspace{10pt}(金銭以外のもので支払われる賞金の価額等)
\begin{description}
\item[第三百二十九条]法第二百十三条第一項第一号ロ(非居住者又は外国法人の所得に係る徴収税額)に規定する政令で定めるところにより計算した金額は、同号ロに規定する金銭以外のものにつき第三百二十一条(金銭以外のもので支払われる賞金の価額)の規定に準じて計算した金額とする。
\item[\rensuji{2}]法第二百十三条第一項第一号ハに規定する政令で定めるところにより計算した金額は、同号ハに規定する支払われる年金の額につき第二百九十六条(生命保険契約等に基づく年金等に係る課税標準)の規定に準じて計算した金額とする。
\item[\rensuji{3}]法第二百十三条第二項第三号に規定する政令で定める金額は、第二百九十八条第一項(内国法人に係る所得税の課税標準)に規定する金額とする。
\end{description}
\noindent\hspace{10pt}(非居住者が源泉徴収の免除を受けるための要件)
\begin{description}
\item[第三百三十条]法第二百十四条第一項(源泉徴収を要しない非居住者の国内源泉所得)に規定する政令で定める要件は、次に掲げる要件とする。
\begin{description}
\item[一]法第二百二十九条(開業等の届出)の規定による届出書を提出していること。
\item[二]納税地に現住しない非居住者については、その者が国税通則法第百十七条第二項(納税管理人)の規定による納税管理人の届出をしていること。
\item[三]その年の前年分の所得税に係る確定申告書を提出していること。
\item[四]法第二百十四条第一項の規定の適用を受けようとする同項に規定する対象国内源泉所得が、法その他所得税に関する法令(法第二条第一項第八号の四ただし書(定義)に規定する条約を含む。)の規定により法第百六十五条第一項(総合課税に係る所得税の課税標準、税額等の計算)に規定する総合課税に係る所得税を課される所得のうちに含まれるものであること。
\item[五]偽りその他不正の行為により所得税を免れたことがないこと。
\item[六]法第二百十四条第一項の規定の適用を受けるために同項の証明書を同項に規定する対象国内源泉所得の支払者に提示する場合において、当該支払者の氏名又は名称及びその住所、事務所、事業所その他当該対象国内源泉所得の支払の場所並びにその提示した年月日を帳簿に記録することが確実であると見込まれること。
\end{description}
\end{description}
\noindent\hspace{10pt}(非居住者が源泉徴収の免除を受けるための手続等)
\begin{description}
\item[第三百三十一条]法第二百十四条第一項(源泉徴収を要しない非居住者の国内源泉所得)の証明書の交付を受けようとする者は、次に掲げる事項を記載した申請書を納税地の所轄税務署長に提出しなければならない。
\begin{description}
\item[一]その者の氏名及び住所並びに国内に居所があるときは当該居所
\item[二]その者の恒久的施設を通じて行う事業に係る事務所、事業所その他これらに準ずるもの(これらが二以上あるときは、そのうち主たるもの。第三百三十三条第一項第一号(非居住者が源泉徴収の免除の要件に該当しなくなつた場合の手続等)において「国内にある事務所等」という。)の名称及び所在地並びにその代表者その他の責任者の氏名並びに国税通則法第百十七条第二項(納税管理人)の規定により届け出た納税管理人が当該責任者と異なるときは、納税管理人の氏名
\item[三]前条第一号に規定する届出書を提出した年月日
\item[四]前条第四号に掲げる要件に該当する事情の概要
\item[五]前条第六号の記録を確実に行う旨
\item[六]当該証明書により法第二百十四条第一項の規定の適用を受けようとする同項に規定する対象国内源泉所得のうち主たるものの支払者の氏名又は名称、その住所、事務所、事業所その他当該対象国内源泉所得の支払の場所及びその支払の宛先並びに当該対象国内源泉所得の種類及び当該対象国内源泉所得の支払を受ける見込期間
\item[七]当該証明書により法第二百十四条第一項の規定の適用を受けようとする国内源泉所得がその者の同項に規定する対象国内源泉所得に該当する事情
\item[八]その他参考となるべき事項
\end{description}
\item[\rensuji{2}]第三百五条第二項及び第三項(外国法人が課税の特例の適用を受けるための手続等)の規定は、非居住者に係る法第二百十四条第一項の証明書について準用する。
\end{description}
\noindent\hspace{10pt}(源泉徴収を免除されない非居住者の国内源泉所得)
\begin{description}
\item[第三百三十二条]法第二百十四条第一項(源泉徴収を要しない非居住者の国内源泉所得)に規定する政令で定める国内源泉所得は、次に掲げる国内源泉所得とする。
\begin{description}
\item[一]法第百六十一条第一項第十一号(国内源泉所得)に掲げる使用料又は対価で法第二百四条第一項第一号(源泉徴収義務)に掲げる報酬又は料金に該当するもの
\item[二]法第百六十一条第一項第十二号イに掲げる報酬で法第二百四条第一項第五号に掲げる人的役務の提供に関する報酬又は料金に該当するもの以外のもの
\item[三]法第百六十一条第一項第十四号に掲げる年金でその支払額が二十五万円以上のもの
\end{description}
\end{description}
\noindent\hspace{10pt}(非居住者が源泉徴収の免除の要件に該当しなくなつた場合の手続等)
\begin{description}
\item[第三百三十三条]法第二百十四条第一項(源泉徴収を要しない非居住者の国内源泉所得)の証明書の交付を受けている者は、同条第二項に規定する場合には、次に掲げる事項を記載した届出書に当該証明書を添付し、これを納税地の所轄税務署長に提出するとともに、その者が当該証明書を提示した国内源泉所得の支払者に対しその旨を遅滞なく通知しなければならない。
\begin{description}
\item[一]その者の国内にある事務所等の名称及び所在地並びにその代表者その他の責任者の氏名並びに国税通則法第百十七条第二項(納税管理人)の規定により届け出た納税管理人が当該責任者と異なるときは、納税管理人の氏名
\item[二]第三百三十条各号(非居住者が源泉徴収の免除を受けるための要件)に掲げる要件に該当しないこととなり、又は恒久的施設を有しないこととなつた事情の詳細
\item[三]その者が当該証明書を提示した国内源泉所得の支払者の氏名又は名称及びその住所、事務所、事業所その他当該国内源泉所得の支払の場所
\item[四]その他参考となるべき事項
\end{description}
\item[\rensuji{2}]前項に規定する者は、同項の証明書に係る第三百三十一条第一項(非居住者が源泉徴収の免除を受けるための手続等)の申請書に記載した同項第一号又は第二号に掲げる事項に変更があつた場合には、遅滞なく、その旨を記載した届出書を納税地の所轄税務署長に提出しなければならない。
\end{description}
\noindent\hspace{10pt}(非居住者の給与又は報酬で源泉徴収が行われたものとみなされるもの)
\begin{description}
\item[第三百三十四条]法第二百十五条(非居住者の人的役務の提供による給与等に係る源泉徴収の特例)の規定により所得税の徴収が行われたものとみなされる給与又は報酬の金額は、法第百六十一条第一項第六号(国内源泉所得)に規定する事業を国内において行う者の当該国内において行う事業につき支払を受けた同号に掲げる対価の総額が当該国内において行う事業のために人的役務の提供をする各非居住者に対しその人的役務の提供につき支払うべき同項第十二号イ又はハに掲げる給与又は報酬の金額の合計額に満たなかつた場合には、当該対価の総額に、当該合計額のうちに当該各非居住者に対し支払うべき当該給与又は報酬の金額の占める割合を乗じて計算した金額とする。
\end{description}
\part*{第五編 雑則}
\addcontentsline{toc}{part}{第五編 雑則}
\noindent\hspace{10pt}(告知義務のない利子等及び公共法人等の範囲)
\begin{description}
\item[第三百三十五条]法第二百二十四条第一項(利子、配当等の受領者の告知)に規定する普通預金の利子その他の政令で定めるものは、次に掲げる利子及び収益の分配とする。
\begin{description}
\item[一]当座預金、普通預金、普通貯金、通知預金、通知貯金及び財務省令で定める別段預金の利子
\item[二]第二条第一号及び第二号(預貯金の範囲)に掲げる貯蓄金及び貯金の利子
\item[三]法第九条第一項第二号(非課税所得)に規定する預貯金の利子又は合同運用信託の収益の分配
\item[四]納税貯蓄組合法(昭和二十六年法律第百四十五号)第二条第二項(定義)に規定する納税貯蓄組合預金の利子及び財務省令で定める納税準備預金の利子
\end{description}
\item[\rensuji{2}]法第二百二十四条第一項に規定する法人税法別表第一(公共法人の表)に掲げる法人その他の政令で定めるものは、国並びに次に掲げる法人及び国際機関(以下この編において「公共法人等」という。)とする。
\begin{description}
\item[一]法人税法別表第一に掲げる法人
\item[二]特別の法律により設立された法人(当該特別の法律において、その法人の名称が定められ、かつ、当該名称として用いられた文字を他の者の名称の文字として用いてはならない旨の定めのあるものに限る。)
\item[三]外国政府、外国の地方公共団体及び第二十三条(職員の給与が非課税とされる国際機関の範囲)に規定する国際機関
\end{description}
\end{description}
\noindent\hspace{10pt}(預貯金、株式等に係る利子、配当等の受領者の告知)
\begin{description}
\item[第三百三十六条]国内において法第二百二十四条第一項(利子、配当等の受領者の告知)に規定する利子等(以下この条において「利子等」という。)又は同項に規定する配当等(以下この条において「配当等」という。)につき支払を受ける者(公共法人等を除く。以下この条において同じ。)は、その利子等又は配当等につきその支払の確定する日までに、その確定の都度、その者の氏名又は名称、住所(国内に住所を有しない者にあつては、同項に規定する財務省令で定める場所。以下この条、次条第三項及び第四項並びに第三百三十八条(貯蓄取扱機関等の営業所の長の確認等)において同じ。)及び個人番号又は法人番号(個人番号及び法人番号を有しない者又は第四項の規定に該当する個人(第三百三十八条第一項及び第二項において「番号既告知者」という。)にあつては、氏名又は名称及び住所。次項において同じ。)を、その利子等又は配当等の支払をする者の営業所、事務所その他これらに準ずるものでその支払事務の取扱いをするものの長(第五項第一号に掲げる者を含む。以下この条において「支払事務取扱者」という。)に告知しなければならない。
\item[\rensuji{2}]利子等又は配当等につき支払を受ける者が次の各号に掲げる場合のいずれかに該当するときは、その者は、その支払を受ける当該各号に定める利子等又は配当等につき前項の規定による告知をしたものとみなす。
\begin{description}
\item[一]利子等又は配当等(法第二十四条第一項(配当所得)に規定する投資信託(第五号に規定する特定株式投資信託及び特定不動産投資信託を除く。)及び特定受益証券発行信託の収益の分配に限る。以下この号から第四号までにおいて同じ。)につき支払を受ける者が、銀行、信託会社その他の財務省令で定める者(以下この条及び第三百三十九条(無記名公社債の利子等に係る告知書等の提出等)において「金融機関」という。)の営業所、事務所その他これらに準ずるもの(以下この条及び第三百三十九条において「営業所等」という。)において当該利子等又は配当等を生ずべき預貯金、合同運用信託(貸付信託を除く。)、公社債又は貸付信託、投資信託若しくは特定受益証券発行信託の受益権(以下この条において「預貯金等」という。)の預入、信託又は購入(以下この条において「預入等」という。)をする場合において、その預入等をする際、その者の氏名又は名称、住所及び個人番号又は法人番号を、その預入等をする金融機関の営業所等の長に告知しているとき
\item[二]利子等又は配当等につき支払を受ける者が、金融機関の営業所等において反復して預貯金等の預入等をすることを約する契約その他の財務省令で定める契約に基づき預貯金等の預入等をする場合において、当該契約に基づき最初にその預入等をする際、その者の氏名又は名称、住所及び個人番号又は法人番号を、当該金融機関の営業所等の長に告知しているとき
\item[三]利子等又は配当等につき支払を受ける者が、金融機関の営業所等において金融機関が社債、株式等の振替に関する法律の規定により備え付ける振替口座簿又は金融機関の営業所等を通じて当該金融機関以外の振替機関等(同法第二条第五項(定義)に規定する振替機関等をいい、同法第四十八条(日本銀行が国債の振替に関する業務を営む場合の特例)の規定により同法第二条第二項に規定する振替機関とみなされる者を含む。)が同法の規定により備え付ける振替口座簿に係る口座の開設を受ける際、その者の氏名又は名称、住所及び個人番号又は法人番号を、当該金融機関の営業所等の長に告知している場合
\item[四]利子等又は配当等につき支払を受ける者が、当該利子等又は配当等を生ずべき預貯金等(法第二百二十四条の二(譲渡性預金の譲渡等に関する告知)に規定する譲渡性預金を除く。)の譲受け又は相続その他の方法による取得をした場合において、当該預貯金等の証書、証券その他これらに類するものの名義の変更又は書換えの請求(当該譲受けにつき当該預貯金等の受入れをする者の承諾を要するときは、その承諾の依頼を含む。)をする際、その者の氏名又は名称、住所及び個人番号又は法人番号を、当該名義の変更又は書換えの請求の取扱いをする金融機関の営業所等の長に告知しているとき
\item[五]特定株式投資信託(信託財産を株式のみに対する投資として運用することを目的とする証券投資信託のうち、投資信託及び投資法人に関する法律第四条第一項(投資信託契約の締結)に規定する委託者指図型投資信託約款(当該証券投資信託が同法第二条第二十四項(定義)に規定する外国投資信託である場合には、当該委託者指図型投資信託約款に類する書類)にイからニまでに掲げる事項の定めがあること、その受益権が金融商品取引所(金融商品取引法第二条第十六項(定義)に規定する金融商品取引所をいう。以下この号において同じ。)に上場されていることその他財務省令で定める要件を満たすものをいう。以下この号及び第三百三十九条第八項において同じ。)又は特定不動産投資信託(証券投資信託以外の投資信託で公社債等運用投資信託に該当しないもののうち、当該投資信託の投資信託約款(投資信託及び投資法人に関する法律第四条第一項に規定する委託者指図型投資信託約款又は同法第四十九条第一項(投資信託契約の締結)に規定する委託者非指図型投資信託約款をいう。)にロ、ハ及びホに掲げる事項の定めがあること、その受益権が金融商品取引所に上場されていることその他財務省令で定める要件を満たすものをいう。以下この号及び第三百三十九条第八項において同じ。)の配当等につき支払を受ける者が、財務省令で定めるところにより、当該配当等につき支払を受けるべき者としてその者の氏名又は名称、住所及び個人番号又は法人番号をその配当等の支払事務取扱者に登録をした場合において、その登録の際、その者の氏名又は名称、住所及び個人番号又は法人番号を、当該支払事務取扱者又は当該登録の取次ぎをする金融機関の営業所等の長に告知しているとき
\begin{description}
\item[イ]信託契約期間を定めないこと(当該投資信託が証券投資信託に該当する投資信託及び投資法人に関する法律第二条第二十四項に規定する外国投資信託(以下この号において「外国証券投資信託」という。)である場合には、信託契約期間を定めないこと又は当該外国証券投資信託の設定がされた国の法令の定めるところにより信託契約期間(財務省令で定める期間に限る。)が定められていること。)。
\item[ロ]当該投資信託の受益権が金融商品取引所に上場することとされていること(当該投資信託が外国証券投資信託である場合には、その受益権が金融商品取引法第二条第八項第三号ロに規定する外国金融商品市場に上場することとされていること。)。
\item[ハ]受益者は、その有する受益権(その証券投資信託の受託者が投資信託及び投資法人に関する法律第十七条第一項第二号(投資信託約款の変更等)に規定する重大な約款の変更等に反対した受益者からの同法第十八条第一項(反対受益者の受益権買取請求)の規定による請求により買い取つた受益権を除く。)について、その信託契約期間中に当該信託契約の一部解約を請求することができないこと。
\item[ニ]信託財産は特定の株価指数(金融商品取引法第二条第十七項に規定する取引所金融商品市場又は同条第八項第三号ロに規定する外国金融商品市場に上場されている株式について多数の銘柄の価格の水準を総合的に表した指数をいう。)に採用されている銘柄の株式に投資を行い、その信託財産の受益権一口当たりの純資産額の変動率を当該特定の株価指数の変動率に一致させることを目的とした運用を行うこと。
\item[ホ]信託財産の総額のうちに占める不動産等(投資信託及び投資法人に関する法律施行令(平成十二年政令第四百八十号)第三条第三号(特定資産の範囲)に掲げる不動産、同条第四号に掲げる不動産の賃借権、同条第五号に掲げる地上権その他財務省令で定める資産(以下この号において「不動産等資産」という。)及び同条第一号に掲げる有価証券のうち金融商品取引法第二条第二項第一号に掲げる受益権で不動産等資産のみを信託する信託に係るものをいう。)の価額の割合として財務省令で定める割合を百分の七十以上とすること。
\end{description}
\item[六]配当等(法第二十四条第一項に規定する投資信託及び特定受益証券発行信託の収益の分配を除く。以下この項において同じ。)につき支払を受ける者が、当該配当等を生ずべき株式(投資信託及び投資法人に関する法律第二条第十四項に規定する投資口を含む。)若しくは法人の社員、会員、組合員その他の出資者の持分(これに類するものを含む。以下この条において「株式等」という。)を払込みにより取得した場合又は株式等を購入若しくは相続その他の方法により取得した場合において、当該払込みにより取得をする際又は当該株式等の名義の変更若しくは書換えの請求をする際、その者の氏名又は名称、住所及び個人番号又は法人番号を、当該株式等に係る配当等の支払事務取扱者に告知しているとき
\item[七]配当等につき支払を受ける者が、金融機関の営業所等において金融機関が社債、株式等の振替に関する法律の規定により備え付ける振替口座簿又は金融機関の営業所等を通じて当該金融機関以外の振替機関等(同法第二条第五項に規定する振替機関等をいう。)が同法の規定により備え付ける振替口座簿に係る口座の開設を受ける際、その者の氏名又は名称、住所及び個人番号又は法人番号を、当該金融機関の営業所等の長に告知している場合
\end{description}
\item[\rensuji{3}]前項の場合において、同項各号に定める利子等又は配当等の支払を受ける者が、同項各号の告知をした後、次の各号に掲げる場合に該当することとなつた場合には、その者は、その該当することとなつた日以後最初に当該利子等又は配当等の支払の確定する日までに、当該各号に掲げる場合の区分に応じ当該各号に定める事項を、当該利子等又は配当等に係る支払事務取扱者又は第五項第二号に掲げる金融機関の営業所等の長に告知しなければならない。
\begin{description}
\item[一]その者の氏名若しくは名称又は住所の変更をした場合
\item[二]その者の個人番号の変更をした場合
\item[三]行政手続における特定の個人を識別するための番号の利用等に関する法律の規定により個人番号又は法人番号が初めて通知された場合
\end{description}
\item[\rensuji{4}]法第二百二十四条第一項に規定する政令で定める者は、利子等又は配当等の支払事務取扱者(次項第二号に掲げる金融機関の営業所等の長を含む。次条及び第三百三十八条において「貯蓄取扱機関等の営業所の長」という。)が、財務省令で定めるところにより、当該利子等又は配当等の支払を受ける個人の氏名、住所及び個人番号その他の事項を記載した帳簿(当該個人の次条第二項第一号に定める書類の提示又は法第二百二十四条第一項に規定する署名用電子証明書等(以下この編において「署名用電子証明書等」という。)の送信を受けて作成されたものに限る。)を備えている場合における当該個人(当該個人の氏名、住所又は個人番号が当該帳簿に記載されている当該個人の氏名、住所又は個人番号と異なる場合における当該個人を除く。)とする。
\item[\rensuji{5}]法第二百二十四条第一項に規定する利子等又は配当等の支払をする者に準ずる者として政令で定めるものは、次に掲げる者とする。
\begin{description}
\item[一]法第二百二十五条第一項第一号及び第二号(支払調書)に規定する支払の取扱者並びに当該支払の取扱者以外の者で法第二百二十八条第一項(名義人受領の配当所得等の調書)に規定する利子等又は配当等の支払を受ける者に該当する者
\item[二]第二項第一号若しくは第二号の預入等をする金融機関の営業所等の長、同項第三号に規定する口座に係る同号の金融機関の営業所等の長、同項第四号に規定する名義の変更若しくは書換えの請求の取扱いをする金融機関の営業所等の長、同項第五号に規定する登録の取次ぎをする金融機関の営業所等の長又は同項第七号に規定する口座に係る同号の金融機関の営業所等の長がこれらの規定に規定する預貯金等に係る利子等又は配当等の支払事務取扱者に該当しない場合における当該金融機関の営業所等の長
\end{description}
\item[\rensuji{6}]利子等又は配当等が法第十条第一項(障害者等の少額預金の利子所得等の非課税)、第十一条第二項(公益信託等に係る非課税)、第百七十六条第一項若しくは第二項(信託財産に係る利子等の課税の特例)若しくは第百八十条の二第一項若しくは第二項(信託財産に係る利子等の課税の特例)の規定又は租税特別措置法第四条第一項(障害者等の少額公債の利子の非課税)、第四条の二第一項(勤労者財産形成住宅貯蓄の利子所得等の非課税)、第四条の三第一項(勤労者財産形成年金貯蓄の利子所得等の非課税)、第四条の五第一項(特定寄附信託の利子所得の非課税)、第八条第一項から第三項まで(金融機関等の受ける利子所得等に対する源泉徴収の不適用)、第九条の四(特定の投資法人等の運用財産等に係る利子等の課税の特例)、第九条の四の二第一項(上場証券投資信託等の償還金等に係る課税の特例)若しくは第九条の五第一項(公募株式等証券投資信託の受益権を買い取つた金融商品取引業者等が支払を受ける収益の分配に係る源泉徴収の特例)の規定の適用を受けるものである場合には、当該利子等又は配当等については、第一項の規定による告知は、要しない。
\end{description}
\noindent\hspace{10pt}(告知に係る住民票の写しその他の書類の提示等)
\begin{description}
\item[第三百三十七条]前条第一項に規定する利子等又は配当等につき支払を受ける者は、同項から同条第三項までの規定による告知をする際、当該告知をする貯蓄取扱機関等の営業所の長に、次項に規定する書類を提示し、又は署名用電子証明書等を送信しなければならない。
\item[\rensuji{2}]法第二百二十四条第一項(利子、配当等の受領者の告知)に規定する政令で定める書類は、次の各号に掲げる者の区分に応じ当該各号に定めるいずれかの書類とする。
\begin{description}
\item[一]個人
\item[二]法人
\end{description}
\item[\rensuji{3}]前条第二項各号の告知をした個人が、同条第三項第一号に掲げる場合に該当することとなつた場合において、同項の規定による告知をするときは、第一項の規定による書類の提示又は署名用電子証明書等の送信に代えて、住所等変更確認書類(当該個人の変更前の氏名又は住所及び変更後の氏名又は住所を証する住民票の写しその他の財務省令で定める書類をいう。次条第一項において同じ。)の提示をすることができる。
\item[\rensuji{4}]前条第一項に規定する利子等又は配当等につき支払を受ける者で財務省令で定めるものが貯蓄取扱機関等の営業所の長に同項から同条第三項までの規定による告知をする場合において、当該貯蓄取扱機関等の営業所の長が、財務省令で定めるところにより、その支払を受ける者の氏名又は名称、住所及び個人番号又は法人番号(個人番号及び法人番号を有しない者にあつては、氏名又は名称及び住所。以下この項において同じ。)その他の事項を記載した帳簿(その者から申請書(その者の第二項各号に定めるいずれかの書類の写しを添付したもの又はその提出の際にその者の署名用電子証明書等の送信を受けているものに限る。)の提出を受けて作成されたものに限る。)を備えているときは、その支払を受ける者は、第一項の規定にかかわらず、当該貯蓄取扱機関等の営業所の長に対しては、同項に規定する書類の提示又は署名用電子証明書等の送信を要しないものとする。
\end{description}
\noindent\hspace{10pt}(貯蓄取扱機関等の営業所の長の確認等)
\begin{description}
\item[第三百三十八条]貯蓄取扱機関等の営業所の長は、第三百三十六条第一項から第三項まで(預貯金、株式等に係る利子、配当等の受領者の告知)の規定による告知があつた場合には、当該告知があつた氏名又は名称、住所及び個人番号又は法人番号(個人番号及び法人番号を有しない者、番号既告知者又は同項の規定による告知をした個人(当該告知の際に前条第三項の規定により住所等変更確認書類を提示した個人に限る。次項において「住所等変更告知者」という。)にあつては、氏名又は名称及び住所。以下この条において同じ。)が、当該告知の際に提示又は送信を受けた前条第二項に規定する書類若しくは住所等変更確認書類又は署名用電子証明書等に記載又は記録がされた氏名又は名称、住所及び個人番号又は法人番号と同じであるかどうかを確認しなければならない。
\item[\rensuji{2}]前項の確認をした貯蓄取扱機関等の営業所の長が当該確認に係る利子等又は配当等の第三百三十六条第一項に規定する支払事務取扱者でないときは、当該貯蓄取扱機関等の営業所の長は、遅滞なく、当該利子等又は配当等に係る当該支払事務取扱者に対し、当該確認をした氏名又は名称、住所及び個人番号又は法人番号並びに当該確認をした旨(番号既告知者又は住所等変更告知者について前項の確認をした場合には、当該確認をした氏名及び住所、当該確認をした旨並びに当該番号既告知者又は住所等変更告知者の個人番号。次項において同じ。)を、通知しなければならない。
\item[\rensuji{3}]貯蓄取扱機関等の営業所の長は、第三百三十六条第一項から第三項までの規定による告知(以下この項において「告知」という。)に係る公社債につき国債に関する法律(明治三十九年法律第三十四号)の規定による登録の取次ぎをする場合又は告知に係る公社債若しくは貸付信託、投資信託、特定受益証券発行信託若しくは特定目的信託の受益権につき社債、株式等の振替に関する法律に規定する振替口座簿への記載若しくは記録に係る振替の取次ぎ若しくは保管の委託の取次ぎをする場合には、その登録の取次ぎ又はその振替の取次ぎ若しくは保管の委託の取次ぎをする際、当該登録の取扱いをする者又は当該振替口座簿に記載若しくは記録をする者若しくは当該保管の委託を受ける者に対し、第一項の確認をした氏名又は名称、住所及び個人番号又は法人番号並びに当該確認をした旨を、通知しなければならない。
\item[\rensuji{4}]貯蓄取扱機関等の営業所の長(前項に規定する登録の取扱いをする者並びに同項に規定する振替口座簿に記載又は記録をする者及び保管の委託を受ける者を含む。)は、第一項の確認をした場合又は前二項の規定による通知を受けた場合には、財務省令で定めるところにより、当該確認又は通知に係る預貯金又は合同運用信託の受入れに関する帳簿、有価証券の振替に関する帳簿、株主名簿その他の有価証券の発行に関する帳簿(これらに類する帳簿又は書類を含む。)に、当該確認をした旨又は当該通知を受けた事実を明らかにし、かつ、これらの帳簿又は当該通知の内容を記載した書類を保存しなければならない。
\item[\rensuji{5}]貯蓄取扱機関等の営業所の長は、前項に規定する預貯金若しくは合同運用信託の受入れ若しくは有価証券の振替又は有価証券の発行に関する事務、第三項に規定する登録又は振替若しくは保管の委託に関する事務その他これらに類する事務の全部を他の貯蓄取扱機関等の営業所の長に移管する場合には、前項の帳簿又は書類を、その移管先の貯蓄取扱機関等の営業所の長に移管しなければならない。
\end{description}
\noindent\hspace{10pt}(無記名公社債の利子等に係る告知書等の提出等)
\begin{description}
\item[第三百三十九条]国内において無記名の公社債、法第二百二十四条第二項(利子、配当等の受領者の告知)の無記名株式等又は無記名の貸付信託、投資信託若しくは特定受益証券発行信託の受益証券(以下この条において「無記名公社債等」という。)に係る利子、法第二十四条第一項(配当所得)に規定する剰余金の配当又は収益の分配(以下この条において「利子等」という。)につき支払を受ける者(公共法人等を除く。以下この条において同じ。)は、その無記名公社債等の利子等についてその者の氏名又は名称、住所及び個人番号又は法人番号(個人番号及び法人番号を有しない者又は既に個人番号を告知している者として財務省令で定める者にあつては、氏名又は名称及び住所)その他の財務省令で定める事項を記載した告知書を、その支払を受ける際、その支払の取扱者に提出しなければならない。
\item[\rensuji{2}]無記名公社債等の利子等につき支払を受ける者が、法第二百二十八条第一項(名義人受領の配当所得等の調書)に規定する者を通じてその支払を受ける場合には、同項に規定する者をその支払の取扱者とみなして、前項の規定を適用する。
\item[\rensuji{3}]無記名公社債等の利子等につき支払を受ける者が、金融機関の営業所等(財務省令で定める金融機関の営業所等が行う保管の委託の取次ぎにより当該利子等を生ずべき無記名公社債等の保管の委託を受けたものを除く。)において当該利子等を生ずべき無記名公社債等の保管の委託に係る契約(当該財務省令で定める金融機関の営業所等が行う保管の委託の取次ぎにより当該利子等を生ずべき無記名公社債等の保管の委託をする場合には、当該保管の委託の取次ぎに係る契約(以下この条において「保管委託取次契約」という。))を締結する際、第一項に規定する告知書に当該契約(当該契約が保管委託取次契約である場合には、当該保管委託取次契約に係る保管の委託の契約。以下この項において同じ。)に基づき保管の委託をする無記名公社債等の種類その他の財務省令で定める事項を記載し、これを当該金融機関の営業所等の長に提出したときは、当該契約に基づき保管の委託をしている無記名公社債等の利子等(当該保管の委託をした日から引き続き保管の委託をしている期間内に支払を受ける利子等で、当該金融機関の営業所等の長がその支払の取扱いをするものに限る。)については、その支払を受ける都度、その支払を受ける際に第一項に規定する告知書の提出があつたものとみなす。
\item[\rensuji{4}]前項の規定による告知書の提出をした者が、当該告知書を提出した後、次の各号に掲げる場合に該当することとなつた場合には、その者は、その該当することとなつた日以後最初に同項の保管の委託をしている無記名公社債等の利子等の支払を受ける日までに、当該保管の委託をしている金融機関の営業所等の長(当該保管の委託が保管委託取次契約に係る保管の委託の契約に基づくものである場合には、当該保管委託取次契約に基づき当該無記名公社債等の保管の委託の取次ぎをした同項に規定する財務省令で定める金融機関の営業所等の長。第六項において同じ。)に当該各号に掲げる場合の区分に応じ当該各号に定める事項を記載した書類の提出をしなければならない。
\begin{description}
\item[一]その者の氏名若しくは名称又は住所の変更をした場合
\item[二]その者の個人番号の変更をした場合
\item[三]行政手続における特定の個人を識別するための番号の利用等に関する法律の規定により個人番号又は法人番号が初めて通知された場合
\end{description}
\item[\rensuji{5}]前項の規定は、同項の規定により同項の書類を提出した者が当該書類を提出した後、再び氏名若しくは名称、住所又は個人番号の変更をしたときについて準用する。
\item[\rensuji{6}]第三項の無記名公社債等の保管の委託を受けた金融機関の営業所等の長は、当該無記名公社債等の保管に関する帳簿(当該保管が保管委託取次契約に係る保管の委託の契約に基づくものである場合には、当該保管委託取次契約に基づく当該無記名公社債等の保管の委託の取次ぎに関する帳簿)を備え、各人別に、当該保管に係る無記名公社債等の種類、前項の書類に記載された事項その他の財務省令で定める事項を記載しなければならない。
\item[\rensuji{7}]無記名公社債等の利子等が法第十条第一項(障害者等の少額預金の利子所得等の非課税)、第十一条第二項(公益信託等に係る非課税)、第百七十六条第一項若しくは第二項(信託財産に係る利子等の課税の特例)若しくは第百八十条の二第一項若しくは第二項(信託財産に係る利子等の課税の特例)の規定又は租税特別措置法第四条第一項(障害者等の少額公債の利子の非課税)、第四条の二第一項(勤労者財産形成住宅貯蓄の利子所得等の非課税)、第四条の三第一項(勤労者財産形成年金貯蓄の利子所得等の非課税)、第四条の五第一項(特定寄附信託の利子所得の非課税)、第八条第一項から第三項まで(金融機関等の受ける利子所得等に対する源泉徴収の不適用)、第九条の四(特定の投資法人等の運用財産等に係る利子等の課税の特例)、第九条の四の二第一項(上場証券投資信託等の償還金等に係る課税の特例)若しくは第九条の五第一項(公募株式等証券投資信託の受益権を買い取つた金融商品取引業者等が支払を受ける収益の分配に係る源泉徴収の特例)の規定の適用を受けるものである場合には、当該無記名公社債等の利子等については、第一項の規定による告知書の提出は、要しない。
\item[\rensuji{8}]無記名の特定株式投資信託又は特定不動産投資信託の受益証券に係る利子等につき支払を受ける者が、財務省令で定めるところにより、当該利子等につき支払を受けるべき者としてその者の氏名又は名称、住所及び個人番号又は法人番号をその利子等の第三百三十六条第一項(預貯金、株式等に係る利子、配当等の受領者の告知)に規定する支払事務取扱者に登録をしている場合には、当該登録がされた無記名の特定株式投資信託又は特定不動産投資信託の受益証券に係る利子等は、無記名の投資信託の受益証券に係る収益の分配でないものとして、前三条の規定を適用する。
\item[\rensuji{9}]第三百三十七条(告知に係る住民票の写しその他の書類の提示等)の規定は第一項に規定する支払を受ける者が同項に規定する告知書の提出若しくは第三項の規定による告知書の提出又は第四項(第五項において準用する場合を含む。)に規定する書類の提出をする場合について、前条の規定は無記名公社債等の利子等の支払の取扱者(第二項の規定により支払の取扱者とみなされる者を含む。)がこれらの告知書又は書類を受理した場合について、それぞれ準用する。
\item[\rensuji{10}]第一項の告知書の様式は、財務省令で定める。
\end{description}
\noindent\hspace{10pt}(譲渡等に関する告知書を提出すべき譲渡性預金)
\begin{description}
\item[第三百四十条]法第二百二十四条の二(譲渡性預金の譲渡等に関する告知)に規定する譲渡禁止の特約のない預貯金で政令で定めるものは、準備預金制度に関する法律施行令(昭和三十二年政令第百三十五号)第四条第二号(指定勘定の区別)に規定する譲渡性預金であつて民法第三編第一章第七節第一款(指図証券)に規定する指図証券、同節第二款(記名式所持人払証券)に規定する記名式所持人払証券、同節第三款(その他の記名証券)に規定するその他の記名証券及び同節第四款(無記名証券)に規定する無記名証券に係る債権並びに電子記録債権法第二条第一項(定義)に規定する電子記録債権以外のものとする。
\end{description}
\noindent\hspace{10pt}(株式等の譲渡の対価に係る告知義務のない公共法人等の範囲)
\begin{description}
\item[第三百四十一条]法第二百二十四条の三第一項(株式等の譲渡の対価の受領者の告知)に規定する法人税法別表第一(公共法人の表)に掲げる法人その他の政令で定めるものは、公共法人等とする。
\end{description}
\noindent\hspace{10pt}(一株又は一口に満たない端数に係る規定)
\begin{description}
\item[第三百四十一条の二]法第二百二十四条の三第一項第三号(株式等の譲渡の対価の受領者の告知)に規定する株式等の競売に係る同号に規定する政令で定める規定は、投資信託及び投資法人に関する法律第八十八条第一項及び第百四十九条の十七第一項(一に満たない端数の処理)の規定並びに会社法第二百三十四条第六項(一に満たない端数の処理)において準用する同条第一項の規定とし、同号に規定する競売以外の方法による売却に係る同号に規定する政令で定める規定は、投資信託及び投資法人に関する法律第八十八条第一項及び第百四十九条の十七第一項の規定並びに会社法第二百三十四条第六項において準用する同条第二項の規定とする。
\end{description}
\noindent\hspace{10pt}(株式等の譲渡の対価の受領者の告知)
\begin{description}
\item[第三百四十二条]国内において法第二百二十四条の三第二項(株式等の譲渡の対価の受領者の告知)に規定する株式等(以下第三百四十四条(株式等の譲渡の対価の支払者の確認等)までにおいて「株式等」という。)の譲渡の対価(法第二百二十四条の三第一項に規定する対価をいう。以下第三百四十四条までにおいて同じ。)につき支払を受ける者(公共法人等を除く。以下この条において同じ。)は、当該株式等の譲渡の対価につきその支払を受けるべき時までに、その都度、その者の氏名又は名称、住所(国内に住所を有しない者にあつては、法第二百二十四条の三第一項に規定する財務省令で定める場所。以下この条、次条第三項及び第四項並びに第三百四十四条第一項において同じ。)及び個人番号又は法人番号(個人番号及び法人番号を有しない者又は第四項の規定に該当する個人(第三百四十四条第一項において「番号既告知者」という。)にあつては、氏名又は名称及び住所。次項において同じ。)を、その株式等の譲渡の対価の法第二百二十四条の三第一項に規定する支払者に告知しなければならない。
\item[\rensuji{2}]株式等の譲渡の対価の支払を受ける者が次の各号に掲げる場合のいずれかに該当するときは、その者は、その支払を受ける当該各号に定める株式等の譲渡の対価につき前項の規定による告知をしたものとみなす。
\begin{description}
\item[一]株式等の譲渡の対価の支払を受ける者が、当該株式等を払込みにより取得した場合又は当該株式等を購入若しくは相続その他の方法により取得した場合において、当該払込みにより取得をする際又は当該株式等の名義の変更若しくは書換えの請求をする際、その者の氏名又は名称、住所及び個人番号又は法人番号を当該対価の支払をする法第二百二十四条の三第一項第二号に掲げる者(次号、第三号及び次項において「金融商品取引業者等」という。)の営業所(営業所又は事務所をいう。以下この条及び第三百四十八条(信託受益権の譲渡の対価の受領者の告知)において同じ。)の長に告知しているとき
\item[二]株式等の譲渡の対価の支払を受ける者が、当該対価の支払をする金融商品取引業者等の営業所において株式等の保管の委託に係る契約を締結する際、その者の氏名又は名称、住所及び個人番号又は法人番号を当該金融商品取引業者等の営業所の長に告知しているとき
\item[三]株式等の譲渡の対価の支払を受ける者が、当該対価の支払をする金融商品取引業者等の営業所において金融商品取引業者等が社債、株式等の振替に関する法律の規定により備え付ける振替口座簿又は金融商品取引業者等の営業所を通じて当該金融商品取引業者等以外の振替機関等(同法第二条第五項(定義)に規定する振替機関等をいう。)が同法の規定により備え付ける振替口座簿に係る口座の開設を受ける際、その者の氏名又は名称、住所及び個人番号又は法人番号を当該金融商品取引業者等の営業所の長に告知しているとき
\item[四]株式等の譲渡の対価の支払を受ける者が、金融商品取引法第百五十六条の二十四第一項(免許及び免許の申請)に規定する信用取引又は発行日取引(有価証券が発行される前にその有価証券の売買を行う取引であつて財務省令で定める取引をいう。)(以下この号において「信用取引等」という。)により当該株式等の譲渡を行う場合において、当該株式等の譲渡の際、その者の氏名又は名称、住所及び個人番号又は法人番号を当該対価の支払をする法第二百二十四条の三第一項第二号に掲げる金融商品取引業者の営業所の長に告知しているとき
\end{description}
\item[\rensuji{3}]前項の場合において、同項各号に定める株式等の譲渡の対価の支払を受ける者が同項各号の告知をした後、次の各号に掲げる場合に該当することとなつた場合には、その者は、その該当することとなつた日以後最初に当該株式等の譲渡に係る対価の支払を受けるべき時までに、当該各号に掲げる場合の区分に応じ当該各号に定める事項を当該対価の支払をする金融商品取引業者等の営業所の長に告知しなければならない。
\begin{description}
\item[一]その者の氏名若しくは名称又は住所の変更をした場合
\item[二]その者の個人番号の変更をした場合
\item[三]行政手続における特定の個人を識別するための番号の利用等に関する法律の規定により個人番号又は法人番号が初めて通知された場合
\end{description}
\item[\rensuji{4}]法第二百二十四条の三第一項に規定する政令で定める者は、株式等の譲渡の対価の同項に規定する支払者が、財務省令で定めるところにより、当該株式等の譲渡の対価の支払を受ける個人の氏名、住所及び個人番号その他の事項を記載した帳簿(当該個人の次条第二項において準用する第三百三十七条第二項第一号(告知に係る住民票の写しその他の書類の提示等)に定める書類の提示又は署名用電子証明書等の送信を受けて作成されたものに限る。)を備えている場合における当該個人(当該個人の氏名、住所又は個人番号が当該帳簿に記載されている当該個人の氏名、住所又は個人番号と異なる場合における当該個人を除く。)とする。
\item[\rensuji{5}]法第二百二十四条の三第一項に規定する同項各号に掲げる者に準ずる者として政令で定めるものは、法第二百二十八条第二項(名義人受領の株式等の譲渡の対価の調書)に規定する株式等の譲渡の対価の同項に規定する支払を受ける者に該当する者とする。
\end{description}
\noindent\hspace{10pt}(株式等の譲渡の対価の受領者の告知に係る住民票の写しその他の書類の提示等)
\begin{description}
\item[第三百四十三条]株式等の譲渡の対価につき支払を受ける者は、前条の規定による告知をする際、当該告知をする当該対価の法第二百二十四条の三第一項(株式等の譲渡の対価の受領者の告知)に規定する支払者(第四項及び次条において「支払者」という。)に、次項において準用する第三百三十七条第二項(告知に係る住民票の写しその他の書類の提示等)に規定する書類を提示し、又は署名用電子証明書等を送信しなければならない。
\item[\rensuji{2}]第三百三十七条第二項の規定は、法第二百二十四条の三第一項に規定する政令で定める書類について準用する。
\item[\rensuji{3}]前条第二項各号の告知をした個人が、同条第三項第一号に掲げる場合に該当することとなつた場合において、同項の規定による告知をするときは、第一項の規定による書類の提示又は署名用電子証明書等の送信に代えて、住所等変更確認書類(当該個人の変更前の氏名又は住所及び変更後の氏名又は住所を証する住民票の写しその他の財務省令で定める書類をいう。次条第一項において同じ。)の提示をすることができる。
\item[\rensuji{4}]株式等の譲渡の対価につき支払を受ける者が当該対価の支払者に前条の規定による告知をする場合において、当該対価の支払者が、財務省令で定めるところにより、その支払を受ける者の氏名又は名称、住所及び個人番号又は法人番号(個人番号及び法人番号を有しない者にあつては、氏名又は名称及び住所。以下この項において同じ。)その他の事項を記載した帳簿(その者から申請書(その者の第二項において準用する第三百三十七条第二項各号に定めるいずれかの書類の写しを添付したもの又はその提出の際にその者の署名用電子証明書等の送信を受けているものに限る。)の提出(当該申請書の提出に代えて行う電子情報処理組織を使用する方法その他の情報通信の技術を利用する方法による当該申請書に記載すべき事項の提供を含む。)を受けて作成されたものに限る。)を備えているときは、その支払を受ける者は、第一項の規定にかかわらず、当該対価の支払者に対しては、同項に規定する書類の提示又は署名用電子証明書等の送信を要しないものとする。
\end{description}
\noindent\hspace{10pt}(株式等の譲渡の対価の支払者の確認等)
\begin{description}
\item[第三百四十四条]株式等の譲渡の対価の支払者は、第三百四十二条(株式等の譲渡の対価の受領者の告知)の規定による告知があつた場合には、当該告知があつた氏名又は名称、住所及び個人番号又は法人番号(個人番号及び法人番号を有しない者、番号既告知者又は同条第三項の規定による告知をした個人(当該告知の際に前条第三項の規定により住所等変更確認書類を提示した個人に限る。)にあつては、氏名又は名称及び住所。以下この項において同じ。)が、当該告知の際に提示又は送信を受けた前条第二項において準用する第三百三十七条第二項(告知に係る住民票の写しその他の書類の提示等)に規定する書類若しくは住所等変更確認書類又は署名用電子証明書等に記載又は記録がされた氏名又は名称、住所及び個人番号又は法人番号と同じであるかどうかを確認しなければならない。
\item[\rensuji{2}]株式等の譲渡の対価の支払者は、前項の確認をした場合には、財務省令で定めるところにより、当該確認に関する帳簿(これに類する帳簿又は書類を含む。)に、当該確認をした旨を明らかにし、かつ、当該帳簿を保存しなければならない。
\end{description}
\noindent\hspace{10pt}(株式等の範囲から除かれる公社債)
\begin{description}
\item[第三百四十四条の二]法第二百二十四条の三第二項第七号(株式等の譲渡の対価の受領者の告知)に規定する政令で定める公社債は、農水産業協同組合貯金保険法(昭和四十八年法律第五十三号)第二条第二項第四号(定義)に規定する農林債及び租税特別措置法第四十一条の十二第七項(償還差益等に係る分離課税等)に規定する償還差益につき同条第一項の規定の適用を受ける同条第七項に規定する割引債とする。
\end{description}
\noindent\hspace{10pt}(交付金銭等の受領者の告知等)
\begin{description}
\item[第三百四十五条]法第二百二十四条の三第三項(交付金銭等の受領者の告知)に規定する政令で定める金銭その他の資産は、次に掲げるものとする。
\begin{description}
\item[一]法人(法人税法第二条第六号(定義)に規定する公益法人等及び人格のない社団等を除く。以下この項及び次項において同じ。)の株主等がその法人の合併(法人課税信託に係る信託の併合を含む。)(当該法人の株主等に第百十二条第一項(合併により取得した株式等の取得価額)に規定する合併法人の株式(投資信託及び投資法人に関する法律第二条第十四項(定義)に規定する投資口を含む。以下この項及び第四項において同じ。)若しくは出資又は第百十二条第一項に規定する合併親法人の株式若しくは出資のいずれか一方の株式又は出資以外の資産(当該株主等に対する株式又は出資に係る剰余金の配当、利益の配当又は剰余金の分配として交付がされたもの及び合併に反対する当該株主等に対するその買取請求に基づく対価として交付がされるものを除く。)の交付がされなかつたものを除く。)により交付を受ける金銭及び金銭以外の資産
\item[二]法人の株主等がその法人の分割(法人税法第二条第十二号の九イに規定する分割対価資産として第百十三条第一項(分割型分割により取得した株式等の取得価額)に規定する分割承継法人の株式若しくは出資又は同項に規定する分割承継親法人の株式若しくは出資のいずれか一方の株式又は出資以外の資産の交付がされなかつたもので、当該株式又は出資が同条第二項に規定する分割法人の発行済株式等(同条第一項に規定する発行済株式等をいう。次号において同じ。)の総数又は総額のうちに占める当該分割法人の各株主等の有する当該分割法人の株式の数又は金額の割合に応じて交付されたものを除く。)により交付を受ける金銭及び金銭以外の資産
\item[三]法人の株主等がその法人の行つた法人税法第二条第十二号の十五の二に規定する株式分配(当該法人の株主等に第百十三条の二第一項(株式分配により取得した株式等の取得価額)に規定する完全子法人の株式又は出資以外の資産の交付がされなかつたもので、当該株式又は出資が同条第三項に規定する現物分配法人の発行済株式等の総数又は総額のうちに占める当該現物分配法人の各株主等の有する当該現物分配法人の株式の数又は金額の割合に応じて交付されたものを除く。)により交付を受ける金銭及び金銭以外の資産
\item[四]法人の株主等がその法人の資本の払戻し(法第二十五条第一項第四号(配当等とみなす金額)に規定する資本の払戻しをいう。)により、又はその法人の解散による残余財産の分配として交付を受ける金銭及び金銭以外の資産
\item[五]法人の株主等がその法人の自己の株式又は出資の取得(第六十一条第一項各号(所有株式に対応する資本金等の額又は連結個別資本金等の額の計算方法等)に掲げる事由による取得及び法第五十七条の四第三項第一号から第三号まで(株式交換等に係る譲渡所得等の特例)に掲げる株式又は出資の同項に規定する場合に該当する場合における取得を除く。)により交付を受ける金銭及び金銭以外の資産
\item[六]法人の株主等がその法人の出資の消却(取得した出資について行うものを除く。)、その法人の出資の払戻し、その法人からの退社若しくは脱退による持分の払戻し又はその法人の株式若しくは出資をその法人が取得することなく消滅させることにより交付を受ける金銭及び金銭以外の資産
\item[七]法人の株主等がその法人の組織変更(当該組織変更に際して当該組織変更をしたその法人の株式又は出資以外の資産の交付がされたものに限る。)により交付を受ける金銭及び金銭以外の資産
\end{description}
\item[\rensuji{2}]法第二百二十四条の三第三項に規定する政令で定める金銭は、法人の新株予約権者(新投資口予約権(投資信託及び投資法人に関する法律第二条第十七項に規定する新投資口予約権をいう。以下この項において同じ。)の新投資口予約権者を含む。以下この項において同じ。)がその法人の合併又は組織変更により当該新株予約権者が有していたその法人の新株予約権(新投資口予約権を含む。)に代えて交付を受ける金銭とする。
\item[\rensuji{3}]国内において法第二百二十四条の三第三項に規定する金銭等(以下この項及び次項において「交付金銭等」という。)の交付を受ける者(公共法人等を除く。次項において同じ。)は、当該交付金銭等につきその交付を受けるべき時までに、その都度、その者の氏名又は名称、住所(国内に住所を有しない者にあつては、同条第一項(株式等の譲渡対価の受領者の告知)に規定する財務省令で定める場所。以下この項において同じ。)及び個人番号又は法人番号(個人番号及び法人番号を有しない者又は第五項の規定により読み替えられた第三百四十二条第四項(株式等の譲渡の対価の受領者の告知)の規定に該当する個人にあつては、氏名又は名称及び住所)を、その交付金銭等の法第二百二十四条の三第三項の規定により読み替えられた同条第一項に規定する交付者に告知しなければならない。
\item[\rensuji{4}]交付金銭等の交付を受ける者が、当該交付金銭等の交付の基因となつた株式又は出資につき、第三百三十六条第二項第六号若しくは第七号(預貯金、株式等に係る利子、配当等の受領者の告知)に掲げる場合若しくは第三百三十九条第三項(無記名公社債の利子等に係る告知書等の提出等)に規定する場合に該当する場合又は当該交付金銭等とともに交付を受ける金銭その他の資産で法第二十四条第一項(配当所得)に規定する配当等に該当するものの受領につき、第三百三十六条第一項の規定による告知をした場合(同条第二項の規定により同条第一項の告知をしたものとみなされる場合を含む。)若しくは第三百三十九条第一項の規定による告知書を提出した場合(同条第三項の規定により同条第一項の告知書の提出があつたものとみなされる場合を含む。)には、その者は、当該交付金銭等につき前項の告知をしたものとみなす。
\item[\rensuji{5}]第三百四十二条第四項の規定は法第二百二十四条の三第三項の規定により読み替えられた同条第一項に規定する政令で定める者について、第三百四十二条第五項の規定は法第二百二十四条の三第三項の規定により読み替えられた同条第一項に規定する金銭等の交付をする者に準ずる者として政令で定めるものについて、それぞれ準用する。
\item[\rensuji{6}]第三百四十三条(第三項を除く。)(株式等の譲渡の対価の受領者の告知に係る住民票の写しその他の書類の提示等)の規定は第三項に規定する交付を受ける者が同項の告知をする場合について、第三百四十四条(株式等の譲渡の対価の支払者の確認等)の規定は同項の告知があつた場合について、それぞれ準用する。
\end{description}
\noindent\hspace{10pt}(償還金等の受領者の告知等)
\begin{description}
\item[第三百四十六条]法第二百二十四条の三第四項第一号(償還金等の受領者の告知)に規定する政令で定める金銭その他の資産は、次に掲げるものとする。
\begin{description}
\item[一]投資信託又は特定受益証券発行信託(以下この号及び第四項において「投資信託等」という。)の終了(当該投資信託等の信託の併合に係るものである場合にあつては、当該投資信託等の受益者に当該信託の併合に係る新たな信託の受益権以外の資産(信託の併合に反対する当該受益者に対するその買取請求に基づく対価として交付される金銭その他の資産を除く。)の交付がされた信託の併合に係るものに限る。)又は一部の解約により交付を受ける金銭及び金銭以外の資産
\item[二]特定受益証券発行信託に係る信託の分割(第五十八条第二項(投資信託等の収益の分配に係る収入金額)に規定する分割信託の受益者に同項に規定する承継信託の受益権以外の資産(信託の分割に反対する当該受益者に対する信託法第百三条第六項(受益権取得請求)に規定する受益権取得請求に基づく対価として交付される金銭その他の資産を除く。)の交付がされたものに限る。)により交付を受ける金銭及び金銭以外の資産
\end{description}
\item[\rensuji{2}]法第二百二十四条の三第四項第一号に規定する政令で定める金額は、次の各号に掲げる金銭及び金銭以外の資産の区分に応じ当該各号に定める金額とする。
\begin{description}
\item[一]前項第一号に掲げる金銭及び金銭以外の資産
\item[二]前項第二号に掲げる金銭及び金銭以外の資産
\end{description}
\item[\rensuji{3}]国内において法第二百二十四条の三第四項に規定する償還金等(以下この項及び次項において「償還金等」という。)の交付を受ける者(公共法人等を除く。次項において同じ。)は、当該償還金等につきその交付を受けるべき時までに、その都度、その者の氏名又は名称、住所(国内に住所を有しない者にあつては、同条第一項に規定する財務省令で定める場所。以下この項において同じ。)及び個人番号又は法人番号(個人番号及び法人番号を有しない者又は第五項の規定により読み替えられた第三百四十二条第四項(株式等の譲渡の対価の受領者の告知)の規定に該当する個人にあつては、氏名又は名称及び住所)を、その償還金等の法第二百二十四条の三第四項の規定により読み替えられた同条第一項に規定する交付者に告知しなければならない。
\item[\rensuji{4}]償還金等の交付を受ける者が、当該償還金等の交付の基因となつた投資信託等の受益権、法第二百二十四条の三第四項第二号の社債的受益権若しくは公社債又は同項第三号に規定する分離利子公社債につき、第三百三十六条第二項第一号から第四号まで(預貯金、株式等に係る利子、配当等の受領者の告知)に掲げる場合若しくは第三百三十九条第三項(無記名公社債の利子等に係る告知書等の提出等)に規定する場合に該当する場合又は当該償還金等とともに交付を受ける金銭その他の資産で法第二十三条第一項(利子所得)に規定する利子等若しくは法第二十四条第一項(配当所得)に規定する配当等に該当するものの受領につき、第三百三十六条第一項の規定による告知をした場合(同条第二項の規定により同条第一項の告知をしたものとみなされる場合を含む。)若しくは第三百三十九条第一項の規定による告知書を提出した場合(同条第三項の規定により同条第一項の告知書の提出があつたものとみなされる場合を含む。)には、その者は、当該償還金等につき前項の告知をしたものとみなす。
\item[\rensuji{5}]第三百四十二条第四項の規定は法第二百二十四条の三第四項の規定により読み替えられた同条第一項に規定する政令で定める者について、第三百四十二条第五項の規定は法第二百二十四条の三第四項の規定により読み替えられた同条第一項に規定する償還金等の交付をする者に準ずる者として政令で定めるものについて、それぞれ準用する。
\item[\rensuji{6}]第三百四十三条(第三項を除く。)(株式等の譲渡の対価の受領者の告知に係る住民票の写しその他の書類の提示等)の規定は第三項に規定する交付を受ける者が同項の告知をする場合について、第三百四十四条(株式等の譲渡の対価の支払者の確認等)の規定は同項の告知があつた場合について、それぞれ準用する。
\end{description}
\noindent\hspace{10pt}(信託受益権の譲渡の対価に係る告知義務のない公共法人等の範囲)
\begin{description}
\item[第三百四十七条]法第二百二十四条の四(信託受益権の譲渡の対価の受領者の告知)に規定する法人税法別表第一(公共法人の表)に掲げる法人その他の政令で定めるものは、公共法人等とする。
\end{description}
\noindent\hspace{10pt}(信託受益権の譲渡の対価の受領者の告知)
\begin{description}
\item[第三百四十八条]国内において法第二百二十四条の四(信託受益権の譲渡の対価の受領者の告知)に規定する信託受益権(以下この条から第三百五十条(信託受益権の譲渡の対価の支払者の確認等)までにおいて「信託受益権」という。)の譲渡の対価につき支払を受ける者(公共法人等を除く。以下この条において同じ。)は、当該信託受益権の譲渡の対価につきその支払を受けるべき時までに、その都度、その者の氏名又は名称、住所(国内に住所を有しない者にあつては、法第二百二十四条の四に規定する財務省令で定める場所。以下この条、次条第三項及び第四項並びに第三百五十条第一項において同じ。)及び個人番号又は法人番号(個人番号及び法人番号を有しない者又は第四項の規定に該当する個人(第三百五十条第一項において「番号既告知者」という。)にあつては、氏名又は名称及び住所。次項において同じ。)を、その信託受益権の譲渡の対価の法第二百二十四条の四に規定する支払者に告知しなければならない。
\item[\rensuji{2}]信託受益権の譲渡の対価の支払を受ける者が次の各号に掲げる場合のいずれかに該当するときは、その者は、その支払を受ける当該各号に定める信託受益権の譲渡の対価につき前項の規定による告知をしたものとみなす。
\begin{description}
\item[一]信託受益権の譲渡の対価の支払を受ける者が、当該信託受益権を購入により取得した場合において、当該購入に係る売買契約の締結をする際、その者の氏名又は名称、住所及び個人番号又は法人番号を当該対価の支払をする法第二百二十四条の四第二号に掲げる金融商品取引業者又は登録金融機関の営業所の長に告知しているとき
\item[二]信託受益権の譲渡の対価の支払を受ける者が、当該信託受益権を相続その他の方法により取得した場合において、当該信託受益権に係る信託の受託者の営業所の長に当該信託受益権の受益者となつた旨の告知をする際、その者の氏名又は名称、住所及び個人番号又は法人番号を当該対価の支払をする当該受託者の営業所の長に告知しているとき
\item[三]信託受益権の譲渡の対価の支払を受ける者が、当該信託受益権に係る信託の契約を締結する際、その者の氏名又は名称、住所及び個人番号又は法人番号を当該対価の支払をする当該信託の受託者の営業所の長に告知しているとき
\end{description}
\item[\rensuji{3}]前項の場合において、同項各号に定める信託受益権の譲渡の対価の支払を受ける者が同項各号の告知をした後、次の各号に掲げる場合に該当することとなつた場合には、その者は、その該当することとなつた日以後最初に当該信託受益権の譲渡に係る対価の支払を受けるべき時までに、当該各号に掲げる場合の区分に応じ当該各号に定める事項を当該対価の支払をする同項各号の金融商品取引業者若しくは登録金融機関又は信託の受託者の営業所の長に告知しなければならない。
\begin{description}
\item[一]その者の氏名若しくは名称又は住所の変更をした場合
\item[二]その者の個人番号の変更をした場合
\item[三]行政手続における特定の個人を識別するための番号の利用等に関する法律の規定により個人番号又は法人番号が初めて通知された場合
\end{description}
\item[\rensuji{4}]法第二百二十四条の四に規定する政令で定める者は、信託受益権の譲渡の対価の同条に規定する支払者が、財務省令で定めるところにより、当該信託受益権の譲渡の対価の支払を受ける個人の氏名、住所及び個人番号その他の事項を記載した帳簿(当該個人の次条第二項において準用する第三百三十七条第二項第一号(告知に係る住民票の写しその他の書類の提示等)に定める書類の提示又は署名用電子証明書等の送信を受けて作成されたものに限る。)を備えている場合における当該個人(当該個人の氏名、住所又は個人番号が当該帳簿に記載されている当該個人の氏名、住所又は個人番号と異なる場合における当該個人を除く。)とする。
\end{description}
\noindent\hspace{10pt}(信託受益権の譲渡の対価の受領者の告知に係る住民票の写しその他の書類の提示等)
\begin{description}
\item[第三百四十九条]信託受益権の譲渡の対価につき支払を受ける者は、前条の規定による告知をする際、当該告知をする当該対価の法第二百二十四条の四(信託受益権の譲渡の対価の受領者の告知)に規定する支払者(第四項及び次条において「支払者」という。)に、次項において準用する第三百三十七条第二項(告知に係る住民票の写しその他の書類の提示等)に規定する書類を提示し、又は署名用電子証明書等を送信しなければならない。
\item[\rensuji{2}]第三百三十七条第二項の規定は、法第二百二十四条の四に規定する政令で定める書類について準用する。
\item[\rensuji{3}]前条第二項各号の告知をした個人が、同条第三項第一号に掲げる場合に該当することとなつた場合において、同項の規定による告知をするときは、第一項の規定による書類の提示又は署名用電子証明書等の送信に代えて、住所等変更確認書類(当該個人の変更前の氏名又は住所及び変更後の氏名又は住所を証する住民票の写しその他の財務省令で定める書類をいう。次条第一項において同じ。)の提示をすることができる。
\item[\rensuji{4}]信託受益権の譲渡の対価につき支払を受ける者が当該対価の支払者に前条の規定による告知をする場合において、当該対価の支払者が、財務省令で定めるところにより、その支払を受ける者の氏名又は名称、住所及び個人番号又は法人番号(個人番号及び法人番号を有しない者にあつては、氏名又は名称及び住所。以下この項において同じ。)その他の事項を記載した帳簿(その者から申請書(その者の第二項において準用する第三百三十七条第二項各号に定めるいずれかの書類の写しを添付したもの又はその提出の際にその者の署名用電子証明書等の送信を受けているものに限る。)の提出を受けて作成されたものに限る。)を備えているときは、その支払を受ける者は、第一項の規定にかかわらず、当該対価の支払者に対しては、同項に規定する書類の提示又は署名用電子証明書等の送信を要しないものとする。
\end{description}
\noindent\hspace{10pt}(信託受益権の譲渡の対価の支払者の確認等)
\begin{description}
\item[第三百五十条]信託受益権の譲渡の対価の支払者は、第三百四十八条(信託受益権の譲渡の対価の受領者の告知)の規定による告知があつた場合には、当該告知があつた氏名又は名称、住所及び個人番号又は法人番号(個人番号及び法人番号を有しない者、番号既告知者又は同条第三項の規定による告知をした個人(当該告知の際に前条第三項の規定により住所等変更確認書類を提示した個人に限る。)にあつては、氏名又は名称及び住所。以下この項において同じ。)が、当該告知の際に提示又は送信を受けた前条第二項において準用する第三百三十七条第二項(告知に係る住民票の写しその他の書類の提示等)に規定する書類若しくは住所等変更確認書類又は署名用電子証明書等に記載又は記録がされた氏名又は名称、住所及び個人番号又は法人番号と同じであるかどうかを確認しなければならない。
\item[\rensuji{2}]信託受益権の譲渡の対価の支払者は、前項の確認をした場合には、財務省令で定めるところにより、当該確認に関する帳簿(これに類する帳簿又は書類を含む。)に、当該確認をした旨を明らかにし、かつ、当該帳簿を保存しなければならない。
\end{description}
\noindent\hspace{10pt}(先物取引の差金等決済に係る告知義務のない者の範囲)
\begin{description}
\item[第三百五十条の二]法第二百二十四条の五第一項(先物取引の差金等決済をする者の告知)に規定する法人税法別表第一(公共法人の表)に掲げる法人その他の政令で定めるものは、公共法人等とする。
\end{description}
\noindent\hspace{10pt}(先物取引の差金等決済をする者の告知)
\begin{description}
\item[第三百五十条の三]国内において法第二百二十四条の五第二項(先物取引の差金等決済をする者の告知)に規定する先物取引(以下この条及び次条において「先物取引」という。)の同項に規定する差金等決済(以下この条及び次条において「差金等決済」という。)をする者(公共法人等を除く。以下この条及び次条において同じ。)は、その差金等決済をする日までに、その都度、その者の氏名又は名称、住所(国内に住所を有しない者にあつては、法第二百二十四条の五第一項に規定する財務省令で定める場所。以下この条から第三百五十条の五(商品先物取引業者等の確認等)までにおいて同じ。)及び個人番号又は法人番号(個人番号及び法人番号を有しない者又は第四項の規定に該当する個人(第三百五十条の五第一項において「番号既告知者」という。)にあつては、氏名又は名称及び住所。次項において同じ。)を、その差金等決済に係る先物取引の法第二百二十四条の五第一項に規定する商品先物取引業者等(以下この条から第三百五十条の五までにおいて「商品先物取引業者等」という。)に告知しなければならない。
\item[\rensuji{2}]先物取引の差金等決済をする者が次の各号に掲げる場合のいずれかに該当するときは、その者は、当該各号に定める先物取引の差金等決済につき前項の規定による告知をしたものとみなす。
\begin{description}
\item[一]商品先物取引(法第二百二十四条の五第一項第一号に規定する商品先物取引をいう。以下この号及び次号において同じ。)又は外国商品市場取引(同項第一号に規定する外国商品市場取引をいう。以下この号において同じ。)の差金等決済をする者が、同項第一号に規定する商品先物取引業者(以下この号及び第三号において「商品先物取引業者」という。)と当該商品先物取引又は外国商品市場取引の委託に係る契約を締結する際、その者の氏名又は名称、住所及び個人番号又は法人番号を当該商品先物取引業者の当該商品先物取引又は外国商品市場取引に係る営業所等(同項第一号に規定する営業所等をいう。以下この号及び第三号において同じ。)の長に(当該商品先物取引又は外国商品市場取引を委託の取次ぎにより行つた場合には、当該委託の取次ぎを引き受けた商品先物取引業者と当該委託の取次ぎに係る契約を締結する際、その者の氏名又は名称、住所及び個人番号又は法人番号を当該商品先物取引業者の当該取次ぎに係る営業所等の長に)告知しているとき
\item[二]商品先物取引の差金等決済をする者が、当該商品先物取引に係る商品市場(法第二百二十四条の五第一項第二号に規定する商品市場をいう。以下この号において同じ。)を開設している商品取引所(同項第二号に規定する商品取引所をいう。以下この号において同じ。)に加入をする際、その者の氏名又は名称、住所及び個人番号又は法人番号を当該商品取引所の長に告知しているとき
\item[三]店頭商品デリバティブ取引(法第二百二十四条の五第一項第三号に規定する店頭商品デリバティブ取引をいう。以下この号において同じ。)の差金等決済をする者が、商品先物取引業者と当該店頭商品デリバティブ取引に係る契約を締結する際、その者の氏名又は名称、住所及び個人番号又は法人番号を当該商品先物取引業者の当該店頭商品デリバティブ取引に係る営業所等の長に(当該店頭商品デリバティブ取引を取次ぎにより行つた場合には、当該取次ぎを引き受けた商品先物取引業者と当該取次ぎに係る契約を締結する際、その者の氏名又は名称、住所及び個人番号又は法人番号を当該商品先物取引業者の当該取次ぎに係る営業所等の長に)告知しているとき
\item[四]市場デリバティブ取引(法第二百二十四条の五第一項第四号に規定する市場デリバティブ取引をいう。以下この号において同じ。)又は外国市場デリバティブ取引(同項第四号に規定する外国市場デリバティブ取引をいう。以下この号において同じ。)の差金等決済をする者が、同項第四号に規定する金融商品取引業者等(以下この号及び第六号において「金融商品取引業者等」という。)と当該市場デリバティブ取引又は外国市場デリバティブ取引の委託に係る契約を締結する際、その者の氏名又は名称、住所及び個人番号又は法人番号を当該金融商品取引業者等の当該市場デリバティブ取引又は外国市場デリバティブ取引に係る営業所の長に(当該市場デリバティブ取引又は外国市場デリバティブ取引を委託の取次ぎにより行つた場合には、当該委託の取次ぎを引き受けた金融商品取引業者等と当該委託の取次ぎに係る契約を締結する際、その者の氏名又は名称、住所及び個人番号又は法人番号を当該金融商品取引業者等の当該取次ぎに係る営業所の長に)告知しているとき
\item[五]市場デリバティブ取引(法第二百二十四条の五第一項第五号に規定する市場デリバティブ取引をいう。以下この号において同じ。)の差金等決済をする者が、当該市場デリバティブ取引に係る取引所金融商品市場(同項第五号に規定する取引所金融商品市場をいう。以下この号において同じ。)を開設している金融商品取引所(同項第五号に規定する金融商品取引所をいう。以下この号において同じ。)に加入をする際、その者の氏名又は名称、住所及び個人番号又は法人番号を当該金融商品取引所の長に告知しているとき
\item[六]店頭デリバティブ取引(法第二百二十四条の五第一項第六号に規定する店頭デリバティブ取引をいう。以下この号において同じ。)の差金等決済をする者が、金融商品取引業者等と当該店頭デリバティブ取引に係る契約を締結する際、その者の氏名又は名称、住所及び個人番号又は法人番号を当該金融商品取引業者等の当該店頭デリバティブ取引に係る営業所の長に(当該店頭デリバティブ取引を取次ぎにより行つた場合には、当該取次ぎを引き受けた金融商品取引業者等と当該取次ぎに係る契約を締結する際、その者の氏名又は名称、住所及び個人番号又は法人番号を当該金融商品取引業者等の当該取次ぎに係る営業所の長に)告知しているとき
\item[七]法第二百二十四条の五第一項第七号に規定する有価証券(以下この項において「有価証券」という。)の差金等決済をする者が、当該有価証券を購入又は相続その他の方法により取得した場合において、当該有価証券の名義の変更又は書換えの請求をする際、その者の氏名又は名称、住所及び個人番号又は法人番号を、その有価証券に表示される権利の行使(同条第二項第三号に規定する行使をいう。次号において同じ。)若しくは放棄に関する事務の取扱いをする同条第一項第四号に規定する金融商品取引業者(以下この項において「金融商品取引業者」という。)の営業所の長又は当該有価証券の譲渡の対価の支払をする金融商品取引業者の営業所の長に告知しているとき
\item[八]有価証券の差金等決済をする者が、当該有価証券に表示される権利の行使若しくは放棄に関する事務の取扱いをする金融商品取引業者の営業所又は当該有価証券の譲渡の対価の支払をする金融商品取引業者の営業所においてこれらの有価証券の保管の委託に係る契約を締結する際、その者の氏名又は名称、住所及び個人番号又は法人番号をこれらの金融商品取引業者の営業所の長に告知しているとき
\end{description}
\item[\rensuji{3}]前項の場合において、同項各号に定める先物取引の差金等決済をする者が同項各号の告知をした後、次の各号に掲げる場合に該当することとなつた場合には、その者は、その該当することとなつた日以後最初に当該先物取引の差金等決済をする日までに、当該各号に掲げる場合の区分に応じ当該各号に定める事項を当該告知に係る商品先物取引業者等に告知しなければならない。
\begin{description}
\item[一]その者の氏名若しくは名称又は住所の変更をした場合
\item[二]その者の個人番号の変更をした場合
\item[三]行政手続における特定の個人を識別するための番号の利用等に関する法律の規定により個人番号又は法人番号が初めて通知された場合
\end{description}
\item[\rensuji{4}]法第二百二十四条の五第一項に規定する政令で定める者は、差金等決済に係る先物取引の商品先物取引業者等が、財務省令で定めるところにより、当該差金等決済をする個人の氏名、住所及び個人番号その他の事項を記載した帳簿(当該個人の次条第二項において準用する第三百三十七条第二項第一号(告知に係る住民票の写しその他の書類の提示等)に定める書類の提示又は署名用電子証明書等の送信を受けて作成されたものに限る。)を備えている場合における当該個人(当該個人の氏名、住所又は個人番号が当該帳簿に記載されている当該個人の氏名、住所又は個人番号と異なる場合における当該個人を除く。)とする。
\end{description}
\noindent\hspace{10pt}(先物取引の差金等決済をする者の告知に係る住民票の写しその他の書類の提示等)
\begin{description}
\item[第三百五十条の四]先物取引の差金等決済をする者は、前条の規定による告知をする際、当該告知をする商品先物取引業者等に、次項において準用する第三百三十七条第二項(告知に係る住民票の写しその他の書類の提示等)に規定する書類を提示し、又は署名用電子証明書等を送信しなければならない。
\item[\rensuji{2}]第三百三十七条第二項の規定は、法第二百二十四条の五第一項(先物取引の差金等決済をする者の告知)に規定する政令で定める書類について準用する。
\item[\rensuji{3}]前条第二項各号の告知をした個人が、同条第三項第一号に掲げる場合に該当することとなつた場合において、同項の規定による告知をするときは、第一項の規定による書類の提示又は署名用電子証明書等の送信に代えて、住所等変更確認書類(当該個人の変更前の氏名又は住所及び変更後の氏名又は住所を証する住民票の写しその他の財務省令で定める書類をいう。次条第一項において同じ。)の提示をすることができる。
\item[\rensuji{4}]先物取引の差金等決済をする者が商品先物取引業者等に前条の規定による告知をする場合において、当該商品先物取引業者等が、財務省令で定めるところにより、その先物取引の差金等決済をする者の氏名又は名称、住所及び個人番号又は法人番号(個人番号及び法人番号を有しない者にあつては、氏名又は名称及び住所。以下この項において同じ。)その他の事項を記載した帳簿(その者から申請書(その者の第二項において準用する第三百三十七条第二項各号に定めるいずれかの書類の写しを添付したもの又はその提出の際にその者の署名用電子証明書等の送信を受けているものに限る。)の提出(当該申請書の提出に代えて行う電子情報処理組織を使用する方法その他の情報通信の技術を利用する方法による当該申請書に記載すべき事項の提供を含む。)を受けて作成されたものに限る。)を備えているときは、その先物取引の差金等決済をする者は、第一項の規定にかかわらず、当該商品先物取引業者等に対しては、同項に規定する書類の提示又は署名用電子証明書等の送信を要しないものとする。
\end{description}
\noindent\hspace{10pt}(商品先物取引業者等の確認等)
\begin{description}
\item[第三百五十条の五]商品先物取引業者等は、第三百五十条の三(先物取引の差金等決済をする者の告知)の規定による告知があつた場合には、当該告知があつた氏名又は名称、住所及び個人番号又は法人番号(個人番号及び法人番号を有しない者、番号既告知者又は同条第三項の規定による告知をした個人(当該告知の際に前条第三項の規定により住所等変更確認書類を提示した個人に限る。)にあつては、氏名又は名称及び住所。以下この項及び次項において同じ。)が、当該告知の際に提示又は送信を受けた前条第二項において準用する第三百三十七条第二項(告知に係る住民票の写しその他の書類の提示等)に規定する書類若しくは住所等変更確認書類又は署名用電子証明書等に記載又は記録がされた氏名又は名称、住所及び個人番号又は法人番号と同じであるかどうかを確認しなければならない。
\item[\rensuji{2}]商品先物取引業者等は、第三百五十条の三の規定による告知があつた場合において、当該告知をした者が前条第四項に規定する帳簿に記載されている者であるとき(同項ただし書に該当するときを除く。)は、前項の規定にかかわらず、当該告知があつた氏名又は名称、住所及び個人番号又は法人番号が当該帳簿に記載されている氏名又は名称、住所及び個人番号又は法人番号と同じであるかどうかを確認しなければならない。
\item[\rensuji{3}]商品先物取引業者等は、前二項の確認をした場合には、財務省令で定めるところにより、当該確認に関する帳簿(これに類する帳簿又は書類を含む。)に、当該確認をした旨を明らかにし、かつ、当該帳簿を保存しなければならない。
\end{description}
\noindent\hspace{10pt}(金地金等の譲渡の対価に係る告知義務のない公共法人等の範囲)
\begin{description}
\item[第三百五十条の六]法第二百二十四条の六(金地金等の譲渡の対価の受領者の告知)に規定する法人税法別表第一(公共法人の表)に掲げる法人その他の政令で定めるものは、公共法人等とする。
\end{description}
\noindent\hspace{10pt}(金地金等の譲渡の対価の受領者の告知を要しない譲渡の対価の上限額)
\begin{description}
\item[第三百五十条の七]法第二百二十四条の六(金地金等の譲渡の対価の受領者の告知)に規定する政令で定める金額は、二百万円とする。
\end{description}
\noindent\hspace{10pt}(金地金等の譲渡の対価の受領者の告知)
\begin{description}
\item[第三百五十条の八]国内において法第二百二十四条の六(金地金等の譲渡の対価の受領者の告知)に規定する金地金等(以下この条から第三百五十条の十(金地金等の譲渡の対価の支払者の確認等)までにおいて「金地金等」という。)の譲渡の対価(法第二百二十四条の六に規定する対価をいう。以下この条から第三百五十条の十までにおいて同じ。)につき支払を受ける者(公共法人等を除く。以下この条及び次条において同じ。)は、その金地金等の譲渡の対価につきその支払を受けるべき時までに、その都度、その者の氏名又は名称、住所(国内に住所を有しない者にあつては、法第二百二十四条の六に規定する財務省令で定める場所。以下この条から第三百五十条の十までにおいて同じ。)及び個人番号又は法人番号(個人番号及び法人番号を有しない者又は第四項の規定に該当する個人(第三百五十条の十第一項において「番号既告知者」という。)にあつては、氏名又は名称及び住所。次項において同じ。)を、その金地金等の譲渡の対価の法第二百二十四条の六に規定する支払者(以下この条から第三百五十条の十までにおいて「支払者」という。)に告知しなければならない。
\item[\rensuji{2}]金地金等の譲渡の対価の支払を受ける者が、当該金地金等を購入により取得した場合において、当該購入に係る売買契約の締結をする際、その者の氏名又は名称、住所及び個人番号又は法人番号を当該対価の支払者の営業所、事務所その他これらに準ずるもの(以下この条において「営業所等」という。)の長に告知しているときは、その者は、その支払を受ける当該金地金等の譲渡の対価につき前項の規定による告知をしたものとみなす。
\item[\rensuji{3}]前項の場合において、同項の金地金等の譲渡の対価の支払を受ける者が同項の告知をした後、次の各号に掲げる場合に該当することとなつた場合には、その者は、その該当することとなつた日以後最初に当該金地金等の譲渡に係る対価の支払を受けるべき時までに、当該各号に掲げる場合の区分に応じ当該各号に定める事項を当該対価の支払者の営業所等の長に告知しなければならない。
\begin{description}
\item[一]その者の氏名若しくは名称又は住所の変更をした場合
\item[二]その者の個人番号の変更をした場合
\item[三]行政手続における特定の個人を識別するための番号の利用等に関する法律の規定により個人番号又は法人番号が初めて通知された場合
\end{description}
\item[\rensuji{4}]法第二百二十四条の六に規定する政令で定める者は、金地金等の譲渡の対価の支払者が、財務省令で定めるところにより、当該金地金等の譲渡の対価の支払を受ける個人の氏名、住所及び個人番号その他の事項を記載した帳簿(当該個人の次条第二項において準用する第三百三十七条第二項第一号(告知に係る住民票の写しその他の書類の提示等)に定める書類の提示又は署名用電子証明書等の送信を受けて作成されたものに限る。)を備えている場合における当該個人(当該個人の氏名、住所又は個人番号が当該帳簿に記載されている当該個人の氏名、住所又は個人番号と異なる場合における当該個人を除く。)とする。
\end{description}
\noindent\hspace{10pt}(金地金等の譲渡の対価の受領者の告知に係る住民票の写しその他の書類の提示等)
\begin{description}
\item[第三百五十条の九]金地金等の譲渡の対価につき支払を受ける者は、前条の規定による告知をする際、当該告知をする当該対価の支払者に、次項において準用する第三百三十七条第二項(告知に係る住民票の写しその他の書類の提示等)に規定する書類を提示し、又は署名用電子証明書等を送信しなければならない。
\item[\rensuji{2}]第三百三十七条第二項の規定は、法第二百二十四条の六(金地金等の譲渡の対価の受領者の告知)に規定する政令で定める書類について準用する。
\item[\rensuji{3}]前条第二項の規定による告知をした個人が、同条第三項第一号に掲げる場合に該当することとなつた場合において、同項の規定による告知をするときは、第一項の規定による書類の提示又は署名用電子証明書等の送信に代えて、住所等変更確認書類(当該個人の変更前の氏名又は住所及び変更後の氏名又は住所を証する住民票の写しその他の財務省令で定める書類をいう。次条第一項において同じ。)の提示をすることができる。
\item[\rensuji{4}]金地金等の譲渡の対価につき支払を受ける者が当該対価の支払者に前条の規定による告知をする場合において、当該対価の支払者が、財務省令で定めるところにより、その支払を受ける者の氏名又は名称、住所及び個人番号又は法人番号(個人番号及び法人番号を有しない者にあつては、氏名又は名称及び住所。以下この項において同じ。)その他の事項を記載した帳簿(その者から申請書(その者の第二項において準用する第三百三十七条第二項各号に定めるいずれかの書類の写しを添付したもの又はその提出の際にその者の署名用電子証明書等の送信を受けているものに限る。)の提出を受けて作成されたものに限る。)を備えているときは、その支払を受ける者は、第一項の規定にかかわらず、当該対価の支払者に対しては、同項に規定する書類の提示又は署名用電子証明書等の送信を要しないものとする。
\end{description}
\noindent\hspace{10pt}(金地金等の譲渡の対価の支払者の確認等)
\begin{description}
\item[第三百五十条の十]金地金等の譲渡の対価の支払者は、第三百五十条の八(金地金等の譲渡の対価の受領者の告知)の規定による告知があつた場合には、当該告知があつた氏名又は名称、住所及び個人番号又は法人番号(個人番号及び法人番号を有しない者、番号既告知者又は同条第三項の規定による告知をした個人(当該告知の際に前条第三項の規定により住所等変更確認書類を提示した個人に限る。)にあつては、氏名又は名称及び住所。以下この項において同じ。)が、当該告知の際に提示又は送信を受けた前条第二項において準用する第三百三十七条第二項(告知に係る住民票の写しその他の書類の提示等)に規定する書類若しくは住所等変更確認書類又は署名用電子証明書等に記載又は記録がされた氏名又は名称、住所及び個人番号又は法人番号と同じであるかどうかを確認しなければならない。
\item[\rensuji{2}]金地金等の譲渡の対価の支払者は、前項の確認をした場合には、財務省令で定めるところにより、当該確認に関する帳簿(これに類する帳簿又は書類を含む。)に、当該確認をした旨を明らかにし、かつ、当該帳簿を保存しなければならない。
\end{description}
\noindent\hspace{10pt}(生命保険金に類する給付等)
\begin{description}
\item[第三百五十一条]法第二百二十五条第一項第四号(支払調書等)に規定する政令で定める給付は、次に掲げるもの(法第二十八条第一項(給与所得)に規定する給与等、法第三十条第一項(退職所得)に規定する退職手当等又は法第三十五条第三項(公的年金等の定義)に規定する公的年金等に該当するものを除く。)とする。
\begin{description}
\item[一]生命保険契約(法第二百二十五条第一項第四号に規定する生命保険契約をいう。次項第一号において同じ。)又は旧簡易生命保険契約(第三十条第一号(非課税とされる保険金、損害賠償金等)に規定する旧簡易生命保険契約をいう。)に基づいて支払う保険金(年金を含む。)及び解約返戻金(法第百七十四条第八号(内国法人に係る所得税の課税標準)に掲げる差益に係るものを除く。)
\item[二]法第七十六条第六項第三号(生命保険料控除)に掲げる契約又は第三百二十六条第二項第二号(生命保険契約等に基づく年金に係る源泉徴収)に掲げる契約に基づいて支払う共済金(共済年金を含む。)及び解約返戻金(法第百七十四条第八号に掲げる差益に係るものを除く。)
\item[三]第七十六条第一項各号又は第二項各号(退職金共済制度等に基づく一時金で退職手当等とみなさないもの)に掲げる給付
\item[四]旧厚生年金保険法第九章(厚生年金基金及び企業年金連合会)の規定に基づく一時金、確定給付企業年金法第三条第一項(確定給付企業年金の実施)に規定する確定給付企業年金に係る規約に基づいて支給を受ける一時金、法人税法附則第二十条第三項(退職年金等積立金に対する法人税の特例)に規定する適格退職年金契約に基づいて支給を受ける一時金又は第七十二条第三項第五号イからハまで(退職手当等とみなす一時金)に掲げる規定に基づいて支給を受ける一時金
\item[五]中小企業退職金共済法第十六条第一項(解約手当金)に規定する解約手当金又は第七十四条第五項(特定退職金共済団体の承認)に規定する特定退職金共済団体が行うこれに類する給付
\item[六]小規模企業共済法第十二条第一項(解約手当金)に規定する解約手当金
\item[七]確定拠出年金法附則第二条の二第二項及び第三条第二項(脱退一時金)に規定する脱退一時金
\item[八]第二十条第二項(非課税とされる業務上の傷害に基づく給付等)に規定する共済制度に係る同項の脱退一時金
\item[九]租税特別措置法第二十九条の三(勤労者が受ける財産形成給付金等に係る課税の特例)に規定する財産形成給付金又は第一種財産形成基金給付金若しくは第二種財産形成基金給付金
\end{description}
\item[\rensuji{2}]法第二百二十五条第一項第五号に規定する政令で定める給付は、次に掲げるものとする。
\begin{description}
\item[一]損害保険契約等(法第七十六条第六項第四号に掲げる契約で生命保険契約以外のもの、法第七十七条第二項各号(地震保険料控除)に掲げる契約及び第三百二十六条第二項各号(第二号を除く。)に掲げる契約をいう。次号において同じ。)及び法第二百二十五条第一項第五号に規定する少額短期保険業者の締結した同号に規定する損害保険契約の第百八十四条第四項(満期返戻金等に係る一時所得の金額の計算上控除する保険料等)に規定する満期返戻金等(法第百七十四条第八号に掲げる差益に係るものを除く。)
\item[二]損害保険契約等に基づく年金である中途返戻金(当該年金に係る損害保険契約等の保険期間の満了後に支払われる満期返戻金を含む。)
\end{description}
\end{description}
\noindent\hspace{10pt}(不動産の貸付け等の支払調書を提出すべき不動産業者)
\begin{description}
\item[第三百五十二条]法第二百二十五条第一項第九号(支払調書等)に規定する政令で定める不動産業者は、宅地建物取引業法(昭和二十七年法律第百七十六号)第二条第二号(定義)に規定する宅地建物取引業を営む者のうち建物の貸借の代理又は媒介を主たる目的とする事業を営む者以外の者とする。
\end{description}
\noindent\hspace{10pt}(償還金等の支払調書の提出範囲)
\begin{description}
\item[第三百五十二条の二]法第二百二十五条第一項第十一号(支払調書等)に規定する政令で定める内国法人は、地方自治法第二百六十条の二第七項(地縁による団体)に規定する認可地縁団体、建物の区分所有等に関する法律(昭和三十七年法律第六十九号)第四十七条第二項(成立等)に規定する管理組合法人及び同法第六十六条(建物の区分所有に関する規定の準用)の規定により読み替えられた同項に規定する団地管理組合法人、政党交付金の交付を受ける政党等に対する法人格の付与に関する法律(平成六年法律第百六号)第七条の二第一項(変更の登記)に規定する法人である政党等、密集市街地における防災街区の整備の促進に関する法律(平成九年法律第四十九号)第百三十三条第一項(法人格)に規定する防災街区整備事業組合、特定非営利活動促進法(平成十年法律第七号)第二条第二項(定義)に規定する特定非営利活動法人並びにマンションの建替え等の円滑化に関する法律第五条第一項(マンション建替事業の施行)に規定するマンション建替組合及び同法第百十六条(マンション敷地売却事業の実施)に規定するマンション敷地売却組合とする。
\item[\rensuji{2}]法第二百二十五条第一項第十一号に規定する政令で定める償還金等は、法第二百二十四条の三第二項第七号(株式等の譲渡の対価の受領者の告知)に掲げる公社債のうち次に掲げるものに係る同条第四項に規定する償還金等とする。
\begin{description}
\item[一]割引の方法により発行されるもの
\item[二]分離元本公社債(公社債で元本に係る部分と利子に係る部分とに分離されてそれぞれ独立して取引されるもののうち、当該元本に係る部分であつた公社債をいう。)
\item[三]分離利子公社債(公社債で元本に係る部分と利子に係る部分とに分離されてそれぞれ独立して取引されるもののうち、当該利子に係る部分であつた公社債をいう。)
\item[四]利子が支払われる公社債で、その発行価額として財務省令で定める金額の額面金額に対する割合が財務省令で定める割合以下であるもの
\end{description}
\end{description}
\noindent\hspace{10pt}(支払通知書を交付すべき支払をする者に準ずる者)
\begin{description}
\item[第三百五十二条の三]法第二百二十五条第二項各号(支払通知書)に規定する政令で定めるものは、法第二百二十七条(信託の計算書)に規定する信託の受託者及び法第二百二十八条第一項(名義人受領の配当所得の調書)に規定する配当等の支払を受ける者に該当する者とする。
\end{description}
\noindent\hspace{10pt}(支払通知書に記載すべき事項の電磁的方法による提供の承諾等)
\begin{description}
\item[第三百五十二条の四]法第二百二十五条第三項(支払通知書)に規定する支払をする者は、同項本文の規定により同項に規定する通知書に記載すべき事項を同項に規定する支払を受ける者に対し提供しようとするときは、財務省令で定めるところにより、あらかじめ、当該支払を受ける者に対し、その用いる電磁的方法(同項に規定する電磁的方法をいう。以下この条、次条及び第三百五十六条(給与等、退職手当等又は公的年金等の支払明細書に記載すべき事項の電磁的方法による提供の承諾等)において同じ。)の種類及び内容を示し、書面又は電磁的方法による承諾を得なければならない。
\item[\rensuji{2}]前項の規定による承諾を得た同項の支払をする者は、同項の支払を受ける者から書面又は電磁的方法により法第二百二十五条第三項本文の規定による電磁的方法による提供を受けない旨の申出があつたときは、当該支払を受ける者に対し、同項に規定する通知書に記載すべき事項の提供を電磁的方法によつてしてはならない。
\end{description}
\noindent\hspace{10pt}(源泉徴収票に記載すべき事項の電磁的方法による提供の承諾等)
\begin{description}
\item[第三百五十三条]居住者に対し国内において法第二百二十六条第一項(源泉徴収票)に規定する給与等(以下この条及び第三百五十六条(給与等、退職手当等又は公的年金等の支払明細書に記載すべき事項の電磁的方法による提供の承諾等)において「給与等」という。)、法第二百二十六条第二項に規定する退職手当等(以下この条及び第三百五十六条において「退職手当等」という。)又は法第二百二十六条第三項に規定する公的年金等(以下この条及び第三百五十六条において「公的年金等」という。)の支払をする者は、法第二百二十六条第四項本文の規定により同項に規定する源泉徴収票に記載すべき事項を提供しようとするときは、財務省令で定めるところにより、あらかじめ、当該給与等、退職手当等又は公的年金等の支払を受ける者に対し、その用いる電磁的方法の種類及び内容を示し、書面又は電磁的方法による承諾を得なければならない。
\item[\rensuji{2}]前項の規定による承諾を得た給与等、退職手当等又は公的年金等の支払をする者は、当該給与等、退職手当等又は公的年金等の支払を受ける者から書面又は電磁的方法により法第二百二十六条第四項本文の規定による電磁的方法による提供を受けない旨の申出があつたときは、当該給与等、退職手当等又は公的年金等の支払を受ける者に対し、同項に規定する源泉徴収票に記載すべき事項の提供を電磁的方法によつてしてはならない。
\end{description}
\noindent\hspace{10pt}(有限責任事業組合等に係る組合員所得に関する計算書)
\begin{description}
\item[第三百五十三条の二]法第二百二十七条の二(有限責任事業組合等に係る組合員所得に関する計算書)に規定する政令で定める日は、同条に規定する投資事業有限責任組合契約において定める同条の計算期間の終了の日の翌日から二月を経過する日とする。
\end{description}
\noindent\hspace{10pt}(新株予約権の行使に関する調書)
\begin{description}
\item[第三百五十四条]法第二百二十八条の二(新株予約権の行使に関する調書)に規定する政令で定める新株予約権は、次に掲げる新株予約権とする。
\begin{description}
\item[一]新株予約権を引き受ける者に特に有利な条件又は金額であることとされる当該新株予約権
\item[二]役務の提供その他の行為に係る対価の全部又は一部として発行又は割当てをすることとされる新株予約権(前号に該当するものを除く。)
\end{description}
\item[\rensuji{2}]法第二百二十八条の二に規定する政令で定める発行又は割当ては、同条に規定する新株予約権の発行又は割当てに係る金銭により払い込まれるべき額と当該新株予約権の行使に際して払い込まれるべき額との合計額を当該新株予約権の行使によつて交付することとなる株式の数で除して計算した金額が当該新株予約権の発行又は割当てに係る同条に規定する決議の時における当該新株予約権を発行又は割当てをした株式会社の株式の一株当たりの価額に相当する金額に満たない場合における当該新株予約権の発行又は割当てとする。
\end{description}
\noindent\hspace{10pt}(著しく低い価額の対価による株式割当て)
\begin{description}
\item[第三百五十四条の二]法第二百二十八条の三(株式無償割当てに関する調書)に規定する政令で定める割当ては、会社法第三百二十二条第一項(ある種類の種類株主に損害を及ぼすおそれがある場合の種類株主総会)の決議(同条第二項の規定による定款の定めを含む。)により株式を引き受ける者の募集に応じて割り当てられる株式につき、当該株式の同法第百九十九条第一項第二号(募集事項の決定)に規定する払込金額が当該株式の取得のために通常要する価額の二分の一に満たない金額である場合における当該株式の割当てとする。
\end{description}
\noindent\hspace{10pt}(外国親会社等が国内の役員等に供与等をした経済的利益に関する調書)
\begin{description}
\item[第三百五十四条の三]法第二百二十八条の三の二(外国親会社等が国内の役員等に供与等をした経済的利益に関する調書)に規定する政令で定める関係は、外国法人が内国法人の発行済株式(議決権のあるものに限る。)又は出資(以下この条において「発行済株式等」という。)の総数又は総額の百分の五十以上の数又は金額の株式(議決権のあるものに限るものとし、出資を含む。以下この項において同じ。)を直接又は間接に保有する関係とする。
\begin{description}
\item[一]当該内国法人の株主等である法人の発行済株式等の総数又は総額の百分の五十以上の数又は金額の株式が当該外国法人により所有されている場合
\item[二]当該内国法人の株主等である法人(前号に掲げる場合に該当する同号の株主等である法人を除く。)と当該外国法人との間にこれらの法人と発行済株式等の所有を通じて連鎖関係にある一又は二以上の法人(以下この号において「出資関連法人」という。)が介在している場合(出資関連法人及び当該株主等である法人がそれぞれその発行済株式等の総数又は総額の百分の五十以上の数又は金額の株式を当該外国法人又は出資関連法人(その発行済株式等の総数又は総額の百分の五十以上の数又は金額の株式が当該外国法人又は他の出資関連法人によつて所有されているものに限る。)によつて所有されている場合に限る。)
\end{description}
\item[\rensuji{2}]法第二百二十八条の三の二に規定する政令で定める権利は、次に掲げる権利とする。
\begin{description}
\item[一]法第二百二十八条の三の二に規定する外国親会社等(同条に規定する役員等と同条の契約を締結したものに限る。以下この項において「外国親会社等」という。)の株式又は当該外国親会社等と資本関係(当該外国親会社等と当該外国親会社等以外の法人のいずれか一方の法人が他方の法人の発行済株式等の総数又は総額の百分の五十以上の数又は金額の株式(議決権のあるものに限るものとし、出資を含む。)を直接又は間接に保有する関係をいう。次項において同じ。)がある法人の株式(以下この項において「外国親会社株式等」と総称する。)を無償又は有利な価額で取得することができる権利
\item[二]外国親会社株式等の価額に相当する額又は当該外国親会社株式等に係る配当に相当する額の金銭その他の経済的利益の支払又は供与を受けることができる権利
\item[三]外国親会社株式等の価額、外国親会社等の業績その他の指標の数値が一定の期間内にあらかじめ定めた基準に達した場合に当該外国親会社株式等、金銭その他の経済的利益の交付、支払又は供与を受けることができる権利
\end{description}
\item[\rensuji{3}]第一項後段の規定は、資本関係があるかどうかの判定について準用する。
\end{description}
\noindent\hspace{10pt}(支払調書等の提出の特例)
\begin{description}
\item[第三百五十五条]法第二百二十八条の四第二項(支払調書等の提出の特例)の承認を受けようとする同項に規定する調書等を提出すべき者は、その者の氏名及び住所又は名称、所在地及び法人番号、その提出しようとする同項に規定する光ディスク等の種類その他の財務省令で定める事項を記載した申請書を、その者の同項に規定する所轄の税務署長(以下この条において「所轄の税務署長」という。)に提出しなければならない。
\item[\rensuji{2}]法第二百二十八条の四第三項の承認を受けようとする同項に規定する調書等を提出すべき者は、その者の氏名及び住所又は名称、所在地及び法人番号、当該調書等の同条第一項に規定する記載事項を提供しようとする税務署長その他の財務省令で定める事項を記載した申請書を、その者の所轄の税務署長に提出しなければならない。
\item[\rensuji{3}]前二項の所轄の税務署長は、これらの規定の申請書の提出があつた場合において、その申請につき承認をし、又は承認をしないこととしたときは、その申請をした者に対し、その旨を書面により通知するものとする。
\item[\rensuji{4}]第一項又は第二項の申請書の提出があつた場合において、その申請書の提出の日から二月を経過する日までにその申請につき承認をし、又は承認をしないこととした旨の通知がなかつたときは、同日においてその承認があつたものとみなす。
\end{description}
\noindent\hspace{10pt}(給与等、退職手当等又は公的年金等の支払明細書に記載すべき事項の電磁的方法による提供の承諾等)
\begin{description}
\item[第三百五十六条]居住者に対し国内において給与等、退職手当等又は公的年金等の支払をする者は、法第二百三十一条第二項本文(給与等、退職手当等又は公的年金等の支払明細書)の規定により同項に規定する給与等、退職手当等又は公的年金等の支払明細書に記載すべき事項を提供しようとするときは、財務省令で定めるところにより、あらかじめ、当該給与等、退職手当等又は公的年金等の支払を受ける者に対し、その用いる電磁的方法の種類及び内容を示し、書面又は電磁的方法による承諾を得なければならない。
\item[\rensuji{2}]前項の規定による承諾を得た給与等、退職手当等又は公的年金等の支払をする者は、当該給与等、退職手当等又は公的年金等の支払を受ける者から書面又は電磁的方法により法第二百三十一条第二項本文の規定による電磁的方法による提供を受けない旨の申出があつたときは、当該給与等、退職手当等又は公的年金等の支払を受ける者に対し、同項に規定する給与等、退職手当等又は公的年金等の支払明細書に記載すべき事項の提供を電磁的方法によつてしてはならない。
\end{description}
\end{document}
